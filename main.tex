% !TeX spellcheck = de_DE
\documentclass[12pt,bibstyle=none,pagenumberinfooter]{ifmdocument}


%general settings
\author{José Luis Muñoz Reyes}
\title{Einarbeitung}
\date{\the\day.\the\month.\the\year}
\headerblock{%
	{\bfseries\normalsize Institut für Mechanik}\\[0.2em]
	\thetitle\\
	\theauthor, \thedate
}

%usercommands
%-----------------------------------------------------------------------
%special user commands and packages
%-----------------------------------------------------------------------
%ENTER USER COMMANDS AND PACKAGES HERE

%-----------------------------------------------------------------------
%standard user commands and additional packages (most in ifmthesis.cls)
%-----------------------------------------------------------------------
%assembly symbol (simple)
\DeclareMathOperator*{\assem}{\text{\Large\ensuremath{\mathbf{\textsf A}}}}
\DeclareMathOperator*{\assemtext}{\text{{\ensuremath{\mathbf{\textsf A}}}}}
%functions, operators
\newcommand{\cof}{\ensuremath{\operatorname{cof}}}
\renewcommand{\div}{\ensuremath{\operatorname{div}}}
\newcommand{\Div}{\ensuremath{\operatorname{Div}}}
\newcommand{\Grad}{\ensuremath{\operatorname{Grad}}}
\newcommand{\grad}{\ensuremath{\operatorname{grad}}}
\newcommand{\Curl}{\ensuremath{\operatorname{Curl}}}
\newcommand{\curl}{\ensuremath{\operatorname{curl}}}
\newcommand{\tr}{\ensuremath{\operatorname{tr}}}
\newcommand{\sign}{\ensuremath{\operatorname{sign}}}
%differential signs
\renewcommand{\d}[1][]{\ensuremath{\,\mathrm{d}#1}}
\newcommand{\D}[1][]{\ensuremath{\mathrm{D}#1}}
\newcommand{\nablaq}[1]{\ensuremath{\nabla_{\hspace{-0.5ex}\mbox{\begin{scriptsize}\vec{#1}\end{scriptsize}}}}\hspace{0.1ex}}
%useful for continuum mechanics fem
\renewcommand{\phi}{\varphi}
\renewcommand{\epsilon}{\varepsilon}
\newcommand{\body}{\ensuremath{\mathcal{B}}}
\newcommand{\refgeb}{\hat{\Omega}}
\newcommand{\he}{\ensuremath{^{h,e}}}
\newcommand{\transp}{\mathrm{T}}
%misc
\newcommand{\degree}{\textdegree}
\def\clap#1{\hbox to 0pt{\hss#1\hss}}
\def\mathclap{\mathpalette\mathclapinternal}
\def\mathclapinternal#1#2{\clap{$\mathsurround=0pt#1{#2}$}}
\newcommand{\matlab}{\textsc{Matlab}}
%floats
\newcommand{\picfontsize}{\small}
\newenvironment{myfigure}{\begin{figure}[htb]\centering\picfontsize}{\end{figure}}
\newenvironment{mytable}{\begin{table}[htb]\centering}{\end{table}}
\hyphenation{in-te-gra-ble in-te-gra-bil-i-ty}% Silbentrennung
%dot and comma to end equations
\newcommand{\eqcomma}{\ensuremath{\text{,}}}
\newcommand{\eqdot}{\ensuremath{\text{.}}}
%vec and tens
\renewcommand{\vec}[1]{\ensuremath{\mbox{\boldmath $\mathrm{#1}$}}}
\newcommand{\tens}[1]{\vec{#1}}
%checkmark and xmark
\usepackage{bbding}
\newcommand{\cmark}{\Checkmark}
\newcommand{\xmark}{\XSolidBrush}
%Tensorcrossproduct
\usepackage{stackengine}
\def\stacktype{L}
\stackMath
\def\newhash{\mathrel{\abovebaseline[-.9ex]{\rotatebox{45}{\stackon[0pt]{%
					\stackon[0pt]{\rule[.28em]{.8em}{.045em}}{\rule[.67em]{.8em}{.045em}}%
				}{\rule[.1em]{.045em}{.8em}\rule{.3em}{0pt}\rule[.1em]{.045em}{.8em}}}}}}
\newcommand{\hhash}{\scalebox{0.75}{\raisebox{.7ex}{$\newhash$}}} 



% Tikz images
%\newlength\figH
%\newlength\figW




\begin{document}
\maketitle
\section{First task}

The following tensors are given:

\begin{gather}
    \vec{a} = \begin{bmatrix}
        3 \\ 1 \\ 2
    \end{bmatrix}, \quad
    \vec{A} = \begin{bmatrix}
        2 & 2 & 1 \\ 2 & -1 & 0 \\ 1 & 0 & 3
    \end{bmatrix} .
\end{gather}

The product $\Vec{b} = \Vec{A}\cdot \Vec{a}$ is defined as:

\begin{equation}
    b_i = \sum\limits_{i=1}^3 \sum\limits_{j=1}^3  A_{i,j} \, a_j
\end{equation}

and yields the vector

\begin{equation}
    \Vec{b} = \begin{bmatrix}
        3\cdot 2 + 2\cdot1 + 1\cdot2 \\ 2\cdot3 + (-1)\cdot1 + 0\cdot2 \\ 1\cdot3 + 0\cdot1 + 2\cdot3
    \end{bmatrix}
    = \begin{bmatrix}
        10 \\ 5 \\ 9
    \end{bmatrix}.
\end{equation}

The symmetric part of $\vec{A}$ is given by:

\begin{align}
    sym(\Vec{A})
     & =\frac{1}{2}\,(\vec{A} + \vec{A}^{T})                                          \\
     & = \frac{1}{2}\, \left(  \begin{bmatrix}
            2 & 2 & 1 \\ 2 & -1 & 0 \\ 1 & 0 & 3
        \end{bmatrix} +  \begin{bmatrix}
            2 & 2 & 1 \\ 2 & -1 & 0 \\ 1 & 0 & 3
        \end{bmatrix}
    \right)                                                                           \\
     & = \begin{bmatrix}
        2 & 2 & 1 \\ 2 & -1 & 0 \\ 1 & 0 & 3
    \end{bmatrix}
\end{align}

and the anti-symmetric part of $\vec{A}$ is given by:

\begin{align}
    skw(\Vec{A})
     & = \frac{1}{2}\left(\Vec{A} - \Vec{A}^T \right)                                  \\
     & = \frac{1}{2}\, \left(  \begin{bmatrix}
            2 & 2 & 1 \\ 2 & -1 & 0 \\ 1 & 0 & 3
        \end{bmatrix} -  \begin{bmatrix}
            2 & 2 & 1 \\ 2 & -1 & 0 \\ 1 & 0 & 3
        \end{bmatrix}
    \right)                                                                            \\
     & = \begin{bmatrix}
        0 & 0 & 0 \\ 0 & 0 & 0 \\ 0 & 0 & 0
    \end{bmatrix}
\end{align}



and hence

\begin{align}
    sym(\Vec{A}):skw(\vec{A})
     & = Sp\left(sym(\Vec{A})^T \cdot skw(\Vec{A}) \right)                            \\
     & = Sp \left(\begin{bmatrix}
            2 & 2 & 1 \\ 2 & -1 & 0 \\ 1 & 0 & 3
        \end{bmatrix} \cdot \begin{bmatrix}
            0 & 0 & 0 \\ 0 & 0 & 0 \\ 0 & 0 & 0
        \end{bmatrix} \right) \\
     & = Sp \begin{bmatrix}
        0 & 0 & 0 \\ 0 & 0 & 0 \\ 0 & 0 & 0
    \end{bmatrix}                                                \\
     & = 0.
\end{align}

The divergence of the tensor $ \Vec{A}$ is given by

\begin{align}
    \Div(\Vec{A})
     & = \Vec{A} \cdot \Vec{\nabla}                                  \\
     & = \begin{bmatrix}
        2 & 2 & 1 \\[2mm] 2 & -1 & 0 \\[2mm] 1 & 0 & 3
    \end{bmatrix} \cdot \begin{bmatrix}
        \frac{\partial}{\partial x_1} \\[2mm] \frac{\partial}{\partial x_2} \\[2mm]  \frac{\partial}{\partial x_3}
    \end{bmatrix} \\[2mm]
     & = \begin{bmatrix}
        \frac{\partial}{\partial x_1}\,(2) + \frac{\partial}{\partial x_2}\, (2) + \frac{\partial 1}{\partial x_3}\, (1)  \\[2mm]
        \frac{\partial}{\partial x_1}\,(2) + \frac{\partial}{\partial x_2}\, (-1) + \frac{\partial 1}{\partial x_3}\, (0) \\[2mm]
        \frac{\partial}{\partial x_1}\,(1) + \frac{\partial}{\partial x_2}\, (0) + \frac{\partial 1}{\partial x_3}\, (3)  \\
    \end{bmatrix}                                  \\[2mm]
     & = \begin{bmatrix}
        0 \\ 0 \\ 0
    \end{bmatrix}
\end{align}


\section{Second Task}

\subsection{System description and derivation of the movement equations using the Newtonian formalism}
\label{sec: NewtonianDerivation}


\begin{figure}[h]
    \centering
    \def\svgwidth{0.4\textwidth}
    \input{Figures/pendel.fig.pdf_tex}
    \caption{Schematic representation of the system}
    \label{fig:skizze}
\end{figure}



The mechanical system considered here consists of a mass point of mass $m$ attached to an ideal inelastic string of length $l$, which is fixed to a rotating bearing as shown in Figure \ref{fig:skizze}.


Since the system evolves in form of circular movement at constant curvature radius, polar coordinates are used to describe the movement. A free body diagram for the mass $m$ is shown in Figure \ref{fig:free body}. The forces considered here are gravity $\Vec{F_g} = m\cdot \Vec{g}$ and the reactive tension of the string $\Vec{F_{\eta}}$. Under these circumstances, moment balances in both radial and tangential direction yield the movement equations:

\begin{equation}

\label{eq: radialbalance}
    m \Ddot{r} = F_{\eta} - \cos{\phi}\,F_g = 0

\end{equation}

\begin{equation}
    \label{eq: tangentialbalance}
    m\, l \, \Ddot{\phi} = -m\, g \,\sin{\phi}
\end{equation}

Equation \ref{eq: radialbalance} can be simplified for small values of $\phi <<\, 1$ by using the approximation $\sin{\phi} \approx \phi $. This yields the linear movement equation:

\begin{equation}
    \Ddot{\phi} = -\frac{g}{l}\,\phi
\end{equation}





\begin{figure}[h]

    \centering
    \def\svgwidth{0.4\textwidth}
    \input{Figures/FreeBodyDiagram.pdf_tex}
    \caption{Free body diagram in polar coordinates}
    \label{fig:free body}
\end{figure}

\subsection{Derivation of the movement equations using the Lagrange formalism}
\label{sec: lagrange}

In Lagrangian mechanics, the movement equations are derived from a potential called the Lagrangian $L$, which can be written as:

\begin{equation}
    \label{eq: Lagrangian}
    L(\Vec{q}, \Vec{\Dot{q}}) = T(\Vec{\Dot{q}}) - V(\Vec{q}),
\end{equation}

where $T$ is the total kinetic energy and $V$ is the potential energy of the system. This function depends on the system describing variables $\Vec{q}$ and their time derivative $\Vec{\Dot{q}}$. The movement equations can then be derived from the Lagrangian equation of second kind:

\begin{equation}
    \label{eq: secondKindLagrangian}
    \frac{d}{dt}\left(\frac{\partial L}{\partial \Dot{q}}\right) - \left(\frac{\partial L}{\partial q}  \right) = \Vec{0}.
\end{equation}

Using the minimal coordinate $q$ = $\phi$, the position vector $\Vec{r}$ can be written as:

\begin{equation}
    \Vec{r} = \begin{bmatrix}
        l\,sin(\phi) \\ l\, cos(\phi)
    \end{bmatrix}
\end{equation}

Furthermore, the kinetic and potential energy are given by:

\begin{equation}
    T = \frac{1}{2}\,\Vec{\Dot{r}}^T\,m\,\Vec{\Dot{r}}
\end{equation}

and
\begin{equation}
    V = -m\,g\,l\,cos(\phi).
\end{equation}

Hence, the Lagrangian for the mathematical pendulum is given by:
\begin{equation}
    L = \frac{1}{2}\,\Vec{\Dot{r}}^T\,m\,\Vec{\Dot{r}} + m\,g\,l\,cos(\phi)
\end{equation}

and the required derivatives for Equation \ref{eq: Lagrangian} result to:

\begin{equation}
    \left(\frac{\partial L}{\partial \Dot{q}}\right) = \frac{\partial}{\partial \Dot{\phi}} \left( \frac{1}{2}\,m\,l^2\,(sin(\phi)^2 + cos(\phi)^2)\,\Dot{\phi}^2 \right) = m\,l^2\,\Dot{\phi}
\end{equation}

\begin{equation}
    \label{eq: LagFirstPart}
    \frac{d}{dt}\left(\frac{\partial L}{\partial \Dot{q}}\right) = m\,l^2\,\Ddot{\phi}
\end{equation}

and

\begin{equation}
    \label{eq: LagSecondPart}
    \left(\frac{\partial L}{\partial q}  \right) = \frac{\partial}{\partial \phi} \left(m\,g\,l\cos(\phi)\right) = -m\,g\,l\,sin(\phi).
\end{equation}

Plugging Equation \ref{eq: LagFirstPart} and \ref{eq: LagSecondPart} into Equation \ref{eq: secondKindLagrangian} yields the movement equation:

\begin{equation}
    m\,l^2\,\Ddot{\phi} + m\,g\,l\,sin(\phi) = 0,
\end{equation}

which is equivalent to Equation \ref{eq: tangentialbalance} derived in Section \ref{sec: NewtonianDerivation}.


\subsection{Analytical solution of the linear movement equation}

The linear movement equation is a ordinary differential equation (ODE) of second order. Due to the presence of constant coefficients, the following approach can be taken:

\begin{equation}
    \label{eq: linear}
    \phi (t) = \exp{(\lambda \, t)}.
\end{equation}

This approach leads to the characteristic equation:

\begin{equation}
    \lambda^2 + \frac{g}{l} = 0.
\end{equation}

The solutions to this equation assuming $\frac{g}{l}>0$ are $\lambda_{1,2} = \pm i\,\sqrt{\frac{g}{l}}$. Therefore, a general solution to Equation \ref{eq: linear} is given by:

\begin{equation}
    \phi (t) = A\,\exp{\left(\,i\sqrt{\frac{g}{l}}\,t\right)} + B\, \exp{\left(-\,i\sqrt{\frac{g}{l}}\,t\right)}
\end{equation}

The starting conditions $\Dot{\phi} = 0$ and $\phi = \phi_o$ at $t_o = 0$ yield the form:

\begin{align}
    \phi(t)
     & = \frac{\phi_o}{2}\, \left(\exp{\left(\,i\sqrt{\frac{g}{l}}\,t\right)} + \exp{\left(-\,i\sqrt{\frac{g}{l}}\,t\right)} \right) \\
     & = \phi_o\,\cos{\left(\sqrt{\frac{g}{l}}\,t \right)}
\end{align}

In the following, $\phi_o = 0.0875\,rad$ is assumed. Figure \ref{fig:LinSolutions} (top) shows the course of the analytical solution as a function of time.


\subsection{Numerical solution of the linear equation of movement}

For the numerical solution, three different integration schemes were used: explicit Euler (EE), implicit Euler (EI) and the mid-point rule (MP). First, the linear equation was expressed in the form of a first order system of PDEs by defining a vector:
\begin{equation}
    \Vec{X} = \begin{bmatrix}
        \phi \\ \Dot{\phi}
    \end{bmatrix},
\end{equation}

which yields the equivalent representation of Equation \ref{eq: linear} in the form:

\begin{equation}
    \Vec{\Dot{X}} = \begin{bmatrix}
        0 & 1 \\ -\frac{g}{L}\ & 0
    \end{bmatrix} \, \Vec{X}.
    \label{eq: firstOrderSysLin}
\end{equation}

Equation \ref{eq: firstOrderSysLin} is a first order equation of the form
\begin{equation}
    \Vec{\Dot{X}} = f(t,\Vec{X}),
    \label{eq: GeneralIntForm}
\end{equation}

which can be integrated by using the different standard numerical integration methods.

For the case of explicit Euler, the integration scheme is derived from a Taylor series expansion of the position and the velocity. This yields the following equations:

\begin{equation}
    \Vec{X}(t+\Delta t) = \Vec{X}(t) + [\Dot{\Vec{X}}]_t\,\Delta t + {O(\Delta t^2)},
\end{equation}

which yields a first order approximation of the tangent:

\begin{equation}
    [\Vec{\Dot{X}}]_t = \frac{\Vec{X}(t+\Delta t)-\Vec{X}(t)}{\Delta t} + O\left(\Delta t\right).
    \label{eq: EulerTanApprox}
\end{equation}

Plugging Equation \ref{eq: EulerTanApprox} into Equation \ref{eq: firstOrderSysLin}leads to the integration scheme:

\begin{equation}
    \Vec{X}(t+\Delta t) \approx \Vec{X}(t) + \begin{bmatrix}
        0 & 1 \\ -\frac{g}{L}\ & 0
    \end{bmatrix} \, \Vec{X}(t) \, \Delta t.
\end{equation}

Contrarily, the implicit Euler method approximates the tangent by taking the values at the end of the interesting time interval. This leads to a first order approximation as well. However, the resulting equation has to be solved explicitly:

\begin{equation}
    \Vec{X}(t+\Delta t) \approx \Vec{X}(t) + \begin{bmatrix}
        0 & 1 \\ -\frac{g}{L}\ & 0
    \end{bmatrix} \, \Vec{X}(t+\Delta t) \, \Delta t.
\end{equation}

Rearranging the equation leads to a linear equation system of the form:

\begin{equation}
    \left(\vec{I} - \begin{bmatrix}
        0 & 1 \\ -\frac{g}{L}\ & 0
    \end{bmatrix}\,\Delta t \right)\,\Vec{X}(t+\Delta t) \approx \Vec{X}(t),
\end{equation}
which can be solved by matrix inversion.

Finally, the midpoint rule approximates the tangent by taking its value at the middle of the interesting time interval:

\begin{equation}
    \Vec{\Dot{X}} = f\left( t+\frac{\Delta t}{2}, \Vec{X}(t) + \frac{\Delta t}{2}\,f\left(t, \Vec{X}(t)\right) \right).
\end{equation}

Applying this to Equation \ref{eq: firstOrderSysLin} yields the integration scheme:

\begin{equation}
    \Vec{X}(t+\Delta t) \approx \vec{X}(t) + \begin{bmatrix}
        0 & 1 \\ -\frac{g}{L}\ & 0
    \end{bmatrix}\,\left(\Vec{X}(t) + \begin{bmatrix}
            0 & 1 \\ -\frac{g}{L}\ & 0
        \end{bmatrix} \Vec{X}(t)\, \frac{\Delta t}{2} \right).
\end{equation}


\begin{figure}[h]
    \centering
    \setlength{\figH}{0.3\textheight}
    \setlength{\figW}{0.6\textwidth}
    % This file was created by matlab2tikz.
%
\definecolor{mycolor1}{rgb}{1.00000,0.00000,1.00000}%
%
\begin{tikzpicture}

\begin{axis}[%
width=0.951\figW,
height=0.5\figH,
at={(0\figW,0\figH)},
scale only axis,
xmin=0,
xmax=30,
xlabel style={font=\color{white!15!black}},
xlabel={Time [s]},
ymin=-0.4,
ymax=0.4,
ylabel style={font=\color{white!15!black}},
ylabel={Angle [rad]},
axis background/.style={fill=white},
legend style={at={(0.03,0.97)}, anchor=north west, legend cell align=left, align=left, draw=white!15!black}
]
\addplot[only marks, mark=*, mark options={}, mark size=0.5000pt, draw=blue] table[row sep=crcr]{%
x	y\\
0	0.0875\\
0.01	0.0875\\
0.02	0.08741\\
0.03	0.08724\\
0.04	0.08699\\
0.05	0.08664\\
0.06	0.08621\\
0.07	0.0857\\
0.08	0.0851\\
0.09	0.08442\\
0.1	0.08365\\
0.11	0.08281\\
0.12	0.08188\\
0.13	0.08086\\
0.14	0.07977\\
0.15	0.0786\\
0.16	0.07735\\
0.17	0.07603\\
0.18	0.07462\\
0.19	0.07315\\
0.2	0.0716\\
0.21	0.06997\\
0.22	0.06828\\
0.23	0.06652\\
0.24	0.06469\\
0.25	0.0628\\
0.26	0.06084\\
0.27	0.05882\\
0.28	0.05675\\
0.29	0.05461\\
0.3	0.05242\\
0.31	0.05017\\
0.32	0.04788\\
0.33	0.04553\\
0.34	0.04314\\
0.35	0.0407\\
0.36	0.03822\\
0.37	0.0357\\
0.38	0.03315\\
0.39	0.03056\\
0.4	0.02793\\
0.41	0.02528\\
0.42	0.0226\\
0.43	0.01989\\
0.44	0.01716\\
0.45	0.01441\\
0.46	0.01165\\
0.47	0.008867\\
0.48	0.006077\\
0.49	0.003277\\
0.5	0.0004719\\
0.51	-0.002337\\
0.52	-0.005146\\
0.53	-0.007952\\
0.54	-0.01075\\
0.55	-0.01355\\
0.56	-0.01633\\
0.57	-0.0191\\
0.58	-0.02186\\
0.59	-0.02459\\
0.6	-0.0273\\
0.61	-0.02999\\
0.62	-0.03266\\
0.63	-0.03529\\
0.64	-0.03789\\
0.65	-0.04046\\
0.66	-0.04299\\
0.67	-0.04548\\
0.68	-0.04793\\
0.69	-0.05033\\
0.7	-0.05269\\
0.71	-0.05499\\
0.72	-0.05725\\
0.73	-0.05945\\
0.74	-0.06159\\
0.75	-0.06368\\
0.76	-0.06571\\
0.77	-0.06767\\
0.78	-0.06957\\
0.79	-0.0714\\
0.8	-0.07317\\
0.81	-0.07486\\
0.82	-0.07649\\
0.83	-0.07803\\
0.84	-0.07951\\
0.85	-0.08091\\
0.86	-0.08223\\
0.87	-0.08347\\
0.88	-0.08463\\
0.89	-0.08571\\
0.9	-0.0867\\
0.91	-0.08761\\
0.92	-0.08844\\
0.93	-0.08918\\
0.94	-0.08983\\
0.95	-0.0904\\
0.96	-0.09087\\
0.97	-0.09126\\
0.98	-0.09156\\
0.99	-0.09177\\
1	-0.09189\\
1.01	-0.09192\\
1.02	-0.09186\\
1.03	-0.09171\\
1.04	-0.09147\\
1.05	-0.09114\\
1.06	-0.09072\\
1.07	-0.09021\\
1.08	-0.08961\\
1.09	-0.08893\\
1.1	-0.08815\\
1.11	-0.08729\\
1.12	-0.08634\\
1.13	-0.08531\\
1.14	-0.08419\\
1.15	-0.08299\\
1.16	-0.0817\\
1.17	-0.08034\\
1.18	-0.07889\\
1.19	-0.07736\\
1.2	-0.07576\\
1.21	-0.07408\\
1.22	-0.07233\\
1.23	-0.0705\\
1.24	-0.06861\\
1.25	-0.06664\\
1.26	-0.06461\\
1.27	-0.06251\\
1.28	-0.06035\\
1.29	-0.05813\\
1.3	-0.05584\\
1.31	-0.0535\\
1.32	-0.05111\\
1.33	-0.04866\\
1.34	-0.04617\\
1.35	-0.04362\\
1.36	-0.04103\\
1.37	-0.0384\\
1.38	-0.03573\\
1.39	-0.03302\\
1.4	-0.03027\\
1.41	-0.02749\\
1.42	-0.02468\\
1.43	-0.02185\\
1.44	-0.01899\\
1.45	-0.01611\\
1.46	-0.01321\\
1.47	-0.0103\\
1.48	-0.007369\\
1.49	-0.004432\\
1.5	-0.001487\\
1.51	0.001463\\
1.52	0.004413\\
1.53	0.007363\\
1.54	0.01031\\
1.55	0.01325\\
1.56	0.01617\\
1.57	0.01909\\
1.58	0.02199\\
1.59	0.02487\\
1.6	0.02773\\
1.61	0.03056\\
1.62	0.03337\\
1.63	0.03614\\
1.64	0.03889\\
1.65	0.0416\\
1.66	0.04427\\
1.67	0.0469\\
1.68	0.04948\\
1.69	0.05202\\
1.7	0.05452\\
1.71	0.05696\\
1.72	0.05934\\
1.73	0.06168\\
1.74	0.06395\\
1.75	0.06616\\
1.76	0.06831\\
1.77	0.0704\\
1.78	0.07241\\
1.79	0.07436\\
1.8	0.07624\\
1.81	0.07805\\
1.82	0.07978\\
1.83	0.08143\\
1.84	0.08301\\
1.85	0.0845\\
1.86	0.08592\\
1.87	0.08725\\
1.88	0.08849\\
1.89	0.08966\\
1.9	0.09073\\
1.91	0.09172\\
1.92	0.09261\\
1.93	0.09342\\
1.94	0.09414\\
1.95	0.09476\\
1.96	0.0953\\
1.97	0.09574\\
1.98	0.09608\\
1.99	0.09633\\
2	0.09649\\
2.01	0.09656\\
2.02	0.09653\\
2.03	0.0964\\
2.04	0.09618\\
2.05	0.09586\\
2.06	0.09545\\
2.07	0.09495\\
2.08	0.09435\\
2.09	0.09366\\
2.1	0.09288\\
2.11	0.09201\\
2.12	0.09104\\
2.13	0.08999\\
2.14	0.08884\\
2.15	0.08761\\
2.16	0.08629\\
2.17	0.08488\\
2.18	0.08339\\
2.19	0.08181\\
2.2	0.08016\\
2.21	0.07842\\
2.22	0.0766\\
2.23	0.07471\\
2.24	0.07274\\
2.25	0.0707\\
2.26	0.06859\\
2.27	0.06641\\
2.28	0.06416\\
2.29	0.06185\\
2.3	0.05947\\
2.31	0.05703\\
2.32	0.05454\\
2.33	0.05199\\
2.34	0.04938\\
2.35	0.04672\\
2.36	0.04402\\
2.37	0.04127\\
2.38	0.03848\\
2.39	0.03564\\
2.4	0.03277\\
2.41	0.02986\\
2.42	0.02692\\
2.43	0.02396\\
2.44	0.02096\\
2.45	0.01794\\
2.46	0.0149\\
2.47	0.01185\\
2.48	0.008776\\
2.49	0.005693\\
2.5	0.002602\\
2.51	-0.0004948\\
2.52	-0.003594\\
2.53	-0.006693\\
2.54	-0.009789\\
2.55	-0.01288\\
2.56	-0.01596\\
2.57	-0.01902\\
2.58	-0.02207\\
2.59	-0.02511\\
2.6	-0.02812\\
2.61	-0.0311\\
2.62	-0.03406\\
2.63	-0.03699\\
2.64	-0.03988\\
2.65	-0.04274\\
2.66	-0.04556\\
2.67	-0.04834\\
2.68	-0.05107\\
2.69	-0.05376\\
2.7	-0.05639\\
2.71	-0.05897\\
2.72	-0.0615\\
2.73	-0.06397\\
2.74	-0.06638\\
2.75	-0.06872\\
2.76	-0.071\\
2.77	-0.07322\\
2.78	-0.07536\\
2.79	-0.07743\\
2.8	-0.07943\\
2.81	-0.08135\\
2.82	-0.0832\\
2.83	-0.08496\\
2.84	-0.08664\\
2.85	-0.08824\\
2.86	-0.08976\\
2.87	-0.09119\\
2.88	-0.09253\\
2.89	-0.09378\\
2.9	-0.09494\\
2.91	-0.096\\
2.92	-0.09698\\
2.93	-0.09786\\
2.94	-0.09864\\
2.95	-0.09933\\
2.96	-0.09992\\
2.97	-0.1004\\
2.98	-0.1008\\
2.99	-0.1011\\
3	-0.1013\\
3.01	-0.1014\\
3.02	-0.1014\\
3.03	-0.1013\\
3.04	-0.1011\\
3.05	-0.1008\\
3.06	-0.1004\\
3.07	-0.09993\\
3.08	-0.09933\\
3.09	-0.09864\\
3.1	-0.09785\\
3.11	-0.09697\\
3.12	-0.09598\\
3.13	-0.09491\\
3.14	-0.09373\\
3.15	-0.09247\\
3.16	-0.09111\\
3.17	-0.08966\\
3.18	-0.08813\\
3.19	-0.0865\\
3.2	-0.08479\\
3.21	-0.083\\
3.22	-0.08112\\
3.23	-0.07916\\
3.24	-0.07712\\
3.25	-0.075\\
3.26	-0.0728\\
3.27	-0.07054\\
3.28	-0.0682\\
3.29	-0.06579\\
3.3	-0.06332\\
3.31	-0.06078\\
3.32	-0.05818\\
3.33	-0.05551\\
3.34	-0.0528\\
3.35	-0.05002\\
3.36	-0.0472\\
3.37	-0.04433\\
3.38	-0.04141\\
3.39	-0.03844\\
3.4	-0.03544\\
3.41	-0.0324\\
3.42	-0.02932\\
3.43	-0.02621\\
3.44	-0.02308\\
3.45	-0.01992\\
3.46	-0.01673\\
3.47	-0.01353\\
3.48	-0.0103\\
3.49	-0.00707\\
3.5	-0.003826\\
3.51	-0.0005741\\
3.52	0.002681\\
3.53	0.005937\\
3.54	0.00919\\
3.55	0.01244\\
3.56	0.01568\\
3.57	0.0189\\
3.58	0.02211\\
3.59	0.0253\\
3.6	0.02847\\
3.61	0.03162\\
3.62	0.03474\\
3.63	0.03782\\
3.64	0.04088\\
3.65	0.04389\\
3.66	0.04687\\
3.67	0.0498\\
3.68	0.05269\\
3.69	0.05552\\
3.7	0.05831\\
3.71	0.06104\\
3.72	0.06372\\
3.73	0.06633\\
3.74	0.06888\\
3.75	0.07137\\
3.76	0.07379\\
3.77	0.07614\\
3.78	0.07841\\
3.79	0.08061\\
3.8	0.08274\\
3.81	0.08479\\
3.82	0.08675\\
3.83	0.08863\\
3.84	0.09043\\
3.85	0.09214\\
3.86	0.09376\\
3.87	0.09529\\
3.88	0.09673\\
3.89	0.09807\\
3.9	0.09932\\
3.91	0.1005\\
3.92	0.1015\\
3.93	0.1025\\
3.94	0.1034\\
3.95	0.1041\\
3.96	0.1048\\
3.97	0.1053\\
3.98	0.1058\\
3.99	0.1061\\
4	0.1064\\
4.01	0.1065\\
4.02	0.1065\\
4.03	0.1065\\
4.04	0.1063\\
4.05	0.106\\
4.06	0.1056\\
4.07	0.1052\\
4.08	0.1046\\
4.09	0.1039\\
4.1	0.1031\\
4.11	0.1022\\
4.12	0.1012\\
4.13	0.1001\\
4.14	0.09888\\
4.15	0.09759\\
4.16	0.0962\\
4.17	0.09471\\
4.18	0.09312\\
4.19	0.09145\\
4.2	0.08968\\
4.21	0.08782\\
4.22	0.08588\\
4.23	0.08385\\
4.24	0.08173\\
4.25	0.07954\\
4.26	0.07726\\
4.27	0.0749\\
4.28	0.07247\\
4.29	0.06997\\
4.3	0.06739\\
4.31	0.06475\\
4.32	0.06203\\
4.33	0.05926\\
4.34	0.05642\\
4.35	0.05353\\
4.36	0.05058\\
4.37	0.04758\\
4.38	0.04453\\
4.39	0.04143\\
4.4	0.03829\\
4.41	0.03511\\
4.42	0.03189\\
4.43	0.02864\\
4.44	0.02535\\
4.45	0.02204\\
4.46	0.0187\\
4.47	0.01534\\
4.48	0.01196\\
4.49	0.00857\\
4.5	0.005165\\
4.51	0.001752\\
4.52	-0.001667\\
4.53	-0.005087\\
4.54	-0.008505\\
4.55	-0.01192\\
4.56	-0.01532\\
4.57	-0.01872\\
4.58	-0.0221\\
4.59	-0.02546\\
4.6	-0.02879\\
4.61	-0.03211\\
4.62	-0.03539\\
4.63	-0.03865\\
4.64	-0.04186\\
4.65	-0.04505\\
4.66	-0.04819\\
4.67	-0.05128\\
4.68	-0.05433\\
4.69	-0.05733\\
4.7	-0.06027\\
4.71	-0.06316\\
4.72	-0.06599\\
4.73	-0.06876\\
4.74	-0.07146\\
4.75	-0.0741\\
4.76	-0.07666\\
4.77	-0.07915\\
4.78	-0.08157\\
4.79	-0.08391\\
4.8	-0.08617\\
4.81	-0.08835\\
4.82	-0.09044\\
4.83	-0.09245\\
4.84	-0.09436\\
4.85	-0.09619\\
4.86	-0.09792\\
4.87	-0.09956\\
4.88	-0.1011\\
4.89	-0.1026\\
4.9	-0.1039\\
4.91	-0.1051\\
4.92	-0.1063\\
4.93	-0.1073\\
4.94	-0.1083\\
4.95	-0.1091\\
4.96	-0.1098\\
4.97	-0.1104\\
4.98	-0.111\\
4.99	-0.1114\\
5	-0.1117\\
5.01	-0.1118\\
5.02	-0.1119\\
5.03	-0.1119\\
5.04	-0.1117\\
5.05	-0.1115\\
5.06	-0.1111\\
5.07	-0.1106\\
5.08	-0.1101\\
5.09	-0.1094\\
5.1	-0.1086\\
5.11	-0.1077\\
5.12	-0.1066\\
5.13	-0.1055\\
5.14	-0.1043\\
5.15	-0.103\\
5.16	-0.1015\\
5.17	-0.1\\
5.18	-0.09839\\
5.19	-0.09666\\
5.2	-0.09484\\
5.21	-0.09292\\
5.22	-0.09091\\
5.23	-0.0888\\
5.24	-0.08661\\
5.25	-0.08433\\
5.26	-0.08197\\
5.27	-0.07952\\
5.28	-0.07699\\
5.29	-0.07439\\
5.3	-0.07171\\
5.31	-0.06895\\
5.32	-0.06613\\
5.33	-0.06323\\
5.34	-0.06028\\
5.35	-0.05726\\
5.36	-0.05418\\
5.37	-0.05105\\
5.38	-0.04786\\
5.39	-0.04462\\
5.4	-0.04134\\
5.41	-0.03801\\
5.42	-0.03464\\
5.43	-0.03123\\
5.44	-0.02779\\
5.45	-0.02432\\
5.46	-0.02082\\
5.47	-0.0173\\
5.48	-0.01376\\
5.49	-0.0102\\
5.5	-0.006629\\
5.51	-0.003046\\
5.52	0.0005434\\
5.53	0.004136\\
5.54	0.007727\\
5.55	0.01132\\
5.56	0.0149\\
5.57	0.01846\\
5.58	0.02202\\
5.59	0.02555\\
5.6	0.02907\\
5.61	0.03256\\
5.62	0.03602\\
5.63	0.03945\\
5.64	0.04284\\
5.65	0.0462\\
5.66	0.04951\\
5.67	0.05278\\
5.68	0.056\\
5.69	0.05917\\
5.7	0.06228\\
5.71	0.06533\\
5.72	0.06833\\
5.73	0.07126\\
5.74	0.07412\\
5.75	0.07691\\
5.76	0.07963\\
5.77	0.08227\\
5.78	0.08484\\
5.79	0.08732\\
5.8	0.08973\\
5.81	0.09204\\
5.82	0.09427\\
5.83	0.09641\\
5.84	0.09846\\
5.85	0.1004\\
5.86	0.1023\\
5.87	0.104\\
5.88	0.1057\\
5.89	0.1072\\
5.9	0.1087\\
5.91	0.11\\
5.92	0.1113\\
5.93	0.1124\\
5.94	0.1134\\
5.95	0.1143\\
5.96	0.1151\\
5.97	0.1158\\
5.98	0.1164\\
5.99	0.1168\\
6	0.1172\\
6.01	0.1174\\
6.02	0.1175\\
6.03	0.1175\\
6.04	0.1174\\
6.05	0.1172\\
6.06	0.1169\\
6.07	0.1164\\
6.08	0.1158\\
6.09	0.1151\\
6.1	0.1143\\
6.11	0.1134\\
6.12	0.1124\\
6.13	0.1113\\
6.14	0.11\\
6.15	0.1087\\
6.16	0.1072\\
6.17	0.1056\\
6.18	0.1039\\
6.19	0.1022\\
6.2	0.1003\\
6.21	0.09829\\
6.22	0.09621\\
6.23	0.09403\\
6.24	0.09176\\
6.25	0.0894\\
6.26	0.08694\\
6.27	0.0844\\
6.28	0.08178\\
6.29	0.07907\\
6.3	0.07628\\
6.31	0.07341\\
6.32	0.07047\\
6.33	0.06745\\
6.34	0.06437\\
6.35	0.06122\\
6.36	0.05801\\
6.37	0.05473\\
6.38	0.0514\\
6.39	0.04802\\
6.4	0.04459\\
6.41	0.04111\\
6.42	0.03758\\
6.43	0.03402\\
6.44	0.03041\\
6.45	0.02678\\
6.46	0.02311\\
6.47	0.01942\\
6.48	0.01571\\
6.49	0.01197\\
6.5	0.008226\\
6.51	0.004465\\
6.52	0.000697\\
6.53	-0.003076\\
6.54	-0.006849\\
6.55	-0.01062\\
6.56	-0.01438\\
6.57	-0.01814\\
6.58	-0.02188\\
6.59	-0.0256\\
6.6	-0.0293\\
6.61	-0.03297\\
6.62	-0.03662\\
6.63	-0.04023\\
6.64	-0.04381\\
6.65	-0.04735\\
6.66	-0.05084\\
6.67	-0.05429\\
6.68	-0.05769\\
6.69	-0.06104\\
6.7	-0.06433\\
6.71	-0.06756\\
6.72	-0.07072\\
6.73	-0.07382\\
6.74	-0.07685\\
6.75	-0.07981\\
6.76	-0.08269\\
6.77	-0.0855\\
6.78	-0.08822\\
6.79	-0.09086\\
6.8	-0.09341\\
6.81	-0.09588\\
6.82	-0.09825\\
6.83	-0.1005\\
6.84	-0.1027\\
6.85	-0.1048\\
6.86	-0.1068\\
6.87	-0.1087\\
6.88	-0.1104\\
6.89	-0.1121\\
6.9	-0.1137\\
6.91	-0.1151\\
6.92	-0.1164\\
6.93	-0.1177\\
6.94	-0.1188\\
6.95	-0.1198\\
6.96	-0.1207\\
6.97	-0.1214\\
6.98	-0.1221\\
6.99	-0.1226\\
7	-0.123\\
7.01	-0.1233\\
7.02	-0.1234\\
7.03	-0.1235\\
7.04	-0.1234\\
7.05	-0.1232\\
7.06	-0.1229\\
7.07	-0.1224\\
7.08	-0.1219\\
7.09	-0.1212\\
7.1	-0.1204\\
7.11	-0.1195\\
7.12	-0.1184\\
7.13	-0.1173\\
7.14	-0.116\\
7.15	-0.1146\\
7.16	-0.1131\\
7.17	-0.1115\\
7.18	-0.1098\\
7.19	-0.1079\\
7.2	-0.106\\
7.21	-0.104\\
7.22	-0.1018\\
7.23	-0.09956\\
7.24	-0.0972\\
7.25	-0.09475\\
7.26	-0.09221\\
7.27	-0.08957\\
7.28	-0.08684\\
7.29	-0.08402\\
7.3	-0.08112\\
7.31	-0.07813\\
7.32	-0.07507\\
7.33	-0.07193\\
7.34	-0.06871\\
7.35	-0.06543\\
7.36	-0.06207\\
7.37	-0.05866\\
7.38	-0.05518\\
7.39	-0.05164\\
7.4	-0.04805\\
7.41	-0.04441\\
7.42	-0.04072\\
7.43	-0.03699\\
7.44	-0.03322\\
7.45	-0.02942\\
7.46	-0.02558\\
7.47	-0.02171\\
7.48	-0.01781\\
7.49	-0.0139\\
7.5	-0.009965\\
7.51	-0.006019\\
7.52	-0.002063\\
7.53	0.001898\\
7.54	0.005862\\
7.55	0.009824\\
7.56	0.01378\\
7.57	0.01773\\
7.58	0.02166\\
7.59	0.02558\\
7.6	0.02947\\
7.61	0.03334\\
7.62	0.03718\\
7.63	0.04099\\
7.64	0.04476\\
7.65	0.04849\\
7.66	0.05218\\
7.67	0.05582\\
7.68	0.05941\\
7.69	0.06294\\
7.7	0.06641\\
7.71	0.06983\\
7.72	0.07318\\
7.73	0.07646\\
7.74	0.07967\\
7.75	0.0828\\
7.76	0.08585\\
7.77	0.08883\\
7.78	0.09172\\
7.79	0.09452\\
7.8	0.09723\\
7.81	0.09985\\
7.82	0.1024\\
7.83	0.1048\\
7.84	0.1071\\
7.85	0.1094\\
7.86	0.1115\\
7.87	0.1135\\
7.88	0.1154\\
7.89	0.1172\\
7.9	0.1188\\
7.91	0.1204\\
7.92	0.1219\\
7.93	0.1232\\
7.94	0.1244\\
7.95	0.1255\\
7.96	0.1264\\
7.97	0.1273\\
7.98	0.128\\
7.99	0.1286\\
8	0.1291\\
8.01	0.1294\\
8.02	0.1296\\
8.03	0.1297\\
8.04	0.1297\\
8.05	0.1295\\
8.06	0.1292\\
8.07	0.1288\\
8.08	0.1282\\
8.09	0.1276\\
8.1	0.1268\\
8.11	0.1258\\
8.12	0.1248\\
8.13	0.1236\\
8.14	0.1223\\
8.15	0.1209\\
8.16	0.1194\\
8.17	0.1177\\
8.18	0.1159\\
8.19	0.1141\\
8.2	0.1121\\
8.21	0.1099\\
8.22	0.1077\\
8.23	0.1054\\
8.24	0.1029\\
8.25	0.1004\\
8.26	0.09776\\
8.27	0.09502\\
8.28	0.09219\\
8.29	0.08926\\
8.3	0.08624\\
8.31	0.08313\\
8.32	0.07994\\
8.33	0.07667\\
8.34	0.07332\\
8.35	0.06989\\
8.36	0.06639\\
8.37	0.06283\\
8.38	0.05919\\
8.39	0.0555\\
8.4	0.05175\\
8.41	0.04794\\
8.42	0.04409\\
8.43	0.04018\\
8.44	0.03623\\
8.45	0.03225\\
8.46	0.02822\\
8.47	0.02417\\
8.48	0.02009\\
8.49	0.01598\\
8.5	0.01186\\
8.51	0.007717\\
8.52	0.003565\\
8.53	-0.0005949\\
8.54	-0.004758\\
8.55	-0.008921\\
8.56	-0.01308\\
8.57	-0.01723\\
8.58	-0.02137\\
8.59	-0.02549\\
8.6	-0.02958\\
8.61	-0.03366\\
8.62	-0.0377\\
8.63	-0.04171\\
8.64	-0.04569\\
8.65	-0.04962\\
8.66	-0.05351\\
8.67	-0.05735\\
8.68	-0.06114\\
8.69	-0.06487\\
8.7	-0.06854\\
8.71	-0.07215\\
8.72	-0.07569\\
8.73	-0.07916\\
8.74	-0.08256\\
8.75	-0.08588\\
8.76	-0.08911\\
8.77	-0.09227\\
8.78	-0.09533\\
8.79	-0.09831\\
8.8	-0.1012\\
8.81	-0.104\\
8.82	-0.1067\\
8.83	-0.1092\\
8.84	-0.1117\\
8.85	-0.1141\\
8.86	-0.1164\\
8.87	-0.1185\\
8.88	-0.1205\\
8.89	-0.1225\\
8.9	-0.1243\\
8.91	-0.1259\\
8.92	-0.1275\\
8.93	-0.1289\\
8.94	-0.1303\\
8.95	-0.1314\\
8.96	-0.1325\\
8.97	-0.1334\\
8.98	-0.1342\\
8.99	-0.1349\\
9	-0.1354\\
9.01	-0.1358\\
9.02	-0.1361\\
9.03	-0.1362\\
9.04	-0.1362\\
9.05	-0.1361\\
9.06	-0.1358\\
9.07	-0.1354\\
9.08	-0.1349\\
9.09	-0.1342\\
9.1	-0.1335\\
9.11	-0.1325\\
9.12	-0.1315\\
9.13	-0.1303\\
9.14	-0.129\\
9.15	-0.1275\\
9.16	-0.1259\\
9.17	-0.1242\\
9.18	-0.1224\\
9.19	-0.1205\\
9.2	-0.1184\\
9.21	-0.1162\\
9.22	-0.1139\\
9.23	-0.1115\\
9.24	-0.109\\
9.25	-0.1064\\
9.26	-0.1036\\
9.27	-0.1008\\
9.28	-0.09785\\
9.29	-0.09481\\
9.3	-0.09167\\
9.31	-0.08843\\
9.32	-0.08511\\
9.33	-0.0817\\
9.34	-0.07821\\
9.35	-0.07463\\
9.36	-0.07098\\
9.37	-0.06726\\
9.38	-0.06347\\
9.39	-0.05961\\
9.4	-0.05569\\
9.41	-0.05171\\
9.42	-0.04768\\
9.43	-0.04359\\
9.44	-0.03946\\
9.45	-0.03528\\
9.46	-0.03107\\
9.47	-0.02682\\
9.48	-0.02255\\
9.49	-0.01824\\
9.5	-0.01391\\
9.51	-0.00957\\
9.52	-0.005212\\
9.53	-0.0008445\\
9.54	0.003528\\
9.55	0.007902\\
9.56	0.01227\\
9.57	0.01663\\
9.58	0.02098\\
9.59	0.02532\\
9.6	0.02963\\
9.61	0.03392\\
9.62	0.03818\\
9.63	0.04241\\
9.64	0.04659\\
9.65	0.05074\\
9.66	0.05484\\
9.67	0.05889\\
9.68	0.06289\\
9.69	0.06683\\
9.7	0.07071\\
9.71	0.07452\\
9.72	0.07827\\
9.73	0.08194\\
9.74	0.08553\\
9.75	0.08904\\
9.76	0.09247\\
9.77	0.09581\\
9.78	0.09907\\
9.79	0.1022\\
9.8	0.1053\\
9.81	0.1082\\
9.82	0.1111\\
9.83	0.1138\\
9.84	0.1165\\
9.85	0.119\\
9.86	0.1214\\
9.87	0.1237\\
9.88	0.1259\\
9.89	0.128\\
9.9	0.1299\\
9.91	0.1317\\
9.92	0.1334\\
9.93	0.1349\\
9.94	0.1364\\
9.95	0.1377\\
9.96	0.1388\\
9.97	0.1398\\
9.98	0.1407\\
9.99	0.1415\\
10	0.1421\\
10.01	0.1425\\
10.02	0.1429\\
10.03	0.1431\\
10.04	0.1431\\
10.05	0.143\\
10.06	0.1428\\
10.07	0.1424\\
10.08	0.1419\\
10.09	0.1413\\
10.1	0.1405\\
10.11	0.1395\\
10.12	0.1385\\
10.13	0.1373\\
10.14	0.1359\\
10.15	0.1345\\
10.16	0.1329\\
10.17	0.1311\\
10.18	0.1293\\
10.19	0.1273\\
10.2	0.1251\\
10.21	0.1229\\
10.22	0.1205\\
10.23	0.118\\
10.24	0.1154\\
10.25	0.1127\\
10.26	0.1098\\
10.27	0.1069\\
10.28	0.1038\\
10.29	0.1007\\
10.3	0.09741\\
10.31	0.09404\\
10.32	0.09059\\
10.33	0.08703\\
10.34	0.08339\\
10.35	0.07967\\
10.36	0.07586\\
10.37	0.07198\\
10.38	0.06802\\
10.39	0.06399\\
10.4	0.05989\\
10.41	0.05573\\
10.42	0.05151\\
10.43	0.04724\\
10.44	0.04291\\
10.45	0.03854\\
10.46	0.03413\\
10.47	0.02968\\
10.48	0.0252\\
10.49	0.02068\\
10.5	0.01615\\
10.51	0.01159\\
10.52	0.007015\\
10.53	0.00243\\
10.54	-0.002162\\
10.55	-0.006756\\
10.56	-0.01135\\
10.57	-0.01593\\
10.58	-0.02051\\
10.59	-0.02507\\
10.6	-0.02961\\
10.61	-0.03412\\
10.62	-0.0386\\
10.63	-0.04306\\
10.64	-0.04747\\
10.65	-0.05184\\
10.66	-0.05617\\
10.67	-0.06044\\
10.68	-0.06466\\
10.69	-0.06882\\
10.7	-0.07291\\
10.71	-0.07694\\
10.72	-0.0809\\
10.73	-0.08478\\
10.74	-0.08858\\
10.75	-0.0923\\
10.76	-0.09593\\
10.77	-0.09948\\
10.78	-0.1029\\
10.79	-0.1063\\
10.8	-0.1095\\
10.81	-0.1127\\
10.82	-0.1157\\
10.83	-0.1186\\
10.84	-0.1214\\
10.85	-0.1241\\
10.86	-0.1267\\
10.87	-0.1292\\
10.88	-0.1315\\
10.89	-0.1337\\
10.9	-0.1358\\
10.91	-0.1377\\
10.92	-0.1395\\
10.93	-0.1412\\
10.94	-0.1428\\
10.95	-0.1442\\
10.96	-0.1454\\
10.97	-0.1465\\
10.98	-0.1475\\
10.99	-0.1484\\
11	-0.149\\
11.01	-0.1496\\
11.02	-0.15\\
11.03	-0.1502\\
11.04	-0.1503\\
11.05	-0.1503\\
11.06	-0.1501\\
11.07	-0.1498\\
11.08	-0.1493\\
11.09	-0.1486\\
11.1	-0.1479\\
11.11	-0.1469\\
11.12	-0.1459\\
11.13	-0.1446\\
11.14	-0.1433\\
11.15	-0.1418\\
11.16	-0.1401\\
11.17	-0.1384\\
11.18	-0.1364\\
11.19	-0.1344\\
11.2	-0.1322\\
11.21	-0.1299\\
11.22	-0.1274\\
11.23	-0.1249\\
11.24	-0.1222\\
11.25	-0.1193\\
11.26	-0.1164\\
11.27	-0.1133\\
11.28	-0.1102\\
11.29	-0.1069\\
11.3	-0.1035\\
11.31	-0.09998\\
11.32	-0.09638\\
11.33	-0.09269\\
11.34	-0.08889\\
11.35	-0.08501\\
11.36	-0.08104\\
11.37	-0.07699\\
11.38	-0.07285\\
11.39	-0.06864\\
11.4	-0.06436\\
11.41	-0.06001\\
11.42	-0.0556\\
11.43	-0.05113\\
11.44	-0.04661\\
11.45	-0.04203\\
11.46	-0.03741\\
11.47	-0.03275\\
11.48	-0.02805\\
11.49	-0.02332\\
11.5	-0.01856\\
11.51	-0.01378\\
11.52	-0.008985\\
11.53	-0.004172\\
11.54	0.0006491\\
11.55	0.005475\\
11.56	0.0103\\
11.57	0.01512\\
11.58	0.01993\\
11.59	0.02472\\
11.6	0.0295\\
11.61	0.03425\\
11.62	0.03897\\
11.63	0.04366\\
11.64	0.04831\\
11.65	0.05292\\
11.66	0.05748\\
11.67	0.06198\\
11.68	0.06644\\
11.69	0.07083\\
11.7	0.07515\\
11.71	0.07941\\
11.72	0.08359\\
11.73	0.08769\\
11.74	0.09172\\
11.75	0.09565\\
11.76	0.0995\\
11.77	0.1033\\
11.78	0.1069\\
11.79	0.1105\\
11.8	0.1139\\
11.81	0.1173\\
11.82	0.1205\\
11.83	0.1236\\
11.84	0.1266\\
11.85	0.1295\\
11.86	0.1322\\
11.87	0.1348\\
11.88	0.1373\\
11.89	0.1397\\
11.9	0.1419\\
11.91	0.144\\
11.92	0.146\\
11.93	0.1478\\
11.94	0.1494\\
11.95	0.151\\
11.96	0.1523\\
11.97	0.1536\\
11.98	0.1546\\
11.99	0.1556\\
12	0.1563\\
12.01	0.1569\\
12.02	0.1574\\
12.03	0.1577\\
12.04	0.1579\\
12.05	0.1579\\
12.06	0.1577\\
12.07	0.1574\\
12.08	0.157\\
12.09	0.1564\\
12.1	0.1556\\
12.11	0.1547\\
12.12	0.1536\\
12.13	0.1524\\
12.14	0.151\\
12.15	0.1495\\
12.16	0.1478\\
12.17	0.146\\
12.18	0.144\\
12.19	0.1419\\
12.2	0.1397\\
12.21	0.1373\\
12.22	0.1347\\
12.23	0.1321\\
12.24	0.1293\\
12.25	0.1264\\
12.26	0.1233\\
12.27	0.1201\\
12.28	0.1169\\
12.29	0.1134\\
12.3	0.1099\\
12.31	0.1063\\
12.32	0.1025\\
12.33	0.09868\\
12.34	0.09473\\
12.35	0.09068\\
12.36	0.08654\\
12.37	0.08231\\
12.38	0.07799\\
12.39	0.0736\\
12.4	0.06913\\
12.41	0.06458\\
12.42	0.05997\\
12.43	0.0553\\
12.44	0.05056\\
12.45	0.04577\\
12.46	0.04094\\
12.47	0.03605\\
12.48	0.03113\\
12.49	0.02617\\
12.5	0.02118\\
12.51	0.01617\\
12.52	0.01113\\
12.53	0.006083\\
12.54	0.001021\\
12.55	-0.004047\\
12.56	-0.009116\\
12.57	-0.01418\\
12.58	-0.01924\\
12.59	-0.02428\\
12.6	-0.0293\\
12.61	-0.0343\\
12.62	-0.03927\\
12.63	-0.04421\\
12.64	-0.04911\\
12.65	-0.05396\\
12.66	-0.05877\\
12.67	-0.06352\\
12.68	-0.06822\\
12.69	-0.07285\\
12.7	-0.07742\\
12.71	-0.08192\\
12.72	-0.08634\\
12.73	-0.09068\\
12.74	-0.09493\\
12.75	-0.0991\\
12.76	-0.1032\\
12.77	-0.1071\\
12.78	-0.111\\
12.79	-0.1148\\
12.8	-0.1184\\
12.81	-0.122\\
12.82	-0.1254\\
12.83	-0.1287\\
12.84	-0.1319\\
12.85	-0.135\\
12.86	-0.1379\\
12.87	-0.1407\\
12.88	-0.1434\\
12.89	-0.1459\\
12.9	-0.1483\\
12.91	-0.1505\\
12.92	-0.1526\\
12.93	-0.1546\\
12.94	-0.1564\\
12.95	-0.158\\
12.96	-0.1595\\
12.97	-0.1609\\
12.98	-0.1621\\
12.99	-0.1631\\
13	-0.164\\
13.01	-0.1647\\
13.02	-0.1652\\
13.03	-0.1656\\
13.04	-0.1658\\
13.05	-0.1659\\
13.06	-0.1658\\
13.07	-0.1655\\
13.08	-0.1651\\
13.09	-0.1645\\
13.1	-0.1637\\
13.11	-0.1628\\
13.12	-0.1617\\
13.13	-0.1605\\
13.14	-0.1591\\
13.15	-0.1576\\
13.16	-0.1559\\
13.17	-0.154\\
13.18	-0.152\\
13.19	-0.1498\\
13.2	-0.1475\\
13.21	-0.1451\\
13.22	-0.1424\\
13.23	-0.1397\\
13.24	-0.1368\\
13.25	-0.1338\\
13.26	-0.1306\\
13.27	-0.1273\\
13.28	-0.1239\\
13.29	-0.1204\\
13.3	-0.1167\\
13.31	-0.1129\\
13.32	-0.109\\
13.33	-0.105\\
13.34	-0.1009\\
13.35	-0.09669\\
13.36	-0.09237\\
13.37	-0.08796\\
13.38	-0.08346\\
13.39	-0.07887\\
13.4	-0.0742\\
13.41	-0.06945\\
13.42	-0.06463\\
13.43	-0.05974\\
13.44	-0.05479\\
13.45	-0.04978\\
13.46	-0.04471\\
13.47	-0.0396\\
13.48	-0.03444\\
13.49	-0.02925\\
13.5	-0.02402\\
13.51	-0.01876\\
13.52	-0.01348\\
13.53	-0.008175\\
13.54	-0.002861\\
13.55	0.002461\\
13.56	0.007786\\
13.57	0.01311\\
13.58	0.01842\\
13.59	0.02372\\
13.6	0.02901\\
13.61	0.03427\\
13.62	0.0395\\
13.63	0.0447\\
13.64	0.04986\\
13.65	0.05497\\
13.66	0.06004\\
13.67	0.06505\\
13.68	0.07001\\
13.69	0.0749\\
13.7	0.07972\\
13.71	0.08447\\
13.72	0.08914\\
13.73	0.09373\\
13.74	0.09823\\
13.75	0.1026\\
13.76	0.1069\\
13.77	0.1112\\
13.78	0.1153\\
13.79	0.1193\\
13.8	0.1231\\
13.81	0.1269\\
13.82	0.1306\\
13.83	0.1341\\
13.84	0.1375\\
13.85	0.1407\\
13.86	0.1439\\
13.87	0.1468\\
13.88	0.1497\\
13.89	0.1524\\
13.9	0.155\\
13.91	0.1574\\
13.92	0.1596\\
13.93	0.1617\\
13.94	0.1637\\
13.95	0.1654\\
13.96	0.1671\\
13.97	0.1685\\
13.98	0.1698\\
13.99	0.171\\
14	0.1719\\
14.01	0.1727\\
14.02	0.1734\\
14.03	0.1738\\
14.04	0.1741\\
14.05	0.1742\\
14.06	0.1742\\
14.07	0.174\\
14.08	0.1736\\
14.09	0.173\\
14.1	0.1723\\
14.11	0.1714\\
14.12	0.1703\\
14.13	0.1691\\
14.14	0.1677\\
14.15	0.1661\\
14.16	0.1643\\
14.17	0.1624\\
14.18	0.1604\\
14.19	0.1582\\
14.2	0.1558\\
14.21	0.1533\\
14.22	0.1506\\
14.23	0.1477\\
14.24	0.1447\\
14.25	0.1416\\
14.26	0.1383\\
14.27	0.1349\\
14.28	0.1314\\
14.29	0.1277\\
14.3	0.1239\\
14.31	0.12\\
14.32	0.1159\\
14.33	0.1117\\
14.34	0.1075\\
14.35	0.1031\\
14.36	0.09857\\
14.37	0.09397\\
14.38	0.08927\\
14.39	0.08448\\
14.4	0.0796\\
14.41	0.07464\\
14.42	0.0696\\
14.43	0.06449\\
14.44	0.05931\\
14.45	0.05406\\
14.46	0.04876\\
14.47	0.04341\\
14.48	0.03801\\
14.49	0.03256\\
14.5	0.02708\\
14.51	0.02156\\
14.52	0.01602\\
14.53	0.01046\\
14.54	0.004883\\
14.55	-0.0007055\\
14.56	-0.006298\\
14.57	-0.01189\\
14.58	-0.01748\\
14.59	-0.02305\\
14.6	-0.02861\\
14.61	-0.03414\\
14.62	-0.03965\\
14.63	-0.04512\\
14.64	-0.05056\\
14.65	-0.05595\\
14.66	-0.06129\\
14.67	-0.06657\\
14.68	-0.0718\\
14.69	-0.07696\\
14.7	-0.08205\\
14.71	-0.08706\\
14.72	-0.09199\\
14.73	-0.09684\\
14.74	-0.1016\\
14.75	-0.1063\\
14.76	-0.1108\\
14.77	-0.1153\\
14.78	-0.1196\\
14.79	-0.1239\\
14.8	-0.128\\
14.81	-0.132\\
14.82	-0.1359\\
14.83	-0.1396\\
14.84	-0.1432\\
14.85	-0.1467\\
14.86	-0.15\\
14.87	-0.1532\\
14.88	-0.1563\\
14.89	-0.1591\\
14.9	-0.1619\\
14.91	-0.1645\\
14.92	-0.1669\\
14.93	-0.1691\\
14.94	-0.1712\\
14.95	-0.1732\\
14.96	-0.1749\\
14.97	-0.1765\\
14.98	-0.178\\
14.99	-0.1792\\
15	-0.1803\\
15.01	-0.1812\\
15.02	-0.1819\\
15.03	-0.1824\\
15.04	-0.1828\\
15.05	-0.183\\
15.06	-0.183\\
15.07	-0.1828\\
15.08	-0.1825\\
15.09	-0.182\\
15.1	-0.1812\\
15.11	-0.1804\\
15.12	-0.1793\\
15.13	-0.178\\
15.14	-0.1766\\
15.15	-0.175\\
15.16	-0.1733\\
15.17	-0.1713\\
15.18	-0.1692\\
15.19	-0.1669\\
15.2	-0.1645\\
15.21	-0.1619\\
15.22	-0.1591\\
15.23	-0.1562\\
15.24	-0.1531\\
15.25	-0.1499\\
15.26	-0.1465\\
15.27	-0.143\\
15.28	-0.1393\\
15.29	-0.1355\\
15.3	-0.1315\\
15.31	-0.1274\\
15.32	-0.1232\\
15.33	-0.1189\\
15.34	-0.1144\\
15.35	-0.1098\\
15.36	-0.1051\\
15.37	-0.1003\\
15.38	-0.09544\\
15.39	-0.09044\\
15.4	-0.08534\\
15.41	-0.08016\\
15.42	-0.0749\\
15.43	-0.06955\\
15.44	-0.06413\\
15.45	-0.05865\\
15.46	-0.0531\\
15.47	-0.04749\\
15.48	-0.04183\\
15.49	-0.03613\\
15.5	-0.03038\\
15.51	-0.0246\\
15.52	-0.01879\\
15.53	-0.01295\\
15.54	-0.0071\\
15.55	-0.001233\\
15.56	0.004641\\
15.57	0.01052\\
15.58	0.01639\\
15.59	0.02225\\
15.6	0.02809\\
15.61	0.03391\\
15.62	0.03971\\
15.63	0.04547\\
15.64	0.05119\\
15.65	0.05687\\
15.66	0.0625\\
15.67	0.06807\\
15.68	0.07358\\
15.69	0.07902\\
15.7	0.08439\\
15.71	0.08969\\
15.72	0.0949\\
15.73	0.1\\
15.74	0.1051\\
15.75	0.11\\
15.76	0.1148\\
15.77	0.1195\\
15.78	0.1241\\
15.79	0.1286\\
15.8	0.133\\
15.81	0.1372\\
15.82	0.1414\\
15.83	0.1453\\
15.84	0.1492\\
15.85	0.1529\\
15.86	0.1564\\
15.87	0.1598\\
15.88	0.1631\\
15.89	0.1662\\
15.9	0.1691\\
15.91	0.1719\\
15.92	0.1745\\
15.93	0.1769\\
15.94	0.1792\\
15.95	0.1812\\
15.96	0.1832\\
15.97	0.1849\\
15.98	0.1864\\
15.99	0.1878\\
16	0.189\\
16.01	0.19\\
16.02	0.1908\\
16.03	0.1915\\
16.04	0.1919\\
16.05	0.1922\\
16.06	0.1923\\
16.07	0.1921\\
16.08	0.1918\\
16.09	0.1913\\
16.1	0.1907\\
16.11	0.1898\\
16.12	0.1887\\
16.13	0.1875\\
16.14	0.1861\\
16.15	0.1844\\
16.16	0.1826\\
16.17	0.1807\\
16.18	0.1785\\
16.19	0.1762\\
16.2	0.1737\\
16.21	0.171\\
16.22	0.1681\\
16.23	0.1651\\
16.24	0.1619\\
16.25	0.1586\\
16.26	0.1551\\
16.27	0.1514\\
16.28	0.1476\\
16.29	0.1437\\
16.3	0.1396\\
16.31	0.1353\\
16.32	0.1309\\
16.33	0.1264\\
16.34	0.1218\\
16.35	0.117\\
16.36	0.1121\\
16.37	0.1071\\
16.38	0.102\\
16.39	0.09677\\
16.4	0.09146\\
16.41	0.08604\\
16.42	0.08054\\
16.43	0.07496\\
16.44	0.06929\\
16.45	0.06355\\
16.46	0.05774\\
16.47	0.05187\\
16.48	0.04595\\
16.49	0.03997\\
16.5	0.03395\\
16.51	0.02789\\
16.52	0.02179\\
16.53	0.01567\\
16.54	0.009526\\
16.55	0.003367\\
16.56	-0.002801\\
16.57	-0.008972\\
16.58	-0.01514\\
16.59	-0.0213\\
16.6	-0.02745\\
16.61	-0.03357\\
16.62	-0.03967\\
16.63	-0.04573\\
16.64	-0.05176\\
16.65	-0.05774\\
16.66	-0.06367\\
16.67	-0.06954\\
16.68	-0.07535\\
16.69	-0.08109\\
16.7	-0.08676\\
16.71	-0.09235\\
16.72	-0.09786\\
16.73	-0.1033\\
16.74	-0.1086\\
16.75	-0.1138\\
16.76	-0.1189\\
16.77	-0.1239\\
16.78	-0.1288\\
16.79	-0.1335\\
16.8	-0.1382\\
16.81	-0.1427\\
16.82	-0.1471\\
16.83	-0.1513\\
16.84	-0.1554\\
16.85	-0.1593\\
16.86	-0.1631\\
16.87	-0.1667\\
16.88	-0.1702\\
16.89	-0.1735\\
16.9	-0.1766\\
16.91	-0.1796\\
16.92	-0.1824\\
16.93	-0.185\\
16.94	-0.1874\\
16.95	-0.1897\\
16.96	-0.1917\\
16.97	-0.1936\\
16.98	-0.1953\\
16.99	-0.1968\\
17	-0.1981\\
17.01	-0.1993\\
17.02	-0.2002\\
17.03	-0.2009\\
17.04	-0.2015\\
17.05	-0.2018\\
17.06	-0.2019\\
17.07	-0.2019\\
17.08	-0.2016\\
17.09	-0.2012\\
17.1	-0.2005\\
17.11	-0.1997\\
17.12	-0.1986\\
17.13	-0.1974\\
17.14	-0.196\\
17.15	-0.1943\\
17.16	-0.1925\\
17.17	-0.1905\\
17.18	-0.1883\\
17.19	-0.1859\\
17.2	-0.1833\\
17.21	-0.1806\\
17.22	-0.1776\\
17.23	-0.1745\\
17.24	-0.1712\\
17.25	-0.1678\\
17.26	-0.1642\\
17.27	-0.1604\\
17.28	-0.1564\\
17.29	-0.1523\\
17.3	-0.1481\\
17.31	-0.1436\\
17.32	-0.1391\\
17.33	-0.1344\\
17.34	-0.1296\\
17.35	-0.1246\\
17.36	-0.1195\\
17.37	-0.1143\\
17.38	-0.109\\
17.39	-0.1035\\
17.4	-0.09796\\
17.41	-0.0923\\
17.42	-0.08656\\
17.43	-0.08072\\
17.44	-0.07479\\
17.45	-0.06879\\
17.46	-0.06271\\
17.47	-0.05657\\
17.48	-0.05036\\
17.49	-0.0441\\
17.5	-0.03779\\
17.51	-0.03144\\
17.52	-0.02505\\
17.53	-0.01862\\
17.54	-0.01218\\
17.55	-0.005713\\
17.56	0.0007635\\
17.57	0.007246\\
17.58	0.01373\\
17.59	0.0202\\
17.6	0.02666\\
17.61	0.0331\\
17.62	0.03952\\
17.63	0.0459\\
17.64	0.05224\\
17.65	0.05854\\
17.66	0.06479\\
17.67	0.07098\\
17.68	0.07711\\
17.69	0.08316\\
17.7	0.08914\\
17.71	0.09504\\
17.72	0.1009\\
17.73	0.1066\\
17.74	0.1122\\
17.75	0.1177\\
17.76	0.1231\\
17.77	0.1284\\
17.78	0.1336\\
17.79	0.1386\\
17.8	0.1435\\
17.81	0.1483\\
17.82	0.1529\\
17.83	0.1574\\
17.84	0.1618\\
17.85	0.166\\
17.86	0.17\\
17.87	0.1738\\
17.88	0.1775\\
17.89	0.1811\\
17.9	0.1844\\
17.91	0.1876\\
17.92	0.1906\\
17.93	0.1934\\
17.94	0.196\\
17.95	0.1985\\
17.96	0.2007\\
17.97	0.2027\\
17.98	0.2046\\
17.99	0.2062\\
18	0.2077\\
18.01	0.2089\\
18.02	0.21\\
18.03	0.2108\\
18.04	0.2115\\
18.05	0.2119\\
18.06	0.2121\\
18.07	0.2121\\
18.08	0.2119\\
18.09	0.2115\\
18.1	0.2109\\
18.11	0.2101\\
18.12	0.209\\
18.13	0.2078\\
18.14	0.2064\\
18.15	0.2047\\
18.16	0.2029\\
18.17	0.2008\\
18.18	0.1986\\
18.19	0.1961\\
18.2	0.1935\\
18.21	0.1907\\
18.22	0.1877\\
18.23	0.1845\\
18.24	0.1811\\
18.25	0.1775\\
18.26	0.1737\\
18.27	0.1698\\
18.28	0.1657\\
18.29	0.1615\\
18.3	0.157\\
18.31	0.1525\\
18.32	0.1477\\
18.33	0.1428\\
18.34	0.1378\\
18.35	0.1326\\
18.36	0.1273\\
18.37	0.1219\\
18.38	0.1163\\
18.39	0.1107\\
18.4	0.1049\\
18.41	0.09897\\
18.42	0.09296\\
18.43	0.08686\\
18.44	0.08066\\
18.45	0.07439\\
18.46	0.06803\\
18.47	0.0616\\
18.48	0.0551\\
18.49	0.04854\\
18.5	0.04193\\
18.51	0.03527\\
18.52	0.02857\\
18.53	0.02183\\
18.54	0.01507\\
18.55	0.008285\\
18.56	0.001486\\
18.57	-0.005322\\
18.58	-0.01213\\
18.59	-0.01894\\
18.6	-0.02573\\
18.61	-0.0325\\
18.62	-0.03925\\
18.63	-0.04597\\
18.64	-0.05264\\
18.65	-0.05928\\
18.66	-0.06586\\
18.67	-0.07238\\
18.68	-0.07884\\
18.69	-0.08523\\
18.7	-0.09154\\
18.71	-0.09776\\
18.72	-0.1039\\
18.73	-0.1099\\
18.74	-0.1159\\
18.75	-0.1217\\
18.76	-0.1274\\
18.77	-0.133\\
18.78	-0.1385\\
18.79	-0.1438\\
18.8	-0.149\\
18.81	-0.1541\\
18.82	-0.159\\
18.83	-0.1638\\
18.84	-0.1684\\
18.85	-0.1729\\
18.86	-0.1771\\
18.87	-0.1813\\
18.88	-0.1852\\
18.89	-0.189\\
18.9	-0.1926\\
18.91	-0.196\\
18.92	-0.1992\\
18.93	-0.2022\\
18.94	-0.205\\
18.95	-0.2076\\
18.96	-0.21\\
18.97	-0.2123\\
18.98	-0.2143\\
18.99	-0.2161\\
19	-0.2177\\
19.01	-0.2191\\
19.02	-0.2202\\
19.03	-0.2212\\
19.04	-0.2219\\
19.05	-0.2224\\
19.06	-0.2227\\
19.07	-0.2228\\
19.08	-0.2227\\
19.09	-0.2223\\
19.1	-0.2218\\
19.11	-0.221\\
19.12	-0.22\\
19.13	-0.2187\\
19.14	-0.2173\\
19.15	-0.2157\\
19.16	-0.2138\\
19.17	-0.2117\\
19.18	-0.2094\\
19.19	-0.2069\\
19.2	-0.2042\\
19.21	-0.2013\\
19.22	-0.1982\\
19.23	-0.1949\\
19.24	-0.1914\\
19.25	-0.1877\\
19.26	-0.1838\\
19.27	-0.1798\\
19.28	-0.1755\\
19.29	-0.1711\\
19.3	-0.1665\\
19.31	-0.1618\\
19.32	-0.1569\\
19.33	-0.1518\\
19.34	-0.1466\\
19.35	-0.1412\\
19.36	-0.1356\\
19.37	-0.13\\
19.38	-0.1242\\
19.39	-0.1183\\
19.4	-0.1122\\
19.41	-0.1061\\
19.42	-0.09978\\
19.43	-0.0934\\
19.44	-0.08693\\
19.45	-0.08036\\
19.46	-0.07371\\
19.47	-0.06698\\
19.48	-0.06018\\
19.49	-0.05331\\
19.5	-0.04638\\
19.51	-0.0394\\
19.52	-0.03238\\
19.53	-0.02532\\
19.54	-0.01822\\
19.55	-0.0111\\
19.56	-0.003963\\
19.57	0.003186\\
19.58	0.01034\\
19.59	0.01749\\
19.6	0.02463\\
19.61	0.03175\\
19.62	0.03885\\
19.63	0.04592\\
19.64	0.05295\\
19.65	0.05993\\
19.66	0.06686\\
19.67	0.07373\\
19.68	0.08054\\
19.69	0.08728\\
19.7	0.09393\\
19.71	0.1005\\
19.72	0.107\\
19.73	0.1134\\
19.74	0.1196\\
19.75	0.1258\\
19.76	0.1318\\
19.77	0.1378\\
19.78	0.1436\\
19.79	0.1492\\
19.8	0.1547\\
19.81	0.1601\\
19.82	0.1653\\
19.83	0.1704\\
19.84	0.1753\\
19.85	0.18\\
19.86	0.1846\\
19.87	0.189\\
19.88	0.1932\\
19.89	0.1972\\
19.9	0.201\\
19.91	0.2047\\
19.92	0.2081\\
19.93	0.2113\\
19.94	0.2144\\
19.95	0.2172\\
19.96	0.2198\\
19.97	0.2222\\
19.98	0.2244\\
19.99	0.2264\\
20	0.2281\\
20.01	0.2296\\
20.02	0.2309\\
20.03	0.232\\
20.04	0.2329\\
20.05	0.2335\\
20.06	0.2339\\
20.07	0.2341\\
20.08	0.234\\
20.09	0.2337\\
20.1	0.2332\\
20.11	0.2324\\
20.12	0.2314\\
20.13	0.2302\\
20.14	0.2288\\
20.15	0.2271\\
20.16	0.2252\\
20.17	0.2231\\
20.18	0.2208\\
20.19	0.2183\\
20.2	0.2155\\
20.21	0.2125\\
20.22	0.2093\\
20.23	0.2059\\
20.24	0.2023\\
20.25	0.1985\\
20.26	0.1945\\
20.27	0.1903\\
20.28	0.1859\\
20.29	0.1813\\
20.3	0.1766\\
20.31	0.1716\\
20.32	0.1665\\
20.33	0.1613\\
20.34	0.1558\\
20.35	0.1502\\
20.36	0.1445\\
20.37	0.1386\\
20.38	0.1325\\
20.39	0.1263\\
20.4	0.12\\
20.41	0.1136\\
20.42	0.107\\
20.43	0.1004\\
20.44	0.09361\\
20.45	0.08674\\
20.46	0.07979\\
20.47	0.07274\\
20.48	0.06562\\
20.49	0.05843\\
20.5	0.05118\\
20.51	0.04386\\
20.52	0.0365\\
20.53	0.02909\\
20.54	0.02165\\
20.55	0.01418\\
20.56	0.006684\\
20.57	-0.000822\\
20.58	-0.008335\\
20.59	-0.01585\\
20.6	-0.02335\\
20.61	-0.03084\\
20.62	-0.0383\\
20.63	-0.04574\\
20.64	-0.05314\\
20.65	-0.06049\\
20.66	-0.06779\\
20.67	-0.07503\\
20.68	-0.08221\\
20.69	-0.08931\\
20.7	-0.09633\\
20.71	-0.1033\\
20.72	-0.1101\\
20.73	-0.1168\\
20.74	-0.1235\\
20.75	-0.13\\
20.76	-0.1364\\
20.77	-0.1426\\
20.78	-0.1488\\
20.79	-0.1548\\
20.8	-0.1606\\
20.81	-0.1663\\
20.82	-0.1718\\
20.83	-0.1772\\
20.84	-0.1824\\
20.85	-0.1875\\
20.86	-0.1923\\
20.87	-0.197\\
20.88	-0.2015\\
20.89	-0.2057\\
20.9	-0.2098\\
20.91	-0.2137\\
20.92	-0.2174\\
20.93	-0.2209\\
20.94	-0.2241\\
20.95	-0.2272\\
20.96	-0.23\\
20.97	-0.2326\\
20.98	-0.235\\
20.99	-0.2371\\
21	-0.239\\
21.01	-0.2407\\
21.02	-0.2422\\
21.03	-0.2434\\
21.04	-0.2443\\
21.05	-0.2451\\
21.06	-0.2456\\
21.07	-0.2458\\
21.08	-0.2458\\
21.09	-0.2456\\
21.1	-0.2451\\
21.11	-0.2444\\
21.12	-0.2435\\
21.13	-0.2423\\
21.14	-0.2409\\
21.15	-0.2392\\
21.16	-0.2373\\
21.17	-0.2352\\
21.18	-0.2328\\
21.19	-0.2302\\
21.2	-0.2274\\
21.21	-0.2243\\
21.22	-0.221\\
21.23	-0.2175\\
21.24	-0.2138\\
21.25	-0.2099\\
21.26	-0.2058\\
21.27	-0.2014\\
21.28	-0.1969\\
21.29	-0.1921\\
21.3	-0.1872\\
21.31	-0.1821\\
21.32	-0.1768\\
21.33	-0.1713\\
21.34	-0.1656\\
21.35	-0.1598\\
21.36	-0.1538\\
21.37	-0.1476\\
21.38	-0.1413\\
21.39	-0.1349\\
21.4	-0.1283\\
21.41	-0.1216\\
21.42	-0.1148\\
21.43	-0.1078\\
21.44	-0.1007\\
21.45	-0.09355\\
21.46	-0.08628\\
21.47	-0.07891\\
21.48	-0.07146\\
21.49	-0.06392\\
21.5	-0.05632\\
21.51	-0.04866\\
21.52	-0.04094\\
21.53	-0.03318\\
21.54	-0.02537\\
21.55	-0.01753\\
21.56	-0.009669\\
21.57	-0.001788\\
21.58	0.006102\\
21.59	0.01399\\
21.6	0.02188\\
21.61	0.02975\\
21.62	0.0376\\
21.63	0.04543\\
21.64	0.05321\\
21.65	0.06095\\
21.66	0.06864\\
21.67	0.07627\\
21.68	0.08383\\
21.69	0.09131\\
21.7	0.09872\\
21.71	0.106\\
21.72	0.1132\\
21.73	0.1204\\
21.74	0.1274\\
21.75	0.1342\\
21.76	0.141\\
21.77	0.1476\\
21.78	0.1541\\
21.79	0.1605\\
21.8	0.1667\\
21.81	0.1727\\
21.82	0.1786\\
21.83	0.1843\\
21.84	0.1898\\
21.85	0.1951\\
21.86	0.2003\\
21.87	0.2053\\
21.88	0.2101\\
21.89	0.2146\\
21.9	0.219\\
21.91	0.2231\\
21.92	0.2271\\
21.93	0.2308\\
21.94	0.2343\\
21.95	0.2376\\
21.96	0.2406\\
21.97	0.2434\\
21.98	0.246\\
21.99	0.2483\\
22	0.2504\\
22.01	0.2523\\
22.02	0.2539\\
22.03	0.2552\\
22.04	0.2563\\
22.05	0.2572\\
22.06	0.2578\\
22.07	0.2582\\
22.08	0.2583\\
22.09	0.2581\\
22.1	0.2577\\
22.11	0.257\\
22.12	0.2561\\
22.13	0.255\\
22.14	0.2535\\
22.15	0.2519\\
22.16	0.25\\
22.17	0.2478\\
22.18	0.2454\\
22.19	0.2427\\
22.2	0.2399\\
22.21	0.2367\\
22.22	0.2334\\
22.23	0.2298\\
22.24	0.2259\\
22.25	0.2219\\
22.26	0.2176\\
22.27	0.2131\\
22.28	0.2084\\
22.29	0.2035\\
22.3	0.1984\\
22.31	0.1931\\
22.32	0.1876\\
22.33	0.1819\\
22.34	0.176\\
22.35	0.1699\\
22.36	0.1637\\
22.37	0.1573\\
22.38	0.1507\\
22.39	0.144\\
22.4	0.1371\\
22.41	0.1301\\
22.42	0.123\\
22.43	0.1157\\
22.44	0.1083\\
22.45	0.1008\\
22.46	0.09321\\
22.47	0.0855\\
22.48	0.0777\\
22.49	0.06982\\
22.5	0.06186\\
22.51	0.05383\\
22.52	0.04574\\
22.53	0.0376\\
22.54	0.02941\\
22.55	0.02119\\
22.56	0.01294\\
22.57	0.004662\\
22.58	-0.003624\\
22.59	-0.01191\\
22.6	-0.0202\\
22.61	-0.02848\\
22.62	-0.03673\\
22.63	-0.04496\\
22.64	-0.05315\\
22.65	-0.0613\\
22.66	-0.06939\\
22.67	-0.07743\\
22.68	-0.08539\\
22.69	-0.09328\\
22.7	-0.1011\\
22.71	-0.1088\\
22.72	-0.1164\\
22.73	-0.1239\\
22.74	-0.1313\\
22.75	-0.1386\\
22.76	-0.1457\\
22.77	-0.1528\\
22.78	-0.1596\\
22.79	-0.1663\\
22.8	-0.1729\\
22.81	-0.1793\\
22.82	-0.1855\\
22.83	-0.1916\\
22.84	-0.1974\\
22.85	-0.2031\\
22.86	-0.2086\\
22.87	-0.2139\\
22.88	-0.219\\
22.89	-0.2239\\
22.9	-0.2285\\
22.91	-0.233\\
22.92	-0.2372\\
22.93	-0.2412\\
22.94	-0.2449\\
22.95	-0.2484\\
22.96	-0.2517\\
22.97	-0.2547\\
22.98	-0.2575\\
22.99	-0.2601\\
23	-0.2623\\
23.01	-0.2644\\
23.02	-0.2661\\
23.03	-0.2677\\
23.04	-0.2689\\
23.05	-0.2699\\
23.06	-0.2706\\
23.07	-0.2711\\
23.08	-0.2713\\
23.09	-0.2712\\
23.1	-0.2709\\
23.11	-0.2703\\
23.12	-0.2694\\
23.13	-0.2683\\
23.14	-0.2669\\
23.15	-0.2652\\
23.16	-0.2633\\
23.17	-0.2611\\
23.18	-0.2586\\
23.19	-0.2559\\
23.2	-0.253\\
23.21	-0.2498\\
23.22	-0.2463\\
23.23	-0.2426\\
23.24	-0.2387\\
23.25	-0.2345\\
23.26	-0.2301\\
23.27	-0.2255\\
23.28	-0.2206\\
23.29	-0.2155\\
23.3	-0.2102\\
23.31	-0.2047\\
23.32	-0.199\\
23.33	-0.1931\\
23.34	-0.1869\\
23.35	-0.1806\\
23.36	-0.1741\\
23.37	-0.1675\\
23.38	-0.1606\\
23.39	-0.1536\\
23.4	-0.1465\\
23.41	-0.1392\\
23.42	-0.1317\\
23.43	-0.1241\\
23.44	-0.1164\\
23.45	-0.1086\\
23.46	-0.1006\\
23.47	-0.09254\\
23.48	-0.08438\\
23.49	-0.07613\\
23.5	-0.06779\\
23.51	-0.05938\\
23.52	-0.05091\\
23.53	-0.04237\\
23.54	-0.03379\\
23.55	-0.02516\\
23.56	-0.0165\\
23.57	-0.00782\\
23.58	0.0008798\\
23.59	0.009587\\
23.6	0.01829\\
23.61	0.02699\\
23.62	0.03567\\
23.63	0.04432\\
23.64	0.05294\\
23.65	0.06152\\
23.66	0.07004\\
23.67	0.0785\\
23.68	0.08689\\
23.69	0.09521\\
23.7	0.1034\\
23.71	0.1116\\
23.72	0.1196\\
23.73	0.1275\\
23.74	0.1354\\
23.75	0.143\\
23.76	0.1506\\
23.77	0.158\\
23.78	0.1653\\
23.79	0.1724\\
23.8	0.1793\\
23.81	0.1861\\
23.82	0.1927\\
23.83	0.1991\\
23.84	0.2053\\
23.85	0.2114\\
23.86	0.2172\\
23.87	0.2228\\
23.88	0.2283\\
23.89	0.2334\\
23.9	0.2384\\
23.91	0.2432\\
23.92	0.2477\\
23.93	0.2519\\
23.94	0.256\\
23.95	0.2597\\
23.96	0.2633\\
23.97	0.2665\\
23.98	0.2696\\
23.99	0.2723\\
24	0.2748\\
24.01	0.277\\
24.02	0.279\\
24.03	0.2806\\
24.04	0.2821\\
24.05	0.2832\\
24.06	0.284\\
24.07	0.2846\\
24.08	0.2849\\
24.09	0.2849\\
24.1	0.2847\\
24.11	0.2841\\
24.12	0.2833\\
24.13	0.2822\\
24.14	0.2808\\
24.15	0.2792\\
24.16	0.2773\\
24.17	0.2751\\
24.18	0.2726\\
24.19	0.2698\\
24.2	0.2668\\
24.21	0.2635\\
24.22	0.26\\
24.23	0.2562\\
24.24	0.2522\\
24.25	0.2479\\
24.26	0.2433\\
24.27	0.2385\\
24.28	0.2335\\
24.29	0.2282\\
24.3	0.2227\\
24.31	0.217\\
24.32	0.2111\\
24.33	0.2049\\
24.34	0.1986\\
24.35	0.192\\
24.36	0.1852\\
24.37	0.1783\\
24.38	0.1712\\
24.39	0.1639\\
24.4	0.1564\\
24.41	0.1488\\
24.42	0.141\\
24.43	0.1331\\
24.44	0.125\\
24.45	0.1168\\
24.46	0.1085\\
24.47	0.1001\\
24.48	0.09153\\
24.49	0.08289\\
24.5	0.07416\\
24.51	0.06536\\
24.52	0.05647\\
24.53	0.04753\\
24.54	0.03853\\
24.55	0.02948\\
24.56	0.0204\\
24.57	0.01128\\
24.58	0.002149\\
24.59	-0.006996\\
24.6	-0.01614\\
24.61	-0.02528\\
24.62	-0.03441\\
24.63	-0.04351\\
24.64	-0.05257\\
24.65	-0.0616\\
24.66	-0.07057\\
24.67	-0.07948\\
24.68	-0.08832\\
24.69	-0.09708\\
24.7	-0.1058\\
24.71	-0.1143\\
24.72	-0.1228\\
24.73	-0.1312\\
24.74	-0.1394\\
24.75	-0.1476\\
24.76	-0.1555\\
24.77	-0.1634\\
24.78	-0.171\\
24.79	-0.1786\\
24.8	-0.1859\\
24.81	-0.1931\\
24.82	-0.2001\\
24.83	-0.2069\\
24.84	-0.2135\\
24.85	-0.2199\\
24.86	-0.2261\\
24.87	-0.2321\\
24.88	-0.2379\\
24.89	-0.2434\\
24.9	-0.2487\\
24.91	-0.2538\\
24.92	-0.2586\\
24.93	-0.2632\\
24.94	-0.2675\\
24.95	-0.2715\\
24.96	-0.2753\\
24.97	-0.2789\\
24.98	-0.2821\\
24.99	-0.2851\\
25	-0.2878\\
25.01	-0.2902\\
25.02	-0.2924\\
25.03	-0.2942\\
25.04	-0.2958\\
25.05	-0.2971\\
25.06	-0.2981\\
25.07	-0.2988\\
25.08	-0.2992\\
25.09	-0.2993\\
25.1	-0.2991\\
25.11	-0.2987\\
25.12	-0.2979\\
25.13	-0.2969\\
25.14	-0.2955\\
25.15	-0.2939\\
25.16	-0.2919\\
25.17	-0.2897\\
25.18	-0.2872\\
25.19	-0.2844\\
25.2	-0.2814\\
25.21	-0.278\\
25.22	-0.2744\\
25.23	-0.2705\\
25.24	-0.2663\\
25.25	-0.2619\\
25.26	-0.2572\\
25.27	-0.2523\\
25.28	-0.2471\\
25.29	-0.2416\\
25.3	-0.2359\\
25.31	-0.23\\
25.32	-0.2238\\
25.33	-0.2174\\
25.34	-0.2108\\
25.35	-0.204\\
25.36	-0.197\\
25.37	-0.1898\\
25.38	-0.1823\\
25.39	-0.1747\\
25.4	-0.167\\
25.41	-0.159\\
25.42	-0.1509\\
25.43	-0.1426\\
25.44	-0.1342\\
25.45	-0.1256\\
25.46	-0.1169\\
25.47	-0.1081\\
25.48	-0.09917\\
25.49	-0.09013\\
25.5	-0.08099\\
25.51	-0.07177\\
25.52	-0.06247\\
25.53	-0.05309\\
25.54	-0.04366\\
25.55	-0.03417\\
25.56	-0.02464\\
25.57	-0.01507\\
25.58	-0.005484\\
25.59	0.004119\\
25.6	0.01373\\
25.61	0.02333\\
25.62	0.03292\\
25.63	0.04249\\
25.64	0.05203\\
25.65	0.06152\\
25.66	0.07096\\
25.67	0.08035\\
25.68	0.08966\\
25.69	0.0989\\
25.7	0.108\\
25.71	0.1171\\
25.72	0.126\\
25.73	0.1349\\
25.74	0.1436\\
25.75	0.1521\\
25.76	0.1606\\
25.77	0.1689\\
25.78	0.177\\
25.79	0.1849\\
25.8	0.1927\\
25.81	0.2003\\
25.82	0.2077\\
25.83	0.215\\
25.84	0.222\\
25.85	0.2288\\
25.86	0.2354\\
25.87	0.2417\\
25.88	0.2479\\
25.89	0.2538\\
25.9	0.2594\\
25.91	0.2648\\
25.92	0.27\\
25.93	0.2748\\
25.94	0.2795\\
25.95	0.2838\\
25.96	0.2879\\
25.97	0.2917\\
25.98	0.2952\\
25.99	0.2984\\
26	0.3014\\
26.01	0.304\\
26.02	0.3064\\
26.03	0.3084\\
26.04	0.3102\\
26.05	0.3117\\
26.06	0.3128\\
26.07	0.3136\\
26.08	0.3142\\
26.09	0.3144\\
26.1	0.3143\\
26.11	0.3139\\
26.12	0.3132\\
26.13	0.3122\\
26.14	0.3109\\
26.15	0.3093\\
26.16	0.3074\\
26.17	0.3052\\
26.18	0.3026\\
26.19	0.2998\\
26.2	0.2967\\
26.21	0.2933\\
26.22	0.2895\\
26.23	0.2855\\
26.24	0.2813\\
26.25	0.2767\\
26.26	0.2719\\
26.27	0.2668\\
26.28	0.2614\\
26.29	0.2557\\
26.3	0.2498\\
26.31	0.2437\\
26.32	0.2373\\
26.33	0.2307\\
26.34	0.2238\\
26.35	0.2167\\
26.36	0.2094\\
26.37	0.2019\\
26.38	0.1942\\
26.39	0.1863\\
26.4	0.1781\\
26.41	0.1698\\
26.42	0.1614\\
26.43	0.1527\\
26.44	0.1439\\
26.45	0.135\\
26.46	0.1259\\
26.47	0.1167\\
26.48	0.1073\\
26.49	0.09788\\
26.5	0.08831\\
26.51	0.07865\\
26.52	0.06891\\
26.53	0.05909\\
26.54	0.04919\\
26.55	0.03925\\
26.56	0.02925\\
26.57	0.01921\\
26.58	0.009148\\
26.59	-0.0009352\\
26.6	-0.01103\\
26.61	-0.02112\\
26.62	-0.0312\\
26.63	-0.04126\\
26.64	-0.05129\\
26.65	-0.06128\\
26.66	-0.07122\\
26.67	-0.08109\\
26.68	-0.0909\\
26.69	-0.1006\\
26.7	-0.1103\\
26.71	-0.1198\\
26.72	-0.1292\\
26.73	-0.1386\\
26.74	-0.1477\\
26.75	-0.1568\\
26.76	-0.1657\\
26.77	-0.1745\\
26.78	-0.1831\\
26.79	-0.1915\\
26.8	-0.1997\\
26.81	-0.2078\\
26.82	-0.2156\\
26.83	-0.2233\\
26.84	-0.2307\\
26.85	-0.2379\\
26.86	-0.2449\\
26.87	-0.2517\\
26.88	-0.2582\\
26.89	-0.2645\\
26.9	-0.2705\\
26.91	-0.2763\\
26.92	-0.2818\\
26.93	-0.287\\
26.94	-0.292\\
26.95	-0.2966\\
26.96	-0.301\\
26.97	-0.3051\\
26.98	-0.3089\\
26.99	-0.3124\\
27	-0.3156\\
27.01	-0.3185\\
27.02	-0.321\\
27.03	-0.3233\\
27.04	-0.3253\\
27.05	-0.3269\\
27.06	-0.3282\\
27.07	-0.3292\\
27.08	-0.3299\\
27.09	-0.3302\\
27.1	-0.3302\\
27.11	-0.3299\\
27.12	-0.3293\\
27.13	-0.3284\\
27.14	-0.3271\\
27.15	-0.3255\\
27.16	-0.3236\\
27.17	-0.3214\\
27.18	-0.3188\\
27.19	-0.3159\\
27.2	-0.3128\\
27.21	-0.3093\\
27.22	-0.3055\\
27.23	-0.3014\\
27.24	-0.297\\
27.25	-0.2923\\
27.26	-0.2873\\
27.27	-0.282\\
27.28	-0.2765\\
27.29	-0.2707\\
27.3	-0.2646\\
27.31	-0.2582\\
27.32	-0.2516\\
27.33	-0.2447\\
27.34	-0.2376\\
27.35	-0.2302\\
27.36	-0.2226\\
27.37	-0.2148\\
27.38	-0.2067\\
27.39	-0.1985\\
27.4	-0.19\\
27.41	-0.1814\\
27.42	-0.1725\\
27.43	-0.1635\\
27.44	-0.1543\\
27.45	-0.145\\
27.46	-0.1355\\
27.47	-0.1258\\
27.48	-0.1161\\
27.49	-0.1062\\
27.5	-0.09616\\
27.51	-0.08604\\
27.52	-0.07583\\
27.53	-0.06554\\
27.54	-0.05517\\
27.55	-0.04474\\
27.56	-0.03426\\
27.57	-0.02373\\
27.58	-0.01317\\
27.59	-0.002579\\
27.6	0.00802\\
27.61	0.01862\\
27.62	0.02922\\
27.63	0.03979\\
27.64	0.05034\\
27.65	0.06085\\
27.66	0.0713\\
27.67	0.0817\\
27.68	0.09203\\
27.69	0.1023\\
27.7	0.1124\\
27.71	0.1225\\
27.72	0.1324\\
27.73	0.1423\\
27.74	0.152\\
27.75	0.1615\\
27.76	0.1709\\
27.77	0.1802\\
27.78	0.1893\\
27.79	0.1982\\
27.8	0.2069\\
27.81	0.2154\\
27.82	0.2237\\
27.83	0.2319\\
27.84	0.2397\\
27.85	0.2474\\
27.86	0.2548\\
27.87	0.262\\
27.88	0.269\\
27.89	0.2756\\
27.9	0.2821\\
27.91	0.2882\\
27.92	0.2941\\
27.93	0.2997\\
27.94	0.305\\
27.95	0.31\\
27.96	0.3147\\
27.97	0.3191\\
27.98	0.3232\\
27.99	0.3269\\
28	0.3304\\
28.01	0.3335\\
28.02	0.3364\\
28.03	0.3388\\
28.04	0.341\\
28.05	0.3428\\
28.06	0.3443\\
28.07	0.3455\\
28.08	0.3463\\
28.09	0.3468\\
28.1	0.3469\\
28.11	0.3467\\
28.12	0.3462\\
28.13	0.3453\\
28.14	0.3441\\
28.15	0.3425\\
28.16	0.3406\\
28.17	0.3384\\
28.18	0.3358\\
28.19	0.3329\\
28.2	0.3297\\
28.21	0.3261\\
28.22	0.3223\\
28.23	0.3181\\
28.24	0.3136\\
28.25	0.3087\\
28.26	0.3036\\
28.27	0.2982\\
28.28	0.2924\\
28.29	0.2864\\
28.3	0.2801\\
28.31	0.2735\\
28.32	0.2666\\
28.33	0.2595\\
28.34	0.2521\\
28.35	0.2444\\
28.36	0.2365\\
28.37	0.2284\\
28.38	0.22\\
28.39	0.2114\\
28.4	0.2026\\
28.41	0.1936\\
28.42	0.1844\\
28.43	0.1749\\
28.44	0.1654\\
28.45	0.1556\\
28.46	0.1457\\
28.47	0.1356\\
28.48	0.1254\\
28.49	0.115\\
28.5	0.1046\\
28.51	0.09396\\
28.52	0.08327\\
28.53	0.07249\\
28.54	0.06162\\
28.55	0.05069\\
28.56	0.03969\\
28.57	0.02865\\
28.58	0.01756\\
28.59	0.006449\\
28.6	-0.004681\\
28.61	-0.01582\\
28.62	-0.02695\\
28.63	-0.03807\\
28.64	-0.04916\\
28.65	-0.06021\\
28.66	-0.07121\\
28.67	-0.08216\\
28.68	-0.09303\\
28.69	-0.1038\\
28.7	-0.1145\\
28.71	-0.1251\\
28.72	-0.1356\\
28.73	-0.146\\
28.74	-0.1562\\
28.75	-0.1663\\
28.76	-0.1763\\
28.77	-0.186\\
28.78	-0.1956\\
28.79	-0.205\\
28.8	-0.2143\\
28.81	-0.2233\\
28.82	-0.2321\\
28.83	-0.2407\\
28.84	-0.2491\\
28.85	-0.2572\\
28.86	-0.2651\\
28.87	-0.2727\\
28.88	-0.2801\\
28.89	-0.2872\\
28.9	-0.294\\
28.91	-0.3006\\
28.92	-0.3069\\
28.93	-0.3128\\
28.94	-0.3185\\
28.95	-0.3239\\
28.96	-0.3289\\
28.97	-0.3336\\
28.98	-0.338\\
28.99	-0.3421\\
29	-0.3459\\
29.01	-0.3493\\
29.02	-0.3524\\
29.03	-0.3551\\
29.04	-0.3575\\
29.05	-0.3595\\
29.06	-0.3612\\
29.07	-0.3625\\
29.08	-0.3635\\
29.09	-0.3641\\
29.1	-0.3644\\
29.11	-0.3643\\
29.12	-0.3639\\
29.13	-0.3631\\
29.14	-0.3619\\
29.15	-0.3604\\
29.16	-0.3585\\
29.17	-0.3563\\
29.18	-0.3537\\
29.19	-0.3508\\
29.2	-0.3475\\
29.21	-0.3439\\
29.22	-0.3399\\
29.23	-0.3356\\
29.24	-0.331\\
29.25	-0.326\\
29.26	-0.3207\\
29.27	-0.3151\\
29.28	-0.3092\\
29.29	-0.303\\
29.3	-0.2965\\
29.31	-0.2896\\
29.32	-0.2825\\
29.33	-0.2751\\
29.34	-0.2674\\
29.35	-0.2595\\
29.36	-0.2513\\
29.37	-0.2428\\
29.38	-0.2341\\
29.39	-0.2251\\
29.4	-0.2159\\
29.41	-0.2065\\
29.42	-0.1969\\
29.43	-0.1871\\
29.44	-0.1771\\
29.45	-0.1669\\
29.46	-0.1565\\
29.47	-0.146\\
29.48	-0.1353\\
29.49	-0.1245\\
29.5	-0.1135\\
29.51	-0.1025\\
29.52	-0.09126\\
29.53	-0.07996\\
29.54	-0.06858\\
29.55	-0.05711\\
29.56	-0.04558\\
29.57	-0.034\\
29.58	-0.02237\\
29.59	-0.0107\\
29.6	0.000986\\
29.61	0.01268\\
29.62	0.02438\\
29.63	0.03606\\
29.64	0.04772\\
29.65	0.05935\\
29.66	0.07092\\
29.67	0.08244\\
29.68	0.09389\\
29.69	0.1053\\
29.7	0.1165\\
29.71	0.1277\\
29.72	0.1388\\
29.73	0.1497\\
29.74	0.1605\\
29.75	0.1712\\
29.76	0.1816\\
29.77	0.192\\
29.78	0.2021\\
29.79	0.2121\\
29.8	0.2218\\
29.81	0.2314\\
29.82	0.2407\\
29.83	0.2498\\
29.84	0.2587\\
29.85	0.2673\\
29.86	0.2757\\
29.87	0.2838\\
29.88	0.2917\\
29.89	0.2992\\
29.9	0.3065\\
29.91	0.3135\\
29.92	0.3202\\
29.93	0.3265\\
29.94	0.3326\\
29.95	0.3383\\
29.96	0.3437\\
29.97	0.3488\\
29.98	0.3536\\
29.99	0.358\\
};
\addlegendentry{EE}

\addplot[only marks, mark=*, mark options={}, mark size=0.5000pt, draw=black] table[row sep=crcr]{%
x	y\\
0	0.0875\\
0.01	0.08741\\
0.02	0.08724\\
0.03	0.08699\\
0.04	0.08664\\
0.05	0.08622\\
0.06	0.08571\\
0.07	0.08511\\
0.08	0.08444\\
0.09	0.08368\\
0.1	0.08284\\
0.11	0.08192\\
0.12	0.08092\\
0.13	0.07984\\
0.14	0.07869\\
0.15	0.07745\\
0.16	0.07615\\
0.17	0.07477\\
0.18	0.07332\\
0.19	0.0718\\
0.2	0.07021\\
0.21	0.06855\\
0.22	0.06682\\
0.23	0.06504\\
0.24	0.06319\\
0.25	0.06128\\
0.26	0.05931\\
0.27	0.05729\\
0.28	0.05521\\
0.29	0.05308\\
0.3	0.0509\\
0.31	0.04867\\
0.32	0.0464\\
0.33	0.04408\\
0.34	0.04173\\
0.35	0.03933\\
0.36	0.0369\\
0.37	0.03443\\
0.38	0.03194\\
0.39	0.02941\\
0.4	0.02686\\
0.41	0.02428\\
0.42	0.02168\\
0.43	0.01907\\
0.44	0.01644\\
0.45	0.01379\\
0.46	0.01113\\
0.47	0.008468\\
0.48	0.005797\\
0.49	0.003123\\
0.5	0.0004493\\
0.51	-0.002223\\
0.52	-0.00489\\
0.53	-0.00755\\
0.54	-0.0102\\
0.55	-0.01284\\
0.56	-0.01546\\
0.57	-0.01806\\
0.58	-0.02065\\
0.59	-0.02321\\
0.6	-0.02574\\
0.61	-0.02825\\
0.62	-0.03073\\
0.63	-0.03318\\
0.64	-0.03559\\
0.65	-0.03796\\
0.66	-0.04029\\
0.67	-0.04259\\
0.68	-0.04484\\
0.69	-0.04704\\
0.7	-0.04919\\
0.71	-0.05129\\
0.72	-0.05335\\
0.73	-0.05534\\
0.74	-0.05728\\
0.75	-0.05917\\
0.76	-0.06099\\
0.77	-0.06275\\
0.78	-0.06445\\
0.79	-0.06608\\
0.8	-0.06765\\
0.81	-0.06915\\
0.82	-0.07058\\
0.83	-0.07194\\
0.84	-0.07322\\
0.85	-0.07444\\
0.86	-0.07558\\
0.87	-0.07664\\
0.88	-0.07763\\
0.89	-0.07854\\
0.9	-0.07938\\
0.91	-0.08013\\
0.92	-0.08081\\
0.93	-0.0814\\
0.94	-0.08192\\
0.95	-0.08236\\
0.96	-0.08271\\
0.97	-0.08298\\
0.98	-0.08317\\
0.99	-0.08328\\
1	-0.08331\\
1.01	-0.08325\\
1.02	-0.08312\\
1.03	-0.0829\\
1.04	-0.0826\\
1.05	-0.08223\\
1.06	-0.08177\\
1.07	-0.08123\\
1.08	-0.08061\\
1.09	-0.07991\\
1.1	-0.07914\\
1.11	-0.07829\\
1.12	-0.07736\\
1.13	-0.07636\\
1.14	-0.07529\\
1.15	-0.07414\\
1.16	-0.07292\\
1.17	-0.07163\\
1.18	-0.07027\\
1.19	-0.06884\\
1.2	-0.06735\\
1.21	-0.06579\\
1.22	-0.06417\\
1.23	-0.06249\\
1.24	-0.06075\\
1.25	-0.05895\\
1.26	-0.0571\\
1.27	-0.05519\\
1.28	-0.05323\\
1.29	-0.05122\\
1.3	-0.04916\\
1.31	-0.04705\\
1.32	-0.0449\\
1.33	-0.04271\\
1.34	-0.04048\\
1.35	-0.03821\\
1.36	-0.03591\\
1.37	-0.03357\\
1.38	-0.0312\\
1.39	-0.02881\\
1.4	-0.02639\\
1.41	-0.02394\\
1.42	-0.02148\\
1.43	-0.01899\\
1.44	-0.01649\\
1.45	-0.01398\\
1.46	-0.01145\\
1.47	-0.008915\\
1.48	-0.006374\\
1.49	-0.003829\\
1.5	-0.001283\\
1.51	0.001261\\
1.52	0.003802\\
1.53	0.006337\\
1.54	0.008863\\
1.55	0.01138\\
1.56	0.01388\\
1.57	0.01636\\
1.58	0.01883\\
1.59	0.02128\\
1.6	0.0237\\
1.61	0.0261\\
1.62	0.02847\\
1.63	0.03081\\
1.64	0.03311\\
1.65	0.03538\\
1.66	0.03762\\
1.67	0.03981\\
1.68	0.04197\\
1.69	0.04408\\
1.7	0.04615\\
1.71	0.04816\\
1.72	0.05013\\
1.73	0.05205\\
1.74	0.05392\\
1.75	0.05573\\
1.76	0.05748\\
1.77	0.05918\\
1.78	0.06082\\
1.79	0.06239\\
1.8	0.06391\\
1.81	0.06536\\
1.82	0.06674\\
1.83	0.06806\\
1.84	0.0693\\
1.85	0.07048\\
1.86	0.07159\\
1.87	0.07263\\
1.88	0.0736\\
1.89	0.07449\\
1.9	0.07531\\
1.91	0.07605\\
1.92	0.07672\\
1.93	0.07731\\
1.94	0.07783\\
1.95	0.07827\\
1.96	0.07863\\
1.97	0.07892\\
1.98	0.07913\\
1.99	0.07926\\
2	0.07931\\
2.01	0.07928\\
2.02	0.07918\\
2.03	0.079\\
2.04	0.07874\\
2.05	0.07841\\
2.06	0.078\\
2.07	0.07751\\
2.08	0.07695\\
2.09	0.07631\\
2.1	0.0756\\
2.11	0.07481\\
2.12	0.07395\\
2.13	0.07302\\
2.14	0.07202\\
2.15	0.07095\\
2.16	0.06982\\
2.17	0.06861\\
2.18	0.06734\\
2.19	0.066\\
2.2	0.0646\\
2.21	0.06314\\
2.22	0.06162\\
2.23	0.06004\\
2.24	0.0584\\
2.25	0.05671\\
2.26	0.05496\\
2.27	0.05316\\
2.28	0.05131\\
2.29	0.04941\\
2.3	0.04746\\
2.31	0.04547\\
2.32	0.04344\\
2.33	0.04137\\
2.34	0.03926\\
2.35	0.03711\\
2.36	0.03493\\
2.37	0.03271\\
2.38	0.03047\\
2.39	0.0282\\
2.4	0.0259\\
2.41	0.02358\\
2.42	0.02124\\
2.43	0.01888\\
2.44	0.0165\\
2.45	0.01411\\
2.46	0.01171\\
2.47	0.009299\\
2.48	0.006882\\
2.49	0.00446\\
2.5	0.002036\\
2.51	-0.0003868\\
2.52	-0.002807\\
2.53	-0.005223\\
2.54	-0.007631\\
2.55	-0.01003\\
2.56	-0.01241\\
2.57	-0.01479\\
2.58	-0.01714\\
2.59	-0.01948\\
2.6	-0.02179\\
2.61	-0.02408\\
2.62	-0.02634\\
2.63	-0.02858\\
2.64	-0.03079\\
2.65	-0.03296\\
2.66	-0.0351\\
2.67	-0.0372\\
2.68	-0.03927\\
2.69	-0.04129\\
2.7	-0.04327\\
2.71	-0.04521\\
2.72	-0.0471\\
2.73	-0.04895\\
2.74	-0.05074\\
2.75	-0.05248\\
2.76	-0.05417\\
2.77	-0.0558\\
2.78	-0.05738\\
2.79	-0.0589\\
2.8	-0.06036\\
2.81	-0.06176\\
2.82	-0.0631\\
2.83	-0.06437\\
2.84	-0.06558\\
2.85	-0.06673\\
2.86	-0.06781\\
2.87	-0.06882\\
2.88	-0.06976\\
2.89	-0.07064\\
2.9	-0.07144\\
2.91	-0.07217\\
2.92	-0.07283\\
2.93	-0.07342\\
2.94	-0.07394\\
2.95	-0.07438\\
2.96	-0.07475\\
2.97	-0.07505\\
2.98	-0.07527\\
2.99	-0.07542\\
3	-0.07549\\
3.01	-0.0755\\
3.02	-0.07542\\
3.03	-0.07527\\
3.04	-0.07505\\
3.05	-0.07476\\
3.06	-0.07439\\
3.07	-0.07395\\
3.08	-0.07344\\
3.09	-0.07286\\
3.1	-0.0722\\
3.11	-0.07148\\
3.12	-0.07069\\
3.13	-0.06982\\
3.14	-0.06889\\
3.15	-0.0679\\
3.16	-0.06684\\
3.17	-0.06571\\
3.18	-0.06452\\
3.19	-0.06327\\
3.2	-0.06196\\
3.21	-0.06058\\
3.22	-0.05916\\
3.23	-0.05767\\
3.24	-0.05613\\
3.25	-0.05453\\
3.26	-0.05289\\
3.27	-0.05119\\
3.28	-0.04944\\
3.29	-0.04765\\
3.3	-0.04581\\
3.31	-0.04393\\
3.32	-0.04201\\
3.33	-0.04005\\
3.34	-0.03805\\
3.35	-0.03602\\
3.36	-0.03395\\
3.37	-0.03185\\
3.38	-0.02973\\
3.39	-0.02757\\
3.4	-0.02539\\
3.41	-0.02319\\
3.42	-0.02097\\
3.43	-0.01873\\
3.44	-0.01647\\
3.45	-0.0142\\
3.46	-0.01192\\
3.47	-0.009625\\
3.48	-0.007325\\
3.49	-0.005021\\
3.5	-0.002714\\
3.51	-0.000407\\
3.52	0.001898\\
3.53	0.0042\\
3.54	0.006495\\
3.55	0.008781\\
3.56	0.01106\\
3.57	0.01332\\
3.58	0.01557\\
3.59	0.0178\\
3.6	0.02001\\
3.61	0.02219\\
3.62	0.02436\\
3.63	0.0265\\
3.64	0.02861\\
3.65	0.03069\\
3.66	0.03274\\
3.67	0.03475\\
3.68	0.03673\\
3.69	0.03867\\
3.7	0.04057\\
3.71	0.04243\\
3.72	0.04424\\
3.73	0.04601\\
3.74	0.04774\\
3.75	0.04941\\
3.76	0.05103\\
3.77	0.05261\\
3.78	0.05413\\
3.79	0.05559\\
3.8	0.057\\
3.81	0.05836\\
3.82	0.05965\\
3.83	0.06088\\
3.84	0.06206\\
3.85	0.06317\\
3.86	0.06422\\
3.87	0.0652\\
3.88	0.06612\\
3.89	0.06697\\
3.9	0.06776\\
3.91	0.06848\\
3.92	0.06913\\
3.93	0.06972\\
3.94	0.07023\\
3.95	0.07068\\
3.96	0.07105\\
3.97	0.07136\\
3.98	0.07159\\
3.99	0.07176\\
4	0.07185\\
4.01	0.07188\\
4.02	0.07183\\
4.03	0.07172\\
4.04	0.07153\\
4.05	0.07127\\
4.06	0.07095\\
4.07	0.07055\\
4.08	0.07009\\
4.09	0.06955\\
4.1	0.06895\\
4.11	0.06829\\
4.12	0.06755\\
4.13	0.06676\\
4.14	0.06589\\
4.15	0.06496\\
4.16	0.06397\\
4.17	0.06292\\
4.18	0.06181\\
4.19	0.06064\\
4.2	0.05941\\
4.21	0.05812\\
4.22	0.05678\\
4.23	0.05538\\
4.24	0.05393\\
4.25	0.05243\\
4.26	0.05088\\
4.27	0.04928\\
4.28	0.04763\\
4.29	0.04594\\
4.3	0.04421\\
4.31	0.04243\\
4.32	0.04061\\
4.33	0.03876\\
4.34	0.03687\\
4.35	0.03494\\
4.36	0.03299\\
4.37	0.031\\
4.38	0.02898\\
4.39	0.02694\\
4.4	0.02487\\
4.41	0.02278\\
4.42	0.02068\\
4.43	0.01855\\
4.44	0.0164\\
4.45	0.01425\\
4.46	0.01208\\
4.47	0.009897\\
4.48	0.00771\\
4.49	0.005518\\
4.5	0.003322\\
4.51	0.001126\\
4.52	-0.00107\\
4.53	-0.003262\\
4.54	-0.005449\\
4.55	-0.007629\\
4.56	-0.009799\\
4.57	-0.01196\\
4.58	-0.0141\\
4.59	-0.01623\\
4.6	-0.01834\\
4.61	-0.02043\\
4.62	-0.0225\\
4.63	-0.02454\\
4.64	-0.02656\\
4.65	-0.02855\\
4.66	-0.03051\\
4.67	-0.03244\\
4.68	-0.03434\\
4.69	-0.03619\\
4.7	-0.03802\\
4.71	-0.0398\\
4.72	-0.04154\\
4.73	-0.04324\\
4.74	-0.0449\\
4.75	-0.04651\\
4.76	-0.04807\\
4.77	-0.04958\\
4.78	-0.05105\\
4.79	-0.05246\\
4.8	-0.05382\\
4.81	-0.05513\\
4.82	-0.05638\\
4.83	-0.05757\\
4.84	-0.05871\\
4.85	-0.05979\\
4.86	-0.0608\\
4.87	-0.06176\\
4.88	-0.06266\\
4.89	-0.06349\\
4.9	-0.06426\\
4.91	-0.06497\\
4.92	-0.06561\\
4.93	-0.06619\\
4.94	-0.0667\\
4.95	-0.06715\\
4.96	-0.06753\\
4.97	-0.06784\\
4.98	-0.06809\\
4.99	-0.06827\\
5	-0.06838\\
5.01	-0.06843\\
5.02	-0.06841\\
5.03	-0.06832\\
5.04	-0.06816\\
5.05	-0.06794\\
5.06	-0.06765\\
5.07	-0.0673\\
5.08	-0.06688\\
5.09	-0.06639\\
5.1	-0.06584\\
5.11	-0.06523\\
5.12	-0.06455\\
5.13	-0.06381\\
5.14	-0.06301\\
5.15	-0.06215\\
5.16	-0.06123\\
5.17	-0.06025\\
5.18	-0.05921\\
5.19	-0.05811\\
5.2	-0.05696\\
5.21	-0.05575\\
5.22	-0.05449\\
5.23	-0.05318\\
5.24	-0.05181\\
5.25	-0.0504\\
5.26	-0.04894\\
5.27	-0.04743\\
5.28	-0.04588\\
5.29	-0.04428\\
5.3	-0.04264\\
5.31	-0.04097\\
5.32	-0.03925\\
5.33	-0.0375\\
5.34	-0.03571\\
5.35	-0.03389\\
5.36	-0.03203\\
5.37	-0.03015\\
5.38	-0.02824\\
5.39	-0.0263\\
5.4	-0.02434\\
5.41	-0.02236\\
5.42	-0.02036\\
5.43	-0.01834\\
5.44	-0.0163\\
5.45	-0.01425\\
5.46	-0.01219\\
5.47	-0.01012\\
5.48	-0.00804\\
5.49	-0.005955\\
5.5	-0.003866\\
5.51	-0.001775\\
5.52	0.0003163\\
5.53	0.002405\\
5.54	0.004489\\
5.55	0.006566\\
5.56	0.008635\\
5.57	0.01069\\
5.58	0.01274\\
5.59	0.01477\\
5.6	0.01679\\
5.61	0.01878\\
5.62	0.02076\\
5.63	0.02271\\
5.64	0.02464\\
5.65	0.02655\\
5.66	0.02842\\
5.67	0.03027\\
5.68	0.03208\\
5.69	0.03387\\
5.7	0.03561\\
5.71	0.03732\\
5.72	0.03899\\
5.73	0.04063\\
5.74	0.04222\\
5.75	0.04376\\
5.76	0.04527\\
5.77	0.04673\\
5.78	0.04814\\
5.79	0.0495\\
5.8	0.05081\\
5.81	0.05207\\
5.82	0.05328\\
5.83	0.05443\\
5.84	0.05553\\
5.85	0.05658\\
5.86	0.05757\\
5.87	0.0585\\
5.88	0.05937\\
5.89	0.06018\\
5.9	0.06094\\
5.91	0.06163\\
5.92	0.06226\\
5.93	0.06284\\
5.94	0.06334\\
5.95	0.06379\\
5.96	0.06417\\
5.97	0.06449\\
5.98	0.06475\\
5.99	0.06494\\
6	0.06507\\
6.01	0.06514\\
6.02	0.06514\\
6.03	0.06507\\
6.04	0.06495\\
6.05	0.06476\\
6.06	0.06451\\
6.07	0.06419\\
6.08	0.06381\\
6.09	0.06337\\
6.1	0.06287\\
6.11	0.0623\\
6.12	0.06168\\
6.13	0.06099\\
6.14	0.06025\\
6.15	0.05945\\
6.16	0.05859\\
6.17	0.05767\\
6.18	0.0567\\
6.19	0.05568\\
6.2	0.0546\\
6.21	0.05347\\
6.22	0.05228\\
6.23	0.05105\\
6.24	0.04977\\
6.25	0.04844\\
6.26	0.04706\\
6.27	0.04564\\
6.28	0.04418\\
6.29	0.04267\\
6.3	0.04113\\
6.31	0.03954\\
6.32	0.03792\\
6.33	0.03626\\
6.34	0.03457\\
6.35	0.03285\\
6.36	0.03109\\
6.37	0.02931\\
6.38	0.0275\\
6.39	0.02566\\
6.4	0.0238\\
6.41	0.02192\\
6.42	0.02003\\
6.43	0.01811\\
6.44	0.01617\\
6.45	0.01423\\
6.46	0.01227\\
6.47	0.0103\\
6.48	0.008321\\
6.49	0.006337\\
6.5	0.004349\\
6.51	0.002359\\
6.52	0.0003678\\
6.53	-0.001621\\
6.54	-0.003607\\
6.55	-0.005587\\
6.56	-0.00756\\
6.57	-0.009523\\
6.58	-0.01148\\
6.59	-0.01341\\
6.6	-0.01534\\
6.61	-0.01725\\
6.62	-0.01913\\
6.63	-0.021\\
6.64	-0.02285\\
6.65	-0.02467\\
6.66	-0.02646\\
6.67	-0.02823\\
6.68	-0.02997\\
6.69	-0.03167\\
6.7	-0.03335\\
6.71	-0.03499\\
6.72	-0.03659\\
6.73	-0.03816\\
6.74	-0.03969\\
6.75	-0.04117\\
6.76	-0.04262\\
6.77	-0.04402\\
6.78	-0.04538\\
6.79	-0.04669\\
6.8	-0.04796\\
6.81	-0.04917\\
6.82	-0.05034\\
6.83	-0.05146\\
6.84	-0.05252\\
6.85	-0.05353\\
6.86	-0.05449\\
6.87	-0.0554\\
6.88	-0.05625\\
6.89	-0.05704\\
6.9	-0.05778\\
6.91	-0.05846\\
6.92	-0.05908\\
6.93	-0.05964\\
6.94	-0.06015\\
6.95	-0.06059\\
6.96	-0.06098\\
6.97	-0.0613\\
6.98	-0.06157\\
6.99	-0.06177\\
7	-0.06191\\
7.01	-0.062\\
7.02	-0.06202\\
7.03	-0.06198\\
7.04	-0.06188\\
7.05	-0.06172\\
7.06	-0.0615\\
7.07	-0.06122\\
7.08	-0.06087\\
7.09	-0.06047\\
7.1	-0.06002\\
7.11	-0.0595\\
7.12	-0.05892\\
7.13	-0.05829\\
7.14	-0.0576\\
7.15	-0.05686\\
7.16	-0.05606\\
7.17	-0.0552\\
7.18	-0.0543\\
7.19	-0.05334\\
7.2	-0.05233\\
7.21	-0.05127\\
7.22	-0.05016\\
7.23	-0.049\\
7.24	-0.04779\\
7.25	-0.04654\\
7.26	-0.04525\\
7.27	-0.04391\\
7.28	-0.04253\\
7.29	-0.04111\\
7.3	-0.03965\\
7.31	-0.03815\\
7.32	-0.03662\\
7.33	-0.03505\\
7.34	-0.03346\\
7.35	-0.03182\\
7.36	-0.03016\\
7.37	-0.02848\\
7.38	-0.02676\\
7.39	-0.02502\\
7.4	-0.02326\\
7.41	-0.02148\\
7.42	-0.01967\\
7.43	-0.01785\\
7.44	-0.01602\\
7.45	-0.01417\\
7.46	-0.01231\\
7.47	-0.01044\\
7.48	-0.008555\\
7.49	-0.006668\\
7.5	-0.004777\\
7.51	-0.002882\\
7.52	-0.0009871\\
7.53	0.0009073\\
7.54	0.002799\\
7.55	0.004686\\
7.56	0.006567\\
7.57	0.008439\\
7.58	0.0103\\
7.59	0.01215\\
7.6	0.01399\\
7.61	0.01581\\
7.62	0.01761\\
7.63	0.0194\\
7.64	0.02116\\
7.65	0.0229\\
7.66	0.02462\\
7.67	0.02631\\
7.68	0.02798\\
7.69	0.02961\\
7.7	0.03122\\
7.71	0.03279\\
7.72	0.03433\\
7.73	0.03583\\
7.74	0.0373\\
7.75	0.03873\\
7.76	0.04012\\
7.77	0.04146\\
7.78	0.04277\\
7.79	0.04403\\
7.8	0.04525\\
7.81	0.04643\\
7.82	0.04755\\
7.83	0.04863\\
7.84	0.04966\\
7.85	0.05065\\
7.86	0.05158\\
7.87	0.05246\\
7.88	0.05328\\
7.89	0.05406\\
7.9	0.05478\\
7.91	0.05544\\
7.92	0.05605\\
7.93	0.05661\\
7.94	0.05711\\
7.95	0.05755\\
7.96	0.05793\\
7.97	0.05826\\
7.98	0.05853\\
7.99	0.05875\\
8	0.0589\\
8.01	0.059\\
8.02	0.05904\\
8.03	0.05902\\
8.04	0.05895\\
8.05	0.05881\\
8.06	0.05862\\
8.07	0.05837\\
8.08	0.05807\\
8.09	0.05771\\
8.1	0.05729\\
8.11	0.05681\\
8.12	0.05629\\
8.13	0.0557\\
8.14	0.05506\\
8.15	0.05437\\
8.16	0.05363\\
8.17	0.05283\\
8.18	0.05199\\
8.19	0.05109\\
8.2	0.05015\\
8.21	0.04915\\
8.22	0.04811\\
8.23	0.04702\\
8.24	0.04589\\
8.25	0.04471\\
8.26	0.0435\\
8.27	0.04223\\
8.28	0.04093\\
8.29	0.0396\\
8.3	0.03822\\
8.31	0.03681\\
8.32	0.03536\\
8.33	0.03388\\
8.34	0.03236\\
8.35	0.03082\\
8.36	0.02925\\
8.37	0.02765\\
8.38	0.02603\\
8.39	0.02438\\
8.4	0.02271\\
8.41	0.02102\\
8.42	0.01931\\
8.43	0.01758\\
8.44	0.01584\\
8.45	0.01408\\
8.46	0.01231\\
8.47	0.01053\\
8.48	0.008747\\
8.49	0.006952\\
8.5	0.005153\\
8.51	0.00335\\
8.52	0.001546\\
8.53	-0.0002578\\
8.54	-0.00206\\
8.55	-0.003858\\
8.56	-0.00565\\
8.57	-0.007436\\
8.58	-0.009212\\
8.59	-0.01098\\
8.6	-0.01273\\
8.61	-0.01447\\
8.62	-0.01619\\
8.63	-0.0179\\
8.64	-0.01958\\
8.65	-0.02125\\
8.66	-0.02289\\
8.67	-0.02451\\
8.68	-0.0261\\
8.69	-0.02767\\
8.7	-0.02921\\
8.71	-0.03071\\
8.72	-0.03219\\
8.73	-0.03363\\
8.74	-0.03504\\
8.75	-0.03641\\
8.76	-0.03775\\
8.77	-0.03905\\
8.78	-0.0403\\
8.79	-0.04152\\
8.8	-0.0427\\
8.81	-0.04383\\
8.82	-0.04492\\
8.83	-0.04596\\
8.84	-0.04696\\
8.85	-0.04791\\
8.86	-0.04881\\
8.87	-0.04966\\
8.88	-0.05047\\
8.89	-0.05122\\
8.9	-0.05192\\
8.91	-0.05257\\
8.92	-0.05317\\
8.93	-0.05372\\
8.94	-0.05421\\
8.95	-0.05465\\
8.96	-0.05504\\
8.97	-0.05537\\
8.98	-0.05564\\
8.99	-0.05586\\
9	-0.05603\\
9.01	-0.05614\\
9.02	-0.0562\\
9.03	-0.0562\\
9.04	-0.05615\\
9.05	-0.05604\\
9.06	-0.05588\\
9.07	-0.05566\\
9.08	-0.05538\\
9.09	-0.05506\\
9.1	-0.05468\\
9.11	-0.05424\\
9.12	-0.05376\\
9.13	-0.05322\\
9.14	-0.05263\\
9.15	-0.05199\\
9.16	-0.0513\\
9.17	-0.05056\\
9.18	-0.04977\\
9.19	-0.04893\\
9.2	-0.04805\\
9.21	-0.04712\\
9.22	-0.04614\\
9.23	-0.04512\\
9.24	-0.04405\\
9.25	-0.04295\\
9.26	-0.0418\\
9.27	-0.04062\\
9.28	-0.03939\\
9.29	-0.03813\\
9.3	-0.03683\\
9.31	-0.03549\\
9.32	-0.03413\\
9.33	-0.03273\\
9.34	-0.0313\\
9.35	-0.02984\\
9.36	-0.02835\\
9.37	-0.02684\\
9.38	-0.0253\\
9.39	-0.02374\\
9.4	-0.02216\\
9.41	-0.02055\\
9.42	-0.01893\\
9.43	-0.01729\\
9.44	-0.01564\\
9.45	-0.01397\\
9.46	-0.01229\\
9.47	-0.0106\\
9.48	-0.0089\\
9.49	-0.007193\\
9.5	-0.005482\\
9.51	-0.003767\\
9.52	-0.002049\\
9.53	-0.0003317\\
9.54	0.001385\\
9.55	0.003098\\
9.56	0.004806\\
9.57	0.006508\\
9.58	0.008202\\
9.59	0.009887\\
9.6	0.01156\\
9.61	0.01322\\
9.62	0.01487\\
9.63	0.01649\\
9.64	0.01811\\
9.65	0.0197\\
9.66	0.02127\\
9.67	0.02282\\
9.68	0.02434\\
9.69	0.02584\\
9.7	0.02732\\
9.71	0.02876\\
9.72	0.03018\\
9.73	0.03156\\
9.74	0.03291\\
9.75	0.03423\\
9.76	0.03551\\
9.77	0.03676\\
9.78	0.03797\\
9.79	0.03914\\
9.8	0.04028\\
9.81	0.04137\\
9.82	0.04242\\
9.83	0.04342\\
9.84	0.04439\\
9.85	0.04531\\
9.86	0.04618\\
9.87	0.04701\\
9.88	0.04779\\
9.89	0.04852\\
9.9	0.04921\\
9.91	0.04985\\
9.92	0.05043\\
9.93	0.05097\\
9.94	0.05146\\
9.95	0.05189\\
9.96	0.05228\\
9.97	0.05261\\
9.98	0.05289\\
9.99	0.05312\\
10	0.05329\\
10.01	0.05342\\
10.02	0.05349\\
10.03	0.05351\\
10.04	0.05348\\
10.05	0.05339\\
10.06	0.05325\\
10.07	0.05306\\
10.08	0.05282\\
10.09	0.05253\\
10.1	0.05218\\
10.11	0.05179\\
10.12	0.05134\\
10.13	0.05084\\
10.14	0.0503\\
10.15	0.0497\\
10.16	0.04906\\
10.17	0.04837\\
10.18	0.04764\\
10.19	0.04686\\
10.2	0.04603\\
10.21	0.04516\\
10.22	0.04424\\
10.23	0.04328\\
10.24	0.04228\\
10.25	0.04125\\
10.26	0.04017\\
10.27	0.03905\\
10.28	0.0379\\
10.29	0.03671\\
10.3	0.03548\\
10.31	0.03422\\
10.32	0.03293\\
10.33	0.03161\\
10.34	0.03026\\
10.35	0.02888\\
10.36	0.02747\\
10.37	0.02604\\
10.38	0.02458\\
10.39	0.0231\\
10.4	0.0216\\
10.41	0.02008\\
10.42	0.01854\\
10.43	0.01699\\
10.44	0.01542\\
10.45	0.01383\\
10.46	0.01224\\
10.47	0.01063\\
10.48	0.009017\\
10.49	0.007395\\
10.5	0.005767\\
10.51	0.004135\\
10.52	0.002501\\
10.53	0.0008653\\
10.54	-0.0007691\\
10.55	-0.002401\\
10.56	-0.004029\\
10.57	-0.005652\\
10.58	-0.007268\\
10.59	-0.008874\\
10.6	-0.01047\\
10.61	-0.01206\\
10.62	-0.01363\\
10.63	-0.01518\\
10.64	-0.01672\\
10.65	-0.01825\\
10.66	-0.01975\\
10.67	-0.02123\\
10.68	-0.02269\\
10.69	-0.02413\\
10.7	-0.02554\\
10.71	-0.02692\\
10.72	-0.02828\\
10.73	-0.02961\\
10.74	-0.0309\\
10.75	-0.03217\\
10.76	-0.0334\\
10.77	-0.0346\\
10.78	-0.03577\\
10.79	-0.03689\\
10.8	-0.03798\\
10.81	-0.03904\\
10.82	-0.04005\\
10.83	-0.04102\\
10.84	-0.04195\\
10.85	-0.04284\\
10.86	-0.04369\\
10.87	-0.04449\\
10.88	-0.04525\\
10.89	-0.04597\\
10.9	-0.04663\\
10.91	-0.04726\\
10.92	-0.04783\\
10.93	-0.04836\\
10.94	-0.04884\\
10.95	-0.04927\\
10.96	-0.04965\\
10.97	-0.04998\\
10.98	-0.05027\\
10.99	-0.0505\\
11	-0.05069\\
11.01	-0.05082\\
11.02	-0.0509\\
11.03	-0.05094\\
11.04	-0.05092\\
11.05	-0.05086\\
11.06	-0.05074\\
11.07	-0.05058\\
11.08	-0.05037\\
11.09	-0.0501\\
11.1	-0.04979\\
11.11	-0.04943\\
11.12	-0.04902\\
11.13	-0.04857\\
11.14	-0.04806\\
11.15	-0.04751\\
11.16	-0.04692\\
11.17	-0.04628\\
11.18	-0.04559\\
11.19	-0.04486\\
11.2	-0.04409\\
11.21	-0.04327\\
11.22	-0.04241\\
11.23	-0.04152\\
11.24	-0.04058\\
11.25	-0.0396\\
11.26	-0.03859\\
11.27	-0.03754\\
11.28	-0.03645\\
11.29	-0.03533\\
11.3	-0.03417\\
11.31	-0.03298\\
11.32	-0.03177\\
11.33	-0.03052\\
11.34	-0.02924\\
11.35	-0.02794\\
11.36	-0.0266\\
11.37	-0.02525\\
11.38	-0.02387\\
11.39	-0.02247\\
11.4	-0.02105\\
11.41	-0.01961\\
11.42	-0.01815\\
11.43	-0.01667\\
11.44	-0.01518\\
11.45	-0.01368\\
11.46	-0.01216\\
11.47	-0.01064\\
11.48	-0.009102\\
11.49	-0.00756\\
11.5	-0.006012\\
11.51	-0.004459\\
11.52	-0.002904\\
11.53	-0.001347\\
11.54	0.0002094\\
11.55	0.001764\\
11.56	0.003316\\
11.57	0.004862\\
11.58	0.006403\\
11.59	0.007935\\
11.6	0.009459\\
11.61	0.01097\\
11.62	0.01247\\
11.63	0.01396\\
11.64	0.01543\\
11.65	0.01688\\
11.66	0.01832\\
11.67	0.01974\\
11.68	0.02114\\
11.69	0.02251\\
11.7	0.02386\\
11.71	0.02519\\
11.72	0.02649\\
11.73	0.02776\\
11.74	0.02901\\
11.75	0.03022\\
11.76	0.03141\\
11.77	0.03256\\
11.78	0.03368\\
11.79	0.03477\\
11.8	0.03582\\
11.81	0.03683\\
11.82	0.03781\\
11.83	0.03875\\
11.84	0.03965\\
11.85	0.04051\\
11.86	0.04133\\
11.87	0.04211\\
11.88	0.04284\\
11.89	0.04354\\
11.9	0.04419\\
11.91	0.04479\\
11.92	0.04536\\
11.93	0.04587\\
11.94	0.04634\\
11.95	0.04677\\
11.96	0.04715\\
11.97	0.04748\\
11.98	0.04777\\
11.99	0.04801\\
12	0.0482\\
12.01	0.04834\\
12.02	0.04844\\
12.03	0.04849\\
12.04	0.04849\\
12.05	0.04844\\
12.06	0.04835\\
12.07	0.04821\\
12.08	0.04802\\
12.09	0.04779\\
12.1	0.04751\\
12.11	0.04718\\
12.12	0.04681\\
12.13	0.04639\\
12.14	0.04592\\
12.15	0.04541\\
12.16	0.04486\\
12.17	0.04427\\
12.18	0.04363\\
12.19	0.04295\\
12.2	0.04222\\
12.21	0.04146\\
12.22	0.04066\\
12.23	0.03982\\
12.24	0.03894\\
12.25	0.03802\\
12.26	0.03706\\
12.27	0.03608\\
12.28	0.03505\\
12.29	0.03399\\
12.3	0.03291\\
12.31	0.03179\\
12.32	0.03063\\
12.33	0.02946\\
12.34	0.02825\\
12.35	0.02701\\
12.36	0.02576\\
12.37	0.02447\\
12.38	0.02317\\
12.39	0.02184\\
12.4	0.02049\\
12.41	0.01913\\
12.42	0.01774\\
12.43	0.01634\\
12.44	0.01493\\
12.45	0.0135\\
12.46	0.01206\\
12.47	0.01062\\
12.48	0.009157\\
12.49	0.007691\\
12.5	0.006219\\
12.51	0.004742\\
12.52	0.003262\\
12.53	0.001781\\
12.54	0.0002986\\
12.55	-0.001182\\
12.56	-0.00266\\
12.57	-0.004135\\
12.58	-0.005603\\
12.59	-0.007065\\
12.6	-0.008518\\
12.61	-0.009962\\
12.62	-0.01139\\
12.63	-0.01281\\
12.64	-0.01422\\
12.65	-0.01561\\
12.66	-0.01698\\
12.67	-0.01834\\
12.68	-0.01968\\
12.69	-0.02099\\
12.7	-0.02229\\
12.71	-0.02356\\
12.72	-0.0248\\
12.73	-0.02603\\
12.74	-0.02722\\
12.75	-0.02839\\
12.76	-0.02952\\
12.77	-0.03063\\
12.78	-0.03171\\
12.79	-0.03275\\
12.8	-0.03376\\
12.81	-0.03474\\
12.82	-0.03568\\
12.83	-0.03659\\
12.84	-0.03746\\
12.85	-0.03829\\
12.86	-0.03909\\
12.87	-0.03984\\
12.88	-0.04055\\
12.89	-0.04123\\
12.9	-0.04186\\
12.91	-0.04245\\
12.92	-0.043\\
12.93	-0.04351\\
12.94	-0.04397\\
12.95	-0.04439\\
12.96	-0.04477\\
12.97	-0.0451\\
12.98	-0.04539\\
12.99	-0.04563\\
13	-0.04583\\
13.01	-0.04598\\
13.02	-0.04609\\
13.03	-0.04615\\
13.04	-0.04617\\
13.05	-0.04614\\
13.06	-0.04607\\
13.07	-0.04595\\
13.08	-0.04578\\
13.09	-0.04557\\
13.1	-0.04532\\
13.11	-0.04502\\
13.12	-0.04468\\
13.13	-0.0443\\
13.14	-0.04387\\
13.15	-0.0434\\
13.16	-0.04289\\
13.17	-0.04234\\
13.18	-0.04174\\
13.19	-0.04111\\
13.2	-0.04043\\
13.21	-0.03972\\
13.22	-0.03897\\
13.23	-0.03818\\
13.24	-0.03735\\
13.25	-0.03649\\
13.26	-0.03559\\
13.27	-0.03466\\
13.28	-0.0337\\
13.29	-0.0327\\
13.3	-0.03168\\
13.31	-0.03062\\
13.32	-0.02954\\
13.33	-0.02842\\
13.34	-0.02728\\
13.35	-0.02612\\
13.36	-0.02492\\
13.37	-0.02371\\
13.38	-0.02248\\
13.39	-0.02122\\
13.4	-0.01994\\
13.41	-0.01865\\
13.42	-0.01734\\
13.43	-0.01601\\
13.44	-0.01467\\
13.45	-0.01331\\
13.46	-0.01195\\
13.47	-0.01057\\
13.48	-0.009185\\
13.49	-0.007792\\
13.5	-0.006392\\
13.51	-0.004987\\
13.52	-0.003579\\
13.53	-0.002169\\
13.54	-0.0007584\\
13.55	0.0006518\\
13.56	0.00206\\
13.57	0.003465\\
13.58	0.004865\\
13.59	0.006259\\
13.6	0.007645\\
13.61	0.009023\\
13.62	0.01039\\
13.63	0.01175\\
13.64	0.01309\\
13.65	0.01442\\
13.66	0.01573\\
13.67	0.01703\\
13.68	0.01831\\
13.69	0.01957\\
13.7	0.02081\\
13.71	0.02202\\
13.72	0.02322\\
13.73	0.02439\\
13.74	0.02553\\
13.75	0.02665\\
13.76	0.02775\\
13.77	0.02881\\
13.78	0.02985\\
13.79	0.03085\\
13.8	0.03182\\
13.81	0.03276\\
13.82	0.03367\\
13.83	0.03455\\
13.84	0.03539\\
13.85	0.03619\\
13.86	0.03696\\
13.87	0.03769\\
13.88	0.03838\\
13.89	0.03904\\
13.9	0.03965\\
13.91	0.04023\\
13.92	0.04077\\
13.93	0.04126\\
13.94	0.04172\\
13.95	0.04213\\
13.96	0.0425\\
13.97	0.04283\\
13.98	0.04312\\
13.99	0.04337\\
14	0.04357\\
14.01	0.04373\\
14.02	0.04385\\
14.03	0.04392\\
14.04	0.04395\\
14.05	0.04394\\
14.06	0.04388\\
14.07	0.04378\\
14.08	0.04364\\
14.09	0.04346\\
14.1	0.04323\\
14.11	0.04296\\
14.12	0.04265\\
14.13	0.0423\\
14.14	0.04191\\
14.15	0.04147\\
14.16	0.041\\
14.17	0.04049\\
14.18	0.03993\\
14.19	0.03934\\
14.2	0.03871\\
14.21	0.03804\\
14.22	0.03734\\
14.23	0.0366\\
14.24	0.03583\\
14.25	0.03502\\
14.26	0.03418\\
14.27	0.0333\\
14.28	0.03239\\
14.29	0.03146\\
14.3	0.03049\\
14.31	0.02949\\
14.32	0.02847\\
14.33	0.02742\\
14.34	0.02634\\
14.35	0.02524\\
14.36	0.02411\\
14.37	0.02296\\
14.38	0.02179\\
14.39	0.0206\\
14.4	0.0194\\
14.41	0.01817\\
14.42	0.01693\\
14.43	0.01567\\
14.44	0.01439\\
14.45	0.01311\\
14.46	0.01181\\
14.47	0.0105\\
14.48	0.009188\\
14.49	0.007864\\
14.5	0.006534\\
14.51	0.005198\\
14.52	0.003858\\
14.53	0.002517\\
14.54	0.001173\\
14.55	-0.0001694\\
14.56	-0.001511\\
14.57	-0.002849\\
14.58	-0.004184\\
14.59	-0.005513\\
14.6	-0.006836\\
14.61	-0.00815\\
14.62	-0.009455\\
14.63	-0.01075\\
14.64	-0.01203\\
14.65	-0.0133\\
14.66	-0.01456\\
14.67	-0.0158\\
14.68	-0.01702\\
14.69	-0.01823\\
14.7	-0.01941\\
14.71	-0.02058\\
14.72	-0.02172\\
14.73	-0.02285\\
14.74	-0.02394\\
14.75	-0.02502\\
14.76	-0.02607\\
14.77	-0.02709\\
14.78	-0.02809\\
14.79	-0.02905\\
14.8	-0.02999\\
14.81	-0.03089\\
14.82	-0.03177\\
14.83	-0.03261\\
14.84	-0.03342\\
14.85	-0.0342\\
14.86	-0.03494\\
14.87	-0.03565\\
14.88	-0.03632\\
14.89	-0.03696\\
14.9	-0.03756\\
14.91	-0.03812\\
14.92	-0.03864\\
14.93	-0.03913\\
14.94	-0.03957\\
14.95	-0.03998\\
14.96	-0.04035\\
14.97	-0.04068\\
14.98	-0.04097\\
14.99	-0.04121\\
15	-0.04142\\
15.01	-0.04158\\
15.02	-0.04171\\
15.03	-0.04179\\
15.04	-0.04184\\
15.05	-0.04184\\
15.06	-0.0418\\
15.07	-0.04172\\
15.08	-0.0416\\
15.09	-0.04144\\
15.1	-0.04123\\
15.11	-0.04099\\
15.12	-0.04071\\
15.13	-0.04039\\
15.14	-0.04003\\
15.15	-0.03963\\
15.16	-0.03919\\
15.17	-0.03871\\
15.18	-0.0382\\
15.19	-0.03765\\
15.2	-0.03706\\
15.21	-0.03644\\
15.22	-0.03578\\
15.23	-0.03509\\
15.24	-0.03436\\
15.25	-0.0336\\
15.26	-0.03281\\
15.27	-0.03199\\
15.28	-0.03113\\
15.29	-0.03025\\
15.3	-0.02934\\
15.31	-0.0284\\
15.32	-0.02743\\
15.33	-0.02644\\
15.34	-0.02542\\
15.35	-0.02438\\
15.36	-0.02332\\
15.37	-0.02223\\
15.38	-0.02112\\
15.39	-0.02\\
15.4	-0.01885\\
15.41	-0.01769\\
15.42	-0.01651\\
15.43	-0.01532\\
15.44	-0.01411\\
15.45	-0.01289\\
15.46	-0.01166\\
15.47	-0.01042\\
15.48	-0.009169\\
15.49	-0.007911\\
15.5	-0.006646\\
15.51	-0.005376\\
15.52	-0.004102\\
15.53	-0.002825\\
15.54	-0.001547\\
15.55	-0.0002683\\
15.56	0.001009\\
15.57	0.002285\\
15.58	0.003557\\
15.59	0.004824\\
15.6	0.006085\\
15.61	0.007339\\
15.62	0.008585\\
15.63	0.009821\\
15.64	0.01105\\
15.65	0.01226\\
15.66	0.01346\\
15.67	0.01464\\
15.68	0.01581\\
15.69	0.01697\\
15.7	0.0181\\
15.71	0.01922\\
15.72	0.02032\\
15.73	0.02139\\
15.74	0.02245\\
15.75	0.02348\\
15.76	0.02448\\
15.77	0.02547\\
15.78	0.02642\\
15.79	0.02735\\
15.8	0.02825\\
15.81	0.02912\\
15.82	0.02997\\
15.83	0.03078\\
15.84	0.03156\\
15.85	0.03231\\
15.86	0.03303\\
15.87	0.03372\\
15.88	0.03437\\
15.89	0.03499\\
15.9	0.03557\\
15.91	0.03611\\
15.92	0.03662\\
15.93	0.0371\\
15.94	0.03754\\
15.95	0.03794\\
15.96	0.0383\\
15.97	0.03862\\
15.98	0.03891\\
15.99	0.03916\\
16	0.03937\\
16.01	0.03954\\
16.02	0.03967\\
16.03	0.03976\\
16.04	0.03982\\
16.05	0.03983\\
16.06	0.03981\\
16.07	0.03975\\
16.08	0.03964\\
16.09	0.0395\\
16.1	0.03932\\
16.11	0.03911\\
16.12	0.03885\\
16.13	0.03856\\
16.14	0.03822\\
16.15	0.03786\\
16.16	0.03745\\
16.17	0.03701\\
16.18	0.03653\\
16.19	0.03602\\
16.2	0.03547\\
16.21	0.03489\\
16.22	0.03427\\
16.23	0.03363\\
16.24	0.03295\\
16.25	0.03223\\
16.26	0.03149\\
16.27	0.03072\\
16.28	0.02991\\
16.29	0.02908\\
16.3	0.02822\\
16.31	0.02734\\
16.32	0.02643\\
16.33	0.02549\\
16.34	0.02453\\
16.35	0.02355\\
16.36	0.02254\\
16.37	0.02151\\
16.38	0.02047\\
16.39	0.0194\\
16.4	0.01832\\
16.41	0.01722\\
16.42	0.0161\\
16.43	0.01497\\
16.44	0.01382\\
16.45	0.01267\\
16.46	0.0115\\
16.47	0.01032\\
16.48	0.00913\\
16.49	0.007935\\
16.5	0.006733\\
16.51	0.005525\\
16.52	0.004313\\
16.53	0.003099\\
16.54	0.001882\\
16.55	0.0006646\\
16.56	-0.0005522\\
16.57	-0.001767\\
16.58	-0.002979\\
16.59	-0.004187\\
16.6	-0.00539\\
16.61	-0.006586\\
16.62	-0.007775\\
16.63	-0.008955\\
16.64	-0.01012\\
16.65	-0.01128\\
16.66	-0.01243\\
16.67	-0.01356\\
16.68	-0.01468\\
16.69	-0.01579\\
16.7	-0.01687\\
16.71	-0.01794\\
16.72	-0.01899\\
16.73	-0.02002\\
16.74	-0.02103\\
16.75	-0.02202\\
16.76	-0.02299\\
16.77	-0.02393\\
16.78	-0.02485\\
16.79	-0.02574\\
16.8	-0.02661\\
16.81	-0.02745\\
16.82	-0.02826\\
16.83	-0.02905\\
16.84	-0.0298\\
16.85	-0.03053\\
16.86	-0.03122\\
16.87	-0.03188\\
16.88	-0.03251\\
16.89	-0.03311\\
16.9	-0.03368\\
16.91	-0.03421\\
16.92	-0.03471\\
16.93	-0.03517\\
16.94	-0.0356\\
16.95	-0.03599\\
16.96	-0.03635\\
16.97	-0.03667\\
16.98	-0.03696\\
16.99	-0.0372\\
17	-0.03742\\
17.01	-0.03759\\
17.02	-0.03773\\
17.03	-0.03783\\
17.04	-0.03789\\
17.05	-0.03792\\
17.06	-0.03791\\
17.07	-0.03786\\
17.08	-0.03778\\
17.09	-0.03766\\
17.1	-0.0375\\
17.11	-0.0373\\
17.12	-0.03707\\
17.13	-0.0368\\
17.14	-0.0365\\
17.15	-0.03616\\
17.16	-0.03579\\
17.17	-0.03538\\
17.18	-0.03493\\
17.19	-0.03446\\
17.2	-0.03395\\
17.21	-0.03341\\
17.22	-0.03283\\
17.23	-0.03222\\
17.24	-0.03159\\
17.25	-0.03092\\
17.26	-0.03022\\
17.27	-0.02949\\
17.28	-0.02874\\
17.29	-0.02796\\
17.3	-0.02715\\
17.31	-0.02631\\
17.32	-0.02545\\
17.33	-0.02457\\
17.34	-0.02366\\
17.35	-0.02273\\
17.36	-0.02178\\
17.37	-0.02081\\
17.38	-0.01982\\
17.39	-0.01881\\
17.4	-0.01779\\
17.41	-0.01674\\
17.42	-0.01569\\
17.43	-0.01461\\
17.44	-0.01353\\
17.45	-0.01243\\
17.46	-0.01132\\
17.47	-0.0102\\
17.48	-0.009073\\
17.49	-0.007937\\
17.5	-0.006795\\
17.51	-0.005647\\
17.52	-0.004494\\
17.53	-0.003339\\
17.54	-0.002181\\
17.55	-0.001022\\
17.56	0.0001365\\
17.57	0.001294\\
17.58	0.002449\\
17.59	0.0036\\
17.6	0.004747\\
17.61	0.005888\\
17.62	0.007022\\
17.63	0.008148\\
17.64	0.009265\\
17.65	0.01037\\
17.66	0.01147\\
17.67	0.01255\\
17.68	0.01362\\
17.69	0.01468\\
17.7	0.01572\\
17.71	0.01674\\
17.72	0.01775\\
17.73	0.01873\\
17.74	0.0197\\
17.75	0.02065\\
17.76	0.02158\\
17.77	0.02248\\
17.78	0.02337\\
17.79	0.02422\\
17.8	0.02506\\
17.81	0.02587\\
17.82	0.02665\\
17.83	0.0274\\
17.84	0.02813\\
17.85	0.02883\\
17.86	0.0295\\
17.87	0.03014\\
17.88	0.03075\\
17.89	0.03133\\
17.9	0.03188\\
17.91	0.0324\\
17.92	0.03289\\
17.93	0.03334\\
17.94	0.03376\\
17.95	0.03414\\
17.96	0.03449\\
17.97	0.03481\\
17.98	0.03509\\
17.99	0.03534\\
18	0.03556\\
18.01	0.03573\\
18.02	0.03588\\
18.03	0.03599\\
18.04	0.03606\\
18.05	0.0361\\
18.06	0.0361\\
18.07	0.03606\\
18.08	0.036\\
18.09	0.03589\\
18.1	0.03575\\
18.11	0.03558\\
18.12	0.03537\\
18.13	0.03513\\
18.14	0.03485\\
18.15	0.03454\\
18.16	0.03419\\
18.17	0.03381\\
18.18	0.0334\\
18.19	0.03296\\
18.2	0.03249\\
18.21	0.03198\\
18.22	0.03144\\
18.23	0.03087\\
18.24	0.03028\\
18.25	0.02965\\
18.26	0.029\\
18.27	0.02831\\
18.28	0.0276\\
18.29	0.02687\\
18.3	0.02611\\
18.31	0.02532\\
18.32	0.02451\\
18.33	0.02368\\
18.34	0.02282\\
18.35	0.02194\\
18.36	0.02104\\
18.37	0.02013\\
18.38	0.01919\\
18.39	0.01823\\
18.4	0.01726\\
18.41	0.01628\\
18.42	0.01527\\
18.43	0.01426\\
18.44	0.01323\\
18.45	0.01219\\
18.46	0.01113\\
18.47	0.01007\\
18.48	0.008999\\
18.49	0.00792\\
18.5	0.006835\\
18.51	0.005744\\
18.52	0.004648\\
18.53	0.003549\\
18.54	0.002447\\
18.55	0.001344\\
18.56	0.0002408\\
18.57	-0.0008616\\
18.58	-0.001962\\
18.59	-0.003059\\
18.6	-0.004153\\
18.61	-0.005241\\
18.62	-0.006323\\
18.63	-0.007398\\
18.64	-0.008464\\
18.65	-0.009521\\
18.66	-0.01057\\
18.67	-0.0116\\
18.68	-0.01263\\
18.69	-0.01364\\
18.7	-0.01463\\
18.71	-0.01561\\
18.72	-0.01658\\
18.73	-0.01752\\
18.74	-0.01845\\
18.75	-0.01936\\
18.76	-0.02025\\
18.77	-0.02112\\
18.78	-0.02196\\
18.79	-0.02279\\
18.8	-0.02359\\
18.81	-0.02437\\
18.82	-0.02512\\
18.83	-0.02585\\
18.84	-0.02655\\
18.85	-0.02723\\
18.86	-0.02788\\
18.87	-0.02849\\
18.88	-0.02909\\
18.89	-0.02965\\
18.9	-0.03018\\
18.91	-0.03068\\
18.92	-0.03116\\
18.93	-0.0316\\
18.94	-0.03201\\
18.95	-0.03238\\
18.96	-0.03273\\
18.97	-0.03304\\
18.98	-0.03332\\
18.99	-0.03357\\
19	-0.03379\\
19.01	-0.03397\\
19.02	-0.03411\\
19.03	-0.03423\\
19.04	-0.03431\\
19.05	-0.03436\\
19.06	-0.03437\\
19.07	-0.03435\\
19.08	-0.03429\\
19.09	-0.03421\\
19.1	-0.03408\\
19.11	-0.03393\\
19.12	-0.03374\\
19.13	-0.03352\\
19.14	-0.03327\\
19.15	-0.03298\\
19.16	-0.03266\\
19.17	-0.03232\\
19.18	-0.03193\\
19.19	-0.03152\\
19.2	-0.03108\\
19.21	-0.03061\\
19.22	-0.03011\\
19.23	-0.02958\\
19.24	-0.02902\\
19.25	-0.02843\\
19.26	-0.02782\\
19.27	-0.02718\\
19.28	-0.02651\\
19.29	-0.02582\\
19.3	-0.0251\\
19.31	-0.02436\\
19.32	-0.02359\\
19.33	-0.02281\\
19.34	-0.022\\
19.35	-0.02117\\
19.36	-0.02032\\
19.37	-0.01946\\
19.38	-0.01857\\
19.39	-0.01767\\
19.4	-0.01675\\
19.41	-0.01581\\
19.42	-0.01486\\
19.43	-0.0139\\
19.44	-0.01292\\
19.45	-0.01193\\
19.46	-0.01094\\
19.47	-0.009928\\
19.48	-0.008911\\
19.49	-0.007886\\
19.5	-0.006855\\
19.51	-0.005818\\
19.52	-0.004776\\
19.53	-0.00373\\
19.54	-0.002682\\
19.55	-0.001632\\
19.56	-0.0005822\\
19.57	0.0004677\\
19.58	0.001516\\
19.59	0.002562\\
19.6	0.003604\\
19.61	0.004642\\
19.62	0.005674\\
19.63	0.0067\\
19.64	0.007718\\
19.65	0.008727\\
19.66	0.009727\\
19.67	0.01072\\
19.68	0.01169\\
19.69	0.01266\\
19.7	0.01361\\
19.71	0.01455\\
19.72	0.01547\\
19.73	0.01638\\
19.74	0.01727\\
19.75	0.01814\\
19.76	0.01899\\
19.77	0.01983\\
19.78	0.02064\\
19.79	0.02143\\
19.8	0.02221\\
19.81	0.02295\\
19.82	0.02368\\
19.83	0.02438\\
19.84	0.02506\\
19.85	0.02571\\
19.86	0.02633\\
19.87	0.02693\\
19.88	0.0275\\
19.89	0.02805\\
19.9	0.02857\\
19.91	0.02905\\
19.92	0.02951\\
19.93	0.02994\\
19.94	0.03034\\
19.95	0.03071\\
19.96	0.03105\\
19.97	0.03136\\
19.98	0.03164\\
19.99	0.03188\\
20	0.0321\\
20.01	0.03228\\
20.02	0.03243\\
20.03	0.03255\\
20.04	0.03264\\
20.05	0.03269\\
20.06	0.03272\\
20.07	0.03271\\
20.08	0.03267\\
20.09	0.0326\\
20.1	0.03249\\
20.11	0.03235\\
20.12	0.03219\\
20.13	0.03199\\
20.14	0.03176\\
20.15	0.03149\\
20.16	0.0312\\
20.17	0.03088\\
20.18	0.03053\\
20.19	0.03015\\
20.2	0.02973\\
20.21	0.02929\\
20.22	0.02883\\
20.23	0.02833\\
20.24	0.02781\\
20.25	0.02726\\
20.26	0.02668\\
20.27	0.02608\\
20.28	0.02545\\
20.29	0.0248\\
20.3	0.02413\\
20.31	0.02343\\
20.32	0.02271\\
20.33	0.02197\\
20.34	0.02121\\
20.35	0.02042\\
20.36	0.01962\\
20.37	0.0188\\
20.38	0.01796\\
20.39	0.01711\\
20.4	0.01624\\
20.41	0.01535\\
20.42	0.01445\\
20.43	0.01354\\
20.44	0.01262\\
20.45	0.01168\\
20.46	0.01073\\
20.47	0.009775\\
20.48	0.00881\\
20.49	0.007836\\
20.5	0.006856\\
20.51	0.005871\\
20.52	0.00488\\
20.53	0.003886\\
20.54	0.002889\\
20.55	0.00189\\
20.56	0.0008903\\
20.57	-0.0001094\\
20.58	-0.001108\\
20.59	-0.002105\\
20.6	-0.003098\\
20.61	-0.004088\\
20.62	-0.005072\\
20.63	-0.006051\\
20.64	-0.007022\\
20.65	-0.007986\\
20.66	-0.008941\\
20.67	-0.009887\\
20.68	-0.01082\\
20.69	-0.01174\\
20.7	-0.01266\\
20.71	-0.01355\\
20.72	-0.01444\\
20.73	-0.0153\\
20.74	-0.01616\\
20.75	-0.01699\\
20.76	-0.01781\\
20.77	-0.01861\\
20.78	-0.01939\\
20.79	-0.02015\\
20.8	-0.02089\\
20.81	-0.02161\\
20.82	-0.02231\\
20.83	-0.02299\\
20.84	-0.02364\\
20.85	-0.02427\\
20.86	-0.02487\\
20.87	-0.02545\\
20.88	-0.026\\
20.89	-0.02653\\
20.9	-0.02703\\
20.91	-0.02751\\
20.92	-0.02795\\
20.93	-0.02837\\
20.94	-0.02876\\
20.95	-0.02912\\
20.96	-0.02946\\
20.97	-0.02976\\
20.98	-0.03003\\
20.99	-0.03028\\
21	-0.03049\\
21.01	-0.03068\\
21.02	-0.03083\\
21.03	-0.03095\\
21.04	-0.03105\\
21.05	-0.03111\\
21.06	-0.03114\\
21.07	-0.03115\\
21.08	-0.03112\\
21.09	-0.03106\\
21.1	-0.03097\\
21.11	-0.03085\\
21.12	-0.0307\\
21.13	-0.03052\\
21.14	-0.03031\\
21.15	-0.03007\\
21.16	-0.0298\\
21.17	-0.0295\\
21.18	-0.02918\\
21.19	-0.02882\\
21.2	-0.02844\\
21.21	-0.02803\\
21.22	-0.0276\\
21.23	-0.02713\\
21.24	-0.02664\\
21.25	-0.02613\\
21.26	-0.02559\\
21.27	-0.02502\\
21.28	-0.02443\\
21.29	-0.02382\\
21.3	-0.02319\\
21.31	-0.02253\\
21.32	-0.02185\\
21.33	-0.02115\\
21.34	-0.02043\\
21.35	-0.0197\\
21.36	-0.01894\\
21.37	-0.01816\\
21.38	-0.01737\\
21.39	-0.01656\\
21.4	-0.01574\\
21.41	-0.0149\\
21.42	-0.01405\\
21.43	-0.01318\\
21.44	-0.01231\\
21.45	-0.01142\\
21.46	-0.01052\\
21.47	-0.009613\\
21.48	-0.008696\\
21.49	-0.007772\\
21.5	-0.006842\\
21.51	-0.005905\\
21.52	-0.004964\\
21.53	-0.004018\\
21.54	-0.00307\\
21.55	-0.002119\\
21.56	-0.001168\\
21.57	-0.0002157\\
21.58	0.0007354\\
21.59	0.001685\\
21.6	0.002632\\
21.61	0.003575\\
21.62	0.004514\\
21.63	0.005448\\
21.64	0.006375\\
21.65	0.007295\\
21.66	0.008207\\
21.67	0.009111\\
21.68	0.01\\
21.69	0.01089\\
21.7	0.01176\\
21.71	0.01262\\
21.72	0.01346\\
21.73	0.01429\\
21.74	0.01511\\
21.75	0.01591\\
21.76	0.0167\\
21.77	0.01746\\
21.78	0.01821\\
21.79	0.01895\\
21.8	0.01966\\
21.81	0.02035\\
21.82	0.02102\\
21.83	0.02167\\
21.84	0.0223\\
21.85	0.0229\\
21.86	0.02349\\
21.87	0.02405\\
21.88	0.02458\\
21.89	0.02509\\
21.9	0.02558\\
21.91	0.02604\\
21.92	0.02647\\
21.93	0.02688\\
21.94	0.02726\\
21.95	0.02761\\
21.96	0.02794\\
21.97	0.02824\\
21.98	0.02851\\
21.99	0.02875\\
22	0.02896\\
22.01	0.02915\\
22.02	0.0293\\
22.03	0.02943\\
22.04	0.02953\\
22.05	0.0296\\
22.06	0.02964\\
22.07	0.02965\\
22.08	0.02964\\
22.09	0.02959\\
22.1	0.02951\\
22.11	0.02941\\
22.12	0.02928\\
22.13	0.02911\\
22.14	0.02892\\
22.15	0.02871\\
22.16	0.02846\\
22.17	0.02819\\
22.18	0.02788\\
22.19	0.02756\\
22.2	0.0272\\
22.21	0.02682\\
22.22	0.02641\\
22.23	0.02598\\
22.24	0.02552\\
22.25	0.02504\\
22.26	0.02453\\
22.27	0.024\\
22.28	0.02345\\
22.29	0.02288\\
22.3	0.02228\\
22.31	0.02166\\
22.32	0.02102\\
22.33	0.02036\\
22.34	0.01968\\
22.35	0.01899\\
22.36	0.01827\\
22.37	0.01754\\
22.38	0.01679\\
22.39	0.01603\\
22.4	0.01525\\
22.41	0.01446\\
22.42	0.01365\\
22.43	0.01283\\
22.44	0.012\\
22.45	0.01116\\
22.46	0.0103\\
22.47	0.009443\\
22.48	0.008573\\
22.49	0.007696\\
22.5	0.006812\\
22.51	0.005922\\
22.52	0.005027\\
22.53	0.004128\\
22.54	0.003226\\
22.55	0.002322\\
22.56	0.001416\\
22.57	0.0005099\\
22.58	-0.0003959\\
22.59	-0.0013\\
22.6	-0.002203\\
22.61	-0.003102\\
22.62	-0.003998\\
22.63	-0.004888\\
22.64	-0.005773\\
22.65	-0.006652\\
22.66	-0.007523\\
22.67	-0.008385\\
22.68	-0.009239\\
22.69	-0.01008\\
22.7	-0.01092\\
22.71	-0.01174\\
22.72	-0.01255\\
22.73	-0.01334\\
22.74	-0.01413\\
22.75	-0.01489\\
22.76	-0.01565\\
22.77	-0.01638\\
22.78	-0.0171\\
22.79	-0.0178\\
22.8	-0.01849\\
22.81	-0.01915\\
22.82	-0.0198\\
22.83	-0.02042\\
22.84	-0.02103\\
22.85	-0.02161\\
22.86	-0.02218\\
22.87	-0.02272\\
22.88	-0.02323\\
22.89	-0.02373\\
22.9	-0.0242\\
22.91	-0.02464\\
22.92	-0.02506\\
22.93	-0.02546\\
22.94	-0.02583\\
22.95	-0.02618\\
22.96	-0.0265\\
22.97	-0.02679\\
22.98	-0.02705\\
22.99	-0.02729\\
23	-0.02751\\
23.01	-0.02769\\
23.02	-0.02785\\
23.03	-0.02798\\
23.04	-0.02809\\
23.05	-0.02816\\
23.06	-0.02821\\
23.07	-0.02823\\
23.08	-0.02822\\
23.09	-0.02819\\
23.1	-0.02812\\
23.11	-0.02803\\
23.12	-0.02792\\
23.13	-0.02777\\
23.14	-0.0276\\
23.15	-0.0274\\
23.16	-0.02718\\
23.17	-0.02692\\
23.18	-0.02665\\
23.19	-0.02634\\
23.2	-0.02601\\
23.21	-0.02566\\
23.22	-0.02528\\
23.23	-0.02487\\
23.24	-0.02445\\
23.25	-0.024\\
23.26	-0.02352\\
23.27	-0.02302\\
23.28	-0.02251\\
23.29	-0.02196\\
23.3	-0.0214\\
23.31	-0.02082\\
23.32	-0.02022\\
23.33	-0.0196\\
23.34	-0.01896\\
23.35	-0.0183\\
23.36	-0.01763\\
23.37	-0.01694\\
23.38	-0.01623\\
23.39	-0.01551\\
23.4	-0.01477\\
23.41	-0.01402\\
23.42	-0.01325\\
23.43	-0.01248\\
23.44	-0.01169\\
23.45	-0.01089\\
23.46	-0.01008\\
23.47	-0.009266\\
23.48	-0.008441\\
23.49	-0.007608\\
23.5	-0.006768\\
23.51	-0.005923\\
23.52	-0.005072\\
23.53	-0.004218\\
23.54	-0.00336\\
23.55	-0.0025\\
23.56	-0.001638\\
23.57	-0.0007754\\
23.58	8.715e-05\\
23.59	0.0009488\\
23.6	0.001809\\
23.61	0.002666\\
23.62	0.00352\\
23.63	0.004369\\
23.64	0.005213\\
23.65	0.006052\\
23.66	0.006884\\
23.67	0.007708\\
23.68	0.008523\\
23.69	0.00933\\
23.7	0.01013\\
23.71	0.01091\\
23.72	0.01169\\
23.73	0.01245\\
23.74	0.0132\\
23.75	0.01393\\
23.76	0.01466\\
23.77	0.01536\\
23.78	0.01605\\
23.79	0.01673\\
23.8	0.01738\\
23.81	0.01802\\
23.82	0.01864\\
23.83	0.01925\\
23.84	0.01983\\
23.85	0.02039\\
23.86	0.02093\\
23.87	0.02145\\
23.88	0.02195\\
23.89	0.02243\\
23.9	0.02289\\
23.91	0.02332\\
23.92	0.02373\\
23.93	0.02411\\
23.94	0.02447\\
23.95	0.02481\\
23.96	0.02512\\
23.97	0.02541\\
23.98	0.02567\\
23.99	0.02591\\
24	0.02612\\
24.01	0.02631\\
24.02	0.02647\\
24.03	0.0266\\
24.04	0.02671\\
24.05	0.02679\\
24.06	0.02684\\
24.07	0.02687\\
24.08	0.02687\\
24.09	0.02685\\
24.1	0.0268\\
24.11	0.02672\\
24.12	0.02662\\
24.13	0.02649\\
24.14	0.02633\\
24.15	0.02615\\
24.16	0.02595\\
24.17	0.02571\\
24.18	0.02546\\
24.19	0.02518\\
24.2	0.02487\\
24.21	0.02454\\
24.22	0.02419\\
24.23	0.02381\\
24.24	0.02341\\
24.25	0.02299\\
24.26	0.02255\\
24.27	0.02208\\
24.28	0.02159\\
24.29	0.02109\\
24.3	0.02056\\
24.31	0.02001\\
24.32	0.01944\\
24.33	0.01886\\
24.34	0.01826\\
24.35	0.01764\\
24.36	0.017\\
24.37	0.01635\\
24.38	0.01568\\
24.39	0.01499\\
24.4	0.0143\\
24.41	0.01359\\
24.42	0.01286\\
24.43	0.01213\\
24.44	0.01138\\
24.45	0.01063\\
24.46	0.009859\\
24.47	0.009084\\
24.48	0.008301\\
24.49	0.00751\\
24.5	0.006713\\
24.51	0.00591\\
24.52	0.005102\\
24.53	0.004289\\
24.54	0.003474\\
24.55	0.002655\\
24.56	0.001835\\
24.57	0.001014\\
24.58	0.000193\\
24.59	-0.0006277\\
24.6	-0.001447\\
24.61	-0.002264\\
24.62	-0.003078\\
24.63	-0.003888\\
24.64	-0.004694\\
24.65	-0.005494\\
24.66	-0.006288\\
24.67	-0.007075\\
24.68	-0.007854\\
24.69	-0.008625\\
24.7	-0.009387\\
24.71	-0.01014\\
24.72	-0.01088\\
24.73	-0.01161\\
24.74	-0.01233\\
24.75	-0.01303\\
24.76	-0.01372\\
24.77	-0.0144\\
24.78	-0.01506\\
24.79	-0.01571\\
24.8	-0.01634\\
24.81	-0.01696\\
24.82	-0.01755\\
24.83	-0.01813\\
24.84	-0.01869\\
24.85	-0.01923\\
24.86	-0.01976\\
24.87	-0.02026\\
24.88	-0.02074\\
24.89	-0.0212\\
24.9	-0.02165\\
24.91	-0.02206\\
24.92	-0.02246\\
24.93	-0.02284\\
24.94	-0.02319\\
24.95	-0.02352\\
24.96	-0.02382\\
24.97	-0.0241\\
24.98	-0.02436\\
24.99	-0.02459\\
25	-0.0248\\
25.01	-0.02499\\
25.02	-0.02515\\
25.03	-0.02528\\
25.04	-0.02539\\
25.05	-0.02548\\
25.06	-0.02554\\
25.07	-0.02557\\
25.08	-0.02558\\
25.09	-0.02557\\
25.1	-0.02553\\
25.11	-0.02546\\
25.12	-0.02537\\
25.13	-0.02526\\
25.14	-0.02512\\
25.15	-0.02496\\
25.16	-0.02477\\
25.17	-0.02456\\
25.18	-0.02432\\
25.19	-0.02406\\
25.2	-0.02378\\
25.21	-0.02347\\
25.22	-0.02314\\
25.23	-0.02279\\
25.24	-0.02242\\
25.25	-0.02202\\
25.26	-0.02161\\
25.27	-0.02117\\
25.28	-0.02072\\
25.29	-0.02024\\
25.3	-0.01974\\
25.31	-0.01923\\
25.32	-0.01869\\
25.33	-0.01814\\
25.34	-0.01757\\
25.35	-0.01699\\
25.36	-0.01639\\
25.37	-0.01577\\
25.38	-0.01514\\
25.39	-0.01449\\
25.4	-0.01383\\
25.41	-0.01316\\
25.42	-0.01248\\
25.43	-0.01178\\
25.44	-0.01108\\
25.45	-0.01036\\
25.46	-0.009632\\
25.47	-0.008897\\
25.48	-0.008154\\
25.49	-0.007403\\
25.5	-0.006646\\
25.51	-0.005884\\
25.52	-0.005116\\
25.53	-0.004344\\
25.54	-0.003568\\
25.55	-0.00279\\
25.56	-0.00201\\
25.57	-0.001228\\
25.58	-0.0004465\\
25.59	0.000335\\
25.6	0.001115\\
25.61	0.001894\\
25.62	0.00267\\
25.63	0.003443\\
25.64	0.004211\\
25.65	0.004975\\
25.66	0.005733\\
25.67	0.006484\\
25.68	0.007229\\
25.69	0.007965\\
25.7	0.008694\\
25.71	0.009412\\
25.72	0.01012\\
25.73	0.01082\\
25.74	0.01151\\
25.75	0.01218\\
25.76	0.01284\\
25.77	0.01349\\
25.78	0.01413\\
25.79	0.01475\\
25.8	0.01536\\
25.81	0.01595\\
25.82	0.01652\\
25.83	0.01708\\
25.84	0.01762\\
25.85	0.01814\\
25.86	0.01864\\
25.87	0.01913\\
25.88	0.0196\\
25.89	0.02004\\
25.9	0.02047\\
25.91	0.02087\\
25.92	0.02126\\
25.93	0.02162\\
25.94	0.02196\\
25.95	0.02228\\
25.96	0.02258\\
25.97	0.02286\\
25.98	0.02311\\
25.99	0.02334\\
26	0.02355\\
26.01	0.02373\\
26.02	0.02389\\
26.03	0.02403\\
26.04	0.02414\\
26.05	0.02423\\
26.06	0.0243\\
26.07	0.02434\\
26.08	0.02436\\
26.09	0.02435\\
26.1	0.02432\\
26.11	0.02427\\
26.12	0.02419\\
26.13	0.02409\\
26.14	0.02396\\
26.15	0.02381\\
26.16	0.02364\\
26.17	0.02345\\
26.18	0.02323\\
26.19	0.02299\\
26.2	0.02273\\
26.21	0.02245\\
26.22	0.02214\\
26.23	0.02181\\
26.24	0.02146\\
26.25	0.0211\\
26.26	0.02071\\
26.27	0.0203\\
26.28	0.01987\\
26.29	0.01942\\
26.3	0.01896\\
26.31	0.01847\\
26.32	0.01797\\
26.33	0.01745\\
26.34	0.01691\\
26.35	0.01636\\
26.36	0.0158\\
26.37	0.01521\\
26.38	0.01462\\
26.39	0.01401\\
26.4	0.01338\\
26.41	0.01275\\
26.42	0.0121\\
26.43	0.01144\\
26.44	0.01077\\
26.45	0.01009\\
26.46	0.009403\\
26.47	0.008706\\
26.48	0.008001\\
26.49	0.007289\\
26.5	0.00657\\
26.51	0.005846\\
26.52	0.005116\\
26.53	0.004383\\
26.54	0.003645\\
26.55	0.002905\\
26.56	0.002163\\
26.57	0.00142\\
26.58	0.0006752\\
26.59	-6.896e-05\\
26.6	-0.0008124\\
26.61	-0.001554\\
26.62	-0.002294\\
26.63	-0.003031\\
26.64	-0.003764\\
26.65	-0.004492\\
26.66	-0.005216\\
26.67	-0.005933\\
26.68	-0.006644\\
26.69	-0.007348\\
26.7	-0.008044\\
26.71	-0.008732\\
26.72	-0.00941\\
26.73	-0.01008\\
26.74	-0.01074\\
26.75	-0.01138\\
26.76	-0.01202\\
26.77	-0.01264\\
26.78	-0.01325\\
26.79	-0.01385\\
26.8	-0.01443\\
26.81	-0.01499\\
26.82	-0.01555\\
26.83	-0.01608\\
26.84	-0.0166\\
26.85	-0.0171\\
26.86	-0.01759\\
26.87	-0.01806\\
26.88	-0.01851\\
26.89	-0.01894\\
26.9	-0.01935\\
26.91	-0.01974\\
26.92	-0.02012\\
26.93	-0.02047\\
26.94	-0.0208\\
26.95	-0.02111\\
26.96	-0.02141\\
26.97	-0.02168\\
26.98	-0.02192\\
26.99	-0.02215\\
27	-0.02235\\
27.01	-0.02254\\
27.02	-0.0227\\
27.03	-0.02283\\
27.04	-0.02295\\
27.05	-0.02304\\
27.06	-0.02311\\
27.07	-0.02316\\
27.08	-0.02318\\
27.09	-0.02319\\
27.1	-0.02316\\
27.11	-0.02312\\
27.12	-0.02305\\
27.13	-0.02297\\
27.14	-0.02285\\
27.15	-0.02272\\
27.16	-0.02257\\
27.17	-0.02239\\
27.18	-0.02219\\
27.19	-0.02197\\
27.2	-0.02172\\
27.21	-0.02146\\
27.22	-0.02118\\
27.23	-0.02087\\
27.24	-0.02055\\
27.25	-0.0202\\
27.26	-0.01984\\
27.27	-0.01946\\
27.28	-0.01905\\
27.29	-0.01863\\
27.3	-0.0182\\
27.31	-0.01774\\
27.32	-0.01727\\
27.33	-0.01678\\
27.34	-0.01628\\
27.35	-0.01576\\
27.36	-0.01522\\
27.37	-0.01467\\
27.38	-0.01411\\
27.39	-0.01353\\
27.4	-0.01294\\
27.41	-0.01234\\
27.42	-0.01173\\
27.43	-0.0111\\
27.44	-0.01047\\
27.45	-0.009826\\
27.46	-0.009173\\
27.47	-0.008512\\
27.48	-0.007843\\
27.49	-0.007167\\
27.5	-0.006485\\
27.51	-0.005798\\
27.52	-0.005105\\
27.53	-0.004408\\
27.54	-0.003707\\
27.55	-0.003003\\
27.56	-0.002297\\
27.57	-0.001589\\
27.58	-0.000881\\
27.59	-0.0001724\\
27.6	0.0005356\\
27.61	0.001242\\
27.62	0.001947\\
27.63	0.00265\\
27.64	0.003349\\
27.65	0.004044\\
27.66	0.004734\\
27.67	0.005419\\
27.68	0.006099\\
27.69	0.006771\\
27.7	0.007436\\
27.71	0.008093\\
27.72	0.008742\\
27.73	0.009382\\
27.74	0.01001\\
27.75	0.01063\\
27.76	0.01124\\
27.77	0.01184\\
27.78	0.01242\\
27.79	0.01299\\
27.8	0.01355\\
27.81	0.01409\\
27.82	0.01462\\
27.83	0.01514\\
27.84	0.01564\\
27.85	0.01612\\
27.86	0.01659\\
27.87	0.01704\\
27.88	0.01748\\
27.89	0.01789\\
27.9	0.01829\\
27.91	0.01867\\
27.92	0.01903\\
27.93	0.01938\\
27.94	0.0197\\
27.95	0.02\\
27.96	0.02029\\
27.97	0.02055\\
27.98	0.02079\\
27.99	0.02102\\
28	0.02122\\
28.01	0.0214\\
28.02	0.02156\\
28.03	0.0217\\
28.04	0.02181\\
28.05	0.02191\\
28.06	0.02198\\
28.07	0.02203\\
28.08	0.02207\\
28.09	0.02207\\
28.1	0.02206\\
28.11	0.02203\\
28.12	0.02197\\
28.13	0.02189\\
28.14	0.0218\\
28.15	0.02168\\
28.16	0.02153\\
28.17	0.02137\\
28.18	0.02119\\
28.19	0.02099\\
28.2	0.02076\\
28.21	0.02052\\
28.22	0.02025\\
28.23	0.01997\\
28.24	0.01967\\
28.25	0.01935\\
28.26	0.01901\\
28.27	0.01865\\
28.28	0.01827\\
28.29	0.01788\\
28.3	0.01747\\
28.31	0.01704\\
28.32	0.01659\\
28.33	0.01613\\
28.34	0.01566\\
28.35	0.01517\\
28.36	0.01466\\
28.37	0.01414\\
28.38	0.01361\\
28.39	0.01307\\
28.4	0.01251\\
28.41	0.01194\\
28.42	0.01136\\
28.43	0.01077\\
28.44	0.01017\\
28.45	0.009561\\
28.46	0.008943\\
28.47	0.008316\\
28.48	0.007682\\
28.49	0.00704\\
28.5	0.006393\\
28.51	0.00574\\
28.52	0.005082\\
28.53	0.004419\\
28.54	0.003753\\
28.55	0.003084\\
28.56	0.002413\\
28.57	0.00174\\
28.58	0.001065\\
28.59	0.0003909\\
28.6	-0.0002834\\
28.61	-0.0009568\\
28.62	-0.001629\\
28.63	-0.002298\\
28.64	-0.002965\\
28.65	-0.003628\\
28.66	-0.004287\\
28.67	-0.004941\\
28.68	-0.005589\\
28.69	-0.006232\\
28.7	-0.006867\\
28.71	-0.007496\\
28.72	-0.008116\\
28.73	-0.008728\\
28.74	-0.00933\\
28.75	-0.009923\\
28.76	-0.01051\\
28.77	-0.01108\\
28.78	-0.01164\\
28.79	-0.01219\\
28.8	-0.01272\\
28.81	-0.01325\\
28.82	-0.01375\\
28.83	-0.01425\\
28.84	-0.01473\\
28.85	-0.0152\\
28.86	-0.01565\\
28.87	-0.01608\\
28.88	-0.0165\\
28.89	-0.0169\\
28.9	-0.01729\\
28.91	-0.01766\\
28.92	-0.01801\\
28.93	-0.01834\\
28.94	-0.01865\\
28.95	-0.01895\\
28.96	-0.01922\\
28.97	-0.01948\\
28.98	-0.01972\\
28.99	-0.01994\\
29	-0.02014\\
29.01	-0.02032\\
29.02	-0.02047\\
29.03	-0.02061\\
29.04	-0.02073\\
29.05	-0.02083\\
29.06	-0.02091\\
29.07	-0.02096\\
29.08	-0.021\\
29.09	-0.02101\\
29.1	-0.02101\\
29.11	-0.02098\\
29.12	-0.02094\\
29.13	-0.02087\\
29.14	-0.02078\\
29.15	-0.02068\\
29.16	-0.02055\\
29.17	-0.0204\\
29.18	-0.02023\\
29.19	-0.02005\\
29.2	-0.01984\\
29.21	-0.01961\\
29.22	-0.01937\\
29.23	-0.0191\\
29.24	-0.01882\\
29.25	-0.01852\\
29.26	-0.0182\\
29.27	-0.01787\\
29.28	-0.01752\\
29.29	-0.01715\\
29.3	-0.01676\\
29.31	-0.01636\\
29.32	-0.01594\\
29.33	-0.01551\\
29.34	-0.01506\\
29.35	-0.0146\\
29.36	-0.01412\\
29.37	-0.01363\\
29.38	-0.01313\\
29.39	-0.01262\\
29.4	-0.01209\\
29.41	-0.01155\\
29.42	-0.011\\
29.43	-0.01044\\
29.44	-0.009876\\
29.45	-0.009298\\
29.46	-0.008712\\
29.47	-0.008118\\
29.48	-0.007517\\
29.49	-0.006908\\
29.5	-0.006294\\
29.51	-0.005674\\
29.52	-0.005049\\
29.53	-0.00442\\
29.54	-0.003787\\
29.55	-0.003151\\
29.56	-0.002512\\
29.57	-0.001872\\
29.58	-0.00123\\
29.59	-0.000588\\
29.6	5.413e-05\\
29.61	0.0006956\\
29.62	0.001336\\
29.63	0.001974\\
29.64	0.002609\\
29.65	0.003242\\
29.66	0.003871\\
29.67	0.004495\\
29.68	0.005114\\
29.69	0.005728\\
29.7	0.006335\\
29.71	0.006936\\
29.72	0.007529\\
29.73	0.008114\\
29.74	0.008691\\
29.75	0.009258\\
29.76	0.009816\\
29.77	0.01036\\
29.78	0.0109\\
29.79	0.01143\\
29.8	0.01194\\
29.81	0.01244\\
29.82	0.01293\\
29.83	0.01341\\
29.84	0.01387\\
29.85	0.01432\\
29.86	0.01475\\
29.87	0.01517\\
29.88	0.01558\\
29.89	0.01597\\
29.9	0.01634\\
29.91	0.01669\\
29.92	0.01703\\
29.93	0.01735\\
29.94	0.01766\\
29.95	0.01795\\
29.96	0.01822\\
29.97	0.01847\\
29.98	0.0187\\
29.99	0.01891\\
};
\addlegendentry{EI}

\addplot[only marks, mark=*, mark options={}, mark size=0.5000pt, draw=mycolor1] table[row sep=crcr]{%
x	y\\
0	0.0875\\
0.01	0.08746\\
0.02	0.08733\\
0.03	0.08711\\
0.04	0.08681\\
0.05	0.08643\\
0.06	0.08596\\
0.07	0.0854\\
0.08	0.08477\\
0.09	0.08405\\
0.1	0.08324\\
0.11	0.08236\\
0.12	0.08139\\
0.13	0.08034\\
0.14	0.07922\\
0.15	0.07802\\
0.16	0.07674\\
0.17	0.07538\\
0.18	0.07395\\
0.19	0.07245\\
0.2	0.07088\\
0.21	0.06924\\
0.22	0.06753\\
0.23	0.06575\\
0.24	0.06391\\
0.25	0.06201\\
0.26	0.06005\\
0.27	0.05802\\
0.28	0.05594\\
0.29	0.05381\\
0.3	0.05162\\
0.31	0.04938\\
0.32	0.0471\\
0.33	0.04476\\
0.34	0.04239\\
0.35	0.03997\\
0.36	0.03751\\
0.37	0.03502\\
0.38	0.03249\\
0.39	0.02993\\
0.4	0.02734\\
0.41	0.02472\\
0.42	0.02208\\
0.43	0.01942\\
0.44	0.01674\\
0.45	0.01404\\
0.46	0.01133\\
0.47	0.008602\\
0.48	0.005871\\
0.49	0.003134\\
0.5	0.0003933\\
0.51	-0.002348\\
0.52	-0.005086\\
0.53	-0.007819\\
0.54	-0.01055\\
0.55	-0.01326\\
0.56	-0.01596\\
0.57	-0.01865\\
0.58	-0.02132\\
0.59	-0.02397\\
0.6	-0.02659\\
0.61	-0.02919\\
0.62	-0.03176\\
0.63	-0.0343\\
0.64	-0.0368\\
0.65	-0.03927\\
0.66	-0.0417\\
0.67	-0.04409\\
0.68	-0.04643\\
0.69	-0.04873\\
0.7	-0.05099\\
0.71	-0.05319\\
0.72	-0.05534\\
0.73	-0.05743\\
0.74	-0.05947\\
0.75	-0.06145\\
0.76	-0.06337\\
0.77	-0.06523\\
0.78	-0.06703\\
0.79	-0.06876\\
0.8	-0.07042\\
0.81	-0.07201\\
0.82	-0.07353\\
0.83	-0.07498\\
0.84	-0.07636\\
0.85	-0.07766\\
0.86	-0.07888\\
0.87	-0.08003\\
0.88	-0.0811\\
0.89	-0.08209\\
0.9	-0.083\\
0.91	-0.08382\\
0.92	-0.08457\\
0.93	-0.08523\\
0.94	-0.08581\\
0.95	-0.0863\\
0.96	-0.08671\\
0.97	-0.08704\\
0.98	-0.08728\\
0.99	-0.08743\\
1	-0.0875\\
1.01	-0.08748\\
1.02	-0.08738\\
1.03	-0.08719\\
1.04	-0.08691\\
1.05	-0.08655\\
1.06	-0.0861\\
1.07	-0.08557\\
1.08	-0.08496\\
1.09	-0.08426\\
1.1	-0.08348\\
1.11	-0.08262\\
1.12	-0.08168\\
1.13	-0.08065\\
1.14	-0.07955\\
1.15	-0.07837\\
1.16	-0.07711\\
1.17	-0.07578\\
1.18	-0.07437\\
1.19	-0.07289\\
1.2	-0.07134\\
1.21	-0.06972\\
1.22	-0.06803\\
1.23	-0.06627\\
1.24	-0.06445\\
1.25	-0.06256\\
1.26	-0.06062\\
1.27	-0.05861\\
1.28	-0.05655\\
1.29	-0.05443\\
1.3	-0.05226\\
1.31	-0.05003\\
1.32	-0.04776\\
1.33	-0.04544\\
1.34	-0.04307\\
1.35	-0.04067\\
1.36	-0.03822\\
1.37	-0.03574\\
1.38	-0.03322\\
1.39	-0.03067\\
1.4	-0.02808\\
1.41	-0.02548\\
1.42	-0.02284\\
1.43	-0.02018\\
1.44	-0.01751\\
1.45	-0.01481\\
1.46	-0.01211\\
1.47	-0.009385\\
1.48	-0.006656\\
1.49	-0.00392\\
1.5	-0.00118\\
1.51	0.001561\\
1.52	0.004301\\
1.53	0.007036\\
1.54	0.009764\\
1.55	0.01248\\
1.56	0.01519\\
1.57	0.01788\\
1.58	0.02055\\
1.59	0.02321\\
1.6	0.02584\\
1.61	0.02845\\
1.62	0.03102\\
1.63	0.03357\\
1.64	0.03609\\
1.65	0.03856\\
1.66	0.04101\\
1.67	0.04341\\
1.68	0.04576\\
1.69	0.04808\\
1.7	0.05034\\
1.71	0.05256\\
1.72	0.05473\\
1.73	0.05684\\
1.74	0.05889\\
1.75	0.06089\\
1.76	0.06283\\
1.77	0.06471\\
1.78	0.06652\\
1.79	0.06827\\
1.8	0.06995\\
1.81	0.07156\\
1.82	0.0731\\
1.83	0.07457\\
1.84	0.07597\\
1.85	0.07729\\
1.86	0.07854\\
1.87	0.07971\\
1.88	0.0808\\
1.89	0.08181\\
1.9	0.08275\\
1.91	0.0836\\
1.92	0.08436\\
1.93	0.08505\\
1.94	0.08565\\
1.95	0.08617\\
1.96	0.08661\\
1.97	0.08695\\
1.98	0.08722\\
1.99	0.0874\\
2	0.08749\\
2.01	0.08749\\
2.02	0.08741\\
2.03	0.08725\\
2.04	0.087\\
2.05	0.08666\\
2.06	0.08624\\
2.07	0.08574\\
2.08	0.08515\\
2.09	0.08447\\
2.1	0.08372\\
2.11	0.08288\\
2.12	0.08196\\
2.13	0.08096\\
2.14	0.07988\\
2.15	0.07872\\
2.16	0.07748\\
2.17	0.07617\\
2.18	0.07479\\
2.19	0.07333\\
2.2	0.07179\\
2.21	0.07019\\
2.22	0.06852\\
2.23	0.06678\\
2.24	0.06498\\
2.25	0.06311\\
2.26	0.06118\\
2.27	0.05919\\
2.28	0.05715\\
2.29	0.05504\\
2.3	0.05289\\
2.31	0.05068\\
2.32	0.04842\\
2.33	0.04611\\
2.34	0.04376\\
2.35	0.04136\\
2.36	0.03893\\
2.37	0.03645\\
2.38	0.03394\\
2.39	0.0314\\
2.4	0.02883\\
2.41	0.02623\\
2.42	0.0236\\
2.43	0.02095\\
2.44	0.01828\\
2.45	0.01559\\
2.46	0.01288\\
2.47	0.01017\\
2.48	0.00744\\
2.49	0.004705\\
2.5	0.001966\\
2.51	-0.0007748\\
2.52	-0.003515\\
2.53	-0.006252\\
2.54	-0.008982\\
2.55	-0.0117\\
2.56	-0.01441\\
2.57	-0.01711\\
2.58	-0.01979\\
2.59	-0.02245\\
2.6	-0.02509\\
2.61	-0.0277\\
2.62	-0.03029\\
2.63	-0.03284\\
2.64	-0.03537\\
2.65	-0.03786\\
2.66	-0.04031\\
2.67	-0.04272\\
2.68	-0.04509\\
2.69	-0.04742\\
2.7	-0.0497\\
2.71	-0.05193\\
2.72	-0.05411\\
2.73	-0.05624\\
2.74	-0.05831\\
2.75	-0.06033\\
2.76	-0.06228\\
2.77	-0.06418\\
2.78	-0.06601\\
2.79	-0.06777\\
2.8	-0.06947\\
2.81	-0.07111\\
2.82	-0.07267\\
2.83	-0.07416\\
2.84	-0.07558\\
2.85	-0.07692\\
2.86	-0.07819\\
2.87	-0.07938\\
2.88	-0.0805\\
2.89	-0.08153\\
2.9	-0.08249\\
2.91	-0.08336\\
2.92	-0.08415\\
2.93	-0.08486\\
2.94	-0.08549\\
2.95	-0.08603\\
2.96	-0.08649\\
2.97	-0.08686\\
2.98	-0.08715\\
2.99	-0.08735\\
3	-0.08747\\
3.01	-0.0875\\
3.02	-0.08745\\
3.03	-0.08731\\
3.04	-0.08708\\
3.05	-0.08677\\
3.06	-0.08637\\
3.07	-0.08589\\
3.08	-0.08532\\
3.09	-0.08467\\
3.1	-0.08394\\
3.11	-0.08313\\
3.12	-0.08223\\
3.13	-0.08125\\
3.14	-0.0802\\
3.15	-0.07906\\
3.16	-0.07785\\
3.17	-0.07656\\
3.18	-0.07519\\
3.19	-0.07375\\
3.2	-0.07224\\
3.21	-0.07066\\
3.22	-0.06901\\
3.23	-0.06729\\
3.24	-0.0655\\
3.25	-0.06366\\
3.26	-0.06174\\
3.27	-0.05977\\
3.28	-0.05774\\
3.29	-0.05565\\
3.3	-0.05351\\
3.31	-0.05132\\
3.32	-0.04907\\
3.33	-0.04678\\
3.34	-0.04444\\
3.35	-0.04206\\
3.36	-0.03963\\
3.37	-0.03717\\
3.38	-0.03467\\
3.39	-0.03214\\
3.4	-0.02957\\
3.41	-0.02698\\
3.42	-0.02436\\
3.43	-0.02171\\
3.44	-0.01905\\
3.45	-0.01636\\
3.46	-0.01366\\
3.47	-0.01095\\
3.48	-0.008223\\
3.49	-0.00549\\
3.5	-0.002752\\
3.51	-1.178e-05\\
3.52	0.002729\\
3.53	0.005467\\
3.54	0.0082\\
3.55	0.01092\\
3.56	0.01364\\
3.57	0.01634\\
3.58	0.01902\\
3.59	0.02169\\
3.6	0.02433\\
3.61	0.02695\\
3.62	0.02955\\
3.63	0.03211\\
3.64	0.03465\\
3.65	0.03715\\
3.66	0.03961\\
3.67	0.04203\\
3.68	0.04442\\
3.69	0.04676\\
3.7	0.04905\\
3.71	0.0513\\
3.72	0.05349\\
3.73	0.05563\\
3.74	0.05772\\
3.75	0.05975\\
3.76	0.06173\\
3.77	0.06364\\
3.78	0.06549\\
3.79	0.06727\\
3.8	0.06899\\
3.81	0.07065\\
3.82	0.07223\\
3.83	0.07374\\
3.84	0.07518\\
3.85	0.07655\\
3.86	0.07784\\
3.87	0.07905\\
3.88	0.08019\\
3.89	0.08124\\
3.9	0.08222\\
3.91	0.08312\\
3.92	0.08394\\
3.93	0.08467\\
3.94	0.08532\\
3.95	0.08589\\
3.96	0.08637\\
3.97	0.08677\\
3.98	0.08708\\
3.99	0.08731\\
4	0.08745\\
4.01	0.0875\\
4.02	0.08747\\
4.03	0.08736\\
4.04	0.08716\\
4.05	0.08687\\
4.06	0.0865\\
4.07	0.08604\\
4.08	0.0855\\
4.09	0.08487\\
4.1	0.08416\\
4.11	0.08337\\
4.12	0.0825\\
4.13	0.08154\\
4.14	0.08051\\
4.15	0.07939\\
4.16	0.0782\\
4.17	0.07694\\
4.18	0.07559\\
4.19	0.07417\\
4.2	0.07268\\
4.21	0.07112\\
4.22	0.06949\\
4.23	0.06779\\
4.24	0.06602\\
4.25	0.06419\\
4.26	0.0623\\
4.27	0.06034\\
4.28	0.05833\\
4.29	0.05626\\
4.3	0.05413\\
4.31	0.05195\\
4.32	0.04972\\
4.33	0.04744\\
4.34	0.04511\\
4.35	0.04274\\
4.36	0.04033\\
4.37	0.03788\\
4.38	0.03539\\
4.39	0.03287\\
4.4	0.03031\\
4.41	0.02772\\
4.42	0.02511\\
4.43	0.02247\\
4.44	0.01981\\
4.45	0.01713\\
4.46	0.01444\\
4.47	0.01173\\
4.48	0.009006\\
4.49	0.006275\\
4.5	0.003539\\
4.51	0.0007983\\
4.52	-0.001943\\
4.53	-0.004682\\
4.54	-0.007416\\
4.55	-0.01014\\
4.56	-0.01286\\
4.57	-0.01557\\
4.58	-0.01825\\
4.59	-0.02093\\
4.6	-0.02358\\
4.61	-0.02621\\
4.62	-0.02881\\
4.63	-0.03138\\
4.64	-0.03392\\
4.65	-0.03643\\
4.66	-0.03891\\
4.67	-0.04134\\
4.68	-0.04374\\
4.69	-0.04609\\
4.7	-0.0484\\
4.71	-0.05066\\
4.72	-0.05287\\
4.73	-0.05503\\
4.74	-0.05713\\
4.75	-0.05918\\
4.76	-0.06117\\
4.77	-0.0631\\
4.78	-0.06497\\
4.79	-0.06677\\
4.8	-0.06851\\
4.81	-0.07018\\
4.82	-0.07178\\
4.83	-0.07332\\
4.84	-0.07478\\
4.85	-0.07616\\
4.86	-0.07747\\
4.87	-0.07871\\
4.88	-0.07987\\
4.89	-0.08095\\
4.9	-0.08195\\
4.91	-0.08287\\
4.92	-0.08371\\
4.93	-0.08447\\
4.94	-0.08514\\
4.95	-0.08573\\
4.96	-0.08624\\
4.97	-0.08666\\
4.98	-0.087\\
4.99	-0.08725\\
5	-0.08742\\
5.01	-0.0875\\
5.02	-0.08749\\
5.03	-0.0874\\
5.04	-0.08722\\
5.05	-0.08696\\
5.06	-0.08661\\
5.07	-0.08618\\
5.08	-0.08566\\
5.09	-0.08506\\
5.1	-0.08437\\
5.11	-0.08361\\
5.12	-0.08276\\
5.13	-0.08183\\
5.14	-0.08081\\
5.15	-0.07972\\
5.16	-0.07855\\
5.17	-0.07731\\
5.18	-0.07599\\
5.19	-0.07459\\
5.2	-0.07312\\
5.21	-0.07158\\
5.22	-0.06997\\
5.23	-0.06829\\
5.24	-0.06654\\
5.25	-0.06473\\
5.26	-0.06285\\
5.27	-0.06091\\
5.28	-0.05891\\
5.29	-0.05686\\
5.3	-0.05475\\
5.31	-0.05258\\
5.32	-0.05037\\
5.33	-0.0481\\
5.34	-0.04579\\
5.35	-0.04343\\
5.36	-0.04103\\
5.37	-0.03859\\
5.38	-0.03611\\
5.39	-0.03359\\
5.4	-0.03105\\
5.41	-0.02847\\
5.42	-0.02586\\
5.43	-0.02323\\
5.44	-0.02058\\
5.45	-0.0179\\
5.46	-0.01521\\
5.47	-0.01251\\
5.48	-0.009788\\
5.49	-0.00706\\
5.5	-0.004324\\
5.51	-0.001585\\
5.52	0.001156\\
5.53	0.003896\\
5.54	0.006632\\
5.55	0.009362\\
5.56	0.01208\\
5.57	0.01479\\
5.58	0.01749\\
5.59	0.02016\\
5.6	0.02282\\
5.61	0.02545\\
5.62	0.02806\\
5.63	0.03065\\
5.64	0.0332\\
5.65	0.03572\\
5.66	0.0382\\
5.67	0.04065\\
5.68	0.04306\\
5.69	0.04542\\
5.7	0.04774\\
5.71	0.05002\\
5.72	0.05224\\
5.73	0.05441\\
5.74	0.05653\\
5.75	0.0586\\
5.76	0.0606\\
5.77	0.06255\\
5.78	0.06444\\
5.79	0.06626\\
5.8	0.06802\\
5.81	0.06971\\
5.82	0.07133\\
5.83	0.07288\\
5.84	0.07436\\
5.85	0.07577\\
5.86	0.07711\\
5.87	0.07836\\
5.88	0.07955\\
5.89	0.08065\\
5.9	0.08167\\
5.91	0.08262\\
5.92	0.08348\\
5.93	0.08426\\
5.94	0.08496\\
5.95	0.08557\\
5.96	0.0861\\
5.97	0.08655\\
5.98	0.08691\\
5.99	0.08719\\
6	0.08738\\
6.01	0.08748\\
6.02	0.0875\\
6.03	0.08744\\
6.04	0.08728\\
6.05	0.08705\\
6.06	0.08672\\
6.07	0.08631\\
6.08	0.08582\\
6.09	0.08524\\
6.1	0.08458\\
6.11	0.08384\\
6.12	0.08301\\
6.13	0.0821\\
6.14	0.08111\\
6.15	0.08004\\
6.16	0.0789\\
6.17	0.07767\\
6.18	0.07637\\
6.19	0.075\\
6.2	0.07355\\
6.21	0.07203\\
6.22	0.07044\\
6.23	0.06878\\
6.24	0.06705\\
6.25	0.06525\\
6.26	0.0634\\
6.27	0.06147\\
6.28	0.05949\\
6.29	0.05745\\
6.3	0.05536\\
6.31	0.05321\\
6.32	0.05101\\
6.33	0.04876\\
6.34	0.04646\\
6.35	0.04411\\
6.36	0.04172\\
6.37	0.03929\\
6.38	0.03682\\
6.39	0.03432\\
6.4	0.03178\\
6.41	0.02921\\
6.42	0.02661\\
6.43	0.02399\\
6.44	0.02134\\
6.45	0.01867\\
6.46	0.01599\\
6.47	0.01328\\
6.48	0.01057\\
6.49	0.007843\\
6.5	0.00511\\
6.51	0.002371\\
6.52	-0.0003697\\
6.53	-0.00311\\
6.54	-0.005848\\
6.55	-0.00858\\
6.56	-0.0113\\
6.57	-0.01402\\
6.58	-0.01671\\
6.59	-0.0194\\
6.6	-0.02206\\
6.61	-0.0247\\
6.62	-0.02732\\
6.63	-0.02991\\
6.64	-0.03247\\
6.65	-0.035\\
6.66	-0.03749\\
6.67	-0.03995\\
6.68	-0.04237\\
6.69	-0.04475\\
6.7	-0.04708\\
6.71	-0.04937\\
6.72	-0.05161\\
6.73	-0.0538\\
6.74	-0.05593\\
6.75	-0.05801\\
6.76	-0.06003\\
6.77	-0.062\\
6.78	-0.0639\\
6.79	-0.06574\\
6.8	-0.06752\\
6.81	-0.06923\\
6.82	-0.07087\\
6.83	-0.07245\\
6.84	-0.07395\\
6.85	-0.07538\\
6.86	-0.07673\\
6.87	-0.07801\\
6.88	-0.07922\\
6.89	-0.08034\\
6.9	-0.08139\\
6.91	-0.08236\\
6.92	-0.08324\\
6.93	-0.08405\\
6.94	-0.08477\\
6.95	-0.08541\\
6.96	-0.08596\\
6.97	-0.08643\\
6.98	-0.08682\\
6.99	-0.08712\\
7	-0.08733\\
7.01	-0.08746\\
7.02	-0.08751\\
7.03	-0.08747\\
7.04	-0.08734\\
7.05	-0.08712\\
7.06	-0.08682\\
7.07	-0.08644\\
7.08	-0.08597\\
7.09	-0.08542\\
7.1	-0.08478\\
7.11	-0.08406\\
7.12	-0.08326\\
7.13	-0.08237\\
7.14	-0.08141\\
7.15	-0.08036\\
7.16	-0.07924\\
7.17	-0.07803\\
7.18	-0.07676\\
7.19	-0.0754\\
7.2	-0.07397\\
7.21	-0.07247\\
7.22	-0.0709\\
7.23	-0.06926\\
7.24	-0.06755\\
7.25	-0.06578\\
7.26	-0.06394\\
7.27	-0.06203\\
7.28	-0.06007\\
7.29	-0.05805\\
7.3	-0.05597\\
7.31	-0.05383\\
7.32	-0.05165\\
7.33	-0.04941\\
7.34	-0.04712\\
7.35	-0.04479\\
7.36	-0.04241\\
7.37	-0.03999\\
7.38	-0.03754\\
7.39	-0.03504\\
7.4	-0.03251\\
7.41	-0.02995\\
7.42	-0.02736\\
7.43	-0.02475\\
7.44	-0.0221\\
7.45	-0.01944\\
7.46	-0.01676\\
7.47	-0.01406\\
7.48	-0.01135\\
7.49	-0.008627\\
7.5	-0.005895\\
7.51	-0.003157\\
7.52	-0.0004168\\
7.53	0.002324\\
7.54	0.005063\\
7.55	0.007797\\
7.56	0.01052\\
7.57	0.01324\\
7.58	0.01594\\
7.59	0.01863\\
7.6	0.0213\\
7.61	0.02395\\
7.62	0.02657\\
7.63	0.02917\\
7.64	0.03174\\
7.65	0.03428\\
7.66	0.03678\\
7.67	0.03925\\
7.68	0.04168\\
7.69	0.04407\\
7.7	0.04642\\
7.71	0.04872\\
7.72	0.05097\\
7.73	0.05317\\
7.74	0.05532\\
7.75	0.05742\\
7.76	0.05946\\
7.77	0.06144\\
7.78	0.06336\\
7.79	0.06522\\
7.8	0.06702\\
7.81	0.06875\\
7.82	0.07041\\
7.83	0.072\\
7.84	0.07353\\
7.85	0.07498\\
7.86	0.07635\\
7.87	0.07765\\
7.88	0.07888\\
7.89	0.08003\\
7.9	0.0811\\
7.91	0.08209\\
7.92	0.083\\
7.93	0.08382\\
7.94	0.08457\\
7.95	0.08523\\
7.96	0.08581\\
7.97	0.08631\\
7.98	0.08672\\
7.99	0.08704\\
8	0.08728\\
8.01	0.08744\\
8.02	0.0875\\
8.03	0.08749\\
8.04	0.08738\\
8.05	0.08719\\
8.06	0.08692\\
8.07	0.08656\\
8.08	0.08612\\
8.09	0.08559\\
8.1	0.08497\\
8.11	0.08428\\
8.12	0.0835\\
8.13	0.08263\\
8.14	0.08169\\
8.15	0.08067\\
8.16	0.07957\\
8.17	0.07839\\
8.18	0.07713\\
8.19	0.0758\\
8.2	0.07439\\
8.21	0.07291\\
8.22	0.07136\\
8.23	0.06974\\
8.24	0.06805\\
8.25	0.06629\\
8.26	0.06447\\
8.27	0.06259\\
8.28	0.06064\\
8.29	0.05863\\
8.3	0.05657\\
8.31	0.05445\\
8.32	0.05228\\
8.33	0.05006\\
8.34	0.04778\\
8.35	0.04546\\
8.36	0.0431\\
8.37	0.04069\\
8.38	0.03825\\
8.39	0.03576\\
8.4	0.03324\\
8.41	0.03069\\
8.42	0.02811\\
8.43	0.0255\\
8.44	0.02287\\
8.45	0.02021\\
8.46	0.01753\\
8.47	0.01484\\
8.48	0.01213\\
8.49	0.009409\\
8.5	0.00668\\
8.51	0.003943\\
8.52	0.001203\\
8.53	-0.001538\\
8.54	-0.004277\\
8.55	-0.007013\\
8.56	-0.009742\\
8.57	-0.01246\\
8.58	-0.01517\\
8.59	-0.01786\\
8.6	-0.02053\\
8.61	-0.02319\\
8.62	-0.02582\\
8.63	-0.02843\\
8.64	-0.031\\
8.65	-0.03355\\
8.66	-0.03607\\
8.67	-0.03855\\
8.68	-0.04099\\
8.69	-0.04339\\
8.7	-0.04575\\
8.71	-0.04806\\
8.72	-0.05033\\
8.73	-0.05255\\
8.74	-0.05471\\
8.75	-0.05683\\
8.76	-0.05888\\
8.77	-0.06088\\
8.78	-0.06282\\
8.79	-0.0647\\
8.8	-0.06651\\
8.81	-0.06826\\
8.82	-0.06994\\
8.83	-0.07155\\
8.84	-0.0731\\
8.85	-0.07457\\
8.86	-0.07597\\
8.87	-0.07729\\
8.88	-0.07854\\
8.89	-0.07971\\
8.9	-0.0808\\
8.91	-0.08181\\
8.92	-0.08274\\
8.93	-0.0836\\
8.94	-0.08437\\
8.95	-0.08505\\
8.96	-0.08566\\
8.97	-0.08617\\
8.98	-0.08661\\
8.99	-0.08696\\
9	-0.08722\\
9.01	-0.0874\\
9.02	-0.08749\\
9.03	-0.0875\\
9.04	-0.08742\\
9.05	-0.08726\\
9.06	-0.08701\\
9.07	-0.08667\\
9.08	-0.08625\\
9.09	-0.08575\\
9.1	-0.08516\\
9.11	-0.08448\\
9.12	-0.08373\\
9.13	-0.08289\\
9.14	-0.08197\\
9.15	-0.08097\\
9.16	-0.07989\\
9.17	-0.07874\\
9.18	-0.0775\\
9.19	-0.07619\\
9.2	-0.0748\\
9.21	-0.07334\\
9.22	-0.07181\\
9.23	-0.07021\\
9.24	-0.06854\\
9.25	-0.0668\\
9.26	-0.065\\
9.27	-0.06313\\
9.28	-0.0612\\
9.29	-0.05922\\
9.3	-0.05717\\
9.31	-0.05507\\
9.32	-0.05291\\
9.33	-0.0507\\
9.34	-0.04844\\
9.35	-0.04613\\
9.36	-0.04378\\
9.37	-0.04139\\
9.38	-0.03895\\
9.39	-0.03648\\
9.4	-0.03397\\
9.41	-0.03143\\
9.42	-0.02885\\
9.43	-0.02625\\
9.44	-0.02362\\
9.45	-0.02097\\
9.46	-0.0183\\
9.47	-0.01561\\
9.48	-0.01291\\
9.49	-0.01019\\
9.5	-0.007464\\
9.51	-0.004729\\
9.52	-0.00199\\
9.53	0.0007513\\
9.54	0.003492\\
9.55	0.006229\\
9.56	0.00896\\
9.57	0.01168\\
9.58	0.01439\\
9.59	0.01709\\
9.6	0.01977\\
9.61	0.02243\\
9.62	0.02507\\
9.63	0.02768\\
9.64	0.03027\\
9.65	0.03282\\
9.66	0.03535\\
9.67	0.03784\\
9.68	0.04029\\
9.69	0.04271\\
9.7	0.04508\\
9.71	0.0474\\
9.72	0.04968\\
9.73	0.05192\\
9.74	0.0541\\
9.75	0.05623\\
9.76	0.0583\\
9.77	0.06031\\
9.78	0.06227\\
9.79	0.06417\\
9.8	0.066\\
9.81	0.06777\\
9.82	0.06947\\
9.83	0.0711\\
9.84	0.07266\\
9.85	0.07415\\
9.86	0.07557\\
9.87	0.07692\\
9.88	0.07819\\
9.89	0.07938\\
9.9	0.08049\\
9.91	0.08153\\
9.92	0.08249\\
9.93	0.08336\\
9.94	0.08415\\
9.95	0.08486\\
9.96	0.08549\\
9.97	0.08604\\
9.98	0.08649\\
9.99	0.08687\\
10	0.08716\\
10.01	0.08736\\
10.02	0.08748\\
10.03	0.08751\\
10.04	0.08746\\
10.05	0.08732\\
10.06	0.08709\\
10.07	0.08678\\
10.08	0.08638\\
10.09	0.0859\\
10.1	0.08534\\
10.11	0.08469\\
10.12	0.08396\\
10.13	0.08314\\
10.14	0.08224\\
10.15	0.08127\\
10.16	0.08021\\
10.17	0.07908\\
10.18	0.07786\\
10.19	0.07657\\
10.2	0.07521\\
10.21	0.07377\\
10.22	0.07226\\
10.23	0.07068\\
10.24	0.06903\\
10.25	0.06731\\
10.26	0.06553\\
10.27	0.06368\\
10.28	0.06177\\
10.29	0.05979\\
10.3	0.05776\\
10.31	0.05568\\
10.32	0.05353\\
10.33	0.05134\\
10.34	0.04909\\
10.35	0.0468\\
10.36	0.04446\\
10.37	0.04208\\
10.38	0.03966\\
10.39	0.03719\\
10.4	0.03469\\
10.41	0.03216\\
10.42	0.0296\\
10.43	0.027\\
10.44	0.02438\\
10.45	0.02174\\
10.46	0.01907\\
10.47	0.01639\\
10.48	0.01369\\
10.49	0.01097\\
10.5	0.008247\\
10.51	0.005514\\
10.52	0.002776\\
10.53	3.534e-05\\
10.54	-0.002706\\
10.55	-0.005444\\
10.56	-0.008177\\
10.57	-0.0109\\
10.58	-0.01362\\
10.59	-0.01632\\
10.6	-0.019\\
10.61	-0.02167\\
10.62	-0.02431\\
10.63	-0.02693\\
10.64	-0.02953\\
10.65	-0.03209\\
10.66	-0.03463\\
10.67	-0.03713\\
10.68	-0.03959\\
10.69	-0.04202\\
10.7	-0.0444\\
10.71	-0.04674\\
10.72	-0.04904\\
10.73	-0.05128\\
10.74	-0.05348\\
10.75	-0.05562\\
10.76	-0.05771\\
10.77	-0.05974\\
10.78	-0.06172\\
10.79	-0.06363\\
10.8	-0.06548\\
10.81	-0.06727\\
10.82	-0.06899\\
10.83	-0.07064\\
10.84	-0.07222\\
10.85	-0.07373\\
10.86	-0.07517\\
10.87	-0.07654\\
10.88	-0.07783\\
10.89	-0.07905\\
10.9	-0.08018\\
10.91	-0.08124\\
10.92	-0.08222\\
10.93	-0.08312\\
10.94	-0.08394\\
10.95	-0.08467\\
10.96	-0.08532\\
10.97	-0.08589\\
10.98	-0.08637\\
10.99	-0.08677\\
11	-0.08708\\
11.01	-0.08731\\
11.02	-0.08745\\
11.03	-0.08751\\
11.04	-0.08748\\
11.05	-0.08737\\
11.06	-0.08716\\
11.07	-0.08688\\
11.08	-0.08651\\
11.09	-0.08605\\
11.1	-0.08551\\
11.11	-0.08488\\
11.12	-0.08417\\
11.13	-0.08338\\
11.14	-0.08251\\
11.15	-0.08156\\
11.16	-0.08052\\
11.17	-0.07941\\
11.18	-0.07822\\
11.19	-0.07695\\
11.2	-0.07561\\
11.21	-0.07419\\
11.22	-0.0727\\
11.23	-0.07114\\
11.24	-0.06951\\
11.25	-0.06781\\
11.26	-0.06605\\
11.27	-0.06421\\
11.28	-0.06232\\
11.29	-0.06037\\
11.3	-0.05835\\
11.31	-0.05628\\
11.32	-0.05415\\
11.33	-0.05197\\
11.34	-0.04974\\
11.35	-0.04746\\
11.36	-0.04514\\
11.37	-0.04277\\
11.38	-0.04036\\
11.39	-0.0379\\
11.4	-0.03541\\
11.41	-0.03289\\
11.42	-0.03033\\
11.43	-0.02775\\
11.44	-0.02514\\
11.45	-0.0225\\
11.46	-0.01984\\
11.47	-0.01716\\
11.48	-0.01446\\
11.49	-0.01175\\
11.5	-0.00903\\
11.51	-0.006299\\
11.52	-0.003562\\
11.53	-0.000822\\
11.54	0.001919\\
11.55	0.004659\\
11.56	0.007393\\
11.57	0.01012\\
11.58	0.01284\\
11.59	0.01554\\
11.6	0.01823\\
11.61	0.02091\\
11.62	0.02356\\
11.63	0.02618\\
11.64	0.02879\\
11.65	0.03136\\
11.66	0.03391\\
11.67	0.03642\\
11.68	0.03889\\
11.69	0.04133\\
11.7	0.04372\\
11.71	0.04608\\
11.72	0.04838\\
11.73	0.05064\\
11.74	0.05285\\
11.75	0.05501\\
11.76	0.05712\\
11.77	0.05917\\
11.78	0.06116\\
11.79	0.06309\\
11.8	0.06496\\
11.81	0.06676\\
11.82	0.0685\\
11.83	0.07017\\
11.84	0.07178\\
11.85	0.07331\\
11.86	0.07477\\
11.87	0.07616\\
11.88	0.07747\\
11.89	0.07871\\
11.9	0.07987\\
11.91	0.08095\\
11.92	0.08195\\
11.93	0.08287\\
11.94	0.08371\\
11.95	0.08447\\
11.96	0.08514\\
11.97	0.08574\\
11.98	0.08624\\
11.99	0.08667\\
12	0.087\\
12.01	0.08726\\
12.02	0.08742\\
12.03	0.0875\\
12.04	0.0875\\
12.05	0.08741\\
12.06	0.08723\\
12.07	0.08697\\
12.08	0.08662\\
12.09	0.08619\\
12.1	0.08567\\
12.11	0.08507\\
12.12	0.08439\\
12.13	0.08362\\
12.14	0.08277\\
12.15	0.08184\\
12.16	0.08083\\
12.17	0.07974\\
12.18	0.07857\\
12.19	0.07733\\
12.2	0.076\\
12.21	0.07461\\
12.22	0.07314\\
12.23	0.0716\\
12.24	0.06999\\
12.25	0.06831\\
12.26	0.06656\\
12.27	0.06475\\
12.28	0.06287\\
12.29	0.06093\\
12.3	0.05894\\
12.31	0.05688\\
12.32	0.05477\\
12.33	0.05261\\
12.34	0.05039\\
12.35	0.04812\\
12.36	0.04581\\
12.37	0.04345\\
12.38	0.04105\\
12.39	0.03861\\
12.4	0.03613\\
12.41	0.03362\\
12.42	0.03107\\
12.43	0.02849\\
12.44	0.02589\\
12.45	0.02326\\
12.46	0.0206\\
12.47	0.01793\\
12.48	0.01524\\
12.49	0.01253\\
12.5	0.009812\\
12.51	0.007084\\
12.52	0.004348\\
12.53	0.001609\\
12.54	-0.001133\\
12.55	-0.003873\\
12.56	-0.006609\\
12.57	-0.009339\\
12.58	-0.01206\\
12.59	-0.01477\\
12.6	-0.01746\\
12.61	-0.02014\\
12.62	-0.0228\\
12.63	-0.02543\\
12.64	-0.02804\\
12.65	-0.03063\\
12.66	-0.03318\\
12.67	-0.0357\\
12.68	-0.03818\\
12.69	-0.04063\\
12.7	-0.04304\\
12.71	-0.0454\\
12.72	-0.04773\\
12.73	-0.05\\
12.74	-0.05223\\
12.75	-0.0544\\
12.76	-0.05652\\
12.77	-0.05858\\
12.78	-0.06059\\
12.79	-0.06254\\
12.8	-0.06443\\
12.81	-0.06625\\
12.82	-0.06801\\
12.83	-0.0697\\
12.84	-0.07132\\
12.85	-0.07288\\
12.86	-0.07436\\
12.87	-0.07577\\
12.88	-0.0771\\
12.89	-0.07836\\
12.9	-0.07954\\
12.91	-0.08065\\
12.92	-0.08167\\
12.93	-0.08262\\
12.94	-0.08348\\
12.95	-0.08426\\
12.96	-0.08496\\
12.97	-0.08558\\
12.98	-0.08611\\
12.99	-0.08655\\
13	-0.08692\\
13.01	-0.08719\\
13.02	-0.08739\\
13.03	-0.08749\\
13.04	-0.08751\\
13.05	-0.08744\\
13.06	-0.08729\\
13.07	-0.08706\\
13.08	-0.08673\\
13.09	-0.08632\\
13.1	-0.08583\\
13.11	-0.08525\\
13.12	-0.08459\\
13.13	-0.08385\\
13.14	-0.08302\\
13.15	-0.08212\\
13.16	-0.08113\\
13.17	-0.08006\\
13.18	-0.07891\\
13.19	-0.07769\\
13.2	-0.07639\\
13.21	-0.07502\\
13.22	-0.07357\\
13.23	-0.07205\\
13.24	-0.07046\\
13.25	-0.0688\\
13.26	-0.06707\\
13.27	-0.06527\\
13.28	-0.06342\\
13.29	-0.0615\\
13.3	-0.05952\\
13.31	-0.05748\\
13.32	-0.05538\\
13.33	-0.05323\\
13.34	-0.05103\\
13.35	-0.04878\\
13.36	-0.04648\\
13.37	-0.04413\\
13.38	-0.04175\\
13.39	-0.03932\\
13.4	-0.03685\\
13.41	-0.03434\\
13.42	-0.03181\\
13.43	-0.02924\\
13.44	-0.02664\\
13.45	-0.02401\\
13.46	-0.02137\\
13.47	-0.0187\\
13.48	-0.01601\\
13.49	-0.01331\\
13.5	-0.01059\\
13.51	-0.007868\\
13.52	-0.005134\\
13.53	-0.002395\\
13.54	0.0003462\\
13.55	0.003087\\
13.56	0.005825\\
13.57	0.008557\\
13.58	0.01128\\
13.59	0.01399\\
13.6	0.01669\\
13.61	0.01937\\
13.62	0.02204\\
13.63	0.02468\\
13.64	0.0273\\
13.65	0.02989\\
13.66	0.03245\\
13.67	0.03498\\
13.68	0.03748\\
13.69	0.03993\\
13.7	0.04235\\
13.71	0.04473\\
13.72	0.04707\\
13.73	0.04935\\
13.74	0.05159\\
13.75	0.05378\\
13.76	0.05592\\
13.77	0.058\\
13.78	0.06002\\
13.79	0.06199\\
13.8	0.06389\\
13.81	0.06573\\
13.82	0.06751\\
13.83	0.06922\\
13.84	0.07087\\
13.85	0.07244\\
13.86	0.07394\\
13.87	0.07537\\
13.88	0.07673\\
13.89	0.07801\\
13.9	0.07921\\
13.91	0.08034\\
13.92	0.08139\\
13.93	0.08235\\
13.94	0.08324\\
13.95	0.08405\\
13.96	0.08477\\
13.97	0.08541\\
13.98	0.08596\\
13.99	0.08644\\
14	0.08682\\
14.01	0.08712\\
14.02	0.08734\\
14.03	0.08747\\
14.04	0.08751\\
14.05	0.08747\\
14.06	0.08735\\
14.07	0.08713\\
14.08	0.08683\\
14.09	0.08645\\
14.1	0.08598\\
14.11	0.08543\\
14.12	0.08479\\
14.13	0.08407\\
14.14	0.08327\\
14.15	0.08239\\
14.16	0.08142\\
14.17	0.08038\\
14.18	0.07925\\
14.19	0.07805\\
14.2	0.07677\\
14.21	0.07542\\
14.22	0.07399\\
14.23	0.07249\\
14.24	0.07092\\
14.25	0.06928\\
14.26	0.06757\\
14.27	0.0658\\
14.28	0.06396\\
14.29	0.06205\\
14.3	0.06009\\
14.31	0.05807\\
14.32	0.05599\\
14.33	0.05386\\
14.34	0.05167\\
14.35	0.04943\\
14.36	0.04714\\
14.37	0.04481\\
14.38	0.04244\\
14.39	0.04002\\
14.4	0.03756\\
14.41	0.03507\\
14.42	0.03254\\
14.43	0.02998\\
14.44	0.02739\\
14.45	0.02477\\
14.46	0.02213\\
14.47	0.01947\\
14.48	0.01678\\
14.49	0.01409\\
14.5	0.01137\\
14.51	0.008651\\
14.52	0.005919\\
14.53	0.003181\\
14.54	0.0004404\\
14.55	-0.002301\\
14.56	-0.00504\\
14.57	-0.007774\\
14.58	-0.0105\\
14.59	-0.01322\\
14.6	-0.01592\\
14.61	-0.01861\\
14.62	-0.02128\\
14.63	-0.02392\\
14.64	-0.02655\\
14.65	-0.02915\\
14.66	-0.03172\\
14.67	-0.03426\\
14.68	-0.03676\\
14.69	-0.03923\\
14.7	-0.04166\\
14.71	-0.04405\\
14.72	-0.0464\\
14.73	-0.0487\\
14.74	-0.05096\\
14.75	-0.05316\\
14.76	-0.05531\\
14.77	-0.05741\\
14.78	-0.05945\\
14.79	-0.06143\\
14.8	-0.06335\\
14.81	-0.06521\\
14.82	-0.06701\\
14.83	-0.06874\\
14.84	-0.0704\\
14.85	-0.072\\
14.86	-0.07352\\
14.87	-0.07497\\
14.88	-0.07635\\
14.89	-0.07765\\
14.9	-0.07888\\
14.91	-0.08002\\
14.92	-0.08109\\
14.93	-0.08209\\
14.94	-0.083\\
14.95	-0.08382\\
14.96	-0.08457\\
14.97	-0.08523\\
14.98	-0.08581\\
14.99	-0.08631\\
15	-0.08672\\
15.01	-0.08705\\
15.02	-0.08729\\
15.03	-0.08744\\
15.04	-0.08751\\
15.05	-0.0875\\
15.06	-0.08739\\
15.07	-0.0872\\
15.08	-0.08693\\
15.09	-0.08657\\
15.1	-0.08613\\
15.11	-0.0856\\
15.12	-0.08498\\
15.13	-0.08429\\
15.14	-0.08351\\
15.15	-0.08265\\
15.16	-0.08171\\
15.17	-0.08069\\
15.18	-0.07958\\
15.19	-0.0784\\
15.2	-0.07715\\
15.21	-0.07582\\
15.22	-0.07441\\
15.23	-0.07293\\
15.24	-0.07138\\
15.25	-0.06976\\
15.26	-0.06807\\
15.27	-0.06631\\
15.28	-0.06449\\
15.29	-0.06261\\
15.3	-0.06066\\
15.31	-0.05866\\
15.32	-0.05659\\
15.33	-0.05447\\
15.34	-0.0523\\
15.35	-0.05008\\
15.36	-0.04781\\
15.37	-0.04549\\
15.38	-0.04312\\
15.39	-0.04072\\
15.4	-0.03827\\
15.41	-0.03579\\
15.42	-0.03327\\
15.43	-0.03072\\
15.44	-0.02813\\
15.45	-0.02552\\
15.46	-0.02289\\
15.47	-0.02023\\
15.48	-0.01756\\
15.49	-0.01486\\
15.5	-0.01215\\
15.51	-0.009433\\
15.52	-0.006704\\
15.53	-0.003967\\
15.54	-0.001227\\
15.55	0.001514\\
15.56	0.004254\\
15.57	0.00699\\
15.58	0.009719\\
15.59	0.01244\\
15.6	0.01515\\
15.61	0.01784\\
15.62	0.02051\\
15.63	0.02317\\
15.64	0.0258\\
15.65	0.02841\\
15.66	0.03098\\
15.67	0.03353\\
15.68	0.03605\\
15.69	0.03853\\
15.7	0.04097\\
15.71	0.04337\\
15.72	0.04573\\
15.73	0.04805\\
15.74	0.05031\\
15.75	0.05253\\
15.76	0.0547\\
15.77	0.05681\\
15.78	0.05887\\
15.79	0.06087\\
15.8	0.06281\\
15.81	0.06469\\
15.82	0.0665\\
15.83	0.06825\\
15.84	0.06993\\
15.85	0.07155\\
15.86	0.07309\\
15.87	0.07456\\
15.88	0.07596\\
15.89	0.07728\\
15.9	0.07853\\
15.91	0.0797\\
15.92	0.0808\\
15.93	0.08181\\
15.94	0.08274\\
15.95	0.0836\\
15.96	0.08437\\
15.97	0.08505\\
15.98	0.08566\\
15.99	0.08618\\
16	0.08661\\
16.01	0.08696\\
16.02	0.08723\\
16.03	0.08741\\
16.04	0.0875\\
16.05	0.08751\\
16.06	0.08743\\
16.07	0.08727\\
16.08	0.08702\\
16.09	0.08668\\
16.1	0.08626\\
16.11	0.08576\\
16.12	0.08517\\
16.13	0.0845\\
16.14	0.08374\\
16.15	0.08291\\
16.16	0.08199\\
16.17	0.08099\\
16.18	0.07991\\
16.19	0.07875\\
16.2	0.07752\\
16.21	0.07621\\
16.22	0.07482\\
16.23	0.07336\\
16.24	0.07183\\
16.25	0.07023\\
16.26	0.06856\\
16.27	0.06682\\
16.28	0.06502\\
16.29	0.06316\\
16.3	0.06123\\
16.31	0.05924\\
16.32	0.05719\\
16.33	0.05509\\
16.34	0.05293\\
16.35	0.05072\\
16.36	0.04846\\
16.37	0.04616\\
16.38	0.04381\\
16.39	0.04141\\
16.4	0.03898\\
16.41	0.0365\\
16.42	0.03399\\
16.43	0.03145\\
16.44	0.02888\\
16.45	0.02628\\
16.46	0.02365\\
16.47	0.021\\
16.48	0.01833\\
16.49	0.01564\\
16.5	0.01293\\
16.51	0.01022\\
16.52	0.007488\\
16.53	0.004753\\
16.54	0.002014\\
16.55	-0.0007278\\
16.56	-0.003468\\
16.57	-0.006206\\
16.58	-0.008937\\
16.59	-0.01166\\
16.6	-0.01437\\
16.61	-0.01707\\
16.62	-0.01975\\
16.63	-0.02241\\
16.64	-0.02505\\
16.65	-0.02766\\
16.66	-0.03025\\
16.67	-0.03281\\
16.68	-0.03533\\
16.69	-0.03782\\
16.7	-0.04027\\
16.71	-0.04269\\
16.72	-0.04506\\
16.73	-0.04739\\
16.74	-0.04967\\
16.75	-0.0519\\
16.76	-0.05408\\
16.77	-0.05621\\
16.78	-0.05829\\
16.79	-0.0603\\
16.8	-0.06226\\
16.81	-0.06415\\
16.82	-0.06599\\
16.83	-0.06776\\
16.84	-0.06946\\
16.85	-0.07109\\
16.86	-0.07266\\
16.87	-0.07415\\
16.88	-0.07557\\
16.89	-0.07691\\
16.9	-0.07818\\
16.91	-0.07938\\
16.92	-0.08049\\
16.93	-0.08153\\
16.94	-0.08249\\
16.95	-0.08336\\
16.96	-0.08415\\
16.97	-0.08487\\
16.98	-0.08549\\
16.99	-0.08604\\
17	-0.0865\\
17.01	-0.08687\\
17.02	-0.08716\\
17.03	-0.08737\\
17.04	-0.08748\\
17.05	-0.08752\\
17.06	-0.08746\\
17.07	-0.08732\\
17.08	-0.0871\\
17.09	-0.08679\\
17.1	-0.08639\\
17.11	-0.08591\\
17.12	-0.08535\\
17.13	-0.0847\\
17.14	-0.08397\\
17.15	-0.08316\\
17.16	-0.08226\\
17.17	-0.08128\\
17.18	-0.08023\\
17.19	-0.07909\\
17.2	-0.07788\\
17.21	-0.07659\\
17.22	-0.07523\\
17.23	-0.07379\\
17.24	-0.07228\\
17.25	-0.0707\\
17.26	-0.06905\\
17.27	-0.06733\\
17.28	-0.06555\\
17.29	-0.0637\\
17.3	-0.06179\\
17.31	-0.05982\\
17.32	-0.05779\\
17.33	-0.0557\\
17.34	-0.05356\\
17.35	-0.05136\\
17.36	-0.04912\\
17.37	-0.04682\\
17.38	-0.04449\\
17.39	-0.0421\\
17.4	-0.03968\\
17.41	-0.03722\\
17.42	-0.03472\\
17.43	-0.03218\\
17.44	-0.02962\\
17.45	-0.02703\\
17.46	-0.02441\\
17.47	-0.02176\\
17.48	-0.0191\\
17.49	-0.01641\\
17.5	-0.01371\\
17.51	-0.011\\
17.52	-0.008271\\
17.53	-0.005538\\
17.54	-0.0028\\
17.55	-5.89e-05\\
17.56	0.002682\\
17.57	0.005421\\
17.58	0.008154\\
17.59	0.01088\\
17.6	0.01359\\
17.61	0.0163\\
17.62	0.01898\\
17.63	0.02165\\
17.64	0.02429\\
17.65	0.02691\\
17.66	0.02951\\
17.67	0.03208\\
17.68	0.03461\\
17.69	0.03711\\
17.7	0.03957\\
17.71	0.042\\
17.72	0.04438\\
17.73	0.04673\\
17.74	0.04902\\
17.75	0.05127\\
17.76	0.05346\\
17.77	0.05561\\
17.78	0.0577\\
17.79	0.05973\\
17.8	0.0617\\
17.81	0.06362\\
17.82	0.06547\\
17.83	0.06726\\
17.84	0.06898\\
17.85	0.07063\\
17.86	0.07221\\
17.87	0.07373\\
17.88	0.07517\\
17.89	0.07654\\
17.9	0.07783\\
17.91	0.07904\\
17.92	0.08018\\
17.93	0.08124\\
17.94	0.08222\\
17.95	0.08312\\
17.96	0.08394\\
17.97	0.08467\\
17.98	0.08532\\
17.99	0.08589\\
18	0.08638\\
18.01	0.08678\\
18.02	0.08709\\
18.03	0.08732\\
18.04	0.08746\\
18.05	0.08752\\
18.06	0.08749\\
18.07	0.08737\\
18.08	0.08717\\
18.09	0.08689\\
18.1	0.08652\\
18.11	0.08606\\
18.12	0.08552\\
18.13	0.0849\\
18.14	0.08419\\
18.15	0.0834\\
18.16	0.08253\\
18.17	0.08157\\
18.18	0.08054\\
18.19	0.07943\\
18.2	0.07824\\
18.21	0.07697\\
18.22	0.07563\\
18.23	0.07421\\
18.24	0.07272\\
18.25	0.07116\\
18.26	0.06953\\
18.27	0.06783\\
18.28	0.06607\\
18.29	0.06424\\
18.3	0.06234\\
18.31	0.06039\\
18.32	0.05837\\
18.33	0.0563\\
18.34	0.05418\\
18.35	0.052\\
18.36	0.04977\\
18.37	0.04749\\
18.38	0.04516\\
18.39	0.04279\\
18.4	0.04038\\
18.41	0.03793\\
18.42	0.03544\\
18.43	0.03292\\
18.44	0.03036\\
18.45	0.02777\\
18.46	0.02516\\
18.47	0.02252\\
18.48	0.01986\\
18.49	0.01718\\
18.5	0.01449\\
18.51	0.01178\\
18.52	0.009054\\
18.53	0.006323\\
18.54	0.003586\\
18.55	0.0008456\\
18.56	-0.001896\\
18.57	-0.004636\\
18.58	-0.007371\\
18.59	-0.0101\\
18.6	-0.01282\\
18.61	-0.01552\\
18.62	-0.01821\\
18.63	-0.02088\\
18.64	-0.02354\\
18.65	-0.02616\\
18.66	-0.02877\\
18.67	-0.03134\\
18.68	-0.03389\\
18.69	-0.0364\\
18.7	-0.03887\\
18.71	-0.04131\\
18.72	-0.04371\\
18.73	-0.04606\\
18.74	-0.04837\\
18.75	-0.05063\\
18.76	-0.05284\\
18.77	-0.055\\
18.78	-0.0571\\
18.79	-0.05915\\
18.8	-0.06114\\
18.81	-0.06308\\
18.82	-0.06495\\
18.83	-0.06675\\
18.84	-0.06849\\
18.85	-0.07016\\
18.86	-0.07177\\
18.87	-0.0733\\
18.88	-0.07476\\
18.89	-0.07615\\
18.9	-0.07747\\
18.91	-0.0787\\
18.92	-0.07986\\
18.93	-0.08095\\
18.94	-0.08195\\
18.95	-0.08287\\
18.96	-0.08371\\
18.97	-0.08447\\
18.98	-0.08515\\
18.99	-0.08574\\
19	-0.08625\\
19.01	-0.08667\\
19.02	-0.08701\\
19.03	-0.08726\\
19.04	-0.08743\\
19.05	-0.08751\\
19.06	-0.08751\\
19.07	-0.08742\\
19.08	-0.08724\\
19.09	-0.08698\\
19.1	-0.08663\\
19.11	-0.0862\\
19.12	-0.08569\\
19.13	-0.08508\\
19.14	-0.0844\\
19.15	-0.08363\\
19.16	-0.08279\\
19.17	-0.08186\\
19.18	-0.08085\\
19.19	-0.07976\\
19.2	-0.07859\\
19.21	-0.07734\\
19.22	-0.07602\\
19.23	-0.07463\\
19.24	-0.07316\\
19.25	-0.07162\\
19.26	-0.07001\\
19.27	-0.06833\\
19.28	-0.06658\\
19.29	-0.06477\\
19.3	-0.06289\\
19.31	-0.06096\\
19.32	-0.05896\\
19.33	-0.0569\\
19.34	-0.05479\\
19.35	-0.05263\\
19.36	-0.05041\\
19.37	-0.04815\\
19.38	-0.04583\\
19.39	-0.04348\\
19.4	-0.04108\\
19.41	-0.03864\\
19.42	-0.03616\\
19.43	-0.03364\\
19.44	-0.0311\\
19.45	-0.02852\\
19.46	-0.02591\\
19.47	-0.02328\\
19.48	-0.02063\\
19.49	-0.01795\\
19.5	-0.01526\\
19.51	-0.01256\\
19.52	-0.009836\\
19.53	-0.007108\\
19.54	-0.004372\\
19.55	-0.001632\\
19.56	0.001109\\
19.57	0.00385\\
19.58	0.006586\\
19.59	0.009317\\
19.6	0.01204\\
19.61	0.01475\\
19.62	0.01744\\
19.63	0.02012\\
19.64	0.02278\\
19.65	0.02541\\
19.66	0.02802\\
19.67	0.03061\\
19.68	0.03316\\
19.69	0.03568\\
19.7	0.03817\\
19.71	0.04061\\
19.72	0.04302\\
19.73	0.04539\\
19.74	0.04771\\
19.75	0.04998\\
19.76	0.05221\\
19.77	0.05439\\
19.78	0.05651\\
19.79	0.05857\\
19.8	0.06058\\
19.81	0.06253\\
19.82	0.06442\\
19.83	0.06624\\
19.84	0.068\\
19.85	0.06969\\
19.86	0.07132\\
19.87	0.07287\\
19.88	0.07435\\
19.89	0.07576\\
19.9	0.0771\\
19.91	0.07836\\
19.92	0.07954\\
19.93	0.08064\\
19.94	0.08167\\
19.95	0.08262\\
19.96	0.08348\\
19.97	0.08426\\
19.98	0.08496\\
19.99	0.08558\\
20	0.08611\\
20.01	0.08656\\
20.02	0.08692\\
20.03	0.0872\\
20.04	0.08739\\
20.05	0.0875\\
20.06	0.08752\\
20.07	0.08745\\
20.08	0.0873\\
20.09	0.08706\\
20.1	0.08674\\
20.11	0.08634\\
20.12	0.08584\\
20.13	0.08527\\
20.14	0.08461\\
20.15	0.08386\\
20.16	0.08304\\
20.17	0.08213\\
20.18	0.08114\\
20.19	0.08008\\
20.2	0.07893\\
20.21	0.07771\\
20.22	0.07641\\
20.23	0.07504\\
20.24	0.07359\\
20.25	0.07207\\
20.26	0.07048\\
20.27	0.06882\\
20.28	0.06709\\
20.29	0.0653\\
20.3	0.06344\\
20.31	0.06152\\
20.32	0.05954\\
20.33	0.0575\\
20.34	0.05541\\
20.35	0.05326\\
20.36	0.05105\\
20.37	0.0488\\
20.38	0.0465\\
20.39	0.04416\\
20.4	0.04177\\
20.41	0.03934\\
20.42	0.03687\\
20.43	0.03437\\
20.44	0.03183\\
20.45	0.02926\\
20.46	0.02666\\
20.47	0.02404\\
20.48	0.02139\\
20.49	0.01872\\
20.5	0.01604\\
20.51	0.01333\\
20.52	0.01062\\
20.53	0.007892\\
20.54	0.005158\\
20.55	0.002419\\
20.56	-0.0003227\\
20.57	-0.003064\\
20.58	-0.005802\\
20.59	-0.008534\\
20.6	-0.01126\\
20.61	-0.01397\\
20.62	-0.01667\\
20.63	-0.01935\\
20.64	-0.02202\\
20.65	-0.02466\\
20.66	-0.02728\\
20.67	-0.02987\\
20.68	-0.03243\\
20.69	-0.03496\\
20.7	-0.03746\\
20.71	-0.03992\\
20.72	-0.04234\\
20.73	-0.04471\\
20.74	-0.04705\\
20.75	-0.04934\\
20.76	-0.05158\\
20.77	-0.05377\\
20.78	-0.0559\\
20.79	-0.05799\\
20.8	-0.06001\\
20.81	-0.06198\\
20.82	-0.06388\\
20.83	-0.06572\\
20.84	-0.0675\\
20.85	-0.06921\\
20.86	-0.07086\\
20.87	-0.07243\\
20.88	-0.07394\\
20.89	-0.07537\\
20.9	-0.07672\\
20.91	-0.078\\
20.92	-0.07921\\
20.93	-0.08034\\
20.94	-0.08138\\
20.95	-0.08235\\
20.96	-0.08324\\
20.97	-0.08405\\
20.98	-0.08477\\
20.99	-0.08541\\
21	-0.08597\\
21.01	-0.08644\\
21.02	-0.08683\\
21.03	-0.08713\\
21.04	-0.08735\\
21.05	-0.08748\\
21.06	-0.08752\\
21.07	-0.08748\\
21.08	-0.08735\\
21.09	-0.08714\\
21.1	-0.08684\\
21.11	-0.08646\\
21.12	-0.08599\\
21.13	-0.08544\\
21.14	-0.08481\\
21.15	-0.08409\\
21.16	-0.08328\\
21.17	-0.0824\\
21.18	-0.08144\\
21.19	-0.08039\\
21.2	-0.07927\\
21.21	-0.07807\\
21.22	-0.07679\\
21.23	-0.07544\\
21.24	-0.07401\\
21.25	-0.07251\\
21.26	-0.07094\\
21.27	-0.0693\\
21.28	-0.06759\\
21.29	-0.06582\\
21.3	-0.06398\\
21.31	-0.06208\\
21.32	-0.06011\\
21.33	-0.05809\\
21.34	-0.05601\\
21.35	-0.05388\\
21.36	-0.05169\\
21.37	-0.04945\\
21.38	-0.04717\\
21.39	-0.04484\\
21.4	-0.04246\\
21.41	-0.04004\\
21.42	-0.03759\\
21.43	-0.03509\\
21.44	-0.03256\\
21.45	-0.03\\
21.46	-0.02741\\
21.47	-0.0248\\
21.48	-0.02215\\
21.49	-0.01949\\
21.5	-0.01681\\
21.51	-0.01411\\
21.52	-0.0114\\
21.53	-0.008675\\
21.54	-0.005943\\
21.55	-0.003205\\
21.56	-0.000464\\
21.57	0.002277\\
21.58	0.005017\\
21.59	0.007751\\
21.6	0.01048\\
21.61	0.01319\\
21.62	0.0159\\
21.63	0.01859\\
21.64	0.02126\\
21.65	0.0239\\
21.66	0.02653\\
21.67	0.02913\\
21.68	0.0317\\
21.69	0.03424\\
21.7	0.03675\\
21.71	0.03922\\
21.72	0.04165\\
21.73	0.04404\\
21.74	0.04638\\
21.75	0.04869\\
21.76	0.05094\\
21.77	0.05314\\
21.78	0.0553\\
21.79	0.05739\\
21.8	0.05944\\
21.81	0.06142\\
21.82	0.06334\\
21.83	0.0652\\
21.84	0.067\\
21.85	0.06873\\
21.86	0.07039\\
21.87	0.07199\\
21.88	0.07351\\
21.89	0.07496\\
21.9	0.07634\\
21.91	0.07765\\
21.92	0.07887\\
21.93	0.08002\\
21.94	0.08109\\
21.95	0.08208\\
21.96	0.083\\
21.97	0.08383\\
21.98	0.08457\\
21.99	0.08524\\
22	0.08582\\
22.01	0.08631\\
22.02	0.08673\\
22.03	0.08705\\
22.04	0.08729\\
22.05	0.08745\\
22.06	0.08752\\
22.07	0.0875\\
22.08	0.0874\\
22.09	0.08721\\
22.1	0.08694\\
22.11	0.08658\\
22.12	0.08614\\
22.13	0.08561\\
22.14	0.085\\
22.15	0.0843\\
22.16	0.08352\\
22.17	0.08266\\
22.18	0.08172\\
22.19	0.0807\\
22.2	0.0796\\
22.21	0.07842\\
22.22	0.07717\\
22.23	0.07583\\
22.24	0.07443\\
22.25	0.07295\\
22.26	0.0714\\
22.27	0.06978\\
22.28	0.06809\\
22.29	0.06633\\
22.3	0.06451\\
22.31	0.06263\\
22.32	0.06068\\
22.33	0.05868\\
22.34	0.05662\\
22.35	0.0545\\
22.36	0.05233\\
22.37	0.0501\\
22.38	0.04783\\
22.39	0.04551\\
22.4	0.04315\\
22.41	0.04074\\
22.42	0.03829\\
22.43	0.03581\\
22.44	0.03329\\
22.45	0.03074\\
22.46	0.02816\\
22.47	0.02555\\
22.48	0.02291\\
22.49	0.02026\\
22.5	0.01758\\
22.51	0.01489\\
22.52	0.01218\\
22.53	0.009458\\
22.54	0.006728\\
22.55	0.003991\\
22.56	0.001251\\
22.57	-0.001491\\
22.58	-0.004231\\
22.59	-0.006967\\
22.6	-0.009696\\
22.61	-0.01242\\
22.62	-0.01512\\
22.63	-0.01782\\
22.64	-0.02049\\
22.65	-0.02315\\
22.66	-0.02578\\
22.67	-0.02839\\
22.68	-0.03097\\
22.69	-0.03351\\
22.7	-0.03603\\
22.71	-0.03851\\
22.72	-0.04095\\
22.73	-0.04336\\
22.74	-0.04572\\
22.75	-0.04803\\
22.76	-0.0503\\
22.77	-0.05252\\
22.78	-0.05469\\
22.79	-0.0568\\
22.8	-0.05886\\
22.81	-0.06086\\
22.82	-0.0628\\
22.83	-0.06468\\
22.84	-0.06649\\
22.85	-0.06824\\
22.86	-0.06992\\
22.87	-0.07154\\
22.88	-0.07308\\
22.89	-0.07456\\
22.9	-0.07596\\
22.91	-0.07728\\
22.92	-0.07853\\
22.93	-0.0797\\
22.94	-0.08079\\
22.95	-0.08181\\
22.96	-0.08274\\
22.97	-0.0836\\
22.98	-0.08437\\
22.99	-0.08506\\
23	-0.08566\\
23.01	-0.08618\\
23.02	-0.08662\\
23.03	-0.08697\\
23.04	-0.08723\\
23.05	-0.08741\\
23.06	-0.08751\\
23.07	-0.08752\\
23.08	-0.08744\\
23.09	-0.08728\\
23.1	-0.08703\\
23.11	-0.08669\\
23.12	-0.08627\\
23.13	-0.08577\\
23.14	-0.08518\\
23.15	-0.08451\\
23.16	-0.08376\\
23.17	-0.08292\\
23.18	-0.082\\
23.19	-0.081\\
23.2	-0.07993\\
23.21	-0.07877\\
23.22	-0.07754\\
23.23	-0.07623\\
23.24	-0.07484\\
23.25	-0.07338\\
23.26	-0.07185\\
23.27	-0.07025\\
23.28	-0.06858\\
23.29	-0.06685\\
23.3	-0.06504\\
23.31	-0.06318\\
23.32	-0.06125\\
23.33	-0.05926\\
23.34	-0.05721\\
23.35	-0.05511\\
23.36	-0.05295\\
23.37	-0.05075\\
23.38	-0.04849\\
23.39	-0.04618\\
23.4	-0.04383\\
23.41	-0.04144\\
23.42	-0.039\\
23.43	-0.03653\\
23.44	-0.03402\\
23.45	-0.03148\\
23.46	-0.0289\\
23.47	-0.0263\\
23.48	-0.02367\\
23.49	-0.02102\\
23.5	-0.01835\\
23.51	-0.01566\\
23.52	-0.01296\\
23.53	-0.01024\\
23.54	-0.007512\\
23.55	-0.004777\\
23.56	-0.002037\\
23.57	0.0007043\\
23.58	0.003445\\
23.59	0.006183\\
23.6	0.008914\\
23.61	0.01164\\
23.62	0.01435\\
23.63	0.01705\\
23.64	0.01973\\
23.65	0.02239\\
23.66	0.02503\\
23.67	0.02764\\
23.68	0.03023\\
23.69	0.03279\\
23.7	0.03531\\
23.71	0.0378\\
23.72	0.04026\\
23.73	0.04267\\
23.74	0.04504\\
23.75	0.04737\\
23.76	0.04965\\
23.77	0.05189\\
23.78	0.05407\\
23.79	0.0562\\
23.8	0.05827\\
23.81	0.06029\\
23.82	0.06225\\
23.83	0.06414\\
23.84	0.06598\\
23.85	0.06775\\
23.86	0.06945\\
23.87	0.07108\\
23.88	0.07265\\
23.89	0.07414\\
23.9	0.07556\\
23.91	0.07691\\
23.92	0.07818\\
23.93	0.07937\\
23.94	0.08049\\
23.95	0.08153\\
23.96	0.08248\\
23.97	0.08336\\
23.98	0.08416\\
23.99	0.08487\\
24	0.0855\\
24.01	0.08604\\
24.02	0.0865\\
24.03	0.08688\\
24.04	0.08717\\
24.05	0.08737\\
24.06	0.08749\\
24.07	0.08752\\
24.08	0.08747\\
24.09	0.08733\\
24.1	0.08711\\
24.11	0.0868\\
24.12	0.0864\\
24.13	0.08593\\
24.14	0.08536\\
24.15	0.08471\\
24.16	0.08398\\
24.17	0.08317\\
24.18	0.08227\\
24.19	0.0813\\
24.2	0.08024\\
24.21	0.07911\\
24.22	0.0779\\
24.23	0.07661\\
24.24	0.07525\\
24.25	0.07381\\
24.26	0.0723\\
24.27	0.07072\\
24.28	0.06907\\
24.29	0.06735\\
24.3	0.06557\\
24.31	0.06372\\
24.32	0.06181\\
24.33	0.05984\\
24.34	0.05781\\
24.35	0.05572\\
24.36	0.05358\\
24.37	0.05139\\
24.38	0.04914\\
24.39	0.04685\\
24.4	0.04451\\
24.41	0.04213\\
24.42	0.0397\\
24.43	0.03724\\
24.44	0.03474\\
24.45	0.03221\\
24.46	0.02964\\
24.47	0.02705\\
24.48	0.02443\\
24.49	0.02179\\
24.5	0.01912\\
24.51	0.01644\\
24.52	0.01373\\
24.53	0.01102\\
24.54	0.008296\\
24.55	0.005562\\
24.56	0.002824\\
24.57	8.247e-05\\
24.58	-0.002659\\
24.59	-0.005398\\
24.6	-0.008131\\
24.61	-0.01086\\
24.62	-0.01357\\
24.63	-0.01627\\
24.64	-0.01896\\
24.65	-0.02163\\
24.66	-0.02427\\
24.67	-0.02689\\
24.68	-0.02949\\
24.69	-0.03206\\
24.7	-0.03459\\
24.71	-0.03709\\
24.72	-0.03956\\
24.73	-0.04198\\
24.74	-0.04437\\
24.75	-0.04671\\
24.76	-0.049\\
24.77	-0.05125\\
24.78	-0.05345\\
24.79	-0.05559\\
24.8	-0.05768\\
24.81	-0.05972\\
24.82	-0.06169\\
24.83	-0.06361\\
24.84	-0.06546\\
24.85	-0.06725\\
24.86	-0.06897\\
24.87	-0.07062\\
24.88	-0.07221\\
24.89	-0.07372\\
24.9	-0.07516\\
24.91	-0.07653\\
24.92	-0.07782\\
24.93	-0.07904\\
24.94	-0.08018\\
24.95	-0.08124\\
24.96	-0.08222\\
24.97	-0.08312\\
24.98	-0.08394\\
24.99	-0.08467\\
25	-0.08533\\
25.01	-0.08589\\
25.02	-0.08638\\
25.03	-0.08678\\
25.04	-0.08709\\
25.05	-0.08732\\
25.06	-0.08747\\
25.07	-0.08753\\
25.08	-0.0875\\
25.09	-0.08738\\
25.1	-0.08718\\
25.11	-0.0869\\
25.12	-0.08653\\
25.13	-0.08607\\
25.14	-0.08553\\
25.15	-0.08491\\
25.16	-0.0842\\
25.17	-0.08341\\
25.18	-0.08254\\
25.19	-0.08159\\
25.2	-0.08056\\
25.21	-0.07944\\
25.22	-0.07825\\
25.23	-0.07699\\
25.24	-0.07565\\
25.25	-0.07423\\
25.26	-0.07274\\
25.27	-0.07118\\
25.28	-0.06955\\
25.29	-0.06785\\
25.3	-0.06609\\
25.31	-0.06426\\
25.32	-0.06236\\
25.33	-0.06041\\
25.34	-0.0584\\
25.35	-0.05633\\
25.36	-0.0542\\
25.37	-0.05202\\
25.38	-0.04979\\
25.39	-0.04751\\
25.4	-0.04519\\
25.41	-0.04282\\
25.42	-0.0404\\
25.43	-0.03795\\
25.44	-0.03546\\
25.45	-0.03294\\
25.46	-0.03038\\
25.47	-0.0278\\
25.48	-0.02518\\
25.49	-0.02255\\
25.5	-0.01989\\
25.51	-0.01721\\
25.52	-0.01451\\
25.53	-0.0118\\
25.54	-0.009078\\
25.55	-0.006347\\
25.56	-0.00361\\
25.57	-0.0008692\\
25.58	0.001873\\
25.59	0.004612\\
25.6	0.007348\\
25.61	0.01008\\
25.62	0.01279\\
25.63	0.0155\\
25.64	0.01819\\
25.65	0.02086\\
25.66	0.02352\\
25.67	0.02614\\
25.68	0.02875\\
25.69	0.03132\\
25.7	0.03387\\
25.71	0.03638\\
25.72	0.03885\\
25.73	0.04129\\
25.74	0.04369\\
25.75	0.04604\\
25.76	0.04835\\
25.77	0.05061\\
25.78	0.05282\\
25.79	0.05498\\
25.8	0.05709\\
25.81	0.05914\\
25.82	0.06113\\
25.83	0.06306\\
25.84	0.06493\\
25.85	0.06674\\
25.86	0.06848\\
25.87	0.07016\\
25.88	0.07176\\
25.89	0.0733\\
25.9	0.07476\\
25.91	0.07615\\
25.92	0.07746\\
25.93	0.0787\\
25.94	0.07986\\
25.95	0.08094\\
25.96	0.08195\\
25.97	0.08287\\
25.98	0.08371\\
25.99	0.08447\\
26	0.08515\\
26.01	0.08574\\
26.02	0.08625\\
26.03	0.08667\\
26.04	0.08701\\
26.05	0.08727\\
26.06	0.08744\\
26.07	0.08752\\
26.08	0.08751\\
26.09	0.08743\\
26.1	0.08725\\
26.11	0.08699\\
26.12	0.08664\\
26.13	0.08621\\
26.14	0.0857\\
26.15	0.0851\\
26.16	0.08441\\
26.17	0.08365\\
26.18	0.0828\\
26.19	0.08187\\
26.2	0.08086\\
26.21	0.07977\\
26.22	0.0786\\
26.23	0.07736\\
26.24	0.07604\\
26.25	0.07464\\
26.26	0.07318\\
26.27	0.07164\\
26.28	0.07003\\
26.29	0.06835\\
26.3	0.0666\\
26.31	0.06479\\
26.32	0.06291\\
26.33	0.06098\\
26.34	0.05898\\
26.35	0.05693\\
26.36	0.05482\\
26.37	0.05265\\
26.38	0.05044\\
26.39	0.04817\\
26.4	0.04586\\
26.41	0.0435\\
26.42	0.0411\\
26.43	0.03866\\
26.44	0.03618\\
26.45	0.03367\\
26.46	0.03112\\
26.47	0.02854\\
26.48	0.02594\\
26.49	0.02331\\
26.5	0.02065\\
26.51	0.01798\\
26.52	0.01529\\
26.53	0.01258\\
26.54	0.009861\\
26.55	0.007132\\
26.56	0.004396\\
26.57	0.001656\\
26.58	-0.001086\\
26.59	-0.003827\\
26.6	-0.006564\\
26.61	-0.009294\\
26.62	-0.01202\\
26.63	-0.01473\\
26.64	-0.01742\\
26.65	-0.0201\\
26.66	-0.02276\\
26.67	-0.02539\\
26.68	-0.028\\
26.69	-0.03059\\
26.7	-0.03314\\
26.71	-0.03566\\
26.72	-0.03815\\
26.73	-0.0406\\
26.74	-0.04301\\
26.75	-0.04537\\
26.76	-0.04769\\
26.77	-0.04997\\
26.78	-0.0522\\
26.79	-0.05437\\
26.8	-0.05649\\
26.81	-0.05856\\
26.82	-0.06057\\
26.83	-0.06252\\
26.84	-0.06441\\
26.85	-0.06623\\
26.86	-0.06799\\
26.87	-0.06968\\
26.88	-0.07131\\
26.89	-0.07286\\
26.9	-0.07435\\
26.91	-0.07576\\
26.92	-0.07709\\
26.93	-0.07835\\
26.94	-0.07954\\
26.95	-0.08064\\
26.96	-0.08167\\
26.97	-0.08261\\
26.98	-0.08348\\
26.99	-0.08426\\
27	-0.08496\\
27.01	-0.08558\\
27.02	-0.08611\\
27.03	-0.08656\\
27.04	-0.08693\\
27.05	-0.0872\\
27.06	-0.0874\\
27.07	-0.0875\\
27.08	-0.08753\\
27.09	-0.08746\\
27.1	-0.08731\\
27.11	-0.08707\\
27.12	-0.08675\\
27.13	-0.08635\\
27.14	-0.08585\\
27.15	-0.08528\\
27.16	-0.08462\\
27.17	-0.08388\\
27.18	-0.08305\\
27.19	-0.08215\\
27.2	-0.08116\\
27.21	-0.08009\\
27.22	-0.07895\\
27.23	-0.07773\\
27.24	-0.07643\\
27.25	-0.07505\\
27.26	-0.07361\\
27.27	-0.07209\\
27.28	-0.0705\\
27.29	-0.06884\\
27.3	-0.06711\\
27.31	-0.06532\\
27.32	-0.06346\\
27.33	-0.06154\\
27.34	-0.05956\\
27.35	-0.05752\\
27.36	-0.05543\\
27.37	-0.05328\\
27.38	-0.05108\\
27.39	-0.04883\\
27.4	-0.04653\\
27.41	-0.04418\\
27.42	-0.04179\\
27.43	-0.03937\\
27.44	-0.0369\\
27.45	-0.03439\\
27.46	-0.03186\\
27.47	-0.02929\\
27.48	-0.02669\\
27.49	-0.02406\\
27.5	-0.02142\\
27.51	-0.01875\\
27.52	-0.01606\\
27.53	-0.01336\\
27.54	-0.01064\\
27.55	-0.007916\\
27.56	-0.005182\\
27.57	-0.002442\\
27.58	0.0002991\\
27.59	0.00304\\
27.6	0.005779\\
27.61	0.008511\\
27.62	0.01124\\
27.63	0.01395\\
27.64	0.01665\\
27.65	0.01933\\
27.66	0.022\\
27.67	0.02464\\
27.68	0.02726\\
27.69	0.02985\\
27.7	0.03241\\
27.71	0.03494\\
27.72	0.03744\\
27.73	0.0399\\
27.74	0.04232\\
27.75	0.0447\\
27.76	0.04703\\
27.77	0.04932\\
27.78	0.05156\\
27.79	0.05375\\
27.8	0.05589\\
27.81	0.05797\\
27.82	0.06\\
27.83	0.06196\\
27.84	0.06387\\
27.85	0.06571\\
27.86	0.06749\\
27.87	0.06921\\
27.88	0.07085\\
27.89	0.07242\\
27.9	0.07393\\
27.91	0.07536\\
27.92	0.07672\\
27.93	0.078\\
27.94	0.07921\\
27.95	0.08033\\
27.96	0.08138\\
27.97	0.08235\\
27.98	0.08324\\
27.99	0.08405\\
28	0.08477\\
28.01	0.08541\\
28.02	0.08597\\
28.03	0.08644\\
28.04	0.08683\\
28.05	0.08713\\
28.06	0.08735\\
28.07	0.08748\\
28.08	0.08753\\
28.09	0.08749\\
28.1	0.08736\\
28.11	0.08715\\
28.12	0.08686\\
28.13	0.08647\\
28.14	0.08601\\
28.15	0.08545\\
28.16	0.08482\\
28.17	0.0841\\
28.18	0.0833\\
28.19	0.08242\\
28.2	0.08145\\
28.21	0.08041\\
28.22	0.07929\\
28.23	0.07809\\
28.24	0.07681\\
28.25	0.07546\\
28.26	0.07403\\
28.27	0.07253\\
28.28	0.07096\\
28.29	0.06932\\
28.3	0.06761\\
28.31	0.06584\\
28.32	0.064\\
28.33	0.0621\\
28.34	0.06014\\
28.35	0.05811\\
28.36	0.05604\\
28.37	0.0539\\
28.38	0.05172\\
28.39	0.04948\\
28.4	0.04719\\
28.41	0.04486\\
28.42	0.04248\\
28.43	0.04007\\
28.44	0.03761\\
28.45	0.03512\\
28.46	0.03259\\
28.47	0.03003\\
28.48	0.02744\\
28.49	0.02482\\
28.5	0.02218\\
28.51	0.01952\\
28.52	0.01683\\
28.53	0.01414\\
28.54	0.01142\\
28.55	0.008699\\
28.56	0.005967\\
28.57	0.003229\\
28.58	0.0004876\\
28.59	-0.002254\\
28.6	-0.004994\\
28.61	-0.007728\\
28.62	-0.01046\\
28.63	-0.01317\\
28.64	-0.01588\\
28.65	-0.01856\\
28.66	-0.02123\\
28.67	-0.02388\\
28.68	-0.02651\\
28.69	-0.02911\\
28.7	-0.03168\\
28.71	-0.03422\\
28.72	-0.03673\\
28.73	-0.0392\\
28.74	-0.04163\\
28.75	-0.04402\\
28.76	-0.04637\\
28.77	-0.04867\\
28.78	-0.05093\\
28.79	-0.05313\\
28.8	-0.05528\\
28.81	-0.05738\\
28.82	-0.05942\\
28.83	-0.06141\\
28.84	-0.06333\\
28.85	-0.06519\\
28.86	-0.06699\\
28.87	-0.06872\\
28.88	-0.07039\\
28.89	-0.07198\\
28.9	-0.07351\\
28.91	-0.07496\\
28.92	-0.07634\\
28.93	-0.07764\\
28.94	-0.07887\\
28.95	-0.08002\\
28.96	-0.08109\\
28.97	-0.08208\\
28.98	-0.083\\
28.99	-0.08383\\
29	-0.08457\\
29.01	-0.08524\\
29.02	-0.08582\\
29.03	-0.08632\\
29.04	-0.08673\\
29.05	-0.08706\\
29.06	-0.0873\\
29.07	-0.08746\\
29.08	-0.08753\\
29.09	-0.08751\\
29.1	-0.08741\\
29.11	-0.08722\\
29.12	-0.08695\\
29.13	-0.08659\\
29.14	-0.08615\\
29.15	-0.08562\\
29.16	-0.08501\\
29.17	-0.08432\\
29.18	-0.08354\\
29.19	-0.08268\\
29.2	-0.08174\\
29.21	-0.08072\\
29.22	-0.07962\\
29.23	-0.07844\\
29.24	-0.07718\\
29.25	-0.07585\\
29.26	-0.07445\\
29.27	-0.07297\\
29.28	-0.07142\\
29.29	-0.0698\\
29.3	-0.06811\\
29.31	-0.06636\\
29.32	-0.06453\\
29.33	-0.06265\\
29.34	-0.06071\\
29.35	-0.0587\\
29.36	-0.05664\\
29.37	-0.05452\\
29.38	-0.05235\\
29.39	-0.05013\\
29.4	-0.04785\\
29.41	-0.04553\\
29.42	-0.04317\\
29.43	-0.04077\\
29.44	-0.03832\\
29.45	-0.03584\\
29.46	-0.03332\\
29.47	-0.03076\\
29.48	-0.02818\\
29.49	-0.02557\\
29.5	-0.02294\\
29.51	-0.02028\\
29.52	-0.01761\\
29.53	-0.01491\\
29.54	-0.0122\\
29.55	-0.009482\\
29.56	-0.006752\\
29.57	-0.004015\\
29.58	-0.001274\\
29.59	0.001467\\
29.6	0.004208\\
29.61	0.006944\\
29.62	0.009674\\
29.63	0.01239\\
29.64	0.0151\\
29.65	0.01779\\
29.66	0.02047\\
29.67	0.02313\\
29.68	0.02576\\
29.69	0.02837\\
29.7	0.03095\\
29.71	0.0335\\
29.72	0.03601\\
29.73	0.03849\\
29.74	0.04094\\
29.75	0.04334\\
29.76	0.0457\\
29.77	0.04802\\
29.78	0.05028\\
29.79	0.0525\\
29.8	0.05467\\
29.81	0.05679\\
29.82	0.05884\\
29.83	0.06085\\
29.84	0.06279\\
29.85	0.06467\\
29.86	0.06648\\
29.87	0.06823\\
29.88	0.06992\\
29.89	0.07153\\
29.9	0.07308\\
29.91	0.07455\\
29.92	0.07595\\
29.93	0.07728\\
29.94	0.07853\\
29.95	0.0797\\
29.96	0.08079\\
29.97	0.08181\\
29.98	0.08274\\
29.99	0.0836\\
};
\addlegendentry{MP}

\end{axis}

\begin{axis}[%
width=0.951\figW,
height=0.5\figH,
at={(0\figW,0.5\figH)},
scale only axis,
xmin=0,
xmax=30,
xtick={0,5,10,15,20,25,30},
xticklabels={\empty},
ymin=-0.4,
ymax=0.4,
ylabel style={font=\color{white!15!black}},
ylabel={Angle [rad]},
axis background/.style={fill=white},
legend style={at={(0.03,0.97)}, anchor=north west, legend cell align=left, align=left, draw=white!15!black}
]
\addplot[only marks, mark=*, mark options={}, mark size=0.5000pt, draw=green] table[row sep=crcr]{%
x	y\\
0	0.0875\\
0.01	0.08733\\
0.02	0.08711\\
0.03	0.08681\\
0.04	0.08643\\
0.05	0.08596\\
0.06	0.08541\\
0.07	0.08477\\
0.08	0.08405\\
0.09	0.08324\\
0.1	0.08236\\
0.11	0.08139\\
0.12	0.08035\\
0.13	0.07922\\
0.14	0.07802\\
0.15	0.07674\\
0.16	0.07539\\
0.17	0.07396\\
0.18	0.07246\\
0.19	0.07089\\
0.2	0.06925\\
0.21	0.06754\\
0.22	0.06576\\
0.23	0.06392\\
0.24	0.06202\\
0.25	0.06006\\
0.26	0.05803\\
0.27	0.05595\\
0.28	0.05382\\
0.29	0.05163\\
0.3	0.0494\\
0.31	0.04711\\
0.32	0.04478\\
0.33	0.0424\\
0.34	0.03998\\
0.35	0.03753\\
0.36	0.03503\\
0.37	0.0325\\
0.38	0.02994\\
0.39	0.02736\\
0.4	0.02474\\
0.41	0.0221\\
0.42	0.01944\\
0.43	0.01676\\
0.44	0.01406\\
0.45	0.01135\\
0.46	0.008623\\
0.47	0.005892\\
0.48	0.003156\\
0.49	0.0004157\\
0.5	-0.002325\\
0.51	-0.005063\\
0.52	-0.007796\\
0.53	-0.01052\\
0.54	-0.01324\\
0.55	-0.01594\\
0.56	-0.01862\\
0.57	-0.02129\\
0.58	-0.02394\\
0.59	-0.02656\\
0.6	-0.02916\\
0.61	-0.03173\\
0.62	-0.03427\\
0.63	-0.03677\\
0.64	-0.03924\\
0.65	-0.04167\\
0.66	-0.04406\\
0.67	-0.04641\\
0.68	-0.04871\\
0.69	-0.05096\\
0.7	-0.05316\\
0.71	-0.05531\\
0.72	-0.05741\\
0.73	-0.05945\\
0.74	-0.06143\\
0.75	-0.06335\\
0.76	-0.06521\\
0.77	-0.067\\
0.78	-0.06873\\
0.79	-0.0704\\
0.8	-0.07199\\
0.81	-0.07351\\
0.82	-0.07496\\
0.83	-0.07634\\
0.84	-0.07764\\
0.85	-0.07887\\
0.86	-0.08001\\
0.87	-0.08108\\
0.88	-0.08207\\
0.89	-0.08298\\
0.9	-0.08381\\
0.91	-0.08456\\
0.92	-0.08522\\
0.93	-0.0858\\
0.94	-0.0863\\
0.95	-0.08671\\
0.96	-0.08703\\
0.97	-0.08727\\
0.98	-0.08743\\
0.99	-0.0875\\
1	-0.08748\\
1.01	-0.08738\\
1.02	-0.08719\\
1.03	-0.08691\\
1.04	-0.08655\\
1.05	-0.08611\\
1.06	-0.08558\\
1.07	-0.08497\\
1.08	-0.08427\\
1.09	-0.0835\\
1.1	-0.08264\\
1.11	-0.08169\\
1.12	-0.08067\\
1.13	-0.07957\\
1.14	-0.07839\\
1.15	-0.07714\\
1.16	-0.07581\\
1.17	-0.0744\\
1.18	-0.07292\\
1.19	-0.07137\\
1.2	-0.06975\\
1.21	-0.06806\\
1.22	-0.06631\\
1.23	-0.06449\\
1.24	-0.0626\\
1.25	-0.06066\\
1.26	-0.05865\\
1.27	-0.05659\\
1.28	-0.05447\\
1.29	-0.0523\\
1.3	-0.05008\\
1.31	-0.04781\\
1.32	-0.04549\\
1.33	-0.04313\\
1.34	-0.04072\\
1.35	-0.03828\\
1.36	-0.03579\\
1.37	-0.03327\\
1.38	-0.03072\\
1.39	-0.02814\\
1.4	-0.02554\\
1.41	-0.0229\\
1.42	-0.02025\\
1.43	-0.01757\\
1.44	-0.01488\\
1.45	-0.01217\\
1.46	-0.00945\\
1.47	-0.006721\\
1.48	-0.003986\\
1.49	-0.001247\\
1.5	0.001494\\
1.51	0.004233\\
1.52	0.006967\\
1.53	0.009695\\
1.54	0.01241\\
1.55	0.01512\\
1.56	0.01781\\
1.57	0.02049\\
1.58	0.02314\\
1.59	0.02577\\
1.6	0.02838\\
1.61	0.03095\\
1.62	0.0335\\
1.63	0.03602\\
1.64	0.0385\\
1.65	0.04094\\
1.66	0.04334\\
1.67	0.0457\\
1.68	0.04801\\
1.69	0.05028\\
1.7	0.0525\\
1.71	0.05467\\
1.72	0.05678\\
1.73	0.05884\\
1.74	0.06083\\
1.75	0.06277\\
1.76	0.06465\\
1.77	0.06647\\
1.78	0.06822\\
1.79	0.0699\\
1.8	0.07151\\
1.81	0.07306\\
1.82	0.07453\\
1.83	0.07593\\
1.84	0.07725\\
1.85	0.0785\\
1.86	0.07967\\
1.87	0.08077\\
1.88	0.08178\\
1.89	0.08272\\
1.9	0.08357\\
1.91	0.08434\\
1.92	0.08503\\
1.93	0.08563\\
1.94	0.08615\\
1.95	0.08659\\
1.96	0.08694\\
1.97	0.08721\\
1.98	0.08739\\
1.99	0.08748\\
2	0.08749\\
2.01	0.08742\\
2.02	0.08725\\
2.03	0.08701\\
2.04	0.08667\\
2.05	0.08625\\
2.06	0.08575\\
2.07	0.08516\\
2.08	0.08449\\
2.09	0.08374\\
2.1	0.0829\\
2.11	0.08199\\
2.12	0.08099\\
2.13	0.07991\\
2.14	0.07876\\
2.15	0.07753\\
2.16	0.07622\\
2.17	0.07483\\
2.18	0.07338\\
2.19	0.07185\\
2.2	0.07025\\
2.21	0.06858\\
2.22	0.06685\\
2.23	0.06504\\
2.24	0.06318\\
2.25	0.06125\\
2.26	0.05927\\
2.27	0.05722\\
2.28	0.05512\\
2.29	0.05297\\
2.3	0.05076\\
2.31	0.0485\\
2.32	0.0462\\
2.33	0.04385\\
2.34	0.04145\\
2.35	0.03902\\
2.36	0.03655\\
2.37	0.03404\\
2.38	0.0315\\
2.39	0.02893\\
2.4	0.02633\\
2.41	0.0237\\
2.42	0.02105\\
2.43	0.01838\\
2.44	0.0157\\
2.45	0.01299\\
2.46	0.01028\\
2.47	0.00755\\
2.48	0.004816\\
2.49	0.002078\\
2.5	-0.0006623\\
2.51	-0.003402\\
2.52	-0.006138\\
2.53	-0.008869\\
2.54	-0.01159\\
2.55	-0.0143\\
2.56	-0.017\\
2.57	-0.01968\\
2.58	-0.02234\\
2.59	-0.02498\\
2.6	-0.02759\\
2.61	-0.03018\\
2.62	-0.03273\\
2.63	-0.03526\\
2.64	-0.03775\\
2.65	-0.0402\\
2.66	-0.04262\\
2.67	-0.04499\\
2.68	-0.04732\\
2.69	-0.0496\\
2.7	-0.05183\\
2.71	-0.05401\\
2.72	-0.05614\\
2.73	-0.05822\\
2.74	-0.06023\\
2.75	-0.06219\\
2.76	-0.06409\\
2.77	-0.06592\\
2.78	-0.06769\\
2.79	-0.0694\\
2.8	-0.07103\\
2.81	-0.0726\\
2.82	-0.07409\\
2.83	-0.07551\\
2.84	-0.07686\\
2.85	-0.07813\\
2.86	-0.07933\\
2.87	-0.08044\\
2.88	-0.08148\\
2.89	-0.08244\\
2.9	-0.08332\\
2.91	-0.08411\\
2.92	-0.08483\\
2.93	-0.08546\\
2.94	-0.08601\\
2.95	-0.08647\\
2.96	-0.08684\\
2.97	-0.08714\\
2.98	-0.08734\\
2.99	-0.08746\\
3	-0.0875\\
3.01	-0.08745\\
3.02	-0.08731\\
3.03	-0.08709\\
3.04	-0.08678\\
3.05	-0.08639\\
3.06	-0.08591\\
3.07	-0.08535\\
3.08	-0.08471\\
3.09	-0.08398\\
3.1	-0.08317\\
3.11	-0.08227\\
3.12	-0.0813\\
3.13	-0.08025\\
3.14	-0.07912\\
3.15	-0.07791\\
3.16	-0.07662\\
3.17	-0.07526\\
3.18	-0.07383\\
3.19	-0.07232\\
3.2	-0.07074\\
3.21	-0.06909\\
3.22	-0.06738\\
3.23	-0.0656\\
3.24	-0.06375\\
3.25	-0.06184\\
3.26	-0.05988\\
3.27	-0.05785\\
3.28	-0.05576\\
3.29	-0.05362\\
3.3	-0.05143\\
3.31	-0.04919\\
3.32	-0.0469\\
3.33	-0.04456\\
3.34	-0.04218\\
3.35	-0.03976\\
3.36	-0.0373\\
3.37	-0.03481\\
3.38	-0.03228\\
3.39	-0.02971\\
3.4	-0.02712\\
3.41	-0.0245\\
3.42	-0.02186\\
3.43	-0.0192\\
3.44	-0.01651\\
3.45	-0.01381\\
3.46	-0.0111\\
3.47	-0.008378\\
3.48	-0.005646\\
3.49	-0.002909\\
3.5	-0.000169\\
3.51	0.002571\\
3.52	0.005309\\
3.53	0.008041\\
3.54	0.01077\\
3.55	0.01348\\
3.56	0.01618\\
3.57	0.01887\\
3.58	0.02153\\
3.59	0.02418\\
3.6	0.0268\\
3.61	0.02939\\
3.62	0.03196\\
3.63	0.0345\\
3.64	0.037\\
3.65	0.03946\\
3.66	0.04189\\
3.67	0.04427\\
3.68	0.04662\\
3.69	0.04891\\
3.7	0.05116\\
3.71	0.05336\\
3.72	0.0555\\
3.73	0.05759\\
3.74	0.05963\\
3.75	0.0616\\
3.76	0.06352\\
3.77	0.06537\\
3.78	0.06716\\
3.79	0.06889\\
3.8	0.07054\\
3.81	0.07213\\
3.82	0.07364\\
3.83	0.07509\\
3.84	0.07646\\
3.85	0.07775\\
3.86	0.07897\\
3.87	0.08011\\
3.88	0.08118\\
3.89	0.08216\\
3.9	0.08306\\
3.91	0.08388\\
3.92	0.08462\\
3.93	0.08528\\
3.94	0.08585\\
3.95	0.08634\\
3.96	0.08674\\
3.97	0.08706\\
3.98	0.08729\\
3.99	0.08744\\
4	0.0875\\
4.01	0.08747\\
4.02	0.08736\\
4.03	0.08717\\
4.04	0.08689\\
4.05	0.08652\\
4.06	0.08607\\
4.07	0.08553\\
4.08	0.08491\\
4.09	0.08421\\
4.1	0.08342\\
4.11	0.08255\\
4.12	0.0816\\
4.13	0.08058\\
4.14	0.07947\\
4.15	0.07828\\
4.16	0.07702\\
4.17	0.07568\\
4.18	0.07427\\
4.19	0.07278\\
4.2	0.07123\\
4.21	0.0696\\
4.22	0.06791\\
4.23	0.06615\\
4.24	0.06432\\
4.25	0.06243\\
4.26	0.06048\\
4.27	0.05847\\
4.28	0.0564\\
4.29	0.05428\\
4.3	0.0521\\
4.31	0.04988\\
4.32	0.0476\\
4.33	0.04528\\
4.34	0.04291\\
4.35	0.0405\\
4.36	0.03805\\
4.37	0.03557\\
4.38	0.03305\\
4.39	0.03049\\
4.4	0.02791\\
4.41	0.0253\\
4.42	0.02266\\
4.43	0.02001\\
4.44	0.01733\\
4.45	0.01463\\
4.46	0.01193\\
4.47	0.009205\\
4.48	0.006476\\
4.49	0.00374\\
4.5	0.001\\
4.51	-0.00174\\
4.52	-0.004479\\
4.53	-0.007213\\
4.54	-0.00994\\
4.55	-0.01266\\
4.56	-0.01536\\
4.57	-0.01805\\
4.58	-0.02073\\
4.59	-0.02338\\
4.6	-0.02601\\
4.61	-0.02861\\
4.62	-0.03119\\
4.63	-0.03373\\
4.64	-0.03624\\
4.65	-0.03872\\
4.66	-0.04116\\
4.67	-0.04355\\
4.68	-0.04591\\
4.69	-0.04822\\
4.7	-0.05048\\
4.71	-0.0527\\
4.72	-0.05486\\
4.73	-0.05697\\
4.74	-0.05902\\
4.75	-0.06101\\
4.76	-0.06295\\
4.77	-0.06482\\
4.78	-0.06663\\
4.79	-0.06837\\
4.8	-0.07005\\
4.81	-0.07165\\
4.82	-0.07319\\
4.83	-0.07466\\
4.84	-0.07605\\
4.85	-0.07737\\
4.86	-0.07861\\
4.87	-0.07978\\
4.88	-0.08086\\
4.89	-0.08187\\
4.9	-0.0828\\
4.91	-0.08364\\
4.92	-0.08441\\
4.93	-0.08509\\
4.94	-0.08568\\
4.95	-0.0862\\
4.96	-0.08663\\
4.97	-0.08697\\
4.98	-0.08723\\
4.99	-0.0874\\
5	-0.08749\\
5.01	-0.08749\\
5.02	-0.08741\\
5.03	-0.08724\\
5.04	-0.08698\\
5.05	-0.08664\\
5.06	-0.08621\\
5.07	-0.0857\\
5.08	-0.08511\\
5.09	-0.08443\\
5.1	-0.08367\\
5.11	-0.08283\\
5.12	-0.0819\\
5.13	-0.0809\\
5.14	-0.07981\\
5.15	-0.07865\\
5.16	-0.07741\\
5.17	-0.0761\\
5.18	-0.07471\\
5.19	-0.07324\\
5.2	-0.07171\\
5.21	-0.0701\\
5.22	-0.06843\\
5.23	-0.06669\\
5.24	-0.06488\\
5.25	-0.06301\\
5.26	-0.06108\\
5.27	-0.05909\\
5.28	-0.05704\\
5.29	-0.05493\\
5.3	-0.05277\\
5.31	-0.05056\\
5.32	-0.0483\\
5.33	-0.04599\\
5.34	-0.04363\\
5.35	-0.04124\\
5.36	-0.0388\\
5.37	-0.03633\\
5.38	-0.03381\\
5.39	-0.03127\\
5.4	-0.0287\\
5.41	-0.02609\\
5.42	-0.02347\\
5.43	-0.02081\\
5.44	-0.01814\\
5.45	-0.01545\\
5.46	-0.01275\\
5.47	-0.01003\\
5.48	-0.007304\\
5.49	-0.00457\\
5.5	-0.001831\\
5.51	0.0009089\\
5.52	0.003648\\
5.53	0.006384\\
5.54	0.009114\\
5.55	0.01183\\
5.56	0.01454\\
5.57	0.01724\\
5.58	0.01992\\
5.59	0.02258\\
5.6	0.02521\\
5.61	0.02782\\
5.62	0.03041\\
5.63	0.03296\\
5.64	0.03548\\
5.65	0.03797\\
5.66	0.04042\\
5.67	0.04283\\
5.68	0.0452\\
5.69	0.04752\\
5.7	0.0498\\
5.71	0.05203\\
5.72	0.05421\\
5.73	0.05633\\
5.74	0.0584\\
5.75	0.06041\\
5.76	0.06237\\
5.77	0.06426\\
5.78	0.06609\\
5.79	0.06785\\
5.8	0.06955\\
5.81	0.07117\\
5.82	0.07273\\
5.83	0.07422\\
5.84	0.07564\\
5.85	0.07698\\
5.86	0.07824\\
5.87	0.07943\\
5.88	0.08054\\
5.89	0.08157\\
5.9	0.08252\\
5.91	0.08339\\
5.92	0.08418\\
5.93	0.08489\\
5.94	0.08551\\
5.95	0.08605\\
5.96	0.0865\\
5.97	0.08687\\
5.98	0.08716\\
5.99	0.08736\\
6	0.08747\\
6.01	0.0875\\
6.02	0.08744\\
6.03	0.0873\\
6.04	0.08707\\
6.05	0.08675\\
6.06	0.08635\\
6.07	0.08587\\
6.08	0.0853\\
6.09	0.08464\\
6.1	0.08391\\
6.11	0.08309\\
6.12	0.08219\\
6.13	0.08121\\
6.14	0.08015\\
6.15	0.07901\\
6.16	0.0778\\
6.17	0.0765\\
6.18	0.07514\\
6.19	0.07369\\
6.2	0.07218\\
6.21	0.0706\\
6.22	0.06894\\
6.23	0.06722\\
6.24	0.06543\\
6.25	0.06358\\
6.26	0.06167\\
6.27	0.0597\\
6.28	0.05766\\
6.29	0.05557\\
6.3	0.05343\\
6.31	0.05123\\
6.32	0.04899\\
6.33	0.04669\\
6.34	0.04435\\
6.35	0.04197\\
6.36	0.03954\\
6.37	0.03708\\
6.38	0.03458\\
6.39	0.03205\\
6.4	0.02948\\
6.41	0.02689\\
6.42	0.02427\\
6.43	0.02162\\
6.44	0.01895\\
6.45	0.01627\\
6.46	0.01357\\
6.47	0.01086\\
6.48	0.008132\\
6.49	0.0054\\
6.5	0.002663\\
6.51	-7.764e-05\\
6.52	-0.002818\\
6.53	-0.005555\\
6.54	-0.008287\\
6.55	-0.01101\\
6.56	-0.01372\\
6.57	-0.01642\\
6.58	-0.01911\\
6.59	-0.02177\\
6.6	-0.02441\\
6.61	-0.02703\\
6.62	-0.02963\\
6.63	-0.03219\\
6.64	-0.03472\\
6.65	-0.03722\\
6.66	-0.03968\\
6.67	-0.0421\\
6.68	-0.04449\\
6.69	-0.04682\\
6.7	-0.04912\\
6.71	-0.05136\\
6.72	-0.05355\\
6.73	-0.05569\\
6.74	-0.05778\\
6.75	-0.05981\\
6.76	-0.06178\\
6.77	-0.06369\\
6.78	-0.06554\\
6.79	-0.06732\\
6.8	-0.06904\\
6.81	-0.07069\\
6.82	-0.07227\\
6.83	-0.07378\\
6.84	-0.07521\\
6.85	-0.07658\\
6.86	-0.07787\\
6.87	-0.07908\\
6.88	-0.08021\\
6.89	-0.08127\\
6.9	-0.08224\\
6.91	-0.08314\\
6.92	-0.08395\\
6.93	-0.08468\\
6.94	-0.08533\\
6.95	-0.0859\\
6.96	-0.08638\\
6.97	-0.08677\\
6.98	-0.08708\\
6.99	-0.08731\\
7	-0.08745\\
7.01	-0.0875\\
7.02	-0.08747\\
7.03	-0.08735\\
7.04	-0.08715\\
7.05	-0.08686\\
7.06	-0.08648\\
7.07	-0.08602\\
7.08	-0.08548\\
7.09	-0.08485\\
7.1	-0.08414\\
7.11	-0.08335\\
7.12	-0.08247\\
7.13	-0.08152\\
7.14	-0.08048\\
7.15	-0.07936\\
7.16	-0.07817\\
7.17	-0.0769\\
7.18	-0.07556\\
7.19	-0.07414\\
7.2	-0.07265\\
7.21	-0.07108\\
7.22	-0.06945\\
7.23	-0.06775\\
7.24	-0.06598\\
7.25	-0.06415\\
7.26	-0.06226\\
7.27	-0.0603\\
7.28	-0.05829\\
7.29	-0.05621\\
7.3	-0.05409\\
7.31	-0.05191\\
7.32	-0.04967\\
7.33	-0.04739\\
7.34	-0.04507\\
7.35	-0.0427\\
7.36	-0.04028\\
7.37	-0.03783\\
7.38	-0.03534\\
7.39	-0.03282\\
7.4	-0.03026\\
7.41	-0.02768\\
7.42	-0.02506\\
7.43	-0.02243\\
7.44	-0.01977\\
7.45	-0.01709\\
7.46	-0.01439\\
7.47	-0.01168\\
7.48	-0.00896\\
7.49	-0.00623\\
7.5	-0.003493\\
7.51	-0.0007537\\
7.52	0.001987\\
7.53	0.004725\\
7.54	0.007459\\
7.55	0.01019\\
7.56	0.0129\\
7.57	0.01561\\
7.58	0.01829\\
7.59	0.02097\\
7.6	0.02362\\
7.61	0.02624\\
7.62	0.02884\\
7.63	0.03142\\
7.64	0.03396\\
7.65	0.03647\\
7.66	0.03894\\
7.67	0.04137\\
7.68	0.04377\\
7.69	0.04612\\
7.7	0.04843\\
7.71	0.05068\\
7.72	0.05289\\
7.73	0.05505\\
7.74	0.05715\\
7.75	0.0592\\
7.76	0.06119\\
7.77	0.06312\\
7.78	0.06498\\
7.79	0.06679\\
7.8	0.06852\\
7.81	0.07019\\
7.82	0.0718\\
7.83	0.07333\\
7.84	0.07479\\
7.85	0.07617\\
7.86	0.07748\\
7.87	0.07872\\
7.88	0.07988\\
7.89	0.08096\\
7.9	0.08196\\
7.91	0.08288\\
7.92	0.08371\\
7.93	0.08447\\
7.94	0.08514\\
7.95	0.08573\\
7.96	0.08624\\
7.97	0.08666\\
7.98	0.087\\
7.99	0.08725\\
8	0.08741\\
8.01	0.08749\\
8.02	0.08749\\
8.03	0.08739\\
8.04	0.08722\\
8.05	0.08695\\
8.06	0.0866\\
8.07	0.08617\\
8.08	0.08565\\
8.09	0.08505\\
8.1	0.08436\\
8.11	0.0836\\
8.12	0.08275\\
8.13	0.08181\\
8.14	0.0808\\
8.15	0.07971\\
8.16	0.07854\\
8.17	0.0773\\
8.18	0.07597\\
8.19	0.07458\\
8.2	0.07311\\
8.21	0.07157\\
8.22	0.06995\\
8.23	0.06827\\
8.24	0.06653\\
8.25	0.06471\\
8.26	0.06284\\
8.27	0.0609\\
8.28	0.0589\\
8.29	0.05685\\
8.3	0.05474\\
8.31	0.05257\\
8.32	0.05036\\
8.33	0.04809\\
8.34	0.04578\\
8.35	0.04342\\
8.36	0.04102\\
8.37	0.03858\\
8.38	0.0361\\
8.39	0.03359\\
8.4	0.03104\\
8.41	0.02846\\
8.42	0.02586\\
8.43	0.02323\\
8.44	0.02057\\
8.45	0.0179\\
8.46	0.01521\\
8.47	0.0125\\
8.48	0.009786\\
8.49	0.007058\\
8.5	0.004324\\
8.51	0.001585\\
8.52	-0.001156\\
8.53	-0.003895\\
8.54	-0.00663\\
8.55	-0.009359\\
8.56	-0.01208\\
8.57	-0.01479\\
8.58	-0.01748\\
8.59	-0.02016\\
8.6	-0.02281\\
8.61	-0.02545\\
8.62	-0.02806\\
8.63	-0.03064\\
8.64	-0.03319\\
8.65	-0.03571\\
8.66	-0.03819\\
8.67	-0.04064\\
8.68	-0.04305\\
8.69	-0.04541\\
8.7	-0.04773\\
8.71	-0.05\\
8.72	-0.05223\\
8.73	-0.0544\\
8.74	-0.05652\\
8.75	-0.05858\\
8.76	-0.06059\\
8.77	-0.06254\\
8.78	-0.06442\\
8.79	-0.06625\\
8.8	-0.068\\
8.81	-0.0697\\
8.82	-0.07132\\
8.83	-0.07287\\
8.84	-0.07435\\
8.85	-0.07576\\
8.86	-0.07709\\
8.87	-0.07835\\
8.88	-0.07953\\
8.89	-0.08064\\
8.9	-0.08166\\
8.91	-0.0826\\
8.92	-0.08347\\
8.93	-0.08425\\
8.94	-0.08495\\
8.95	-0.08556\\
8.96	-0.08609\\
8.97	-0.08654\\
8.98	-0.0869\\
8.99	-0.08718\\
9	-0.08737\\
9.01	-0.08748\\
9.02	-0.0875\\
9.03	-0.08743\\
9.04	-0.08728\\
9.05	-0.08704\\
9.06	-0.08672\\
9.07	-0.08631\\
9.08	-0.08582\\
9.09	-0.08524\\
9.1	-0.08458\\
9.11	-0.08384\\
9.12	-0.08301\\
9.13	-0.08211\\
9.14	-0.08112\\
9.15	-0.08005\\
9.16	-0.0789\\
9.17	-0.07768\\
9.18	-0.07638\\
9.19	-0.07501\\
9.2	-0.07356\\
9.21	-0.07204\\
9.22	-0.07045\\
9.23	-0.06879\\
9.24	-0.06706\\
9.25	-0.06527\\
9.26	-0.06341\\
9.27	-0.06149\\
9.28	-0.05951\\
9.29	-0.05748\\
9.3	-0.05538\\
9.31	-0.05323\\
9.32	-0.05103\\
9.33	-0.04878\\
9.34	-0.04648\\
9.35	-0.04414\\
9.36	-0.04175\\
9.37	-0.03932\\
9.38	-0.03686\\
9.39	-0.03435\\
9.4	-0.03182\\
9.41	-0.02925\\
9.42	-0.02665\\
9.43	-0.02403\\
9.44	-0.02138\\
9.45	-0.01871\\
9.46	-0.01603\\
9.47	-0.01333\\
9.48	-0.01061\\
9.49	-0.007887\\
9.5	-0.005154\\
9.51	-0.002416\\
9.52	0.0003243\\
9.53	0.003064\\
9.54	0.005801\\
9.55	0.008532\\
9.56	0.01126\\
9.57	0.01397\\
9.58	0.01667\\
9.59	0.01935\\
9.6	0.02201\\
9.61	0.02465\\
9.62	0.02727\\
9.63	0.02986\\
9.64	0.03242\\
9.65	0.03495\\
9.66	0.03744\\
9.67	0.0399\\
9.68	0.04232\\
9.69	0.0447\\
9.7	0.04703\\
9.71	0.04932\\
9.72	0.05156\\
9.73	0.05375\\
9.74	0.05588\\
9.75	0.05796\\
9.76	0.05999\\
9.77	0.06195\\
9.78	0.06386\\
9.79	0.0657\\
9.8	0.06748\\
9.81	0.06919\\
9.82	0.07083\\
9.83	0.07241\\
9.84	0.07391\\
9.85	0.07534\\
9.86	0.0767\\
9.87	0.07798\\
9.88	0.07918\\
9.89	0.08031\\
9.9	0.08136\\
9.91	0.08233\\
9.92	0.08321\\
9.93	0.08402\\
9.94	0.08474\\
9.95	0.08539\\
9.96	0.08594\\
9.97	0.08641\\
9.98	0.0868\\
9.99	0.08711\\
10	0.08732\\
10.01	0.08745\\
10.02	0.0875\\
10.03	0.08746\\
10.04	0.08733\\
10.05	0.08712\\
10.06	0.08683\\
10.07	0.08644\\
10.08	0.08598\\
10.09	0.08543\\
10.1	0.08479\\
10.11	0.08407\\
10.12	0.08327\\
10.13	0.08239\\
10.14	0.08143\\
10.15	0.08038\\
10.16	0.07926\\
10.17	0.07806\\
10.18	0.07678\\
10.19	0.07543\\
10.2	0.07401\\
10.21	0.07251\\
10.22	0.07094\\
10.23	0.0693\\
10.24	0.06759\\
10.25	0.06582\\
10.26	0.06398\\
10.27	0.06208\\
10.28	0.06012\\
10.29	0.0581\\
10.3	0.05602\\
10.31	0.05389\\
10.32	0.05171\\
10.33	0.04947\\
10.34	0.04719\\
10.35	0.04486\\
10.36	0.04248\\
10.37	0.04006\\
10.38	0.03761\\
10.39	0.03512\\
10.4	0.03259\\
10.41	0.03003\\
10.42	0.02744\\
10.43	0.02483\\
10.44	0.02219\\
10.45	0.01953\\
10.46	0.01684\\
10.47	0.01415\\
10.48	0.01144\\
10.49	0.008714\\
10.5	0.005984\\
10.51	0.003247\\
10.52	0.000507\\
10.53	-0.002233\\
10.54	-0.004971\\
10.55	-0.007705\\
10.56	-0.01043\\
10.57	-0.01315\\
10.58	-0.01585\\
10.59	-0.01854\\
10.6	-0.0212\\
10.61	-0.02385\\
10.62	-0.02648\\
10.63	-0.02908\\
10.64	-0.03165\\
10.65	-0.03419\\
10.66	-0.03669\\
10.67	-0.03916\\
10.68	-0.04159\\
10.69	-0.04398\\
10.7	-0.04633\\
10.71	-0.04863\\
10.72	-0.05088\\
10.73	-0.05309\\
10.74	-0.05524\\
10.75	-0.05734\\
10.76	-0.05938\\
10.77	-0.06136\\
10.78	-0.06329\\
10.79	-0.06515\\
10.8	-0.06695\\
10.81	-0.06868\\
10.82	-0.07034\\
10.83	-0.07194\\
10.84	-0.07346\\
10.85	-0.07491\\
10.86	-0.07629\\
10.87	-0.0776\\
10.88	-0.07883\\
10.89	-0.07998\\
10.9	-0.08105\\
10.91	-0.08204\\
10.92	-0.08295\\
10.93	-0.08379\\
10.94	-0.08453\\
10.95	-0.0852\\
10.96	-0.08578\\
10.97	-0.08628\\
10.98	-0.08669\\
10.99	-0.08702\\
11	-0.08727\\
11.01	-0.08742\\
11.02	-0.0875\\
11.03	-0.08748\\
11.04	-0.08738\\
11.05	-0.0872\\
11.06	-0.08692\\
11.07	-0.08657\\
11.08	-0.08613\\
11.09	-0.0856\\
11.1	-0.08499\\
11.11	-0.0843\\
11.12	-0.08352\\
11.13	-0.08267\\
11.14	-0.08173\\
11.15	-0.08071\\
11.16	-0.07961\\
11.17	-0.07843\\
11.18	-0.07718\\
11.19	-0.07585\\
11.2	-0.07445\\
11.21	-0.07297\\
11.22	-0.07142\\
11.23	-0.06981\\
11.24	-0.06812\\
11.25	-0.06637\\
11.26	-0.06455\\
11.27	-0.06267\\
11.28	-0.06072\\
11.29	-0.05872\\
11.3	-0.05666\\
11.31	-0.05454\\
11.32	-0.05237\\
11.33	-0.05015\\
11.34	-0.04788\\
11.35	-0.04557\\
11.36	-0.04321\\
11.37	-0.0408\\
11.38	-0.03836\\
11.39	-0.03588\\
11.4	-0.03336\\
11.41	-0.03081\\
11.42	-0.02823\\
11.43	-0.02562\\
11.44	-0.02299\\
11.45	-0.02033\\
11.46	-0.01766\\
11.47	-0.01497\\
11.48	-0.01226\\
11.49	-0.009541\\
11.5	-0.006813\\
11.51	-0.004077\\
11.52	-0.001338\\
11.53	0.001402\\
11.54	0.004141\\
11.55	0.006876\\
11.56	0.009605\\
11.57	0.01232\\
11.58	0.01503\\
11.59	0.01772\\
11.6	0.0204\\
11.61	0.02305\\
11.62	0.02568\\
11.63	0.02829\\
11.64	0.03087\\
11.65	0.03342\\
11.66	0.03593\\
11.67	0.03842\\
11.68	0.04086\\
11.69	0.04326\\
11.7	0.04562\\
11.71	0.04794\\
11.72	0.05021\\
11.73	0.05243\\
11.74	0.05459\\
11.75	0.05671\\
11.76	0.05877\\
11.77	0.06077\\
11.78	0.06271\\
11.79	0.06459\\
11.8	0.06641\\
11.81	0.06816\\
11.82	0.06984\\
11.83	0.07146\\
11.84	0.07301\\
11.85	0.07448\\
11.86	0.07588\\
11.87	0.07721\\
11.88	0.07846\\
11.89	0.07964\\
11.9	0.08073\\
11.91	0.08175\\
11.92	0.08269\\
11.93	0.08354\\
11.94	0.08432\\
11.95	0.08501\\
11.96	0.08561\\
11.97	0.08614\\
11.98	0.08658\\
11.99	0.08693\\
12	0.0872\\
12.01	0.08738\\
12.02	0.08748\\
12.03	0.08749\\
12.04	0.08742\\
12.05	0.08726\\
12.06	0.08702\\
12.07	0.08669\\
12.08	0.08627\\
12.09	0.08577\\
12.1	0.08519\\
12.11	0.08452\\
12.12	0.08377\\
12.13	0.08293\\
12.14	0.08202\\
12.15	0.08102\\
12.16	0.07995\\
12.17	0.0788\\
12.18	0.07757\\
12.19	0.07626\\
12.2	0.07488\\
12.21	0.07343\\
12.22	0.0719\\
12.23	0.0703\\
12.24	0.06864\\
12.25	0.0669\\
12.26	0.06511\\
12.27	0.06324\\
12.28	0.06132\\
12.29	0.05933\\
12.3	0.05729\\
12.31	0.05519\\
12.32	0.05304\\
12.33	0.05083\\
12.34	0.04858\\
12.35	0.04627\\
12.36	0.04393\\
12.37	0.04154\\
12.38	0.0391\\
12.39	0.03663\\
12.4	0.03413\\
12.41	0.03159\\
12.42	0.02902\\
12.43	0.02642\\
12.44	0.02379\\
12.45	0.02114\\
12.46	0.01847\\
12.47	0.01579\\
12.48	0.01308\\
12.49	0.01037\\
12.5	0.007641\\
12.51	0.004908\\
12.52	0.002169\\
12.53	-0.0005709\\
12.54	-0.003311\\
12.55	-0.006047\\
12.56	-0.008778\\
12.57	-0.0115\\
12.58	-0.01421\\
12.59	-0.01691\\
12.6	-0.01959\\
12.61	-0.02225\\
12.62	-0.02489\\
12.63	-0.0275\\
12.64	-0.03009\\
12.65	-0.03265\\
12.66	-0.03517\\
12.67	-0.03767\\
12.68	-0.04012\\
12.69	-0.04254\\
12.7	-0.04491\\
12.71	-0.04724\\
12.72	-0.04952\\
12.73	-0.05176\\
12.74	-0.05394\\
12.75	-0.05607\\
12.76	-0.05815\\
12.77	-0.06017\\
12.78	-0.06213\\
12.79	-0.06403\\
12.8	-0.06586\\
12.81	-0.06763\\
12.82	-0.06934\\
12.83	-0.07098\\
12.84	-0.07255\\
12.85	-0.07404\\
12.86	-0.07547\\
12.87	-0.07682\\
12.88	-0.07809\\
12.89	-0.07929\\
12.9	-0.08041\\
12.91	-0.08145\\
12.92	-0.08241\\
12.93	-0.08329\\
12.94	-0.08409\\
12.95	-0.08481\\
12.96	-0.08544\\
12.97	-0.08599\\
12.98	-0.08645\\
12.99	-0.08683\\
13	-0.08713\\
13.01	-0.08734\\
13.02	-0.08746\\
13.03	-0.0875\\
13.04	-0.08745\\
13.05	-0.08732\\
13.06	-0.0871\\
13.07	-0.08679\\
13.08	-0.0864\\
13.09	-0.08593\\
13.1	-0.08537\\
13.11	-0.08473\\
13.12	-0.084\\
13.13	-0.0832\\
13.14	-0.08231\\
13.15	-0.08133\\
13.16	-0.08028\\
13.17	-0.07916\\
13.18	-0.07795\\
13.19	-0.07667\\
13.2	-0.07531\\
13.21	-0.07388\\
13.22	-0.07237\\
13.23	-0.0708\\
13.24	-0.06915\\
13.25	-0.06744\\
13.26	-0.06566\\
13.27	-0.06382\\
13.28	-0.06191\\
13.29	-0.05994\\
13.3	-0.05792\\
13.31	-0.05583\\
13.32	-0.0537\\
13.33	-0.05151\\
13.34	-0.04927\\
13.35	-0.04698\\
13.36	-0.04464\\
13.37	-0.04227\\
13.38	-0.03985\\
13.39	-0.03739\\
13.4	-0.03489\\
13.41	-0.03236\\
13.42	-0.0298\\
13.43	-0.02721\\
13.44	-0.02459\\
13.45	-0.02195\\
13.46	-0.01928\\
13.47	-0.0166\\
13.48	-0.0139\\
13.49	-0.01119\\
13.5	-0.008469\\
13.51	-0.005737\\
13.52	-0.003\\
13.53	-0.0002604\\
13.54	0.00248\\
13.55	0.005218\\
13.56	0.00795\\
13.57	0.01068\\
13.58	0.01339\\
13.59	0.01609\\
13.6	0.01878\\
13.61	0.02144\\
13.62	0.02409\\
13.63	0.02671\\
13.64	0.02931\\
13.65	0.03188\\
13.66	0.03441\\
13.67	0.03691\\
13.68	0.03938\\
13.69	0.04181\\
13.7	0.04419\\
13.71	0.04654\\
13.72	0.04884\\
13.73	0.05109\\
13.74	0.05329\\
13.75	0.05543\\
13.76	0.05753\\
13.77	0.05956\\
13.78	0.06154\\
13.79	0.06346\\
13.8	0.06531\\
13.81	0.0671\\
13.82	0.06883\\
13.83	0.07049\\
13.84	0.07208\\
13.85	0.0736\\
13.86	0.07504\\
13.87	0.07641\\
13.88	0.07771\\
13.89	0.07893\\
13.9	0.08008\\
13.91	0.08114\\
13.92	0.08213\\
13.93	0.08303\\
13.94	0.08386\\
13.95	0.0846\\
13.96	0.08526\\
13.97	0.08583\\
13.98	0.08632\\
13.99	0.08673\\
14	0.08705\\
14.01	0.08728\\
14.02	0.08743\\
14.03	0.0875\\
14.04	0.08748\\
14.05	0.08737\\
14.06	0.08717\\
14.07	0.0869\\
14.08	0.08653\\
14.09	0.08608\\
14.1	0.08555\\
14.11	0.08493\\
14.12	0.08423\\
14.13	0.08345\\
14.14	0.08258\\
14.15	0.08164\\
14.16	0.08061\\
14.17	0.07951\\
14.18	0.07832\\
14.19	0.07706\\
14.2	0.07573\\
14.21	0.07432\\
14.22	0.07284\\
14.23	0.07128\\
14.24	0.06966\\
14.25	0.06796\\
14.26	0.06621\\
14.27	0.06438\\
14.28	0.06249\\
14.29	0.06055\\
14.3	0.05854\\
14.31	0.05647\\
14.32	0.05435\\
14.33	0.05218\\
14.34	0.04995\\
14.35	0.04768\\
14.36	0.04536\\
14.37	0.04299\\
14.38	0.04058\\
14.39	0.03814\\
14.4	0.03565\\
14.41	0.03313\\
14.42	0.03058\\
14.43	0.028\\
14.44	0.02539\\
14.45	0.02275\\
14.46	0.02009\\
14.47	0.01742\\
14.48	0.01472\\
14.49	0.01202\\
14.5	0.009296\\
14.51	0.006567\\
14.52	0.003831\\
14.53	0.001092\\
14.54	-0.001649\\
14.55	-0.004388\\
14.56	-0.007122\\
14.57	-0.00985\\
14.58	-0.01257\\
14.59	-0.01527\\
14.6	-0.01796\\
14.61	-0.02064\\
14.62	-0.02329\\
14.63	-0.02592\\
14.64	-0.02852\\
14.65	-0.0311\\
14.66	-0.03365\\
14.67	-0.03616\\
14.68	-0.03864\\
14.69	-0.04108\\
14.7	-0.04348\\
14.71	-0.04583\\
14.72	-0.04814\\
14.73	-0.05041\\
14.74	-0.05262\\
14.75	-0.05479\\
14.76	-0.0569\\
14.77	-0.05895\\
14.78	-0.06095\\
14.79	-0.06288\\
14.8	-0.06476\\
14.81	-0.06657\\
14.82	-0.06831\\
14.83	-0.06999\\
14.84	-0.0716\\
14.85	-0.07314\\
14.86	-0.07461\\
14.87	-0.07601\\
14.88	-0.07733\\
14.89	-0.07857\\
14.9	-0.07974\\
14.91	-0.08083\\
14.92	-0.08184\\
14.93	-0.08277\\
14.94	-0.08361\\
14.95	-0.08438\\
14.96	-0.08506\\
14.97	-0.08567\\
14.98	-0.08618\\
14.99	-0.08661\\
15	-0.08696\\
15.01	-0.08722\\
15.02	-0.0874\\
15.03	-0.08749\\
15.04	-0.08749\\
15.05	-0.08741\\
15.06	-0.08724\\
15.07	-0.08699\\
15.08	-0.08665\\
15.09	-0.08623\\
15.1	-0.08572\\
15.11	-0.08513\\
15.12	-0.08445\\
15.13	-0.0837\\
15.14	-0.08285\\
15.15	-0.08193\\
15.16	-0.08093\\
15.17	-0.07985\\
15.18	-0.07869\\
15.19	-0.07745\\
15.2	-0.07614\\
15.21	-0.07475\\
15.22	-0.07329\\
15.23	-0.07176\\
15.24	-0.07016\\
15.25	-0.06848\\
15.26	-0.06675\\
15.27	-0.06494\\
15.28	-0.06307\\
15.29	-0.06114\\
15.3	-0.05915\\
15.31	-0.0571\\
15.32	-0.055\\
15.33	-0.05284\\
15.34	-0.05063\\
15.35	-0.04837\\
15.36	-0.04607\\
15.37	-0.04371\\
15.38	-0.04132\\
15.39	-0.03888\\
15.4	-0.03641\\
15.41	-0.0339\\
15.42	-0.03136\\
15.43	-0.02878\\
15.44	-0.02618\\
15.45	-0.02355\\
15.46	-0.0209\\
15.47	-0.01823\\
15.48	-0.01554\\
15.49	-0.01284\\
15.5	-0.01012\\
15.51	-0.007395\\
15.52	-0.004661\\
15.53	-0.001923\\
15.54	0.0008176\\
15.55	0.003557\\
15.56	0.006293\\
15.57	0.009023\\
15.58	0.01174\\
15.59	0.01445\\
15.6	0.01715\\
15.61	0.01983\\
15.62	0.02249\\
15.63	0.02512\\
15.64	0.02774\\
15.65	0.03032\\
15.66	0.03288\\
15.67	0.0354\\
15.68	0.03789\\
15.69	0.04034\\
15.7	0.04275\\
15.71	0.04512\\
15.72	0.04745\\
15.73	0.04973\\
15.74	0.05196\\
15.75	0.05414\\
15.76	0.05626\\
15.77	0.05833\\
15.78	0.06035\\
15.79	0.0623\\
15.8	0.06419\\
15.81	0.06603\\
15.82	0.06779\\
15.83	0.06949\\
15.84	0.07112\\
15.85	0.07268\\
15.86	0.07417\\
15.87	0.07559\\
15.88	0.07693\\
15.89	0.0782\\
15.9	0.07939\\
15.91	0.0805\\
15.92	0.08154\\
15.93	0.08249\\
15.94	0.08337\\
15.95	0.08416\\
15.96	0.08487\\
15.97	0.08549\\
15.98	0.08603\\
15.99	0.08649\\
16	0.08686\\
16.01	0.08715\\
16.02	0.08735\\
16.03	0.08747\\
16.04	0.0875\\
16.05	0.08744\\
16.06	0.0873\\
16.07	0.08708\\
16.08	0.08676\\
16.09	0.08637\\
16.1	0.08588\\
16.11	0.08532\\
16.12	0.08467\\
16.13	0.08393\\
16.14	0.08312\\
16.15	0.08222\\
16.16	0.08124\\
16.17	0.08019\\
16.18	0.07905\\
16.19	0.07784\\
16.2	0.07655\\
16.21	0.07518\\
16.22	0.07374\\
16.23	0.07223\\
16.24	0.07065\\
16.25	0.069\\
16.26	0.06728\\
16.27	0.0655\\
16.28	0.06365\\
16.29	0.06173\\
16.3	0.05976\\
16.31	0.05773\\
16.32	0.05564\\
16.33	0.0535\\
16.34	0.05131\\
16.35	0.04906\\
16.36	0.04677\\
16.37	0.04443\\
16.38	0.04205\\
16.39	0.03963\\
16.4	0.03716\\
16.41	0.03466\\
16.42	0.03213\\
16.43	0.02957\\
16.44	0.02697\\
16.45	0.02435\\
16.46	0.02171\\
16.47	0.01904\\
16.48	0.01636\\
16.49	0.01366\\
16.5	0.01095\\
16.51	0.008223\\
16.52	0.005491\\
16.53	0.002754\\
16.54	1.373e-05\\
16.55	-0.002726\\
16.56	-0.005464\\
16.57	-0.008196\\
16.58	-0.01092\\
16.59	-0.01363\\
16.6	-0.01633\\
16.61	-0.01902\\
16.62	-0.02168\\
16.63	-0.02433\\
16.64	-0.02695\\
16.65	-0.02954\\
16.66	-0.03211\\
16.67	-0.03464\\
16.68	-0.03714\\
16.69	-0.0396\\
16.7	-0.04202\\
16.71	-0.04441\\
16.72	-0.04675\\
16.73	-0.04904\\
16.74	-0.05129\\
16.75	-0.05348\\
16.76	-0.05562\\
16.77	-0.05771\\
16.78	-0.05974\\
16.79	-0.06171\\
16.8	-0.06363\\
16.81	-0.06548\\
16.82	-0.06726\\
16.83	-0.06898\\
16.84	-0.07063\\
16.85	-0.07222\\
16.86	-0.07373\\
16.87	-0.07517\\
16.88	-0.07653\\
16.89	-0.07782\\
16.9	-0.07904\\
16.91	-0.08018\\
16.92	-0.08123\\
16.93	-0.08221\\
16.94	-0.08311\\
16.95	-0.08393\\
16.96	-0.08466\\
16.97	-0.08531\\
16.98	-0.08588\\
16.99	-0.08636\\
17	-0.08676\\
17.01	-0.08707\\
17.02	-0.0873\\
17.03	-0.08744\\
17.04	-0.0875\\
17.05	-0.08747\\
17.06	-0.08735\\
17.07	-0.08715\\
17.08	-0.08687\\
17.09	-0.0865\\
17.1	-0.08604\\
17.11	-0.0855\\
17.12	-0.08487\\
17.13	-0.08416\\
17.14	-0.08337\\
17.15	-0.0825\\
17.16	-0.08155\\
17.17	-0.08052\\
17.18	-0.0794\\
17.19	-0.07821\\
17.2	-0.07695\\
17.21	-0.0756\\
17.22	-0.07419\\
17.23	-0.0727\\
17.24	-0.07114\\
17.25	-0.06951\\
17.26	-0.06781\\
17.27	-0.06604\\
17.28	-0.06421\\
17.29	-0.06232\\
17.3	-0.06037\\
17.31	-0.05835\\
17.32	-0.05628\\
17.33	-0.05416\\
17.34	-0.05198\\
17.35	-0.04975\\
17.36	-0.04747\\
17.37	-0.04515\\
17.38	-0.04278\\
17.39	-0.04036\\
17.4	-0.03791\\
17.41	-0.03543\\
17.42	-0.0329\\
17.43	-0.03035\\
17.44	-0.02776\\
17.45	-0.02515\\
17.46	-0.02251\\
17.47	-0.01985\\
17.48	-0.01718\\
17.49	-0.01448\\
17.5	-0.01177\\
17.51	-0.009051\\
17.52	-0.006321\\
17.53	-0.003585\\
17.54	-0.000845\\
17.55	0.001895\\
17.56	0.004634\\
17.57	0.007368\\
17.58	0.01009\\
17.59	0.01281\\
17.6	0.01552\\
17.61	0.01821\\
17.62	0.02088\\
17.63	0.02353\\
17.64	0.02615\\
17.65	0.02876\\
17.66	0.03133\\
17.67	0.03387\\
17.68	0.03638\\
17.69	0.03886\\
17.7	0.04129\\
17.71	0.04369\\
17.72	0.04604\\
17.73	0.04835\\
17.74	0.05061\\
17.75	0.05282\\
17.76	0.05498\\
17.77	0.05708\\
17.78	0.05913\\
17.79	0.06112\\
17.8	0.06305\\
17.81	0.06492\\
17.82	0.06673\\
17.83	0.06847\\
17.84	0.07014\\
17.85	0.07174\\
17.86	0.07328\\
17.87	0.07474\\
17.88	0.07613\\
17.89	0.07744\\
17.9	0.07868\\
17.91	0.07984\\
17.92	0.08092\\
17.93	0.08192\\
17.94	0.08285\\
17.95	0.08369\\
17.96	0.08445\\
17.97	0.08512\\
17.98	0.08572\\
17.99	0.08622\\
18	0.08665\\
18.01	0.08699\\
18.02	0.08724\\
18.03	0.08741\\
18.04	0.08749\\
18.05	0.08749\\
18.06	0.0874\\
18.07	0.08722\\
18.08	0.08696\\
18.09	0.08662\\
18.1	0.08619\\
18.11	0.08567\\
18.12	0.08507\\
18.13	0.08439\\
18.14	0.08362\\
18.15	0.08278\\
18.16	0.08185\\
18.17	0.08084\\
18.18	0.07975\\
18.19	0.07858\\
18.2	0.07734\\
18.21	0.07602\\
18.22	0.07462\\
18.23	0.07316\\
18.24	0.07162\\
18.25	0.07001\\
18.26	0.06833\\
18.27	0.06659\\
18.28	0.06478\\
18.29	0.0629\\
18.3	0.06097\\
18.31	0.05897\\
18.32	0.05692\\
18.33	0.05481\\
18.34	0.05265\\
18.35	0.05043\\
18.36	0.04817\\
18.37	0.04586\\
18.38	0.0435\\
18.39	0.0411\\
18.4	0.03866\\
18.41	0.03618\\
18.42	0.03367\\
18.43	0.03113\\
18.44	0.02855\\
18.45	0.02595\\
18.46	0.02332\\
18.47	0.02066\\
18.48	0.01799\\
18.49	0.0153\\
18.5	0.01259\\
18.51	0.009877\\
18.52	0.00715\\
18.53	0.004415\\
18.54	0.001676\\
18.55	-0.001064\\
18.56	-0.003804\\
18.57	-0.006539\\
18.58	-0.009269\\
18.59	-0.01199\\
18.6	-0.0147\\
18.61	-0.01739\\
18.62	-0.02007\\
18.63	-0.02273\\
18.64	-0.02536\\
18.65	-0.02797\\
18.66	-0.03055\\
18.67	-0.03311\\
18.68	-0.03563\\
18.69	-0.03811\\
18.7	-0.04056\\
18.71	-0.04297\\
18.72	-0.04533\\
18.73	-0.04765\\
18.74	-0.04993\\
18.75	-0.05215\\
18.76	-0.05433\\
18.77	-0.05645\\
18.78	-0.05852\\
18.79	-0.06053\\
18.8	-0.06247\\
18.81	-0.06436\\
18.82	-0.06619\\
18.83	-0.06795\\
18.84	-0.06964\\
18.85	-0.07126\\
18.86	-0.07282\\
18.87	-0.0743\\
18.88	-0.07571\\
18.89	-0.07705\\
18.9	-0.07831\\
18.91	-0.0795\\
18.92	-0.0806\\
18.93	-0.08163\\
18.94	-0.08257\\
18.95	-0.08344\\
18.96	-0.08422\\
18.97	-0.08493\\
18.98	-0.08554\\
18.99	-0.08608\\
19	-0.08653\\
19.01	-0.08689\\
19.02	-0.08717\\
19.03	-0.08737\\
19.04	-0.08747\\
19.05	-0.0875\\
19.06	-0.08743\\
19.07	-0.08729\\
19.08	-0.08705\\
19.09	-0.08673\\
19.1	-0.08633\\
19.11	-0.08584\\
19.12	-0.08526\\
19.13	-0.0846\\
19.14	-0.08386\\
19.15	-0.08304\\
19.16	-0.08214\\
19.17	-0.08115\\
19.18	-0.08009\\
19.19	-0.07894\\
19.2	-0.07772\\
19.21	-0.07643\\
19.22	-0.07506\\
19.23	-0.07361\\
19.24	-0.07209\\
19.25	-0.0705\\
19.26	-0.06885\\
19.27	-0.06712\\
19.28	-0.06533\\
19.29	-0.06348\\
19.3	-0.06156\\
19.31	-0.05958\\
19.32	-0.05755\\
19.33	-0.05545\\
19.34	-0.05331\\
19.35	-0.05111\\
19.36	-0.04886\\
19.37	-0.04656\\
19.38	-0.04422\\
19.39	-0.04183\\
19.4	-0.03941\\
19.41	-0.03694\\
19.42	-0.03444\\
19.43	-0.0319\\
19.44	-0.02933\\
19.45	-0.02674\\
19.46	-0.02412\\
19.47	-0.02147\\
19.48	-0.0188\\
19.49	-0.01612\\
19.5	-0.01342\\
19.51	-0.0107\\
19.52	-0.007978\\
19.53	-0.005245\\
19.54	-0.002507\\
19.55	0.0002329\\
19.56	0.002973\\
19.57	0.00571\\
19.58	0.008442\\
19.59	0.01116\\
19.6	0.01388\\
19.61	0.01658\\
19.62	0.01926\\
19.63	0.02192\\
19.64	0.02456\\
19.65	0.02718\\
19.66	0.02977\\
19.67	0.03233\\
19.68	0.03486\\
19.69	0.03736\\
19.7	0.03982\\
19.71	0.04224\\
19.72	0.04462\\
19.73	0.04696\\
19.74	0.04924\\
19.75	0.05149\\
19.76	0.05368\\
19.77	0.05581\\
19.78	0.0579\\
19.79	0.05992\\
19.8	0.06189\\
19.81	0.0638\\
19.82	0.06564\\
19.83	0.06742\\
19.84	0.06913\\
19.85	0.07078\\
19.86	0.07236\\
19.87	0.07386\\
19.88	0.07529\\
19.89	0.07665\\
19.9	0.07794\\
19.91	0.07914\\
19.92	0.08027\\
19.93	0.08132\\
19.94	0.0823\\
19.95	0.08319\\
19.96	0.084\\
19.97	0.08472\\
19.98	0.08537\\
19.99	0.08593\\
20	0.0864\\
20.01	0.08679\\
20.02	0.0871\\
20.03	0.08732\\
20.04	0.08745\\
20.05	0.0875\\
20.06	0.08746\\
20.07	0.08734\\
20.08	0.08713\\
20.09	0.08684\\
20.1	0.08646\\
20.11	0.08599\\
20.12	0.08544\\
20.13	0.08481\\
20.14	0.0841\\
20.15	0.0833\\
20.16	0.08242\\
20.17	0.08146\\
20.18	0.08042\\
20.19	0.0793\\
20.2	0.0781\\
20.21	0.07683\\
20.22	0.07548\\
20.23	0.07406\\
20.24	0.07256\\
20.25	0.07099\\
20.26	0.06936\\
20.27	0.06765\\
20.28	0.06588\\
20.29	0.06405\\
20.3	0.06215\\
20.31	0.06019\\
20.32	0.05817\\
20.33	0.05609\\
20.34	0.05396\\
20.35	0.05178\\
20.36	0.04955\\
20.37	0.04726\\
20.38	0.04493\\
20.39	0.04256\\
20.4	0.04015\\
20.41	0.03769\\
20.42	0.0352\\
20.43	0.03267\\
20.44	0.03012\\
20.45	0.02753\\
20.46	0.02491\\
20.47	0.02228\\
20.48	0.01961\\
20.49	0.01693\\
20.5	0.01424\\
20.51	0.01153\\
20.52	0.008805\\
20.53	0.006075\\
20.54	0.003338\\
20.55	0.0005984\\
20.56	-0.002142\\
20.57	-0.00488\\
20.58	-0.007614\\
20.59	-0.01034\\
20.6	-0.01306\\
20.61	-0.01576\\
20.62	-0.01845\\
20.63	-0.02112\\
20.64	-0.02376\\
20.65	-0.02639\\
20.66	-0.02899\\
20.67	-0.03156\\
20.68	-0.0341\\
20.69	-0.03661\\
20.7	-0.03908\\
20.71	-0.04151\\
20.72	-0.0439\\
20.73	-0.04625\\
20.74	-0.04855\\
20.75	-0.05081\\
20.76	-0.05302\\
20.77	-0.05517\\
20.78	-0.05727\\
20.79	-0.05931\\
20.8	-0.0613\\
20.81	-0.06322\\
20.82	-0.06509\\
20.83	-0.06689\\
20.84	-0.06862\\
20.85	-0.07029\\
20.86	-0.07188\\
20.87	-0.07341\\
20.88	-0.07487\\
20.89	-0.07625\\
20.9	-0.07756\\
20.91	-0.07879\\
20.92	-0.07994\\
20.93	-0.08101\\
20.94	-0.08201\\
20.95	-0.08293\\
20.96	-0.08376\\
20.97	-0.08451\\
20.98	-0.08518\\
20.99	-0.08576\\
21	-0.08627\\
21.01	-0.08668\\
21.02	-0.08701\\
21.03	-0.08726\\
21.04	-0.08742\\
21.05	-0.08749\\
21.06	-0.08748\\
21.07	-0.08739\\
21.08	-0.0872\\
21.09	-0.08694\\
21.1	-0.08658\\
21.11	-0.08614\\
21.12	-0.08562\\
21.13	-0.08501\\
21.14	-0.08432\\
21.15	-0.08355\\
21.16	-0.08269\\
21.17	-0.08176\\
21.18	-0.08074\\
21.19	-0.07965\\
21.2	-0.07847\\
21.21	-0.07722\\
21.22	-0.0759\\
21.23	-0.0745\\
21.24	-0.07302\\
21.25	-0.07148\\
21.26	-0.06986\\
21.27	-0.06818\\
21.28	-0.06643\\
21.29	-0.06461\\
21.3	-0.06273\\
21.31	-0.06079\\
21.32	-0.05879\\
21.33	-0.05673\\
21.34	-0.05462\\
21.35	-0.05245\\
21.36	-0.05023\\
21.37	-0.04796\\
21.38	-0.04565\\
21.39	-0.04329\\
21.4	-0.04088\\
21.41	-0.03844\\
21.42	-0.03596\\
21.43	-0.03344\\
21.44	-0.0309\\
21.45	-0.02832\\
21.46	-0.02571\\
21.47	-0.02308\\
21.48	-0.02042\\
21.49	-0.01775\\
21.5	-0.01506\\
21.51	-0.01235\\
21.52	-0.009632\\
21.53	-0.006904\\
21.54	-0.004169\\
21.55	-0.00143\\
21.56	0.001311\\
21.57	0.00405\\
21.58	0.006785\\
21.59	0.009514\\
21.6	0.01223\\
21.61	0.01494\\
21.62	0.01763\\
21.63	0.02031\\
21.64	0.02296\\
21.65	0.0256\\
21.66	0.0282\\
21.67	0.03078\\
21.68	0.03333\\
21.69	0.03585\\
21.7	0.03833\\
21.71	0.04078\\
21.72	0.04318\\
21.73	0.04554\\
21.74	0.04786\\
21.75	0.05013\\
21.76	0.05235\\
21.77	0.05452\\
21.78	0.05664\\
21.79	0.0587\\
21.8	0.0607\\
21.81	0.06265\\
21.82	0.06453\\
21.83	0.06635\\
21.84	0.0681\\
21.85	0.06979\\
21.86	0.07141\\
21.87	0.07296\\
21.88	0.07443\\
21.89	0.07584\\
21.9	0.07717\\
21.91	0.07842\\
21.92	0.0796\\
21.93	0.0807\\
21.94	0.08172\\
21.95	0.08266\\
21.96	0.08351\\
21.97	0.08429\\
21.98	0.08499\\
21.99	0.0856\\
22	0.08612\\
22.01	0.08656\\
22.02	0.08692\\
22.03	0.08719\\
22.04	0.08738\\
22.05	0.08748\\
22.06	0.0875\\
22.07	0.08742\\
22.08	0.08727\\
22.09	0.08703\\
22.1	0.0867\\
22.11	0.08629\\
22.12	0.08579\\
22.13	0.08521\\
22.14	0.08454\\
22.15	0.08379\\
22.16	0.08296\\
22.17	0.08205\\
22.18	0.08106\\
22.19	0.07999\\
22.2	0.07884\\
22.21	0.07761\\
22.22	0.07631\\
22.23	0.07493\\
22.24	0.07348\\
22.25	0.07195\\
22.26	0.07036\\
22.27	0.06869\\
22.28	0.06696\\
22.29	0.06517\\
22.3	0.06331\\
22.31	0.06138\\
22.32	0.0594\\
22.33	0.05736\\
22.34	0.05526\\
22.35	0.05311\\
22.36	0.05091\\
22.37	0.04865\\
22.38	0.04635\\
22.39	0.04401\\
22.4	0.04162\\
22.41	0.03918\\
22.42	0.03672\\
22.43	0.03421\\
22.44	0.03167\\
22.45	0.0291\\
22.46	0.0265\\
22.47	0.02388\\
22.48	0.02123\\
22.49	0.01856\\
22.5	0.01588\\
22.51	0.01317\\
22.52	0.01046\\
22.53	0.007732\\
22.54	0.004999\\
22.55	0.002261\\
22.56	-0.0004796\\
22.57	-0.003219\\
22.58	-0.005956\\
22.59	-0.008687\\
22.6	-0.01141\\
22.61	-0.01412\\
22.62	-0.01682\\
22.63	-0.0195\\
22.64	-0.02216\\
22.65	-0.0248\\
22.66	-0.02742\\
22.67	-0.03\\
22.68	-0.03256\\
22.69	-0.03509\\
22.7	-0.03758\\
22.71	-0.04004\\
22.72	-0.04246\\
22.73	-0.04483\\
22.74	-0.04716\\
22.75	-0.04945\\
22.76	-0.05168\\
22.77	-0.05387\\
22.78	-0.056\\
22.79	-0.05808\\
22.8	-0.0601\\
22.81	-0.06206\\
22.82	-0.06396\\
22.83	-0.0658\\
22.84	-0.06758\\
22.85	-0.06928\\
22.86	-0.07092\\
22.87	-0.07249\\
22.88	-0.07399\\
22.89	-0.07542\\
22.9	-0.07677\\
22.91	-0.07805\\
22.92	-0.07925\\
22.93	-0.08037\\
22.94	-0.08142\\
22.95	-0.08238\\
22.96	-0.08326\\
22.97	-0.08406\\
22.98	-0.08478\\
22.99	-0.08542\\
23	-0.08597\\
23.01	-0.08644\\
23.02	-0.08682\\
23.03	-0.08712\\
23.04	-0.08733\\
23.05	-0.08746\\
23.06	-0.0875\\
23.07	-0.08746\\
23.08	-0.08732\\
23.09	-0.08711\\
23.1	-0.08681\\
23.11	-0.08642\\
23.12	-0.08595\\
23.13	-0.08539\\
23.14	-0.08475\\
23.15	-0.08403\\
23.16	-0.08322\\
23.17	-0.08234\\
23.18	-0.08137\\
23.19	-0.08032\\
23.2	-0.07919\\
23.21	-0.07799\\
23.22	-0.07671\\
23.23	-0.07535\\
23.24	-0.07392\\
23.25	-0.07242\\
23.26	-0.07085\\
23.27	-0.06921\\
23.28	-0.0675\\
23.29	-0.06572\\
23.3	-0.06388\\
23.31	-0.06197\\
23.32	-0.06001\\
23.33	-0.05799\\
23.34	-0.0559\\
23.35	-0.05377\\
23.36	-0.05158\\
23.37	-0.04934\\
23.38	-0.04706\\
23.39	-0.04472\\
23.4	-0.04235\\
23.41	-0.03993\\
23.42	-0.03747\\
23.43	-0.03497\\
23.44	-0.03245\\
23.45	-0.02988\\
23.46	-0.02729\\
23.47	-0.02468\\
23.48	-0.02204\\
23.49	-0.01937\\
23.5	-0.01669\\
23.51	-0.01399\\
23.52	-0.01128\\
23.53	-0.00856\\
23.54	-0.005829\\
23.55	-0.003092\\
23.56	-0.0003518\\
23.57	0.002389\\
23.58	0.005126\\
23.59	0.007859\\
23.6	0.01058\\
23.61	0.0133\\
23.62	0.016\\
23.63	0.01869\\
23.64	0.02136\\
23.65	0.024\\
23.66	0.02663\\
23.67	0.02922\\
23.68	0.03179\\
23.69	0.03433\\
23.7	0.03683\\
23.71	0.0393\\
23.72	0.04173\\
23.73	0.04412\\
23.74	0.04646\\
23.75	0.04876\\
23.76	0.05101\\
23.77	0.05321\\
23.78	0.05536\\
23.79	0.05746\\
23.8	0.05949\\
23.81	0.06147\\
23.82	0.06339\\
23.83	0.06525\\
23.84	0.06705\\
23.85	0.06877\\
23.86	0.07043\\
23.87	0.07203\\
23.88	0.07355\\
23.89	0.07499\\
23.9	0.07637\\
23.91	0.07767\\
23.92	0.07889\\
23.93	0.08004\\
23.94	0.08111\\
23.95	0.0821\\
23.96	0.083\\
23.97	0.08383\\
23.98	0.08457\\
23.99	0.08524\\
24	0.08581\\
24.01	0.08631\\
24.02	0.08671\\
24.03	0.08704\\
24.04	0.08728\\
24.05	0.08743\\
24.06	0.0875\\
24.07	0.08748\\
24.08	0.08737\\
24.09	0.08718\\
24.1	0.08691\\
24.11	0.08655\\
24.12	0.0861\\
24.13	0.08557\\
24.14	0.08495\\
24.15	0.08426\\
24.16	0.08348\\
24.17	0.08261\\
24.18	0.08167\\
24.19	0.08065\\
24.2	0.07954\\
24.21	0.07836\\
24.22	0.07711\\
24.23	0.07577\\
24.24	0.07437\\
24.25	0.07289\\
24.26	0.07133\\
24.27	0.06971\\
24.28	0.06802\\
24.29	0.06626\\
24.3	0.06444\\
24.31	0.06256\\
24.32	0.06061\\
24.33	0.05861\\
24.34	0.05654\\
24.35	0.05442\\
24.36	0.05225\\
24.37	0.05003\\
24.38	0.04775\\
24.39	0.04543\\
24.4	0.04307\\
24.41	0.04066\\
24.42	0.03822\\
24.43	0.03573\\
24.44	0.03322\\
24.45	0.03066\\
24.46	0.02808\\
24.47	0.02547\\
24.48	0.02284\\
24.49	0.02018\\
24.5	0.01751\\
24.51	0.01481\\
24.52	0.01211\\
24.53	0.009387\\
24.54	0.006658\\
24.55	0.003922\\
24.56	0.001183\\
24.57	-0.001557\\
24.58	-0.004296\\
24.59	-0.007031\\
24.6	-0.009759\\
24.61	-0.01248\\
24.62	-0.01518\\
24.63	-0.01787\\
24.64	-0.02055\\
24.65	-0.0232\\
24.66	-0.02583\\
24.67	-0.02844\\
24.68	-0.03101\\
24.69	-0.03356\\
24.7	-0.03608\\
24.71	-0.03855\\
24.72	-0.041\\
24.73	-0.0434\\
24.74	-0.04575\\
24.75	-0.04807\\
24.76	-0.05033\\
24.77	-0.05255\\
24.78	-0.05472\\
24.79	-0.05683\\
24.8	-0.05888\\
24.81	-0.06088\\
24.82	-0.06282\\
24.83	-0.0647\\
24.84	-0.06651\\
24.85	-0.06826\\
24.86	-0.06994\\
24.87	-0.07155\\
24.88	-0.07309\\
24.89	-0.07456\\
24.9	-0.07596\\
24.91	-0.07728\\
24.92	-0.07853\\
24.93	-0.0797\\
24.94	-0.08079\\
24.95	-0.0818\\
24.96	-0.08274\\
24.97	-0.08359\\
24.98	-0.08436\\
24.99	-0.08504\\
25	-0.08565\\
25.01	-0.08617\\
25.02	-0.0866\\
25.03	-0.08695\\
25.04	-0.08721\\
25.05	-0.08739\\
25.06	-0.08749\\
25.07	-0.08749\\
25.08	-0.08741\\
25.09	-0.08725\\
25.1	-0.087\\
25.11	-0.08666\\
25.12	-0.08624\\
25.13	-0.08574\\
25.14	-0.08515\\
25.15	-0.08448\\
25.16	-0.08372\\
25.17	-0.08288\\
25.18	-0.08197\\
25.19	-0.08097\\
25.2	-0.07989\\
25.21	-0.07873\\
25.22	-0.0775\\
25.23	-0.07619\\
25.24	-0.0748\\
25.25	-0.07334\\
25.26	-0.07181\\
25.27	-0.07021\\
25.28	-0.06854\\
25.29	-0.0668\\
25.3	-0.065\\
25.31	-0.06314\\
25.32	-0.06121\\
25.33	-0.05922\\
25.34	-0.05717\\
25.35	-0.05507\\
25.36	-0.05291\\
25.37	-0.05071\\
25.38	-0.04845\\
25.39	-0.04614\\
25.4	-0.04379\\
25.41	-0.0414\\
25.42	-0.03896\\
25.43	-0.03649\\
25.44	-0.03398\\
25.45	-0.03144\\
25.46	-0.02887\\
25.47	-0.02627\\
25.48	-0.02364\\
25.49	-0.02099\\
25.5	-0.01832\\
25.51	-0.01563\\
25.52	-0.01293\\
25.53	-0.01021\\
25.54	-0.007486\\
25.55	-0.004753\\
25.56	-0.002014\\
25.57	0.0007262\\
25.58	0.003466\\
25.59	0.006202\\
25.6	0.008932\\
25.61	0.01165\\
25.62	0.01436\\
25.63	0.01706\\
25.64	0.01974\\
25.65	0.0224\\
25.66	0.02504\\
25.67	0.02765\\
25.68	0.03024\\
25.69	0.03279\\
25.7	0.03532\\
25.71	0.03781\\
25.72	0.04026\\
25.73	0.04267\\
25.74	0.04504\\
25.75	0.04737\\
25.76	0.04965\\
25.77	0.05188\\
25.78	0.05406\\
25.79	0.05619\\
25.8	0.05827\\
25.81	0.06028\\
25.82	0.06224\\
25.83	0.06413\\
25.84	0.06597\\
25.85	0.06773\\
25.86	0.06943\\
25.87	0.07107\\
25.88	0.07263\\
25.89	0.07412\\
25.9	0.07554\\
25.91	0.07689\\
25.92	0.07816\\
25.93	0.07935\\
25.94	0.08047\\
25.95	0.08151\\
25.96	0.08246\\
25.97	0.08334\\
25.98	0.08413\\
25.99	0.08484\\
26	0.08547\\
26.01	0.08602\\
26.02	0.08648\\
26.03	0.08685\\
26.04	0.08714\\
26.05	0.08735\\
26.06	0.08747\\
26.07	0.0875\\
26.08	0.08745\\
26.09	0.08731\\
26.1	0.08708\\
26.11	0.08677\\
26.12	0.08638\\
26.13	0.0859\\
26.14	0.08534\\
26.15	0.08469\\
26.16	0.08396\\
26.17	0.08315\\
26.18	0.08225\\
26.19	0.08128\\
26.2	0.08022\\
26.21	0.07909\\
26.22	0.07788\\
26.23	0.07659\\
26.24	0.07523\\
26.25	0.07379\\
26.26	0.07228\\
26.27	0.0707\\
26.28	0.06906\\
26.29	0.06734\\
26.3	0.06556\\
26.31	0.06371\\
26.32	0.0618\\
26.33	0.05983\\
26.34	0.0578\\
26.35	0.05571\\
26.36	0.05357\\
26.37	0.05138\\
26.38	0.04914\\
26.39	0.04685\\
26.4	0.04451\\
26.41	0.04213\\
26.42	0.03971\\
26.43	0.03725\\
26.44	0.03475\\
26.45	0.03222\\
26.46	0.02965\\
26.47	0.02706\\
26.48	0.02444\\
26.49	0.0218\\
26.5	0.01913\\
26.51	0.01645\\
26.52	0.01375\\
26.53	0.01104\\
26.54	0.008314\\
26.55	0.005582\\
26.56	0.002845\\
26.57	0.0001051\\
26.58	-0.002635\\
26.59	-0.005373\\
26.6	-0.008105\\
26.61	-0.01083\\
26.62	-0.01354\\
26.63	-0.01624\\
26.64	-0.01893\\
26.65	-0.02159\\
26.66	-0.02424\\
26.67	-0.02686\\
26.68	-0.02945\\
26.69	-0.03202\\
26.7	-0.03455\\
26.71	-0.03706\\
26.72	-0.03952\\
26.73	-0.04194\\
26.74	-0.04433\\
26.75	-0.04667\\
26.76	-0.04896\\
26.77	-0.05121\\
26.78	-0.05341\\
26.79	-0.05555\\
26.8	-0.05764\\
26.81	-0.05968\\
26.82	-0.06165\\
26.83	-0.06356\\
26.84	-0.06542\\
26.85	-0.0672\\
26.86	-0.06893\\
26.87	-0.07058\\
26.88	-0.07216\\
26.89	-0.07368\\
26.9	-0.07512\\
26.91	-0.07649\\
26.92	-0.07778\\
26.93	-0.079\\
26.94	-0.08014\\
26.95	-0.0812\\
26.96	-0.08218\\
26.97	-0.08308\\
26.98	-0.0839\\
26.99	-0.08464\\
27	-0.08529\\
27.01	-0.08586\\
27.02	-0.08635\\
27.03	-0.08675\\
27.04	-0.08706\\
27.05	-0.08729\\
27.06	-0.08744\\
27.07	-0.0875\\
27.08	-0.08747\\
27.09	-0.08736\\
27.1	-0.08716\\
27.11	-0.08688\\
27.12	-0.08651\\
27.13	-0.08606\\
27.14	-0.08552\\
27.15	-0.0849\\
27.16	-0.08419\\
27.17	-0.0834\\
27.18	-0.08253\\
27.19	-0.08158\\
27.2	-0.08055\\
27.21	-0.07944\\
27.22	-0.07825\\
27.23	-0.07699\\
27.24	-0.07565\\
27.25	-0.07424\\
27.26	-0.07275\\
27.27	-0.07119\\
27.28	-0.06956\\
27.29	-0.06787\\
27.3	-0.0661\\
27.31	-0.06428\\
27.32	-0.06238\\
27.33	-0.06043\\
27.34	-0.05842\\
27.35	-0.05635\\
27.36	-0.05423\\
27.37	-0.05205\\
27.38	-0.04982\\
27.39	-0.04755\\
27.4	-0.04522\\
27.41	-0.04286\\
27.42	-0.04045\\
27.43	-0.038\\
27.44	-0.03551\\
27.45	-0.03299\\
27.46	-0.03043\\
27.47	-0.02785\\
27.48	-0.02524\\
27.49	-0.0226\\
27.5	-0.01994\\
27.51	-0.01727\\
27.52	-0.01457\\
27.53	-0.01186\\
27.54	-0.009141\\
27.55	-0.006412\\
27.56	-0.003676\\
27.57	-0.0009364\\
27.58	0.001804\\
27.59	0.004543\\
27.6	0.007277\\
27.61	0.01\\
27.62	0.01272\\
27.63	0.01543\\
27.64	0.01812\\
27.65	0.02079\\
27.66	0.02344\\
27.67	0.02607\\
27.68	0.02867\\
27.69	0.03125\\
27.7	0.03379\\
27.71	0.0363\\
27.72	0.03878\\
27.73	0.04121\\
27.74	0.04361\\
27.75	0.04596\\
27.76	0.04827\\
27.77	0.05054\\
27.78	0.05275\\
27.79	0.05491\\
27.8	0.05701\\
27.81	0.05906\\
27.82	0.06106\\
27.83	0.06299\\
27.84	0.06486\\
27.85	0.06667\\
27.86	0.06841\\
27.87	0.07009\\
27.88	0.07169\\
27.89	0.07323\\
27.9	0.07469\\
27.91	0.07608\\
27.92	0.0774\\
27.93	0.07864\\
27.94	0.0798\\
27.95	0.08089\\
27.96	0.08189\\
27.97	0.08282\\
27.98	0.08366\\
27.99	0.08442\\
28	0.0851\\
28.01	0.0857\\
28.02	0.08621\\
28.03	0.08663\\
28.04	0.08698\\
28.05	0.08723\\
28.06	0.0874\\
28.07	0.08749\\
28.08	0.08749\\
28.09	0.0874\\
28.1	0.08723\\
28.11	0.08697\\
28.12	0.08663\\
28.13	0.0862\\
28.14	0.08569\\
28.15	0.08509\\
28.16	0.08441\\
28.17	0.08365\\
28.18	0.0828\\
28.19	0.08188\\
28.2	0.08087\\
28.21	0.07979\\
28.22	0.07862\\
28.23	0.07738\\
28.24	0.07606\\
28.25	0.07467\\
28.26	0.07321\\
28.27	0.07167\\
28.28	0.07006\\
28.29	0.06839\\
28.3	0.06665\\
28.31	0.06484\\
28.32	0.06296\\
28.33	0.06103\\
28.34	0.05904\\
28.35	0.05699\\
28.36	0.05488\\
28.37	0.05272\\
28.38	0.05051\\
28.39	0.04824\\
28.4	0.04593\\
28.41	0.04358\\
28.42	0.04118\\
28.43	0.03874\\
28.44	0.03627\\
28.45	0.03376\\
28.46	0.03121\\
28.47	0.02864\\
28.48	0.02603\\
28.49	0.0234\\
28.5	0.02075\\
28.51	0.01808\\
28.52	0.01539\\
28.53	0.01269\\
28.54	0.009968\\
28.55	0.007241\\
28.56	0.004506\\
28.57	0.001768\\
28.58	-0.0009728\\
28.59	-0.003712\\
28.6	-0.006448\\
28.61	-0.009178\\
28.62	-0.0119\\
28.63	-0.01461\\
28.64	-0.0173\\
28.65	-0.01998\\
28.66	-0.02264\\
28.67	-0.02527\\
28.68	-0.02788\\
28.69	-0.03047\\
28.7	-0.03302\\
28.71	-0.03554\\
28.72	-0.03803\\
28.73	-0.04048\\
28.74	-0.04289\\
28.75	-0.04525\\
28.76	-0.04758\\
28.77	-0.04985\\
28.78	-0.05208\\
28.79	-0.05426\\
28.8	-0.05638\\
28.81	-0.05845\\
28.82	-0.06046\\
28.83	-0.06241\\
28.84	-0.0643\\
28.85	-0.06613\\
28.86	-0.06789\\
28.87	-0.06958\\
28.88	-0.07121\\
28.89	-0.07277\\
28.9	-0.07426\\
28.91	-0.07567\\
28.92	-0.07701\\
28.93	-0.07827\\
28.94	-0.07946\\
28.95	-0.08057\\
28.96	-0.08159\\
28.97	-0.08254\\
28.98	-0.08341\\
28.99	-0.0842\\
29	-0.0849\\
29.01	-0.08552\\
29.02	-0.08606\\
29.03	-0.08651\\
29.04	-0.08688\\
29.05	-0.08716\\
29.06	-0.08736\\
29.07	-0.08747\\
29.08	-0.0875\\
29.09	-0.08744\\
29.1	-0.08729\\
29.11	-0.08706\\
29.12	-0.08674\\
29.13	-0.08634\\
29.14	-0.08585\\
29.15	-0.08528\\
29.16	-0.08463\\
29.17	-0.08389\\
29.18	-0.08307\\
29.19	-0.08217\\
29.2	-0.08119\\
29.21	-0.08012\\
29.22	-0.07898\\
29.23	-0.07777\\
29.24	-0.07647\\
29.25	-0.0751\\
29.26	-0.07366\\
29.27	-0.07214\\
29.28	-0.07056\\
29.29	-0.0689\\
29.3	-0.06718\\
29.31	-0.06539\\
29.32	-0.06354\\
29.33	-0.06162\\
29.34	-0.05965\\
29.35	-0.05761\\
29.36	-0.05552\\
29.37	-0.05338\\
29.38	-0.05118\\
29.39	-0.04893\\
29.4	-0.04664\\
29.41	-0.0443\\
29.42	-0.04191\\
29.43	-0.03949\\
29.44	-0.03702\\
29.45	-0.03452\\
29.46	-0.03199\\
29.47	-0.02942\\
29.48	-0.02683\\
29.49	-0.0242\\
29.5	-0.02156\\
29.51	-0.01889\\
29.52	-0.01621\\
29.53	-0.01351\\
29.54	-0.01079\\
29.55	-0.008069\\
29.56	-0.005336\\
29.57	-0.002599\\
29.58	0.0001415\\
29.59	0.002882\\
29.6	0.005619\\
29.61	0.008351\\
29.62	0.01107\\
29.63	0.01379\\
29.64	0.01649\\
29.65	0.01917\\
29.66	0.02183\\
29.67	0.02448\\
29.68	0.02709\\
29.69	0.02969\\
29.7	0.03225\\
29.71	0.03478\\
29.72	0.03728\\
29.73	0.03974\\
29.74	0.04216\\
29.75	0.04454\\
29.76	0.04688\\
29.77	0.04917\\
29.78	0.05141\\
29.79	0.0536\\
29.8	0.05574\\
29.81	0.05783\\
29.82	0.05986\\
29.83	0.06182\\
29.84	0.06373\\
29.85	0.06558\\
29.86	0.06736\\
29.87	0.06908\\
29.88	0.07073\\
29.89	0.0723\\
29.9	0.07381\\
29.91	0.07525\\
29.92	0.07661\\
29.93	0.0779\\
29.94	0.07911\\
29.95	0.08024\\
29.96	0.08129\\
29.97	0.08227\\
29.98	0.08316\\
29.99	0.08397\\
};
\addlegendentry{ANA}

\end{axis}

\begin{axis}[%
width=1.227\figW,
height=1.227\figH,
at={(-0.16\figW,-0.135\figH)},
scale only axis,
xmin=0,
xmax=1,
ymin=0,
ymax=1,
axis line style={draw=none},
ticks=none,
axis x line*=bottom,
axis y line*=left
]
\end{axis}
\end{tikzpicture}%
    \caption{Top: analytical (ANA) solution of the linear movement equation. Bottom: Numerical solution of the linear movement equation using explicit Euler (EE), implicit Euler (EI) and the middle point rule (MP). The studied time interval was $30\,s$ at a time step of $0.01\,s$ for all integrators.}
    \label{fig:LinSolutions}
\end{figure}

Figure \ref{fig:LinSolutions} shows both the analytical and the numerical solutions of Equation \ref{eq: linear}. The initial conditions were chosen to $\phi_o = 0.0875\,rad$ and $\Dot{\phi_o}= 0$. The total simulation time was chosen to $T = 30\,s$ at a time step of $0.01\,s$. For the numerical case, the EE, EI, and MP integration schemes were compared. As the Figure shows, the angle coordinate varied periodically in time. For the analytical case, the amplitude and period of the oscillation remained unchanged within the simulated time interval. In the case of the numerical methods, the period of the oscillation was maintained. However, the amplitude increased dramatically for the EE method, decreased for the EI method and remained almost unchanged for the MP method.

Further, the course of the total energy of the system for the different integration methods was studied. In the mathematical pendulum system, the total energy is given by the sum of the (gravitational) potential and the kinetic energy. Taking the fixing point as reference level for the potential energy, the total energy of the system can be written as:

\begin{equation}
    E(t) = E_{kin}(t) + E_{pot}(t) = \frac{1}{2}\,m\,l^2\,\Dot{\phi}(t) - m\,g\,l\,cos\left(\phi(t)\right).
    \label{eq: energy}
\end{equation}

Since this is a conservative system, total energy should be conserved in time. As Figure \ref{fig: EnergyPlot} shows, this is not the case for the EE method for which total energy increases as the simulation elapses. Contrarily, the EI method causes a decrease in energy, which correlates with the decrease in the oscillation amplitude seen in Figure \ref{fig:LinSolutions}. On the other hand, the middle-point rule remains close to the constant total energy observed for the analytical solution.

\begin{figure}[h]
    \centering
    \setlength{\figH}{0.3\textheight}
    \setlength{\figW}{0.6\textwidth}
    % This file was created by matlab2tikz.
%
%The latest updates can be retrieved from
%  http://www.mathworks.com/matlabcentral/fileexchange/22022-matlab2tikz-matlab2tikz
%where you can also make suggestions and rate matlab2tikz.
%
\definecolor{mycolor1}{rgb}{1.00000,0.00000,1.00000}%
%
\begin{tikzpicture}

\begin{axis}[%
width=4.521in,
height=3.566in,
at={(0.758in,0.481in)},
scale only axis,
xmin=0,
xmax=30,
xlabel style={font=\color{white!15!black}},
xlabel={Time [s]},
ymin=-9.9,
ymax=-9.1,
ylabel style={font=\color{white!15!black}},
ylabel={Energy [J]},
axis background/.style={fill=white},
legend style={at={(0.03,0.97)}, anchor=north west, legend cell align=left, align=left, draw=white!15!black}
]
\addplot [color=red]
  table[row sep=crcr]{%
0	-9.77247004781056\\
0.01	-9.77243320742853\\
0.02	-9.77239623706052\\
0.03	-9.77235913749847\\
0.04	-9.77232191044417\\
0.05	-9.77228455848775\\
0.06	-9.77224708507427\\
0.07	-9.77220949445906\\
0.08	-9.77217179165218\\
0.09	-9.77213398235306\\
0.1	-9.77209607287603\\
0.11	-9.77205807006807\\
0.12	-9.77201998121984\\
0.13	-9.77198181397125\\
0.14	-9.7719435762132\\
0.15	-9.77190527598666\\
0.16	-9.77186692138076\\
0.17	-9.77182852043138\\
0.18	-9.77179008102167\\
0.19	-9.77175161078596\\
0.2	-9.77171311701863\\
0.21	-9.7716746065891\\
0.22	-9.7716360858642\\
0.23	-9.77159756063919\\
0.24	-9.77155903607822\\
0.25	-9.77152051666525\\
0.26	-9.771482006166\\
0.27	-9.77144350760141\\
0.28	-9.77140502323309\\
0.29	-9.77136655456067\\
0.3	-9.77132810233121\\
0.31	-9.77128966656029\\
0.32	-9.77125124656438\\
0.33	-9.77121284100394\\
0.34	-9.77117444793642\\
0.35	-9.77113606487819\\
0.36	-9.77109768887445\\
0.37	-9.77105931657578\\
0.38	-9.77102094432014\\
0.39	-9.77098256821881\\
0.4	-9.77094418424495\\
0.41	-9.77090578832309\\
0.42	-9.77086737641829\\
0.43	-9.77082894462317\\
0.44	-9.77079048924159\\
0.45	-9.77075200686743\\
0.46	-9.77071349445706\\
0.47	-9.77067494939444\\
0.48	-9.77063636954758\\
0.49	-9.77059775331541\\
0.5	-9.77055909966423\\
0.51	-9.77052040815317\\
0.52	-9.77048167894794\\
0.53	-9.7704429128229\\
0.54	-9.77040411115106\\
0.55	-9.77036527588228\\
0.56	-9.77032640950995\\
0.57	-9.77028751502655\\
0.58	-9.77024859586883\\
0.59	-9.77020965585346\\
0.6	-9.77017069910416\\
0.61	-9.77013172997135\\
0.62	-9.77009275294588\\
0.63	-9.77005377256789\\
0.64	-9.77001479333255\\
0.65	-9.76997581959415\\
0.66	-9.76993685547007\\
0.67	-9.76989790474628\\
0.68	-9.76985897078602\\
0.69	-9.76982005644308\\
0.7	-9.76978116398136\\
0.71	-9.76974229500192\\
0.72	-9.76970345037902\\
0.73	-9.76966463020615\\
0.74	-9.76962583375323\\
0.75	-9.76958705943572\\
0.76	-9.76954830479637\\
0.77	-9.76950956650012\\
0.78	-9.76947084034238\\
0.79	-9.76943212127085\\
0.8	-9.76939340342058\\
0.81	-9.7693546801622\\
0.82	-9.76931594416244\\
0.83	-9.76927718745653\\
0.84	-9.76923840153126\\
0.85	-9.76919957741783\\
0.86	-9.76916070579297\\
0.87	-9.76912177708721\\
0.88	-9.76908278159855\\
0.89	-9.76904370960998\\
0.9	-9.76900455150924\\
0.91	-9.76896529790893\\
0.92	-9.7689259397653\\
0.93	-9.76888646849406\\
0.94	-9.76884687608124\\
0.95	-9.76880715518774\\
0.96	-9.76876729924582\\
0.97	-9.76872730254608\\
0.98	-9.76868716031369\\
0.99	-9.76864686877271\\
1	-9.76860642519739\\
1.01	-9.76856582794975\\
1.02	-9.76852507650281\\
1.03	-9.76848417144912\\
1.04	-9.76844311449434\\
1.05	-9.76840190843598\\
1.06	-9.76836055712763\\
1.07	-9.76831906542905\\
1.08	-9.76827743914294\\
1.09	-9.76823568493924\\
1.1	-9.76819381026811\\
1.11	-9.76815182326283\\
1.12	-9.768109732634\\
1.13	-9.76806754755671\\
1.14	-9.76802527755229\\
1.15	-9.76798293236628\\
1.16	-9.76794052184464\\
1.17	-9.76789805580984\\
1.18	-9.76785554393878\\
1.19	-9.76781299564423\\
1.2	-9.76777041996169\\
1.21	-9.76772782544315\\
1.22	-9.76768522005947\\
1.23	-9.7676426111127\\
1.24	-9.76760000515959\\
1.25	-9.76755740794747\\
1.26	-9.7675148243633\\
1.27	-9.76747225839663\\
1.28	-9.76742971311691\\
1.29	-9.76738719066542\\
1.3	-9.76734469226175\\
1.31	-9.76730221822481\\
1.32	-9.76725976800763\\
1.33	-9.76721734024555\\
1.34	-9.76717493281683\\
1.35	-9.76713254291447\\
1.36	-9.76709016712826\\
1.37	-9.76704780153543\\
1.38	-9.76700544179838\\
1.39	-9.76696308326794\\
1.4	-9.76692072109029\\
1.41	-9.76687835031573\\
1.42	-9.76683596600762\\
1.43	-9.76679356334934\\
1.44	-9.76675113774779\\
1.45	-9.76670868493147\\
1.46	-9.76666620104147\\
1.47	-9.76662368271392\\
1.48	-9.7665811271524\\
1.49	-9.76653853218905\\
1.5	-9.76649589633331\\
1.51	-9.76645321880748\\
1.52	-9.76641049956835\\
1.53	-9.76636773931446\\
1.54	-9.76632493947892\\
1.55	-9.76628210220762\\
1.56	-9.76623923032338\\
1.57	-9.76619632727626\\
1.58	-9.76615339708096\\
1.59	-9.76611044424232\\
1.6	-9.76606747366982\\
1.61	-9.76602449058277\\
1.62	-9.76598150040743\\
1.63	-9.76593850866789\\
1.64	-9.76589552087236\\
1.65	-9.76585254239689\\
1.66	-9.76580957836823\\
1.67	-9.76576663354802\\
1.68	-9.76572371222003\\
1.69	-9.76568081808255\\
1.7	-9.76563795414765\\
1.71	-9.76559512264914\\
1.72	-9.76555232496088\\
1.73	-9.7655095615268\\
1.74	-9.76546683180414\\
1.75	-9.76542413422086\\
1.76	-9.76538146614819\\
1.77	-9.76533882388897\\
1.78	-9.76529620268226\\
1.79	-9.76525359672439\\
1.8	-9.76521099920621\\
1.81	-9.76516840236656\\
1.82	-9.76512579756094\\
1.83	-9.76508317534498\\
1.84	-9.76504052557128\\
1.85	-9.76499783749864\\
1.86	-9.76495509991194\\
1.87	-9.76491230125127\\
1.88	-9.76486942974825\\
1.89	-9.76482647356781\\
1.9	-9.76478342095325\\
1.91	-9.76474026037258\\
1.92	-9.7646969806639\\
1.93	-9.76465357117785\\
1.94	-9.7646100219148\\
1.95	-9.76456632365492\\
1.96	-9.76452246807916\\
1.97	-9.7644784478792\\
1.98	-9.76443425685484\\
1.99	-9.76438988999727\\
2	-9.764345343557\\
2.01	-9.7643006150953\\
2.02	-9.76425570351843\\
2.03	-9.76421060909408\\
2.04	-9.76416533344962\\
2.05	-9.76411987955236\\
2.06	-9.76407425167181\\
2.07	-9.76402845532468\\
2.08	-9.76398249720321\\
2.09	-9.763936385088\\
2.1	-9.76389012774655\\
2.11	-9.763843734819\\
2.12	-9.76379721669268\\
2.13	-9.76375058436741\\
2.14	-9.76370384931357\\
2.15	-9.76365702332479\\
2.16	-9.76361011836779\\
2.17	-9.76356314643131\\
2.18	-9.76351611937647\\
2.19	-9.7634690487908\\
2.2	-9.76342194584802\\
2.21	-9.76337482117568\\
2.22	-9.76332768473259\\
2.23	-9.76328054569779\\
2.24	-9.76323341237274\\
2.25	-9.76318629209802\\
2.26	-9.76313919118582\\
2.27	-9.76309211486905\\
2.28	-9.76304506726782\\
2.29	-9.76299805137357\\
2.3	-9.76295106905108\\
2.31	-9.76290412105805\\
2.32	-9.76285720708198\\
2.33	-9.76281032579341\\
2.34	-9.76276347491475\\
2.35	-9.76271665130329\\
2.36	-9.76266985104714\\
2.37	-9.76262306957216\\
2.38	-9.76257630175836\\
2.39	-9.76252954206353\\
2.4	-9.76248278465217\\
2.41	-9.76243602352741\\
2.42	-9.7623892526638\\
2.43	-9.7623424661386\\
2.44	-9.76229565825949\\
2.45	-9.76224882368634\\
2.46	-9.76220195754519\\
2.47	-9.76215505553232\\
2.48	-9.76210811400664\\
2.49	-9.76206113006893\\
2.5	-9.76201410162638\\
2.51	-9.76196702744135\\
2.52	-9.76191990716345\\
2.53	-9.7618727413444\\
2.54	-9.76182553143509\\
2.55	-9.76177827976511\\
2.56	-9.7617309895047\\
2.57	-9.76168366460992\\
2.58	-9.76163630975155\\
2.59	-9.7615889302291\\
2.6	-9.76154153187107\\
2.61	-9.76149412092312\\
2.62	-9.76144670392593\\
2.63	-9.76139928758458\\
2.64	-9.76135187863187\\
2.65	-9.76130448368741\\
2.66	-9.76125710911524\\
2.67	-9.76120976088188\\
2.68	-9.76116244441766\\
2.69	-9.76111516448329\\
2.7	-9.76106792504415\\
2.71	-9.76102072915451\\
2.72	-9.76097357885359\\
2.73	-9.76092647507542\\
2.74	-9.76087941757414\\
2.75	-9.76083240486624\\
2.76	-9.76078543419085\\
2.77	-9.76073850148904\\
2.78	-9.76069160140273\\
2.79	-9.76064472729357\\
2.8	-9.76059787128172\\
2.81	-9.76055102430439\\
2.82	-9.76050417619332\\
2.83	-9.76045731577046\\
2.84	-9.76041043096057\\
2.85	-9.76036350891934\\
2.86	-9.76031653617514\\
2.87	-9.76026949878276\\
2.88	-9.76022238248657\\
2.89	-9.76017517289125\\
2.9	-9.76012785563727\\
2.91	-9.76008041657883\\
2.92	-9.7600328419615\\
2.93	-9.75998511859714\\
2.94	-9.7599372340333\\
2.95	-9.7598891767148\\
2.96	-9.75984093613492\\
2.97	-9.75979250297403\\
2.98	-9.75974386922348\\
2.99	-9.75969502829294\\
3	-9.75964597509951\\
3.01	-9.75959670613721\\
3.02	-9.75954721952581\\
3.03	-9.75949751503814\\
3.04	-9.75944759410546\\
3.05	-9.75939745980074\\
3.06	-9.75934711680005\\
3.07	-9.75929657132257\\
3.08	-9.75924583105006\\
3.09	-9.75919490502691\\
3.1	-9.75914380354233\\
3.11	-9.75909253799621\\
3.12	-9.75904112075081\\
3.13	-9.75898956497029\\
3.14	-9.75893788445061\\
3.15	-9.75888609344212\\
3.16	-9.75883420646764\\
3.17	-9.75878223813856\\
3.18	-9.75873020297189\\
3.19	-9.75867811521068\\
3.2	-9.75862598865072\\
3.21	-9.75857383647589\\
3.22	-9.75852167110471\\
3.23	-9.75846950405019\\
3.24	-9.75841734579515\\
3.25	-9.75836520568469\\
3.26	-9.7583130918374\\
3.27	-9.75826101107649\\
3.28	-9.75820896888179\\
3.29	-9.75815696936314\\
3.3	-9.75810501525557\\
3.31	-9.75805310793594\\
3.32	-9.75800124746083\\
3.33	-9.75794943262481\\
3.34	-9.75789766103794\\
3.35	-9.75784592922126\\
3.36	-9.75779423271847\\
3.37	-9.75774256622186\\
3.38	-9.75769092371039\\
3.39	-9.75763929859755\\
3.4	-9.75758768388642\\
3.41	-9.75753607232935\\
3.42	-9.75748445658961\\
3.43	-9.75743282940212\\
3.44	-9.75738118373068\\
3.45	-9.757329512919\\
3.46	-9.75727781083289\\
3.47	-9.75722607199125\\
3.48	-9.75717429168365\\
3.49	-9.75712246607227\\
3.5	-9.75707059227679\\
3.51	-9.75701866844023\\
3.52	-9.75696669377504\\
3.53	-9.75691466858823\\
3.54	-9.75686259428513\\
3.55	-9.75681047335163\\
3.56	-9.75675830931497\\
3.57	-9.75670610668378\\
3.58	-9.756653870868\\
3.59	-9.75660160808019\\
3.6	-9.75654932521946\\
3.61	-9.75649702974004\\
3.62	-9.75644472950655\\
3.63	-9.75639243263817\\
3.64	-9.75634014734442\\
3.65	-9.756287881755\\
3.66	-9.75623564374675\\
3.67	-9.75618344077033\\
3.68	-9.75613127967975\\
3.69	-9.75607916656744\\
3.7	-9.75602710660777\\
3.71	-9.75597510391175\\
3.72	-9.75592316139537\\
3.73	-9.755871280664\\
3.74	-9.75581946191492\\
3.75	-9.75576770385994\\
3.76	-9.75571600366947\\
3.77	-9.7556643569395\\
3.78	-9.75561275768222\\
3.79	-9.75556119834075\\
3.8	-9.75550966982833\\
3.81	-9.75545816159152\\
3.82	-9.75540666169689\\
3.83	-9.75535515694025\\
3.84	-9.75530363297698\\
3.85	-9.75525207447183\\
3.86	-9.75520046526627\\
3.87	-9.75514878856089\\
3.88	-9.75509702711059\\
3.89	-9.75504516342962\\
3.9	-9.75499318000358\\
3.91	-9.75494105950535\\
3.92	-9.7548887850119\\
3.93	-9.75483634021863\\
3.94	-9.75478370964819\\
3.95	-9.75473087885072\\
3.96	-9.7546778345925\\
3.97	-9.75462456503005\\
3.98	-9.75457105986731\\
3.99	-9.75451731049333\\
4	-9.75446331009833\\
4.01	-9.75440905376655\\
4.02	-9.75435453854425\\
4.03	-9.75429976348183\\
4.04	-9.75424472964939\\
4.05	-9.7541894401254\\
4.06	-9.75413389995866\\
4.07	-9.75407811610384\\
4.08	-9.75402209733178\\
4.09	-9.7539658541156\\
4.1	-9.75390939849431\\
4.11	-9.75385274391603\\
4.12	-9.75379590506293\\
4.13	-9.75373889766057\\
4.14	-9.75368173827441\\
4.15	-9.75362444409652\\
4.16	-9.75356703272558\\
4.17	-9.7535095219435\\
4.18	-9.75345192949187\\
4.19	-9.7533942728517\\
4.2	-9.75333656902941\\
4.21	-9.75327883435264\\
4.22	-9.75322108427843\\
4.23	-9.75316333321692\\
4.24	-9.75310559437295\\
4.25	-9.75304787960794\\
4.26	-9.75299019932388\\
4.27	-9.7529325623712\\
4.28	-9.75287497598171\\
4.29	-9.75281744572735\\
4.3	-9.75275997550539\\
4.31	-9.75270256754993\\
4.32	-9.75264522246943\\
4.33	-9.75258793930937\\
4.34	-9.75253071563894\\
4.35	-9.75247354766014\\
4.36	-9.75241643033729\\
4.37	-9.75235935754483\\
4.38	-9.75230232223078\\
4.39	-9.75224531659302\\
4.4	-9.75218833226551\\
4.41	-9.75213136051115\\
4.42	-9.75207439241815\\
4.43	-9.75201741909645\\
4.44	-9.75196043187096\\
4.45	-9.75190342246833\\
4.46	-9.75184638319398\\
4.47	-9.75178930709657\\
4.48	-9.75173218811682\\
4.49	-9.75167502121837\\
4.5	-9.75161780249829\\
4.51	-9.75156052927523\\
4.52	-9.75150320015375\\
4.53	-9.75144581506359\\
4.54	-9.75138837527305\\
4.55	-9.75133088337613\\
4.56	-9.75127334325359\\
4.57	-9.75121576000826\\
4.58	-9.75115813987569\\
4.59	-9.75110049011139\\
4.6	-9.75104281885639\\
4.61	-9.75098513498327\\
4.62	-9.75092744792508\\
4.63	-9.75086976748982\\
4.64	-9.75081210366357\\
4.65	-9.75075446640542\\
4.66	-9.75069686543757\\
4.67	-9.75063931003404\\
4.68	-9.75058180881159\\
4.69	-9.75052436952625\\
4.7	-9.75046699887901\\
4.71	-9.75040970233402\\
4.72	-9.75035248395234\\
4.73	-9.75029534624439\\
4.74	-9.75023829004363\\
4.75	-9.75018131440384\\
4.76	-9.75012441652218\\
4.77	-9.75006759168943\\
4.78	-9.75001083326881\\
4.79	-9.74995413270408\\
4.8	-9.74989747955724\\
4.81	-9.7498408615757\\
4.82	-9.74978426478837\\
4.83	-9.74972767362958\\
4.84	-9.74967107108927\\
4.85	-9.74961443888776\\
4.86	-9.74955775767248\\
4.87	-9.74950100723428\\
4.88	-9.74944416674017\\
4.89	-9.74938721497925\\
4.9	-9.74933013061834\\
4.91	-9.74927289246375\\
4.92	-9.74921547972517\\
4.93	-9.74915787227807\\
4.94	-9.7491000509206\\
4.95	-9.74904199762127\\
4.96	-9.74898369575367\\
4.97	-9.74892513031479\\
4.98	-9.7488662881236\\
4.99	-9.74880715799689\\
5	-9.74874773089972\\
5.01	-9.74868800006814\\
5.02	-9.74862796110227\\
5.03	-9.74856761202829\\
5.04	-9.74850695332836\\
5.05	-9.74844598793779\\
5.06	-9.74838472120963\\
5.07	-9.74832316084695\\
5.08	-9.74826131680379\\
5.09	-9.74819920115622\\
5.1	-9.74813682794533\\
5.11	-9.74807421299438\\
5.12	-9.74801137370284\\
5.13	-9.74794832882039\\
5.14	-9.74788509820397\\
5.15	-9.7478217025618\\
5.16	-9.74775816318787\\
5.17	-9.74769450169094\\
5.18	-9.747630739722\\
5.19	-9.74756689870439\\
5.2	-9.74750299957027\\
5.21	-9.74743906250754\\
5.22	-9.74737510672099\\
5.23	-9.74731115021083\\
5.24	-9.74724720957229\\
5.25	-9.7471832998188\\
5.26	-9.74711943423152\\
5.27	-9.74705562423716\\
5.28	-9.74699187931596\\
5.29	-9.74692820694076\\
5.3	-9.74686461254816\\
5.31	-9.74680109954175\\
5.32	-9.7467376693272\\
5.33	-9.7466743213784\\
5.34	-9.74661105333332\\
5.35	-9.74654786111789\\
5.36	-9.74648473909562\\
5.37	-9.74642168024034\\
5.38	-9.74635867632917\\
5.39	-9.74629571815224\\
5.4	-9.74623279573579\\
5.41	-9.74616989857467\\
5.42	-9.74610701587044\\
5.43	-9.74604413677088\\
5.44	-9.7459812506071\\
5.45	-9.74591834712392\\
5.46	-9.74585541669989\\
5.47	-9.745792450553\\
5.48	-9.74572944092865\\
5.49	-9.74566638126673\\
5.5	-9.74560326634468\\
5.51	-9.74554009239422\\
5.52	-9.74547685718967\\
5.53	-9.74541356010612\\
5.54	-9.74535020214631\\
5.55	-9.74528678593587\\
5.56	-9.74522331568641\\
5.57	-9.74515979712728\\
5.58	-9.74509623740669\\
5.59	-9.74503264496377\\
5.6	-9.74496902937352\\
5.61	-9.74490540116703\\
5.62	-9.74484177162982\\
5.63	-9.74477815258149\\
5.64	-9.74471455614027\\
5.65	-9.74465099447625\\
5.66	-9.74458747955733\\
5.67	-9.74452402289212\\
5.68	-9.74446063527394\\
5.69	-9.74439732653041\\
5.7	-9.74433410528266\\
5.71	-9.74427097871844\\
5.72	-9.74420795238297\\
5.73	-9.74414502999128\\
5.74	-9.7440822132654\\
5.75	-9.74401950179941\\
5.76	-9.74395689295497\\
5.77	-9.74389438178945\\
5.78	-9.74383196101836\\
5.79	-9.74376962101318\\
5.8	-9.74370734983514\\
5.81	-9.743645133305\\
5.82	-9.74358295510839\\
5.83	-9.74352079693539\\
5.84	-9.74345863865297\\
5.85	-9.7433964585079\\
5.86	-9.74333423335769\\
5.87	-9.74327193892626\\
5.88	-9.74320955008099\\
5.89	-9.74314704112716\\
5.9	-9.7430843861157\\
5.91	-9.74302155915979\\
5.92	-9.74295853475582\\
5.93	-9.74289528810387\\
5.94	-9.74283179542329\\
5.95	-9.74276803425839\\
5.96	-9.74270398377002\\
5.97	-9.74263962500848\\
5.98	-9.74257494116372\\
5.99	-9.7425099177891\\
6	-9.74244454299524\\
6.01	-9.742378807611\\
6.02	-9.74231270530924\\
6.03	-9.74224623269502\\
6.04	-9.74217938935537\\
6.05	-9.74211217786919\\
6.06	-9.7420446037775\\
6.07	-9.74197667551411\\
6.08	-9.74190840429772\\
6.09	-9.74183980398697\\
6.1	-9.74177089090051\\
6.11	-9.74170168360468\\
6.12	-9.74163220267185\\
6.13	-9.7415624704131\\
6.14	-9.74149251058891\\
6.15	-9.74142234810247\\
6.16	-9.74135200867974\\
6.17	-9.74128151854134\\
6.18	-9.74121090407086\\
6.19	-9.74114019148466\\
6.2	-9.74106940650794\\
6.21	-9.74099857406191\\
6.22	-9.74092771796661\\
6.23	-9.74085686066378\\
6.24	-9.74078602296386\\
6.25	-9.74071522382069\\
6.26	-9.74064448013718\\
6.27	-9.74057380660479\\
6.28	-9.74050321557888\\
6.29	-9.74043271699172\\
6.3	-9.74036231830412\\
6.31	-9.74029202449625\\
6.32	-9.74022183809737\\
6.33	-9.74015175925374\\
6.34	-9.74008178583341\\
6.35	-9.74001191356576\\
6.36	-9.73994213621345\\
6.37	-9.73987244577366\\
6.38	-9.73980283270508\\
6.39	-9.73973328617693\\
6.4	-9.73966379433553\\
6.41	-9.73959434458406\\
6.42	-9.73952492387064\\
6.43	-9.73945551897999\\
6.44	-9.73938611682354\\
6.45	-9.73931670472327\\
6.46	-9.73924727068429\\
6.47	-9.7391778036517\\
6.48	-9.73910829374714\\
6.49	-9.73903873248116\\
6.5	-9.73896911293757\\
6.51	-9.73889942992674\\
6.52	-9.73882968010492\\
6.53	-9.73875986205759\\
6.54	-9.73868997634516\\
6.55	-9.73862002551011\\
6.56	-9.73855001404523\\
6.57	-9.7384799483232\\
6.58	-9.73840983648864\\
6.59	-9.73833968831391\\
6.6	-9.73826951502122\\
6.61	-9.73819932907342\\
6.62	-9.73812914393707\\
6.63	-9.73805897382131\\
6.64	-9.73798883339691\\
6.65	-9.73791873749991\\
6.66	-9.73784870082474\\
6.67	-9.73777873761187\\
6.68	-9.73770886133515\\
6.69	-9.73763908439403\\
6.7	-9.73756941781598\\
6.71	-9.73749987097398\\
6.72	-9.73743045132415\\
6.73	-9.73736116416809\\
6.74	-9.737292012444\\
6.75	-9.7372229965507\\
6.76	-9.73715411420765\\
6.77	-9.73708536035391\\
6.78	-9.73701672708821\\
6.79	-9.73694820365172\\
6.8	-9.73687977645447\\
6.81	-9.73681142914567\\
6.82	-9.7367431427275\\
6.83	-9.73667489571125\\
6.84	-9.73660666431409\\
6.85	-9.73653842269399\\
6.86	-9.73647014321975\\
6.87	-9.73640179677266\\
6.88	-9.73633335307545\\
6.89	-9.73626478104425\\
6.9	-9.73619604915831\\
6.91	-9.73612712584242\\
6.92	-9.73605797985645\\
6.93	-9.73598858068633\\
6.94	-9.73591889893082\\
6.95	-9.73584890667828\\
6.96	-9.73577857786804\\
6.97	-9.7357078886307\\
6.98	-9.73563681760264\\
6.99	-9.73556534620959\\
7	-9.73549345891532\\
7.01	-9.7354211434314\\
7.02	-9.73534839088494\\
7.03	-9.73527519594161\\
7.04	-9.73520155688206\\
7.05	-9.73512747563035\\
7.06	-9.73505295773397\\
7.07	-9.73497801229554\\
7.08	-9.73490265185708\\
7.09	-9.73482689223851\\
7.1	-9.73475075233261\\
7.11	-9.73467425385941\\
7.12	-9.73459742108364\\
7.13	-9.73452028049925\\
7.14	-9.73444286048579\\
7.15	-9.73436519094148\\
7.16	-9.73428730289863\\
7.17	-9.73420922812689\\
7.18	-9.73413099873027\\
7.19	-9.73405264674398\\
7.2	-9.73397420373678\\
7.21	-9.73389570042503\\
7.22	-9.7338171663039\\
7.23	-9.73373862930124\\
7.24	-9.7336601154591\\
7.25	-9.73358164864771\\
7.26	-9.7335032503158\\
7.27	-9.73342493928095\\
7.28	-9.73334673156293\\
7.29	-9.73326864026211\\
7.3	-9.73319067548473\\
7.31	-9.73311284431554\\
7.32	-9.73303515083816\\
7.33	-9.73295759620226\\
7.34	-9.7328801787362\\
7.35	-9.73280289410298\\
7.36	-9.73272573549669\\
7.37	-9.73264869387592\\
7.38	-9.73257175823007\\
7.39	-9.73249491587402\\
7.4	-9.7324181527661\\
7.41	-9.73234145384388\\
7.42	-9.73226480337217\\
7.43	-9.73218818529731\\
7.44	-9.73211158360163\\
7.45	-9.73203498265229\\
7.46	-9.7319583675384\\
7.47	-9.73188172439074\\
7.48	-9.73180504067873\\
7.49	-9.73172830547932\\
7.5	-9.73165150971344\\
7.51	-9.73157464634575\\
7.52	-9.7314977105441\\
7.53	-9.73142069979604\\
7.54	-9.73134361398\\
7.55	-9.73126645538987\\
7.56	-9.73118922871223\\
7.57	-9.73111194095634\\
7.58	-9.73103460133788\\
7.59	-9.73095722111798\\
7.6	-9.73087981340005\\
7.61	-9.7308023928875\\
7.62	-9.73072497560607\\
7.63	-9.73064757859532\\
7.64	-9.73057021957407\\
7.65	-9.73049291658521\\
7.66	-9.73041568762577\\
7.67	-9.73033855026814\\
7.68	-9.73026152127887\\
7.69	-9.73018461624125\\
7.7	-9.73010784918814\\
7.71	-9.73003123225124\\
7.72	-9.72995477533282\\
7.73	-9.72987848580561\\
7.74	-9.72980236824631\\
7.75	-9.72972642420739\\
7.76	-9.72965065203155\\
7.77	-9.72957504671253\\
7.78	-9.72949959980506\\
7.79	-9.72942429938633\\
7.8	-9.72934913007024\\
7.81	-9.72927407307512\\
7.82	-9.72919910634457\\
7.83	-9.72912420472051\\
7.84	-9.72904934016639\\
7.85	-9.72897448203798\\
7.86	-9.72889959739819\\
7.87	-9.7288246513719\\
7.88	-9.72874960753591\\
7.89	-9.72867442833855\\
7.9	-9.72859907554322\\
7.91	-9.72852351068952\\
7.92	-9.72844769556518\\
7.93	-9.72837159268226\\
7.94	-9.7282951657504\\
7.95	-9.72821838014038\\
7.96	-9.72814120333095\\
7.97	-9.72806360533246\\
7.98	-9.72798555908086\\
7.99	-9.72790704079618\\
8	-9.72782803030003\\
8.01	-9.72774851128733\\
8.02	-9.72766847154809\\
8.03	-9.72758790313572\\
8.04	-9.72750680247933\\
8.05	-9.72742517043812\\
8.06	-9.72734301229689\\
8.07	-9.72726033770256\\
8.08	-9.72717716054257\\
8.09	-9.72709349876672\\
8.1	-9.7270093741551\\
8.11	-9.72692481203524\\
8.12	-9.72683984095279\\
8.13	-9.7267544923004\\
8.14	-9.72666879991027\\
8.15	-9.72658279961635\\
8.16	-9.72649652879271\\
8.17	-9.72641002587469\\
8.18	-9.72632332987021\\
8.19	-9.7262364798682\\
8.2	-9.72614951455149\\
8.21	-9.72606247172144\\
8.22	-9.72597538784117\\
8.23	-9.72588829760423\\
8.24	-9.72580123353492\\
8.25	-9.7257142256262\\
8.26	-9.72562730102028\\
8.27	-9.72554048373662\\
8.28	-9.72545379445101\\
8.29	-9.72536725032884\\
8.3	-9.72528086491473\\
8.31	-9.72519464807975\\
8.32	-9.7251086060267\\
8.33	-9.7250227413528\\
8.34	-9.72493705316845\\
8.35	-9.72485153726966\\
8.36	-9.72476618636106\\
8.37	-9.72468099032536\\
8.38	-9.7245959365347\\
8.39	-9.72451101019844\\
8.4	-9.72442619474142\\
8.41	-9.7243414722062\\
8.42	-9.72425682367255\\
8.43	-9.72417222968698\\
8.44	-9.72408767069507\\
8.45	-9.72400312746926\\
8.46	-9.72391858152488\\
8.47	-9.72383401551733\\
8.48	-9.72374941361361\\
8.49	-9.72366476183187\\
8.5	-9.72358004834321\\
8.51	-9.72349526373034\\
8.52	-9.72341040119872\\
8.53	-9.72332545673625\\
8.54	-9.72324042921882\\
8.55	-9.72315532045936\\
8.56	-9.72307013519955\\
8.57	-9.72298488104386\\
8.58	-9.72289956833671\\
8.59	-9.72281420998452\\
8.6	-9.7227288212252\\
8.61	-9.72264341934871\\
8.62	-9.72255802337291\\
8.63	-9.72247265367992\\
8.64	-9.72238733161872\\
8.65	-9.72230207908043\\
8.66	-9.72221691805311\\
8.67	-9.7221318701634\\
8.68	-9.72204695621254\\
8.69	-9.72196219571452\\
8.7	-9.72187760644398\\
8.71	-9.72179320400171\\
8.72	-9.721709001405\\
8.73	-9.72162500871007\\
8.74	-9.72154123267324\\
8.75	-9.72145767645682\\
8.76	-9.72137433938532\\
8.77	-9.72129121675667\\
8.78	-9.72120829971214\\
8.79	-9.72112557516822\\
8.8	-9.72104302581225\\
8.81	-9.72096063016306\\
8.82	-9.7208783626965\\
8.83	-9.72079619403495\\
8.84	-9.72071409119894\\
8.85	-9.72063201791775\\
8.86	-9.72054993499525\\
8.87	-9.72046780072609\\
8.88	-9.72038557135674\\
8.89	-9.72030320158495\\
8.9	-9.72022064509077\\
8.91	-9.72013785509153\\
8.92	-9.72005478491281\\
8.93	-9.71997138856733\\
8.94	-9.71988762133318\\
8.95	-9.71980344032292\\
8.96	-9.71971880503532\\
8.97	-9.71963367788134\\
8.98	-9.71954802467667\\
8.99	-9.71946181509339\\
9	-9.71937502306385\\
9.01	-9.71928762713079\\
9.02	-9.71919961073821\\
9.03	-9.71911096245853\\
9.04	-9.7190216761525\\
9.05	-9.71893175105929\\
9.06	-9.71884119181511\\
9.07	-9.71875000840012\\
9.08	-9.71865821601398\\
9.09	-9.71856583488186\\
9.1	-9.71847288999362\\
9.11	-9.71837941077978\\
9.12	-9.71828543072914\\
9.13	-9.71819098695337\\
9.14	-9.71809611970521\\
9.15	-9.71800087185712\\
9.16	-9.71790528834818\\
9.17	-9.71780941560738\\
9.18	-9.71771330096178\\
9.19	-9.7176169920383\\
9.2	-9.71752053616801\\
9.21	-9.71742397980164\\
9.22	-9.71732736794502\\
9.23	-9.7172307436227\\
9.24	-9.71713414737762\\
9.25	-9.71703761681414\\
9.26	-9.71694118619105\\
9.27	-9.71684488607047\\
9.28	-9.71674874302743\\
9.29	-9.71665277942442\\
9.3	-9.71655701325357\\
9.31	-9.71646145804877\\
9.32	-9.71636612286819\\
9.33	-9.71627101234727\\
9.34	-9.7161761268205\\
9.35	-9.71608146250976\\
9.36	-9.71598701177547\\
9.37	-9.71589276342626\\
9.38	-9.71579870308152\\
9.39	-9.71570481358057\\
9.4	-9.7156110754316\\
9.41	-9.7155174672924\\
9.42	-9.71542396647498\\
9.43	-9.71533054946535\\
9.44	-9.71523719244972\\
9.45	-9.71514387183817\\
9.46	-9.71505056477696\\
9.47	-9.71495724964074\\
9.48	-9.71486390649627\\
9.49	-9.7147705175298\\
9.5	-9.71467706743065\\
9.51	-9.71458354372451\\
9.52	-9.71448993705047\\
9.53	-9.714396241377\\
9.54	-9.71430245415292\\
9.55	-9.71420857639047\\
9.56	-9.71411461267874\\
9.57	-9.71402057112696\\
9.58	-9.71392646323794\\
9.59	-9.71383230371377\\
9.6	-9.71373811019612\\
9.61	-9.7136439029456\\
9.62	-9.71354970446468\\
9.63	-9.71345553907034\\
9.64	-9.71336143242329\\
9.65	-9.71326741102103\\
9.66	-9.71317350166331\\
9.67	-9.71307973089846\\
9.68	-9.71298612445972\\
9.69	-9.71289270670109\\
9.7	-9.71279950004185\\
9.71	-9.71270652442943\\
9.72	-9.71261379682954\\
9.73	-9.71252133075245\\
9.74	-9.71242913582386\\
9.75	-9.71233721740781\\
9.76	-9.71224557628867\\
9.77	-9.7121542084183\\
9.78	-9.71206310473328\\
9.79	-9.71197225104636\\
9.8	-9.71188162801498\\
9.81	-9.71179121118844\\
9.82	-9.71170097113431\\
9.83	-9.71161087364328\\
9.84	-9.71152088001045\\
9.85	-9.71143094738981\\
9.86	-9.71134102921762\\
9.87	-9.7112510756993\\
9.88	-9.71116103435306\\
9.89	-9.71107085060316\\
9.9	-9.71098046841444\\
9.91	-9.71088983095909\\
9.92	-9.7107988813063\\
9.93	-9.71070756312482\\
9.94	-9.71061582138814\\
9.95	-9.71052360307225\\
9.96	-9.7104308578353\\
9.97	-9.71033753866955\\
9.98	-9.71024360251558\\
9.99	-9.71014901082968\\
10	-9.71005373009591\\
10.01	-9.70995773227504\\
10.02	-9.70986099518342\\
10.03	-9.70976350279614\\
10.04	-9.70966524546957\\
10.05	-9.70956622007981\\
10.06	-9.70946643007479\\
10.07	-9.70936588543893\\
10.08	-9.70926460257069\\
10.09	-9.70916260407467\\
10.1	-9.70905991847096\\
10.11	-9.70895657982613\\
10.12	-9.70885262731086\\
10.13	-9.70874810469099\\
10.14	-9.70864305975905\\
10.15	-9.7085375437149\\
10.16	-9.70843161050437\\
10.17	-9.70832531612586\\
10.18	-9.70821871791487\\
10.19	-9.70811187381733\\
10.2	-9.70800484166216\\
10.21	-9.70789767844406\\
10.22	-9.70779043962691\\
10.23	-9.70768317847822\\
10.24	-9.70757594544425\\
10.25	-9.70746878757501\\
10.26	-9.70736174800748\\
10.27	-9.70725486551452\\
10.28	-9.70714817412588\\
10.29	-9.70704170282656\\
10.3	-9.70693547533665\\
10.31	-9.70682950997544\\
10.32	-9.70672381961123\\
10.33	-9.70661841169705\\
10.34	-9.70651328839095\\
10.35	-9.70640844675835\\
10.36	-9.70630387905264\\
10.37	-9.70619957306878\\
10.38	-9.70609551256382\\
10.39	-9.70599167773685\\
10.4	-9.70588804576017\\
10.41	-9.70578459135268\\
10.42	-9.70568128738549\\
10.43	-9.70557810550977\\
10.44	-9.70547501679591\\
10.45	-9.70537199237334\\
10.46	-9.7052690040601\\
10.47	-9.70516602497149\\
10.48	-9.70506303009743\\
10.49	-9.70495999683871\\
10.5	-9.7048569054929\\
10.51	-9.70475373968155\\
10.52	-9.70465048671127\\
10.53	-9.70454713786225\\
10.54	-9.70444368859916\\
10.55	-9.70434013870035\\
10.56	-9.70423649230288\\
10.57	-9.70413275786217\\
10.58	-9.70402894802628\\
10.59	-9.70392507942679\\
10.6	-9.70382117238887\\
10.61	-9.70371725056519\\
10.62	-9.70361334049916\\
10.63	-9.70350947112433\\
10.64	-9.70340567320797\\
10.65	-9.70330197874758\\
10.66	-9.70319842033024\\
10.67	-9.70309503046507\\
10.68	-9.70299184089993\\
10.69	-9.70288888193341\\
10.7	-9.70278618173379\\
10.71	-9.70268376567638\\
10.72	-9.70258165571042\\
10.73	-9.70247986976649\\
10.74	-9.70237842121477\\
10.75	-9.70227731838376\\
10.76	-9.70217656414821\\
10.77	-9.70207615559392\\
10.78	-9.70197608376608\\
10.79	-9.70187633350641\\
10.8	-9.70177688338309\\
10.81	-9.70167770571594\\
10.82	-9.70157876669808\\
10.83	-9.70148002661344\\
10.84	-9.70138144014839\\
10.85	-9.70128295679391\\
10.86	-9.70118452133359\\
10.87	-9.70108607441114\\
10.88	-9.70098755317002\\
10.89	-9.70088889195641\\
10.9	-9.70079002307589\\
10.91	-9.7006908775932\\
10.92	-9.7005913861637\\
10.93	-9.70049147988466\\
10.94	-9.70039109115393\\
10.95	-9.70029015452369\\
10.96	-9.70018860753656\\
10.97	-9.70008639153177\\
10.98	-9.69998345240949\\
10.99	-9.69987974134181\\
11	-9.69977521541972\\
11.01	-9.6996698382263\\
11.02	-9.69956358032743\\
11.03	-9.69945641967244\\
11.04	-9.69934834189847\\
11.05	-9.69923934053382\\
11.06	-9.69912941709687\\
11.07	-9.69901858108891\\
11.08	-9.69890684988057\\
11.09	-9.69879424849346\\
11.1	-9.69868080927979\\
11.11	-9.69856657150462\\
11.12	-9.69845158083669\\
11.13	-9.69833588875523\\
11.14	-9.69821955188132\\
11.15	-9.69810263124363\\
11.16	-9.69798519148945\\
11.17	-9.69786730005228\\
11.18	-9.69774902628871\\
11.19	-9.69763044059695\\
11.2	-9.69751161353024\\
11.21	-9.69739261491819\\
11.22	-9.69727351300894\\
11.23	-9.69715437364497\\
11.24	-9.6970352594844\\
11.25	-9.69691622927943\\
11.26	-9.69679733722234\\
11.27	-9.69667863236842\\
11.28	-9.69656015814427\\
11.29	-9.69644195194823\\
11.3	-9.69632404484843\\
11.31	-9.69620646138255\\
11.32	-9.6960892194614\\
11.33	-9.69597233037723\\
11.34	-9.69585579891572\\
11.35	-9.69573962356908\\
11.36	-9.69562379684617\\
11.37	-9.69550830567377\\
11.38	-9.69539313188201\\
11.39	-9.69527825276533\\
11.4	-9.69516364170934\\
11.41	-9.69504926887284\\
11.42	-9.69493510191327\\
11.43	-9.69482110674338\\
11.44	-9.69470724830624\\
11.45	-9.69459349135536\\
11.46	-9.69447980122688\\
11.47	-9.6943661445906\\
11.48	-9.69425249016708\\
11.49	-9.69413880939865\\
11.5	-9.69402507706285\\
11.51	-9.69391127181764\\
11.52	-9.69379737666906\\
11.53	-9.69368337935296\\
11.54	-9.69356927262419\\
11.55	-9.69345505444775\\
11.56	-9.69334072808835\\
11.57	-9.69322630209618\\
11.58	-9.69311179018879\\
11.59	-9.69299721103012\\
11.6	-9.69288258791014\\
11.61	-9.69276794832954\\
11.62	-9.69265332349603\\
11.63	-9.69253874773991\\
11.64	-9.69242425785831\\
11.65	-9.69230989239842\\
11.66	-9.69219569089128\\
11.67	-9.69208169304866\\
11.68	-9.69196793793602\\
11.69	-9.69185446313527\\
11.7	-9.6917413039113\\
11.71	-9.69162849239616\\
11.72	-9.6915160568047\\
11.73	-9.69140402069522\\
11.74	-9.69129240228789\\
11.75	-9.69118121385287\\
11.76	-9.69107046117936\\
11.77	-9.69096014313522\\
11.78	-9.69085025132577\\
11.79	-9.69074076985867\\
11.8	-9.69063167522035\\
11.81	-9.69052293626765\\
11.82	-9.69041451433647\\
11.83	-9.69030636346772\\
11.84	-9.69019843074862\\
11.85	-9.69009065676592\\
11.86	-9.68998297616572\\
11.87	-9.68987531831281\\
11.88	-9.6897676080409\\
11.89	-9.68965976648383\\
11.9	-9.68955171197603\\
11.91	-9.68944336101002\\
11.92	-9.68933462923708\\
11.93	-9.68922543249702\\
11.94	-9.68911568786205\\
11.95	-9.68900531467973\\
11.96	-9.68889423559944\\
11.97	-9.68878237756765\\
11.98	-9.68866967277685\\
11.99	-9.68855605955422\\
12	-9.68844148317668\\
12.01	-9.68832589659995\\
12.02	-9.68820926109061\\
12.03	-9.68809154675152\\
12.04	-9.68797273293235\\
12.05	-9.687852808519\\
12.06	-9.68773177209696\\
12.07	-9.6876096319862\\
12.08	-9.68748640614633\\
12.09	-9.6873621219534\\
12.1	-9.68723681585123\\
12.11	-9.68711053288203\\
12.12	-9.68698332610325\\
12.13	-9.68685525589882\\
12.14	-9.686726389195\\
12.15	-9.68659679859213\\
12.16	-9.68646656142518\\
12.17	-9.68633575876676\\
12.18	-9.68620447438737\\
12.19	-9.68607279368828\\
12.2	-9.68594080262269\\
12.21	-9.68580858662111\\
12.22	-9.68567622953697\\
12.23	-9.68554381262784\\
12.24	-9.68541141358729\\
12.25	-9.68527910564164\\
12.26	-9.68514695672475\\
12.27	-9.68501502874282\\
12.28	-9.68488337693976\\
12.29	-9.68475204937207\\
12.3	-9.68462108650067\\
12.31	-9.68449052090486\\
12.32	-9.68436037712216\\
12.33	-9.68423067161544\\
12.34	-9.68410141286694\\
12.35	-9.68397260159662\\
12.36	-9.68384423110059\\
12.37	-9.68371628770302\\
12.38	-9.68358875131365\\
12.39	-9.68346159608097\\
12.4	-9.68333479112972\\
12.41	-9.68320830137006\\
12.42	-9.68308208836462\\
12.43	-9.6829561112386\\
12.44	-9.68283032761755\\
12.45	-9.68270469457685\\
12.46	-9.68257916958688\\
12.47	-9.68245371143782\\
12.48	-9.68232828112827\\
12.49	-9.68220284270279\\
12.5	-9.68207736402377\\
12.51	-9.68195181746458\\
12.52	-9.68182618051188\\
12.53	-9.68170043626662\\
12.54	-9.68157457383477\\
12.55	-9.68144858860088\\
12.56	-9.68132248237899\\
12.57	-9.68119626343798\\
12.58	-9.68106994640015\\
12.59	-9.68094355201392\\
12.6	-9.68081710680394\\
12.61	-9.68069064260341\\
12.62	-9.68056419597608\\
12.63	-9.68043780753644\\
12.64	-9.68031152117918\\
12.65	-9.68018538322993\\
12.66	-9.680059441531\\
12.67	-9.67993374447687\\
12.68	-9.67980834001528\\
12.69	-9.67968327463024\\
12.7	-9.67955859232384\\
12.71	-9.67943433361399\\
12.72	-9.67931053456481\\
12.73	-9.67918722586643\\
12.74	-9.67906443197995\\
12.75	-9.67894217036259\\
12.76	-9.67882045078691\\
12.77	-9.67869927476656\\
12.78	-9.67857863509941\\
12.79	-9.67845851553732\\
12.8	-9.67833889058957\\
12.81	-9.67821972546542\\
12.82	-9.67810097615857\\
12.83	-9.67798258967448\\
12.84	-9.67786450439923\\
12.85	-9.6777466506062\\
12.86	-9.67762895109494\\
12.87	-9.67751132195427\\
12.88	-9.67739367343968\\
12.89	-9.67727591095364\\
12.9	-9.67715793611502\\
12.91	-9.67703964790307\\
12.92	-9.6769209438598\\
12.93	-9.67680172133348\\
12.94	-9.67668187874555\\
12.95	-9.67656131686248\\
12.96	-9.67643994005397\\
12.97	-9.67631765751915\\
12.98	-9.67619438446237\\
12.99	-9.67607004320141\\
13	-9.67594456419124\\
13.01	-9.67581788694814\\
13.02	-9.67568996086\\
13.03	-9.67556074587064\\
13.04	-9.67543021302741\\
13.05	-9.67529834488369\\
13.06	-9.6751651357498\\
13.07	-9.67503059178822\\
13.08	-9.67489473095126\\
13.09	-9.67475758276161\\
13.1	-9.67461918793879\\
13.11	-9.67447959787649\\
13.12	-9.67433887397832\\
13.13	-9.67419708686156\\
13.14	-9.67405431544027\\
13.15	-9.67391064590142\\
13.16	-9.67376617058867\\
13.17	-9.67362098681065\\
13.18	-9.67347519559083\\
13.19	-9.67332890037775\\
13.2	-9.67318220573453\\
13.21	-9.67303521602695\\
13.22	-9.67288803412961\\
13.23	-9.67274076016916\\
13.24	-9.67259349032304\\
13.25	-9.67244631569157\\
13.26	-9.6722993212597\\
13.27	-9.67215258496355\\
13.28	-9.67200617687529\\
13.29	-9.67186015851789\\
13.3	-9.6717145823193\\
13.31	-9.67156949121348\\
13.32	-9.67142491839347\\
13.33	-9.67128088721898\\
13.34	-9.67113741127889\\
13.35	-9.67099449460659\\
13.36	-9.67085213204328\\
13.37	-9.67071030974257\\
13.38	-9.67056900580705\\
13.39	-9.67042819104563\\
13.4	-9.67028782983838\\
13.41	-9.67014788109407\\
13.42	-9.67000829928392\\
13.43	-9.66986903553398\\
13.44	-9.66973003875763\\
13.45	-9.66959125680891\\
13.46	-9.6694526376372\\
13.47	-9.66931413042363\\
13.48	-9.66917568667996\\
13.49	-9.66903726129116\\
13.5	-9.66889881348398\\
13.51	-9.66876030770482\\
13.52	-9.66862171439181\\
13.53	-9.66848301062756\\
13.54	-9.66834418066116\\
13.55	-9.66820521628998\\
13.56	-9.66806611709418\\
13.57	-9.66792689051929\\
13.58	-9.66778755180446\\
13.59	-9.66764812375706\\
13.6	-9.66750863637612\\
13.61	-9.66736912633031\\
13.62	-9.66722963629817\\
13.63	-9.66709021418068\\
13.64	-9.6669509121987\\
13.65	-9.66681178588926\\
13.66	-9.66667289301714\\
13.67	-9.66653429241907\\
13.68	-9.66639604279933\\
13.69	-9.66625820149682\\
13.7	-9.6661208232436\\
13.71	-9.66598395893591\\
13.72	-9.66584765443814\\
13.73	-9.66571194944026\\
13.74	-9.6655768763882\\
13.75	-9.66544245950594\\
13.76	-9.66530871392659\\
13.77	-9.66517564494838\\
13.78	-9.66504324742925\\
13.79	-9.66491150533219\\
13.8	-9.66478039143068\\
13.81	-9.66464986718153\\
13.82	-9.6645198827696\\
13.83	-9.66439037732643\\
13.84	-9.6642612793219\\
13.85	-9.66413250712555\\
13.86	-9.66400396973126\\
13.87	-9.66387556763678\\
13.88	-9.66374719386663\\
13.89	-9.66361873512513\\
13.9	-9.66349007306368\\
13.91	-9.66336108564498\\
13.92	-9.66323164858498\\
13.93	-9.66310163685204\\
13.94	-9.66297092620192\\
13.95	-9.66283939472627\\
13.96	-9.66270692439221\\
13.97	-9.66257340255021\\
13.98	-9.66243872338829\\
13.99	-9.66230278931066\\
14	-9.66216551222042\\
14.01	-9.66202681468698\\
14.02	-9.66188663098059\\
14.03	-9.66174490795821\\
14.04	-9.66160160578717\\
14.05	-9.66145669849535\\
14.06	-9.66131017433928\\
14.07	-9.66116203598409\\
14.08	-9.66101230049217\\
14.09	-9.66086099912016\\
14.1	-9.66070817692682\\
14.11	-9.66055389219705\\
14.12	-9.66039821569034\\
14.13	-9.66024122972419\\
14.14	-9.66008302710594\\
14.15	-9.65992370992846\\
14.16	-9.65976338824737\\
14.17	-9.65960217865914\\
14.18	-9.659440202801\\
14.19	-9.6592775857949\\
14.2	-9.65911445465822\\
14.21	-9.65895093670496\\
14.22	-9.65878715796086\\
14.23	-9.65862324161585\\
14.24	-9.65845930653673\\
14.25	-9.65829546586171\\
14.26	-9.65813182569772\\
14.27	-9.6579684839392\\
14.28	-9.65780552922552\\
14.29	-9.65764304005211\\
14.3	-9.65748108404758\\
14.31	-9.65731971742693\\
14.32	-9.65715898462792\\
14.33	-9.65699891813488\\
14.34	-9.65683953849137\\
14.35	-9.65668085449988\\
14.36	-9.65652286360419\\
14.37	-9.65636555244666\\
14.38	-9.65620889759038\\
14.39	-9.65605286639322\\
14.4	-9.65589741801846\\
14.41	-9.65574250456459\\
14.42	-9.65558807229483\\
14.43	-9.65543406294547\\
14.44	-9.65528041509073\\
14.45	-9.65512706554099\\
14.46	-9.65497395075067\\
14.47	-9.65482100821197\\
14.48	-9.65466817781067\\
14.49	-9.65451540312107\\
14.5	-9.65436263261791\\
14.51	-9.65420982078448\\
14.52	-9.65405692909797\\
14.53	-9.65390392687473\\
14.54	-9.6537507919608\\
14.55	-9.6535975112553\\
14.56	-9.65344408105699\\
14.57	-9.65329050722745\\
14.58	-9.653136805167\\
14.59	-9.65298299960285\\
14.6	-9.65282912419187\\
14.61	-9.65267522094357\\
14.62	-9.65252133947189\\
14.63	-9.65236753608708\\
14.64	-9.65221387274203\\
14.65	-9.65206041584954\\
14.66	-9.65190723498942\\
14.67	-9.65175440152631\\
14.68	-9.65160198716068\\
14.69	-9.65145006243657\\
14.7	-9.6512986952309\\
14.71	-9.6511479492494\\
14.72	-9.65099788255425\\
14.73	-9.65084854614865\\
14.74	-9.6506999826423\\
14.75	-9.65055222502113\\
14.76	-9.65040529554282\\
14.77	-9.6502592047782\\
14.78	-9.65011395081609\\
14.79	-9.64996951864694\\
14.8	-9.64982587973801\\
14.81	-9.64968299180954\\
14.82	-9.64954079881887\\
14.83	-9.64939923115564\\
14.84	-9.64925820604844\\
14.85	-9.64911762817963\\
14.86	-9.64897739050198\\
14.87	-9.64883737524759\\
14.88	-9.6486974551162\\
14.89	-9.64855749462774\\
14.9	-9.64841735162049\\
14.91	-9.64827687887465\\
14.92	-9.64813592583838\\
14.93	-9.64799434043209\\
14.94	-9.64785197090505\\
14.95	-9.6477086677177\\
14.96	-9.64756428542212\\
14.97	-9.64741868451333\\
14.98	-9.64727173322399\\
14.99	-9.64712330923606\\
15	-9.64697330128381\\
15.01	-9.64682161062418\\
15.02	-9.64666815235246\\
15.03	-9.64651285654308\\
15.04	-9.64635566919837\\
15.05	-9.64619655299033\\
15.06	-9.64603548778412\\
15.07	-9.64587247093458\\
15.08	-9.6457075173509\\
15.09	-9.64554065932786\\
15.1	-9.64537194614536\\
15.11	-9.64520144344188\\
15.12	-9.64502923237029\\
15.13	-9.64485540854835\\
15.14	-9.64468008081878\\
15.15	-9.64450336983709\\
15.16	-9.64432540650766\\
15.17	-9.64414633029105\\
15.18	-9.64396628740743\\
15.19	-9.64378542896251\\
15.2	-9.64360390902374\\
15.21	-9.64342188267502\\
15.22	-9.64323950407875\\
15.23	-9.64305692457375\\
15.24	-9.64287429083716\\
15.25	-9.64269174313726\\
15.26	-9.64250941370289\\
15.27	-9.64232742523331\\
15.28	-9.64214588957002\\
15.29	-9.64196490654975\\
15.3	-9.64178456305479\\
15.31	-9.64160493227388\\
15.32	-9.64142607318346\\
15.33	-9.64124803025585\\
15.34	-9.64107083339699\\
15.35	-9.64089449811316\\
15.36	-9.64071902590213\\
15.37	-9.64054440486091\\
15.38	-9.64037061049879\\
15.39	-9.64019760674077\\
15.4	-9.64002534710413\\
15.41	-9.63985377602733\\
15.42	-9.63968283032871\\
15.43	-9.63951244076987\\
15.44	-9.63934253369721\\
15.45	-9.63917303273366\\
15.46	-9.63900386049209\\
15.47	-9.63883494028116\\
15.48	-9.63866619777472\\
15.49	-9.63849756261645\\
15.5	-9.63832896993234\\
15.51	-9.63816036172508\\
15.52	-9.63799168812651\\
15.53	-9.6378229084863\\
15.54	-9.63765399227794\\
15.55	-9.63748491980585\\
15.56	-9.63731568270088\\
15.57	-9.63714628419468\\
15.58	-9.6369767391673\\
15.59	-9.63680707396592\\
15.6	-9.63663732599643\\
15.61	-9.63646754309342\\
15.62	-9.63629778267778\\
15.63	-9.63612811071465\\
15.64	-9.6359586004879\\
15.65	-9.63578933121038\\
15.66	-9.63562038649213\\
15.67	-9.63545185269105\\
15.68	-9.63528381717288\\
15.69	-9.63511636650879\\
15.7	-9.63494958464033\\
15.71	-9.63478355104218\\
15.72	-9.63461833891334\\
15.73	-9.63445401342739\\
15.74	-9.63429063007174\\
15.75	-9.63412823310432\\
15.76	-9.6339668541551\\
15.77	-9.6338065109971\\
15.78	-9.63364720650966\\
15.79	-9.6334889278535\\
15.8	-9.63333164587407\\
15.81	-9.63317531474625\\
15.82	-9.63301987186968\\
15.83	-9.63286523802029\\
15.84	-9.63271131775943\\
15.85	-9.63255800009821\\
15.86	-9.63240515941054\\
15.87	-9.63225265658441\\
15.88	-9.63210034039724\\
15.89	-9.63194804909761\\
15.9	-9.63179561217213\\
15.91	-9.63164285227344\\
15.92	-9.63148958728251\\
15.93	-9.63133563247619\\
15.94	-9.63118080276925\\
15.95	-9.63102491499841\\
15.96	-9.63086779021563\\
15.97	-9.63070925595679\\
15.98	-9.63054914845274\\
15.99	-9.63038731474987\\
16	-9.63022361470871\\
16.01	-9.6300579228507\\
16.02	-9.6298901300254\\
16.03	-9.62972014487279\\
16.04	-9.62954789505856\\
16.05	-9.62937332826317\\
16.06	-9.62919641290956\\
16.07	-9.62901713861767\\
16.08	-9.62883551637842\\
16.09	-9.62865157844379\\
16.1	-9.62846537793381\\
16.11	-9.62827698816556\\
16.12	-9.62808650171354\\
16.13	-9.6278940292146\\
16.14	-9.62769969793475\\
16.15	-9.62750365011837\\
16.16	-9.6273060411442\\
16.17	-9.62710703751473\\
16.18	-9.62690681470882\\
16.19	-9.62670555492895\\
16.2	-9.62650344477633\\
16.21	-9.62630067288833\\
16.22	-9.62609742757277\\
16.23	-9.62589389447435\\
16.24	-9.62569025430736\\
16.25	-9.62548668068817\\
16.26	-9.62528333809926\\
16.27	-9.62508038001464\\
16.28	-9.62487794721386\\
16.29	-9.62467616630896\\
16.3	-9.62447514850535\\
16.31	-9.624274988614\\
16.32	-9.62407576432827\\
16.33	-9.62387753577475\\
16.34	-9.6236803453429\\
16.35	-9.62348421779421\\
16.36	-9.62328916064692\\
16.37	-9.62309516482813\\
16.38	-9.62290220558087\\
16.39	-9.62271024360966\\
16.4	-9.62251922644419\\
16.41	-9.62232908999754\\
16.42	-9.62213976029192\\
16.43	-9.62195115532245\\
16.44	-9.62176318702727\\
16.45	-9.6215757633304\\
16.46	-9.6213887902228\\
16.47	-9.62120217384617\\
16.48	-9.62101582254443\\
16.49	-9.62082964884762\\
16.5	-9.62064357135481\\
16.51	-9.62045751648362\\
16.52	-9.62027142005638\\
16.53	-9.62008522869548\\
16.54	-9.61989890100373\\
16.55	-9.61971240850866\\
16.56	-9.61952573635395\\
16.57	-9.61933888372502\\
16.58	-9.61915186400029\\
16.59	-9.61896470462408\\
16.6	-9.61877744670171\\
16.61	-9.61859014432202\\
16.62	-9.61840286361699\\
16.63	-9.61821568157258\\
16.64	-9.61802868460912\\
16.65	-9.61784196695353\\
16.66	-9.61765562882913\\
16.67	-9.61746977449199\\
16.68	-9.6172845101458\\
16.69	-9.61709994176894\\
16.7	-9.6169161728896\\
16.71	-9.61673330234574\\
16.72	-9.61655142206731\\
16.73	-9.61637061491788\\
16.74	-9.61619095263266\\
16.75	-9.61601249388809\\
16.76	-9.61583528253691\\
16.77	-9.61565934603987\\
16.78	-9.61548469412256\\
16.79	-9.61531131768267\\
16.8	-9.61513918796875\\
16.81	-9.6149682560481\\
16.82	-9.61479845257642\\
16.83	-9.61462968787758\\
16.84	-9.61446185233699\\
16.85	-9.61429481710704\\
16.86	-9.61412843511854\\
16.87	-9.61396254238671\\
16.88	-9.61379695959637\\
16.89	-9.61363149394579\\
16.9	-9.61346594122495\\
16.91	-9.61330008810007\\
16.92	-9.61313371457278\\
16.93	-9.61296659657948\\
16.94	-9.61279850869395\\
16.95	-9.61262922689462\\
16.96	-9.61245853135622\\
16.97	-9.61228620922554\\
16.98	-9.61211205734024\\
16.99	-9.61193588485084\\
17	-9.61175751570696\\
17.01	-9.61157679097068\\
17.02	-9.61139357092236\\
17.03	-9.6112077369271\\
17.04	-9.61101919303351\\
17.05	-9.61082786728036\\
17.06	-9.6106337126907\\
17.07	-9.61043670793801\\
17.08	-9.61023685767332\\
17.09	-9.61003419250781\\
17.1	-9.60982876864984\\
17.11	-9.60962066720127\\
17.12	-9.60940999312236\\
17.13	-9.60919687387997\\
17.14	-9.60898145779838\\
17.15	-9.60876391213644\\
17.16	-9.60854442091915\\
17.17	-9.60832318255525\\
17.18	-9.60810040727575\\
17.19	-9.60787631443112\\
17.2	-9.60765112968683\\
17.21	-9.60742508215873\\
17.22	-9.60719840153036\\
17.23	-9.6069713151949\\
17.24	-9.60674404546382\\
17.25	-9.60651680688326\\
17.26	-9.60628980369779\\
17.27	-9.60606322749852\\
17.28	-9.60583725508986\\
17.29	-9.60561204660593\\
17.3	-9.60538774390345\\
17.31	-9.60516446925388\\
17.32	-9.60494232435277\\
17.33	-9.60472138965928\\
17.34	-9.60450172407372\\
17.35	-9.60428336495568\\
17.36	-9.6040663284798\\
17.37	-9.60385061032098\\
17.38	-9.60363618665572\\
17.39	-9.60342301546103\\
17.4	-9.60321103808789\\
17.41	-9.60300018108167\\
17.42	-9.60279035821795\\
17.43	-9.60258147271906\\
17.44	-9.60237341961326\\
17.45	-9.60216608819655\\
17.46	-9.6019593645554\\
17.47	-9.60175313410743\\
17.48	-9.60154728411704\\
17.49	-9.60134170614328\\
17.5	-9.60113629837823\\
17.51	-9.600930967836\\
17.52	-9.60072563235481\\
17.53	-9.60052022237764\\
17.54	-9.60031468248049\\
17.55	-9.60010897262138\\
17.56	-9.59990306908777\\
17.57	-9.59969696512501\\
17.58	-9.59949067123358\\
17.59	-9.59928421512842\\
17.6	-9.59907764135914\\
17.61	-9.59887101059556\\
17.62	-9.5986643985885\\
17.63	-9.59845789482135\\
17.64	-9.59825160087302\\
17.65	-9.5980456285177\\
17.66	-9.59784009759164\\
17.67	-9.59763513366091\\
17.68	-9.59743086552779\\
17.69	-9.5972274226163\\
17.7	-9.59702493227982\\
17.71	-9.59682351707505\\
17.72	-9.59662329204781\\
17.73	-9.59642436207631\\
17.74	-9.59622681931666\\
17.75	-9.59603074079467\\
17.76	-9.59583618618552\\
17.77	-9.59564319582059\\
17.78	-9.59545178895724\\
17.79	-9.59526196234356\\
17.8	-9.59507368910578\\
17.81	-9.59488691798093\\
17.82	-9.59470157291242\\
17.83	-9.59451755302026\\
17.84	-9.59433473295236\\
17.85	-9.5941529636169\\
17.86	-9.59397207329019\\
17.87	-9.59379186908849\\
17.88	-9.59361213878631\\
17.89	-9.59343265295852\\
17.9	-9.59325316741809\\
17.91	-9.59307342591676\\
17.92	-9.59289316307147\\
17.93	-9.59271210747577\\
17.94	-9.59252998495213\\
17.95	-9.59234652189882\\
17.96	-9.5921614486829\\
17.97	-9.59197450303037\\
17.98	-9.59178543336348\\
17.99	-9.59159400203658\\
18	-9.59139998842216\\
18.01	-9.59120319180147\\
18.02	-9.59100343401638\\
18.03	-9.59080056184241\\
18.04	-9.59059444904729\\
18.05	-9.59038499810322\\
18.06	-9.59017214152683\\
18.07	-9.58995584282559\\
18.08	-9.58973609703565\\
18.09	-9.58951293084195\\
18.1	-9.58928640227766\\
18.11	-9.58905660000636\\
18.12	-9.58882364219644\\
18.13	-9.58858767500348\\
18.14	-9.58834887068222\\
18.15	-9.58810742535525\\
18.16	-9.58786355647081\\
18.17	-9.58761749998668\\
18.18	-9.5873695073216\\
18.19	-9.58711984211866\\
18.2	-9.58686877686857\\
18.21	-9.58661658944239\\
18.22	-9.58636355958491\\
18.23	-9.58610996542039\\
18.24	-9.58585608002223\\
18.25	-9.58560216809708\\
18.26	-9.58534848283215\\
18.27	-9.58509526295199\\
18.28	-9.58484273002773\\
18.29	-9.58459108607772\\
18.3	-9.58434051149434\\
18.31	-9.58409116332608\\
18.32	-9.58384317393905\\
18.33	-9.58359665007569\\
18.34	-9.58335167232242\\
18.35	-9.58310829499151\\
18.36	-9.58286654641592\\
18.37	-9.58262642964925\\
18.38	-9.58238792355676\\
18.39	-9.58215098427695\\
18.4	-9.58191554702752\\
18.41	-9.58168152822387\\
18.42	-9.58144882787345\\
18.43	-9.58121733220484\\
18.44	-9.5809869164865\\
18.45	-9.58075744798736\\
18.46	-9.58052878902877\\
18.47	-9.58030080007624\\
18.48	-9.58007334281835\\
18.49	-9.57984628318053\\
18.5	-9.57961949422263\\
18.51	-9.5793928588706\\
18.52	-9.57916627243564\\
18.53	-9.57893964487753\\
18.54	-9.57871290277266\\
18.55	-9.57848599095253\\
18.56	-9.57825887378353\\
18.57	-9.57803153606465\\
18.58	-9.57780398352624\\
18.59	-9.57757624291926\\
18.6	-9.57734836169127\\
18.61	-9.57712040725235\\
18.62	-9.57689246584073\\
18.63	-9.57666464100489\\
18.64	-9.57643705172501\\
18.65	-9.57620983020302\\
18.66	-9.57598311935604\\
18.67	-9.57575707005311\\
18.68	-9.5755318381397\\
18.69	-9.57530758129825\\
18.7	-9.57508445579601\\
18.71	-9.57486261317386\\
18.72	-9.57464219693099\\
18.73	-9.57442333926095\\
18.74	-9.57420615789423\\
18.75	-9.57399075310125\\
18.76	-9.57377720490755\\
18.77	-9.57356557056998\\
18.78	-9.573355882359\\
18.79	-9.57314814568768\\
18.8	-9.57294233762284\\
18.81	-9.57273840580809\\
18.82	-9.57253626782208\\
18.83	-9.57233581098896\\
18.84	-9.57213689265077\\
18.85	-9.57193934090445\\
18.86	-9.57174295579907\\
18.87	-9.57154751098127\\
18.88	-9.57135275577046\\
18.89	-9.57115841763784\\
18.9	-9.57096420505738\\
18.91	-9.57076981069057\\
18.92	-9.57057491486161\\
18.93	-9.57037918927455\\
18.94	-9.57018230092029\\
18.95	-9.5699839161176\\
18.96	-9.56978370463019\\
18.97	-9.56958134380018\\
18.98	-9.56937652263769\\
18.99	-9.56916894580646\\
19	-9.56895833744674\\
19.01	-9.56874444477853\\
19.02	-9.56852704143166\\
19.03	-9.5683059304523\\
19.04	-9.56808094694084\\
19.05	-9.56785196028083\\
19.06	-9.56761887592469\\
19.07	-9.56738163670863\\
19.08	-9.56714022367575\\
19.09	-9.56689465639379\\
19.1	-9.56664499276148\\
19.11	-9.5663913283049\\
19.12	-9.56613379497299\\
19.13	-9.56587255944889\\
19.14	-9.56560782100089\\
19.15	-9.56533980890384\\
19.16	-9.56506877946837\\
19.17	-9.56479501272102\\
19.18	-9.56451880878393\\
19.19	-9.56424048400703\\
19.2	-9.56396036690974\\
19.21	-9.56367879399174\\
19.22	-9.56339610547474\\
19.23	-9.56311264103783\\
19.24	-9.5628287356096\\
19.25	-9.56254471527867\\
19.26	-9.56226089338321\\
19.27	-9.5619775668365\\
19.28	-9.56169501274255\\
19.29	-9.56141348535085\\
19.3	-9.56113321339434\\
19.31	-9.56085439784849\\
19.32	-9.56057721014307\\
19.33	-9.56030179085068\\
19.34	-9.56002824886924\\
19.35	-9.55975666110727\\
19.36	-9.55948707267326\\
19.37	-9.55921949756234\\
19.38	-9.55895391982571\\
19.39	-9.55869029520039\\
19.4	-9.55842855316982\\
19.41	-9.55816859941885\\
19.42	-9.55791031864038\\
19.43	-9.55765357764523\\
19.44	-9.55739822872204\\
19.45	-9.55714411318996\\
19.46	-9.55689106508366\\
19.47	-9.5566389149082\\
19.48	-9.55638749339989\\
19.49	-9.55613663522946\\
19.5	-9.55588618258441\\
19.51	-9.55563598856956\\
19.52	-9.5553859203676\\
19.53	-9.55513586210524\\
19.54	-9.55488571737558\\
19.55	-9.55463541137238\\
19.56	-9.55438489259882\\
19.57	-9.55413413411969\\
19.58	-9.55388313433385\\
19.59	-9.55363191725133\\
19.6	-9.55338053226788\\
19.61	-9.55312905343793\\
19.62	-9.55287757825511\\
19.63	-9.55262622595806\\
19.64	-9.55237513538688\\
19.65	-9.55212446242325\\
19.66	-9.55187437705457\\
19.67	-9.55162506010844\\
19.68	-9.55137669971028\\
19.69	-9.551129487521\\
19.7	-9.55088361481637\\
19.71	-9.55063926847248\\
19.72	-9.55039662692367\\
19.73	-9.55015585616053\\
19.74	-9.54991710583537\\
19.75	-9.54968050554134\\
19.76	-9.54944616132922\\
19.77	-9.54921415252275\\
19.78	-9.54898452888881\\
19.79	-9.54875730821397\\
19.8	-9.54853247433258\\
19.81	-9.54830997564501\\
19.82	-9.54808972415721\\
19.83	-9.54787159506464\\
19.84	-9.54765542689553\\
19.85	-9.54744102221956\\
19.86	-9.54722814891938\\
19.87	-9.5470165420135\\
19.88	-9.54680590601066\\
19.89	-9.54659591776689\\
19.9	-9.54638622980911\\
19.91	-9.54617647408108\\
19.92	-9.54596626606094\\
19.93	-9.54575520919355\\
19.94	-9.5455428995753\\
19.95	-9.54532893082505\\
19.96	-9.54511289907138\\
19.97	-9.54489440798415\\
19.98	-9.54467307377703\\
19.99	-9.54444853010797\\
20	-9.54422043280513\\
20.01	-9.54398846434857\\
20.02	-9.54375233804062\\
20.03	-9.54351180180277\\
20.04	-9.54326664154157\\
20.05	-9.54301668403276\\
20.06	-9.54276179927923\\
20.07	-9.54250190230661\\
20.08	-9.54223695436807\\
20.09	-9.54196696353892\\
20.1	-9.5416919846904\\
20.11	-9.54141211884146\\
20.12	-9.54112751189644\\
20.13	-9.54083835278583\\
20.14	-9.54054487103624\\
20.15	-9.54024733380414\\
20.16	-9.53994604241636\\
20.17	-9.53964132846732\\
20.18	-9.53933354953006\\
20.19	-9.53902308454383\\
20.2	-9.53871032894592\\
20.21	-9.53839568961937\\
20.22	-9.53807957973094\\
20.23	-9.53776241353542\\
20.24	-9.53744460122297\\
20.25	-9.53712654388531\\
20.26	-9.53680862867497\\
20.27	-9.53649122422866\\
20.28	-9.53617467642192\\
20.29	-9.53585930451671\\
20.3	-9.53554539775808\\
20.31	-9.53523321246842\\
20.32	-9.53492296968057\\
20.33	-9.53461485334222\\
20.34	-9.53430900911548\\
20.35	-9.53400554378582\\
20.36	-9.53370452528516\\
20.37	-9.5334059833242\\
20.38	-9.53310991061946\\
20.39	-9.53281626469101\\
20.4	-9.53252497019782\\
20.41	-9.53223592176928\\
20.42	-9.53194898728323\\
20.43	-9.53166401153397\\
20.44	-9.53138082022716\\
20.45	-9.53109922423373\\
20.46	-9.53081902403003\\
20.47	-9.53054001424916\\
20.48	-9.53026198826601\\
20.49	-9.52998474273818\\
20.5	-9.52970808202565\\
20.51	-9.52943182241383\\
20.52	-9.52915579606779\\
20.53	-9.5288798546499\\
20.54	-9.52860387253827\\
20.55	-9.52832774959007\\
20.56	-9.52805141340107\\
20.57	-9.52777482102097\\
20.58	-9.52749796009293\\
20.59	-9.5272208493952\\
20.6	-9.52694353877262\\
20.61	-9.52666610845563\\
20.62	-9.5263886677749\\
20.63	-9.5261113532894\\
20.64	-9.52583432635599\\
20.65	-9.52555777017741\\
20.66	-9.52528188637525\\
20.67	-9.52500689114166\\
20.68	-9.52473301103178\\
20.69	-9.52446047846462\\
20.7	-9.52418952700568\\
20.71	-9.52392038650863\\
20.72	-9.52365327819648\\
20.73	-9.52338840976399\\
20.74	-9.52312597058367\\
20.75	-9.52286612709671\\
20.76	-9.52260901846771\\
20.77	-9.52235475257867\\
20.78	-9.52210340243292\\
20.79	-9.52185500303366\\
20.8	-9.52160954879475\\
20.81	-9.52136699153355\\
20.82	-9.52112723908679\\
20.83	-9.52089015458081\\
20.84	-9.52065555637774\\
20.85	-9.52042321870861\\
20.86	-9.52019287299366\\
20.87	-9.51996420983957\\
20.88	-9.51973688169263\\
20.89	-9.5195105061165\\
20.9	-9.51928466965325\\
20.91	-9.5190589322173\\
20.92	-9.51883283196281\\
20.93	-9.51860589055793\\
20.94	-9.51837761879204\\
20.95	-9.51814752243689\\
20.96	-9.5179151082778\\
20.97	-9.51767989022824\\
20.98	-9.51744139543881\\
20.99	-9.51719917031163\\
21	-9.51695278633156\\
21.01	-9.51670184562812\\
21.02	-9.51644598618548\\
21.03	-9.51618488662267\\
21.04	-9.51591827047207\\
21.05	-9.51564590989159\\
21.06	-9.51536762875388\\
21.07	-9.51508330506502\\
21.08	-9.51479287267505\\
21.09	-9.51449632225276\\
21.1	-9.51419370150854\\
21.11	-9.51388511465961\\
21.12	-9.51357072114385\\
21.13	-9.51325073359907\\
21.14	-9.51292541513607\\
21.15	-9.51259507594403\\
21.16	-9.5122600692771\\
21.17	-9.51192078688027\\
21.18	-9.51157765392101\\
21.19	-9.51123112350093\\
21.2	-9.51088167082773\\
21.21	-9.51052978713313\\
21.22	-9.51017597342639\\
21.23	-9.50982073417528\\
21.24	-9.50946457100769\\
21.25	-9.50910797652681\\
21.26	-9.50875142833082\\
21.27	-9.50839538332504\\
21.28	-9.50804027241007\\
21.29	-9.50768649562339\\
21.3	-9.50733441780508\\
21.31	-9.50698436484997\\
21.32	-9.50663662059967\\
21.33	-9.50629142441761\\
21.34	-9.50594896947983\\
21.35	-9.50560940180286\\
21.36	-9.50527282001846\\
21.37	-9.50493927589329\\
21.38	-9.50460877557988\\
21.39	-9.50428128157338\\
21.4	-9.50395671533783\\
21.41	-9.50363496055463\\
21.42	-9.50331586693623\\
21.43	-9.50299925453886\\
21.44	-9.5026849185002\\
21.45	-9.50237263412106\\
21.46	-9.50206216220435\\
21.47	-9.50175325456084\\
21.48	-9.50144565958798\\
21.49	-9.50113912782725\\
21.5	-9.50083341740535\\
21.51	-9.50052829926674\\
21.52	-9.50022356210807\\
21.53	-9.49991901692986\\
21.54	-9.49961450112707\\
21.55	-9.49930988204742\\
21.56	-9.4990050599551\\
21.57	-9.49869997034725\\
21.58	-9.49839458558078\\
21.59	-9.49808891577894\\
21.6	-9.49778300899843\\
21.61	-9.49747695065024\\
21.62	-9.49717086217972\\
21.63	-9.49686489902382\\
21.64	-9.49655924787532\\
21.65	-9.49625412329572\\
21.66	-9.49594976372945\\
21.67	-9.49564642698221\\
21.68	-9.49534438523568\\
21.69	-9.49504391967888\\
21.7	-9.49474531484337\\
21.71	-9.49444885273503\\
21.72	-9.49415480685916\\
21.73	-9.49386343623826\\
21.74	-9.49357497952266\\
21.75	-9.49328964929351\\
21.76	-9.49300762665554\\
21.77	-9.4927290562129\\
21.78	-9.49245404151619\\
21.79	-9.49218264106209\\
21.8	-9.49191486491858\\
21.81	-9.49165067203982\\
21.82	-9.491389968324\\
21.83	-9.49113260545644\\
21.84	-9.49087838056813\\
21.85	-9.49062703672721\\
21.86	-9.49037826426827\\
21.87	-9.49013170295106\\
21.88	-9.48988694492727\\
21.89	-9.48964353848148\\
21.9	-9.48940099249984\\
21.91	-9.48915878160873\\
21.92	-9.48891635191463\\
21.93	-9.48867312726695\\
21.94	-9.48842851595656\\
21.95	-9.48818191775599\\
21.96	-9.48793273120091\\
21.97	-9.48768036100837\\
21.98	-9.4874242255245\\
21.99	-9.48716376409304\\
22	-9.48689844423676\\
22.01	-9.48662776854579\\
22.02	-9.48635128117078\\
22.03	-9.48606857382414\\
22.04	-9.48577929119923\\
22.05	-9.485483135726\\
22.06	-9.4851798715906\\
22.07	-9.48486932795756\\
22.08	-9.48455140134448\\
22.09	-9.48422605711153\\
22.1	-9.48389333004151\\
22.11	-9.48355332399904\\
22.12	-9.48320621067167\\
22.13	-9.48285222740917\\
22.14	-9.48249167419046\\
22.15	-9.48212490976139\\
22.16	-9.4817523469982\\
22.17	-9.48137444756385\\
22.18	-9.48099171593477\\
22.19	-9.48060469288503\\
22.2	-9.48021394852346\\
22.21	-9.47982007498585\\
22.22	-9.47942367888959\\
22.23	-9.47902537366191\\
22.24	-9.47862577185469\\
22.25	-9.47822547755917\\
22.26	-9.4778250790321\\
22.27	-9.47742514164201\\
22.28	-9.477026201239\\
22.29	-9.47662875804506\\
22.3	-9.47623327115406\\
22.31	-9.4758401537206\\
22.32	-9.47544976890679\\
22.33	-9.47506242664371\\
22.34	-9.47467838125198\\
22.35	-9.47429782995203\\
22.36	-9.47392091228102\\
22.37	-9.47354771041892\\
22.38	-9.47317825041184\\
22.39	-9.47281250426681\\
22.4	-9.47245039287798\\
22.41	-9.47209178973133\\
22.42	-9.47173652532238\\
22.43	-9.47138439221004\\
22.44	-9.4710351506195\\
22.45	-9.47068853449831\\
22.46	-9.47034425792225\\
22.47	-9.47000202174224\\
22.48	-9.4696615203591\\
22.49	-9.46932244851135\\
22.5	-9.46898450796047\\
22.51	-9.46864741395989\\
22.52	-9.46831090139726\\
22.53	-9.46797473050468\\
22.54	-9.46763869203879\\
22.55	-9.46730261184077\\
22.56	-9.46696635469658\\
22.57	-9.46662982742926\\
22.58	-9.46629298116713\\
22.59	-9.46595581274592\\
22.6	-9.46561836521648\\
22.61	-9.46528072744477\\
22.62	-9.46494303280582\\
22.63	-9.46460545698837\\
22.64	-9.46426821494181\\
22.65	-9.46393155701134\\
22.66	-9.46359576432113\\
22.67	-9.46326114347818\\
22.68	-9.4629280206809\\
22.69	-9.46259673532729\\
22.7	-9.46226763322628\\
22.71	-9.46194105952311\\
22.72	-9.4616173514552\\
22.73	-9.4612968310586\\
22.74	-9.46097979794696\\
22.75	-9.46066652228476\\
22.76	-9.46035723807428\\
22.77	-9.46005213687195\\
22.78	-9.45975136204368\\
22.79	-9.4594550036609\\
22.8	-9.4591630941301\\
22.81	-9.45887560463725\\
22.82	-9.45859244247665\\
22.83	-9.45831344932014\\
22.84	-9.45803840046822\\
22.85	-9.4577670051097\\
22.86	-9.45749890760085\\
22.87	-9.45723368975891\\
22.88	-9.45697087414947\\
22.89	-9.45670992833124\\
22.9	-9.45645027000659\\
22.91	-9.45619127301221\\
22.92	-9.45593227407035\\
22.93	-9.45567258020909\\
22.94	-9.45541147674903\\
22.95	-9.45514823574449\\
22.96	-9.45488212475937\\
22.97	-9.45461241585221\\
22.98	-9.45433839464058\\
22.99	-9.45405936931321\\
23	-9.45377467945793\\
23.01	-9.45348370457569\\
23.02	-9.45318587215455\\
23.03	-9.45288066518373\\
23.04	-9.45256762899531\\
23.05	-9.45224637733066\\
23.06	-9.45191659753974\\
23.07	-9.45157805483399\\
23.08	-9.45123059552711\\
23.09	-9.45087414921291\\
23.1	-9.45050872984504\\
23.11	-9.45013443569952\\
23.12	-9.44975144821766\\
23.13	-9.44936002974347\\
23.14	-9.44896052018626\\
23.15	-9.44855333265514\\
23.16	-9.44813894812757\\
23.17	-9.44771790922866\\
23.18	-9.44729081321135\\
23.19	-9.44685830423951\\
23.2	-9.4464210650869\\
23.21	-9.44597980837358\\
23.22	-9.44553526746827\\
23.23	-9.44508818719079\\
23.24	-9.44463931445122\\
23.25	-9.44418938896373\\
23.26	-9.44373913417186\\
23.27	-9.44328924851879\\
23.28	-9.44284039719077\\
23.29	-9.4423932044548\\
23.3	-9.44194824670217\\
23.31	-9.44150604629893\\
23.32	-9.4410670663314\\
23.33	-9.44063170632129\\
23.34	-9.44020029896964\\
23.35	-9.43977310797277\\
23.36	-9.4393503269365\\
23.37	-9.43893207939797\\
23.38	-9.43851841994642\\
23.39	-9.43810933641731\\
23.4	-9.4377047531171\\
23.41	-9.43730453501939\\
23.42	-9.43690849285782\\
23.43	-9.43651638902689\\
23.44	-9.43612794418855\\
23.45	-9.43574284447142\\
23.46	-9.43536074913965\\
23.47	-9.43498129860085\\
23.48	-9.43460412261707\\
23.49	-9.43422884857921\\
23.5	-9.43385510970424\\
23.51	-9.43348255301567\\
23.52	-9.43311084697112\\
23.53	-9.43273968860638\\
23.54	-9.4323688100731\\
23.55	-9.43199798445717\\
23.56	-9.43162703077608\\
23.57	-9.4312558180674\\
23.58	-9.430884268495\\
23.59	-9.43051235941566\\
23.6	-9.43014012436606\\
23.61	-9.42976765294772\\
23.62	-9.42939508960572\\
23.63	-9.42902263131547\\
23.64	-9.42865052421\\
23.65	-9.42827905919807\\
23.66	-9.42790856664027\\
23.67	-9.42753941016659\\
23.68	-9.42717197973347\\
23.69	-9.42680668403162\\
23.7	-9.42644394236755\\
23.71	-9.426084176151\\
23.72	-9.42572780012821\\
23.73	-9.42537521350615\\
23.74	-9.42502679111561\\
23.75	-9.42468287476173\\
23.76	-9.42434376490884\\
23.77	-9.42400971284202\\
23.78	-9.42368091344161\\
23.79	-9.42335749869785\\
23.8	-9.42303953208252\\
23.81	-9.42272700388124\\
23.82	-9.42241982757615\\
23.83	-9.42211783735252\\
23.84	-9.42182078678576\\
23.85	-9.42152834874732\\
23.86	-9.42124011654895\\
23.87	-9.42095560632569\\
23.88	-9.42067426063866\\
23.89	-9.42039545325956\\
23.9	-9.42011849508012\\
23.91	-9.41984264107202\\
23.92	-9.41956709820594\\
23.93	-9.419291034223\\
23.94	-9.41901358713801\\
23.95	-9.41873387534181\\
23.96	-9.41845100816006\\
23.97	-9.41816409671764\\
23.98	-9.41787226495244\\
23.99	-9.41757466061882\\
24	-9.41727046612022\\
24.01	-9.41695890901192\\
24.02	-9.41663927201903\\
24.03	-9.41631090242111\\
24.04	-9.41597322066337\\
24.05	-9.41562572806532\\
24.06	-9.4152680135107\\
24.07	-9.41489975901665\\
24.08	-9.41452074409682\\
24.09	-9.4141308488505\\
24.1	-9.41373005572836\\
24.11	-9.413318449945\\
24.12	-9.41289621852872\\
24.13	-9.41246364801863\\
24.14	-9.41202112083985\\
24.15	-9.41156911040699\\
24.16	-9.41110817502533\\
24.17	-9.41063895067691\\
24.18	-9.41016214279589\\
24.19	-9.40967851715251\\
24.2	-9.40918888997865\\
24.21	-9.40869411747935\\
24.22	-9.40819508488428\\
24.23	-9.40769269519981\\
24.24	-9.40718785782738\\
24.25	-9.40668147721549\\
24.26	-9.4061744417126\\
24.27	-9.40566761278483\\
24.28	-9.40516181475704\\
24.29	-9.40465782522771\\
24.3	-9.40415636629767\\
24.31	-9.40365809674028\\
24.32	-9.40316360522582\\
24.33	-9.40267340469681\\
24.34	-9.40218792797282\\
24.35	-9.40170752464408\\
24.36	-9.40123245929328\\
24.37	-9.40076291106375\\
24.38	-9.40029897457129\\
24.39	-9.39984066213552\\
24.4	-9.39938790728572\\
24.41	-9.398940569476\\
24.42	-9.39849843992523\\
24.43	-9.39806124847926\\
24.44	-9.39762867137647\\
24.45	-9.39720033978334\\
24.46	-9.39677584895385\\
24.47	-9.39635476785695\\
24.48	-9.39593664910813\\
24.49	-9.39552103903632\\
24.5	-9.39510748771505\\
24.51	-9.39469555878698\\
24.52	-9.39428483891425\\
24.53	-9.39387494669307\\
24.54	-9.39346554087922\\
24.55	-9.39305632778253\\
24.56	-9.39264706770174\\
24.57	-9.39223758028649\\
24.58	-9.39182774873101\\
24.59	-9.39141752272299\\
24.6	-9.39100692009147\\
24.61	-9.39059602711931\\
24.62	-9.39018499750746\\
24.63	-9.38977405000114\\
24.64	-9.3893634647098\\
24.65	-9.3889535781753\\
24.66	-9.38854477726304\\
24.67	-9.38813749197171\\
24.68	-9.38773218727493\\
24.69	-9.38732935412559\\
24.7	-9.38692949976783\\
24.71	-9.38653313751464\\
24.72	-9.38614077615835\\
24.73	-9.38575290918917\\
24.74	-9.38537000400111\\
24.75	-9.38499249126628\\
24.76	-9.38462075465742\\
24.77	-9.38425512109415\\
24.78	-9.38389585168172\\
24.79	-9.3835431335009\\
24.8	-9.38319707239574\\
24.81	-9.38285768689089\\
24.82	-9.38252490335349\\
24.83	-9.38219855249582\\
24.84	-9.38187836729428\\
24.85	-9.38156398237891\\
24.86	-9.38125493492474\\
24.87	-9.38095066705304\\
24.88	-9.38065052972712\\
24.89	-9.38035378810391\\
24.9	-9.38005962827946\\
24.91	-9.37976716534495\\
24.92	-9.37947545264833\\
24.93	-9.37918349213783\\
24.94	-9.37889024564601\\
24.95	-9.37859464695739\\
24.96	-9.37829561448987\\
24.97	-9.3779920644096\\
24.98	-9.37768292399101\\
24.99	-9.37736714502893\\
25	-9.37704371710764\\
25.01	-9.37671168053255\\
25.02	-9.3763701387342\\
25.03	-9.37601826996103\\
25.04	-9.37565533808701\\
25.05	-9.3752807023724\\
25.06	-9.3748938260309\\
25.07	-9.37449428347318\\
25.08	-9.37408176611635\\
25.09	-9.37365608666923\\
25.1	-9.37321718182588\\
25.11	-9.37276511332318\\
25.12	-9.37230006734233\\
25.13	-9.37182235225852\\
25.14	-9.37133239476785\\
25.15	-9.37083073444452\\
25.16	-9.37031801680476\\
25.17	-9.36979498497659\\
25.18	-9.3692624700952\\
25.19	-9.36872138056313\\
25.2	-9.36817269033141\\
25.21	-9.367617426373\\
25.22	-9.36705665553174\\
25.23	-9.36649147093996\\
25.24	-9.3659229782043\\
25.25	-9.36535228156295\\
25.26	-9.36478047021807\\
25.27	-9.36420860504455\\
25.28	-9.36363770587041\\
25.29	-9.36306873951571\\
25.3	-9.36250260876484\\
25.31	-9.36194014243298\\
25.32	-9.36138208667047\\
25.33	-9.36082909762965\\
25.34	-9.36028173559751\\
25.35	-9.35974046067484\\
25.36	-9.359205630058\\
25.37	-9.35867749695461\\
25.38	-9.35815621113867\\
25.39	-9.35764182112458\\
25.4	-9.35713427791412\\
25.41	-9.35663344024534\\
25.42	-9.35613908124843\\
25.43	-9.35565089639103\\
25.44	-9.35516851257506\\
25.45	-9.3546914982281\\
25.46	-9.35421937421678\\
25.47	-9.35375162539591\\
25.48	-9.35328771259679\\
25.49	-9.35282708485083\\
25.5	-9.35236919164059\\
25.51	-9.35191349496971\\
25.52	-9.35145948104585\\
25.53	-9.35100667137681\\
25.54	-9.35055463308944\\
25.55	-9.35010298829327\\
25.56	-9.34965142232636\\
25.57	-9.34919969073871\\
25.58	-9.34874762488959\\
25.59	-9.34829513605726\\
25.6	-9.34784221798437\\
25.61	-9.34738894780794\\
25.62	-9.34693548534951\\
25.63	-9.34648207076823\\
25.64	-9.34602902060717\\
25.65	-9.34557672228969\\
25.66	-9.34512562714931\\
25.67	-9.34467624210074\\
25.68	-9.34422912008369\\
25.69	-9.34378484943169\\
25.7	-9.34334404233717\\
25.71	-9.34290732260035\\
25.72	-9.34247531286231\\
25.73	-9.34204862153279\\
25.74	-9.34162782962985\\
25.75	-9.34121347775164\\
25.76	-9.34080605340007\\
25.77	-9.34040597887233\\
25.78	-9.34001359992876\\
25.79	-9.33962917543468\\
25.8	-9.33925286815992\\
25.81	-9.33888473690266\\
25.82	-9.33852473008449\\
25.83	-9.33817268094147\\
25.84	-9.33782830441156\\
25.85	-9.33749119579314\\
25.86	-9.33716083122161\\
25.87	-9.33683656998347\\
25.88	-9.33651765865827\\
25.89	-9.33620323705054\\
25.9	-9.33589234584561\\
25.91	-9.33558393589601\\
25.92	-9.3352768790195\\
25.93	-9.33496998016518\\
25.94	-9.33466199078271\\
25.95	-9.33435162320947\\
25.96	-9.33403756587414\\
25.97	-9.33371849910105\\
25.98	-9.33339311128936\\
25.99	-9.33306011523361\\
26	-9.33271826434904\\
26.01	-9.33236636856446\\
26.02	-9.33200330964966\\
26.03	-9.33162805575076\\
26.04	-9.33123967491818\\
26.05	-9.3308373474252\\
26.06	-9.33042037669246\\
26.07	-9.32998819865335\\
26.08	-9.32954038941785\\
26.09	-9.32907667111677\\
26.1	-9.3285969158351\\
26.11	-9.32810114757096\\
26.12	-9.32758954218584\\
26.13	-9.32706242534134\\
26.14	-9.32652026844759\\
26.15	-9.32596368267787\\
26.16	-9.32539341113319\\
26.17	-9.32481031926778\\
26.18	-9.32421538371296\\
26.19	-9.32360967966058\\
26.2	-9.32299436698954\\
26.21	-9.32237067533707\\
26.22	-9.32173988833354\\
26.23	-9.32110332723139\\
26.24	-9.32046233416881\\
26.25	-9.31981825531389\\
26.26	-9.3191724241375\\
26.27	-9.31852614506067\\
26.28	-9.31788067771712\\
26.29	-9.31723722206201\\
26.3	-9.31659690454508\\
26.31	-9.31596076554998\\
26.32	-9.31532974828228\\
26.33	-9.31470468926583\\
26.34	-9.31408631058275\\
26.35	-9.31347521396457\\
26.36	-9.31287187681366\\
26.37	-9.31227665020345\\
26.38	-9.31168975887509\\
26.39	-9.31111130321662\\
26.4	-9.31054126317927\\
26.41	-9.30997950405499\\
26.42	-9.3094257840094\\
26.43	-9.30887976323647\\
26.44	-9.30834101457521\\
26.45	-9.30780903540498\\
26.46	-9.30728326061549\\
26.47	-9.30676307642993\\
26.48	-9.30624783484577\\
26.49	-9.30573686844762\\
26.5	-9.30522950533996\\
26.51	-9.30472508394566\\
26.52	-9.3042229674178\\
26.53	-9.30372255741851\\
26.54	-9.30322330702837\\
26.55	-9.30272473256388\\
26.56	-9.30222642409827\\
26.57	-9.30172805450157\\
26.58	-9.30122938684029\\
26.59	-9.30073028000363\\
26.6	-9.30023069245243\\
26.61	-9.29973068401772\\
26.62	-9.29923041570827\\
26.63	-9.29873014751925\\
26.64	-9.29823023426781\\
26.65	-9.297731119514\\
26.66	-9.29723332765806\\
26.67	-9.29673745433572\\
26.68	-9.29624415526233\\
26.69	-9.29575413370328\\
26.7	-9.29526812677218\\
26.71	-9.29478689077894\\
26.72	-9.29431118586721\\
26.73	-9.29384176019402\\
26.74	-9.29337933391396\\
26.75	-9.29292458323539\\
26.76	-9.29247812481702\\
26.77	-9.29204050076996\\
26.78	-9.29161216452245\\
26.79	-9.29119346779277\\
26.8	-9.29078464889999\\
26.81	-9.29038582262254\\
26.82	-9.2899969717916\\
26.83	-9.28961794078036\\
26.84	-9.2892484310211\\
26.85	-9.28888799865151\\
26.86	-9.28853605435842\\
26.87	-9.28819186545362\\
26.88	-9.28785456018148\\
26.89	-9.28752313422351\\
26.9	-9.28719645933032\\
26.91	-9.28687329397835\\
26.92	-9.28655229591623\\
26.93	-9.28623203643606\\
26.94	-9.28591101617674\\
26.95	-9.28558768224213\\
26.96	-9.28526044639488\\
26.97	-9.28492770406905\\
26.98	-9.2845878539303\\
26.99	-9.28423931770253\\
27	-9.28388055997397\\
27.01	-9.28351010769423\\
27.02	-9.28312656907685\\
27.03	-9.28272865162891\\
27.04	-9.28231517904116\\
27.05	-9.28188510668709\\
27.06	-9.28143753549947\\
27.07	-9.28097172401527\\
27.08	-9.28048709840655\\
27.09	-9.27998326034363\\
27.1	-9.27945999256859\\
27.11	-9.27891726209047\\
27.12	-9.27835522094835\\
27.13	-9.27777420452454\\
27.14	-9.27717472742603\\
27.15	-9.2765574769887\\
27.16	-9.27592330449393\\
27.17	-9.27527321422166\\
27.18	-9.27460835049599\\
27.19	-9.27392998290997\\
27.2	-9.2732394899434\\
27.21	-9.27253834121187\\
27.22	-9.27182807860599\\
27.23	-9.27111029659685\\
27.24	-9.27038662199628\\
27.25	-9.26965869346927\\
27.26	-9.26892814109952\\
27.27	-9.26819656630859\\
27.28	-9.26746552242417\\
27.29	-9.26673649618247\\
27.3	-9.26601089043651\\
27.31	-9.26529000832271\\
27.32	-9.26457503911643\\
27.33	-9.26386704598082\\
27.34	-9.26316695578393\\
27.35	-9.26247555112714\\
27.36	-9.2617934646933\\
27.37	-9.26112117598676\\
27.38	-9.26045901049983\\
27.39	-9.25980714130226\\
27.4	-9.2591655930115\\
27.41	-9.25853424806451\\
27.42	-9.2579128551743\\
27.43	-9.25730103982043\\
27.44	-9.25669831658929\\
27.45	-9.25610410315075\\
27.46	-9.25551773563083\\
27.47	-9.25493848511772\\
27.48	-9.25436557501957\\
27.49	-9.25379819897863\\
27.5	-9.25323553903684\\
27.51	-9.25267678374343\\
27.52	-9.25212114589577\\
27.53	-9.25156787961025\\
27.54	-9.25101629643042\\
27.55	-9.25046578019501\\
27.56	-9.24991580040841\\
27.57	-9.24936592388038\\
27.58	-9.24881582442987\\
27.59	-9.24826529047958\\
27.6	-9.24771423040227\\
27.61	-9.247162675517\\
27.62	-9.24661078067232\\
27.63	-9.24605882239329\\
27.64	-9.24550719461011\\
27.65	-9.24495640202626\\
27.66	-9.24440705122401\\
27.67	-9.24385983964341\\
27.68	-9.24331554260689\\
27.69	-9.24277499859545\\
27.7	-9.2422390930125\\
27.71	-9.24170874069837\\
27.72	-9.24118486748054\\
27.73	-9.24066839106298\\
27.74	-9.24016020157097\\
27.75	-9.23966114207554\\
27.76	-9.23917198942494\\
27.77	-9.23869343570769\\
27.78	-9.23822607066413\\
27.79	-9.23777036535088\\
27.8	-9.23732665734439\\
27.81	-9.23689513774762\\
27.82	-9.23647584023706\\
27.83	-9.23606863235673\\
27.84	-9.23567320923159\\
27.85	-9.23528908983605\\
27.86	-9.23491561591392\\
27.87	-9.23455195360523\\
27.88	-9.23419709779344\\
27.89	-9.23384987914411\\
27.9	-9.23350897376411\\
27.91	-9.23317291536932\\
27.92	-9.23284010980901\\
27.93	-9.23250885175791\\
27.94	-9.23217734335222\\
27.95	-9.2318437145146\\
27.96	-9.23150604468554\\
27.97	-9.23116238565526\\
27.98	-9.23081078517169\\
27.99	-9.23044931098607\\
28	-9.23007607498906\\
28.01	-9.22968925708688\\
28.02	-9.22928712846855\\
28.03	-9.22886807392274\\
28.04	-9.22843061287474\\
28.05	-9.2279734188316\\
28.06	-9.22749533694533\\
28.07	-9.22699539943099\\
28.08	-9.22647283860654\\
28.09	-9.22592709735605\\
28.1	-9.22535783685473\\
28.11	-9.22476494143454\\
28.12	-9.22414852051051\\
28.13	-9.22350890753146\\
28.14	-9.22284665596247\\
28.15	-9.22216253235037\\
28.16	-9.221457506567\\
28.17	-9.22073273936673\\
28.18	-9.2199895674352\\
28.19	-9.21922948614359\\
28.2	-9.21845413025728\\
28.21	-9.21766525287886\\
28.22	-9.21686470293194\\
28.23	-9.21605440151459\\
28.24	-9.21523631746854\\
28.25	-9.21441244252239\\
28.26	-9.21358476637381\\
28.27	-9.21275525207693\\
28.28	-9.21192581209661\\
28.29	-9.2110982853815\\
28.3	-9.21027441579224\\
28.31	-9.20945583220069\\
28.32	-9.20864403055029\\
28.33	-9.20784035813775\\
28.34	-9.20704600034186\\
28.35	-9.20626196998719\\
28.36	-9.2054890994896\\
28.37	-9.20472803588712\\
28.38	-9.20397923881404\\
28.39	-9.20324298143055\\
28.4	-9.2025193542728\\
28.41	-9.20180827194244\\
28.42	-9.20110948250941\\
28.43	-9.2004225794584\\
28.44	-9.19974701596849\\
28.45	-9.19908212127811\\
28.46	-9.19842711885359\\
28.47	-9.19778114605006\\
28.48	-9.19714327492939\\
28.49	-9.19651253388031\\
28.5	-9.19588792967259\\
28.51	-9.19526846956975\\
28.52	-9.19465318312279\\
28.53	-9.19404114327289\\
28.54	-9.19343148640083\\
28.55	-9.19282343097848\\
28.56	-9.19221629449951\\
28.57	-9.19160950839453\\
28.58	-9.19100263066855\\
28.59	-9.19039535603592\\
28.6	-9.1897875233687\\
28.61	-9.18917912031894\\
28.62	-9.18857028502184\\
28.63	-9.1879613048352\\
28.64	-9.18735261212018\\
28.65	-9.18674477711745\\
28.66	-9.18613849802224\\
28.67	-9.18553458840851\\
28.68	-9.18493396219813\\
28.69	-9.18433761641255\\
28.7	-9.18374661198322\\
28.71	-9.18316205293075\\
28.72	-9.18258506425209\\
28.73	-9.18201676887863\\
28.74	-9.18145826408608\\
28.75	-9.18091059774876\\
28.76	-9.18037474483643\\
28.77	-9.17985158455086\\
28.78	-9.17934187849194\\
28.79	-9.1788462502296\\
28.8	-9.17836516663781\\
28.81	-9.17789892132175\\
28.82	-9.17744762043773\\
28.83	-9.17701117117027\\
28.84	-9.17658927308977\\
28.85	-9.17618141257099\\
28.86	-9.17578686040508\\
28.87	-9.1754046726889\\
28.88	-9.17503369502456\\
28.89	-9.17467257001075\\
28.9	-9.17431974795578\\
28.91	-9.17397350069201\\
28.92	-9.17363193832252\\
28.93	-9.17329302868421\\
28.94	-9.17295461926863\\
28.95	-9.1726144613023\\
28.96	-9.17227023565318\\
28.97	-9.17191958020017\\
28.98	-9.17156011827792\\
28.99	-9.17118948779037\\
29	-9.170805370574\\
29.01	-9.17040552158542\\
29.02	-9.16998779748807\\
29.03	-9.16955018421937\\
29.04	-9.16909082313252\\
29.05	-9.16860803532641\\
29.06	-9.16810034380176\\
29.07	-9.16756649311238\\
29.08	-9.16700546621594\\
29.09	-9.16641649826852\\
29.1	-9.16579908715176\\
29.11	-9.16515300056842\\
29.12	-9.16447827959251\\
29.13	-9.16377523861185\\
29.14	-9.1630444616542\\
29.15	-9.16228679514117\\
29.16	-9.16150333716736\\
29.17	-9.16069542345344\\
29.18	-9.15986461017183\\
29.19	-9.15901265388997\\
29.2	-9.15814148891999\\
29.21	-9.15725320240213\\
29.22	-9.15635000748388\\
29.23	-9.15543421498578\\
29.24	-9.15450820396763\\
29.25	-9.15357439162636\\
29.26	-9.15263520296685\\
29.27	-9.15169304069108\\
29.28	-9.15075025574785\\
29.29	-9.14980911897594\\
29.3	-9.14887179425662\\
29.31	-9.14794031356923\\
29.32	-9.14701655431406\\
29.33	-9.14610221923238\\
29.34	-9.14519881921321\\
29.35	-9.14430765923206\\
29.36	-9.14342982761816\\
29.37	-9.14256618879468\\
29.38	-9.1417173795824\\
29.39	-9.140883809101\\
29.4	-9.14006566224557\\
29.41	-9.13926290665934\\
29.42	-9.13847530306786\\
29.43	-9.13770241878649\\
29.44	-9.13694364416173\\
29.45	-9.1361982116603\\
29.46	-9.13546521727617\\
29.47	-9.13474364388887\\
29.48	-9.13403238617383\\
29.49	-9.13333027664001\\
29.5	-9.13263611235131\\
29.51	-9.1319486818763\\
29.52	-9.13126679200653\\
29.53	-9.13058929378671\\
29.54	-9.12991510741063\\
29.55	-9.12924324555437\\
29.56	-9.12857283474367\\
29.57	-9.12790313438383\\
29.58	-9.12723355311822\\
29.59	-9.12656366222561\\
29.6	-9.12589320581447\\
29.61	-9.12522210762563\\
29.62	-9.12455047431021\\
29.63	-9.12387859510882\\
29.64	-9.12320693791742\\
29.65	-9.12253614178618\\
29.66	-9.12186700595718\\
29.67	-9.12120047560566\\
29.68	-9.12053762450534\\
29.69	-9.11987963489104\\
29.7	-9.11922777484001\\
29.71	-9.11858337353722\\
29.72	-9.11794779482665\\
29.73	-9.11732240948242\\
29.74	-9.11670856665735\\
29.75	-9.11610756498346\\
29.76	-9.11552062380816\\
29.77	-9.11494885505135\\
29.78	-9.11439323616167\\
29.79	-9.11385458463679\\
29.8	-9.11333353454991\\
29.81	-9.11283051549648\\
29.82	-9.11234573433893\\
29.83	-9.11187916008562\\
29.84	-9.11143051219273\\
29.85	-9.11099925252559\\
29.86	-9.11058458116007\\
29.87	-9.11018543614512\\
29.88	-9.10980049728679\\
29.89	-9.10942819395097\\
29.9	-9.10906671682006\\
29.91	-9.10871403347638\\
29.92	-9.10836790762554\\
29.93	-9.10802592171498\\
29.94	-9.10768550264971\\
29.95	-9.10734395025717\\
29.96	-9.10699846810947\\
29.97	-9.10664619627241\\
29.98	-9.10628424551893\\
29.99	-9.10590973251959\\
};
\addlegendentry{EE}

\addplot [color=green]
  table[row sep=crcr]{%
0	-9.77247004781056\\
0.01	-9.77250675833386\\
0.02	-9.77254334004203\\
0.03	-9.77257979487992\\
0.04	-9.77261612566675\\
0.05	-9.77265233605759\\
0.06	-9.77268843049412\\
0.07	-9.77272441414519\\
0.08	-9.77276029283815\\
0.09	-9.77279607298218\\
0.1	-9.77283176148443\\
0.11	-9.77286736566055\\
0.12	-9.7729028931408\\
0.13	-9.77293835177312\\
0.14	-9.77297374952466\\
0.15	-9.77300909438314\\
0.16	-9.77304439425957\\
0.17	-9.77307965689364\\
0.18	-9.77311488976316\\
0.19	-9.77315009999887\\
0.2	-9.7731852943058\\
0.21	-9.77322047889221\\
0.22	-9.77325565940709\\
0.23	-9.7732908408871\\
0.24	-9.77332602771351\\
0.25	-9.77336122357971\\
0.26	-9.77339643146963\\
0.27	-9.7734316536472\\
0.28	-9.77346689165687\\
0.29	-9.77350214633502\\
0.3	-9.77353741783187\\
0.31	-9.77357270564345\\
0.32	-9.77360800865295\\
0.33	-9.77364332518057\\
0.34	-9.77367865304108\\
0.35	-9.77371398960799\\
0.36	-9.77374933188315\\
0.37	-9.77378467657064\\
0.38	-9.77382002015365\\
0.39	-9.7738553589731\\
0.4	-9.77389068930659\\
0.41	-9.77392600744644\\
0.42	-9.7739613097755\\
0.43	-9.77399659283954\\
0.44	-9.77403185341501\\
0.45	-9.774067088571\\
0.46	-9.77410229572461\\
0.47	-9.77413747268867\\
0.48	-9.7741726177112\\
0.49	-9.77420772950582\\
0.5	-9.774242807273\\
0.51	-9.77427785071142\\
0.52	-9.7743128600197\\
0.53	-9.77434783588823\\
0.54	-9.77438277948153\\
0.55	-9.77441769241122\\
0.56	-9.7744525767004\\
0.57	-9.77448743473977\\
0.58	-9.77452226923653\\
0.59	-9.77455708315669\\
0.6	-9.77459187966204\\
0.61	-9.77462666204264\\
0.62	-9.77466143364611\\
0.63	-9.77469619780486\\
0.64	-9.77473095776254\\
0.65	-9.77476571660087\\
0.66	-9.77480047716823\\
0.67	-9.77483524201107\\
0.68	-9.77487001330941\\
0.69	-9.77490479281744\\
0.7	-9.77493958181039\\
0.71	-9.77497438103839\\
0.72	-9.77500919068827\\
0.73	-9.77504401035397\\
0.74	-9.77507883901598\\
0.75	-9.77511367503045\\
0.76	-9.77514851612795\\
0.77	-9.77518335942216\\
0.78	-9.77521820142836\\
0.79	-9.77525303809159\\
0.8	-9.77528786482392\\
0.81	-9.77532267655055\\
0.82	-9.77535746776397\\
0.83	-9.77539223258541\\
0.84	-9.77542696483276\\
0.85	-9.77546165809381\\
0.86	-9.77549630580401\\
0.87	-9.77553090132732\\
0.88	-9.77556543803913\\
0.89	-9.77559990940999\\
0.9	-9.7756343090889\\
0.91	-9.77566863098492\\
0.92	-9.77570286934588\\
0.93	-9.77573701883316\\
0.94	-9.77577107459117\\
0.95	-9.7758050323108\\
0.96	-9.77583888828575\\
0.97	-9.77587263946087\\
0.98	-9.77590628347198\\
0.99	-9.77593981867645\\
1	-9.77597324417421\\
1.01	-9.77600655981875\\
1.02	-9.77603976621819\\
1.03	-9.77607286472626\\
1.04	-9.77610585742339\\
1.05	-9.77613874708828\\
1.06	-9.7761715371603\\
1.07	-9.77620423169345\\
1.08	-9.77623683530239\\
1.09	-9.77626935310153\\
1.1	-9.77630179063807\\
1.11	-9.77633415381993\\
1.12	-9.77636644883965\\
1.13	-9.77639868209559\\
1.14	-9.77643086011121\\
1.15	-9.77646298945407\\
1.16	-9.77649507665535\\
1.17	-9.77652712813128\\
1.18	-9.77655915010752\\
1.19	-9.7765911485476\\
1.2	-9.7766231290864\\
1.21	-9.7766550969696\\
1.22	-9.77668705699987\\
1.23	-9.77671901349069\\
1.24	-9.77675097022806\\
1.25	-9.77678293044097\\
1.26	-9.77681489678056\\
1.27	-9.77684687130846\\
1.28	-9.77687885549416\\
1.29	-9.77691085022138\\
1.3	-9.77694285580326\\
1.31	-9.77697487200586\\
1.32	-9.77700689807961\\
1.33	-9.77703893279802\\
1.34	-9.7770709745029\\
1.35	-9.77710302115535\\
1.36	-9.77713507039154\\
1.37	-9.77716711958233\\
1.38	-9.77719916589572\\
1.39	-9.77723120636107\\
1.4	-9.77726323793396\\
1.41	-9.77729525756068\\
1.42	-9.77732726224122\\
1.43	-9.77735924908979\\
1.44	-9.77739121539186\\
1.45	-9.77742315865672\\
1.46	-9.77745507666492\\
1.47	-9.7774869675096\\
1.48	-9.77751882963126\\
1.49	-9.77755066184528\\
1.5	-9.77758246336194\\
1.51	-9.77761423379851\\
1.52	-9.77764597318331\\
1.53	-9.77767768195175\\
1.54	-9.77770936093438\\
1.55	-9.77774101133733\\
1.56	-9.77777263471527\\
1.57	-9.77780423293766\\
1.58	-9.77783580814865\\
1.59	-9.77786736272147\\
1.6	-9.77789889920799\\
1.61	-9.77793042028443\\
1.62	-9.77796192869407\\
1.63	-9.77799342718789\\
1.64	-9.7780249184643\\
1.65	-9.77805640510881\\
1.66	-9.77808788953478\\
1.67	-9.77811937392616\\
1.68	-9.77815086018332\\
1.69	-9.77818234987273\\
1.7	-9.77821384418151\\
1.71	-9.7782453438775\\
1.72	-9.77827684927565\\
1.73	-9.77830836021124\\
1.74	-9.77833987602052\\
1.75	-9.77837139552897\\
1.76	-9.77840291704769\\
1.77	-9.77843443837772\\
1.78	-9.7784659568226\\
1.79	-9.7784974692088\\
1.8	-9.77852897191386\\
1.81	-9.77856046090194\\
1.82	-9.77859193176606\\
1.83	-9.77862337977667\\
1.84	-9.77865479993576\\
1.85	-9.77868618703563\\
1.86	-9.77871753572161\\
1.87	-9.77874884055773\\
1.88	-9.77878009609438\\
1.89	-9.77881129693697\\
1.9	-9.77884243781459\\
1.91	-9.77887351364762\\
1.92	-9.77890451961328\\
1.93	-9.77893545120827\\
1.94	-9.7789663043074\\
1.95	-9.77899707521745\\
1.96	-9.77902776072549\\
1.97	-9.77905835814089\\
1.98	-9.77908886533041\\
1.99	-9.77911928074596\\
2	-9.77914960344445\\
2.01	-9.77917983309971\\
2.02	-9.77920997000606\\
2.03	-9.77924001507366\\
2.04	-9.77926996981573\\
2.05	-9.77929983632779\\
2.06	-9.77932961725921\\
2.07	-9.77935931577769\\
2.08	-9.77938893552704\\
2.09	-9.77941848057901\\
2.1	-9.77944795537982\\
2.11	-9.77947736469228\\
2.12	-9.77950671353434\\
2.13	-9.77953600711491\\
2.14	-9.77956525076809\\
2.15	-9.77959444988649\\
2.16	-9.77962360985495\\
2.17	-9.77965273598532\\
2.18	-9.77968183345347\\
2.19	-9.77971090723924\\
2.2	-9.77973996207039\\
2.21	-9.77976900237105\\
2.22	-9.77979803221569\\
2.23	-9.77982705528893\\
2.24	-9.77985607485194\\
2.25	-9.77988509371567\\
2.26	-9.77991411422136\\
2.27	-9.7799431382284\\
2.28	-9.77997216710966\\
2.29	-9.78000120175431\\
2.3	-9.78003024257784\\
2.31	-9.78005928953915\\
2.32	-9.78008834216422\\
2.33	-9.78011739957597\\
2.34	-9.78014646052969\\
2.35	-9.78017552345338\\
2.36	-9.78020458649239\\
2.37	-9.78023364755737\\
2.38	-9.7802627043749\\
2.39	-9.78029175453987\\
2.4	-9.78032079556863\\
2.41	-9.78034982495221\\
2.42	-9.78037884020855\\
2.43	-9.78040783893299\\
2.44	-9.78043681884614\\
2.45	-9.78046577783841\\
2.46	-9.78049471401043\\
2.47	-9.78052362570878\\
2.48	-9.78055251155638\\
2.49	-9.78058137047715\\
2.5	-9.78061020171457\\
2.51	-9.7806390048438\\
2.52	-9.78066777977724\\
2.53	-9.7806965267635\\
2.54	-9.78072524637978\\
2.55	-9.78075393951785\\
2.56	-9.78078260736386\\
2.57	-9.78081125137248\\
2.58	-9.78083987323559\\
2.59	-9.7808684748463\\
2.6	-9.78089705825879\\
2.61	-9.78092562564466\\
2.62	-9.78095417924665\\
2.63	-9.78098272133035\\
2.64	-9.78101125413485\\
2.65	-9.78103977982312\\
2.66	-9.78106830043293\\
2.67	-9.7810968178292\\
2.68	-9.78112533365855\\
2.69	-9.78115384930683\\
2.7	-9.78118236586036\\
2.71	-9.78121088407156\\
2.72	-9.78123940432951\\
2.73	-9.78126792663602\\
2.74	-9.78129645058766\\
2.75	-9.78132497536395\\
2.76	-9.78135349972217\\
2.77	-9.7813820219987\\
2.78	-9.78141054011704\\
2.79	-9.78143905160245\\
2.8	-9.78146755360301\\
2.81	-9.78149604291672\\
2.82	-9.78152451602448\\
2.83	-9.78155296912828\\
2.84	-9.78158139819415\\
2.85	-9.78160979899927\\
2.86	-9.78163816718243\\
2.87	-9.7816664982973\\
2.88	-9.78169478786749\\
2.89	-9.78172303144282\\
2.9	-9.78175122465579\\
2.91	-9.78177936327748\\
2.92	-9.78180744327207\\
2.93	-9.78183546084921\\
2.94	-9.78186341251332\\
2.95	-9.78189129510926\\
2.96	-9.78191910586364\\
2.97	-9.78194684242106\\
2.98	-9.78197450287497\\
2.99	-9.7820020857925\\
3	-9.78202959023297\\
3.01	-9.78205701575995\\
3.02	-9.78208436244642\\
3.03	-9.7821116308733\\
3.04	-9.78213882212112\\
3.05	-9.78216593775518\\
3.06	-9.78219297980423\\
3.07	-9.78221995073326\\
3.08	-9.78224685341056\\
3.09	-9.78227369106974\\
3.1	-9.78230046726721\\
3.11	-9.78232718583576\\
3.12	-9.78235385083503\\
3.13	-9.7823804664994\\
3.14	-9.78240703718434\\
3.15	-9.78243356731181\\
3.16	-9.78246006131559\\
3.17	-9.78248652358735\\
3.18	-9.78251295842421\\
3.19	-9.78253936997852\\
3.2	-9.78256576221059\\
3.21	-9.78259213884515\\
3.22	-9.78261850333183\\
3.23	-9.78264485881058\\
3.24	-9.78267120808216\\
3.25	-9.78269755358426\\
3.26	-9.78272389737343\\
3.27	-9.78275024111309\\
3.28	-9.78277658606766\\
3.29	-9.78280293310276\\
3.3	-9.78282928269152\\
3.31	-9.78285563492668\\
3.32	-9.78288198953825\\
3.33	-9.78290834591641\\
3.34	-9.78293470313908\\
3.35	-9.78296106000382\\
3.36	-9.78298741506336\\
3.37	-9.78301376666411\\
3.38	-9.78304011298713\\
3.39	-9.78306645209076\\
3.4	-9.78309278195409\\
3.41	-9.7831191005208\\
3.42	-9.78314540574236\\
3.43	-9.78317169562011\\
3.44	-9.78319796824531\\
3.45	-9.78322422183679\\
3.46	-9.78325045477528\\
3.47	-9.78327666563412\\
3.48	-9.78330285320576\\
3.49	-9.78332901652363\\
3.5	-9.78335515487908\\
3.51	-9.78338126783316\\
3.52	-9.78340735522299\\
3.53	-9.78343341716274\\
3.54	-9.78345945403914\\
3.55	-9.7834854665018\\
3.56	-9.78351145544822\\
3.57	-9.7835374220041\\
3.58	-9.7835633674991\\
3.59	-9.7835892934385\\
3.6	-9.78361520147131\\
3.61	-9.78364109335532\\
3.62	-9.78366697091975\\
3.63	-9.78369283602606\\
3.64	-9.78371869052761\\
3.65	-9.78374453622893\\
3.66	-9.78377037484519\\
3.67	-9.78379620796253\\
3.68	-9.78382203700013\\
3.69	-9.78384786317437\\
3.7	-9.78387368746594\\
3.71	-9.7838995105903\\
3.72	-9.78392533297213\\
3.73	-9.78395115472408\\
3.74	-9.78397697563035\\
3.75	-9.78400279513524\\
3.76	-9.78402861233704\\
3.77	-9.78405442598732\\
3.78	-9.78408023449571\\
3.79	-9.78410603594008\\
3.8	-9.78413182808204\\
3.81	-9.78415760838766\\
3.82	-9.78418337405291\\
3.83	-9.78420912203363\\
3.84	-9.78423484907959\\
3.85	-9.78426055177206\\
3.86	-9.78428622656442\\
3.87	-9.78431186982522\\
3.88	-9.78433747788296\\
3.89	-9.78436304707207\\
3.9	-9.7843885737793\\
3.91	-9.7844140544899\\
3.92	-9.78443948583283\\
3.93	-9.78446486462448\\
3.94	-9.78449018791003\\
3.95	-9.7845154530021\\
3.96	-9.78454065751588\\
3.97	-9.78456579940037\\
3.98	-9.78459087696531\\
3.99	-9.78461588890319\\
4	-9.78464083430632\\
4.01	-9.78466571267845\\
4.02	-9.78469052394099\\
4.03	-9.78471526843359\\
4.04	-9.78473994690922\\
4.05	-9.78476456052371\\
4.06	-9.78478911082007\\
4.07	-9.78481359970766\\
4.08	-9.78483802943669\\
4.09	-9.7848624025684\\
4.1	-9.78488672194127\\
4.11	-9.78491099063399\\
4.12	-9.78493521192544\\
4.13	-9.78495938925265\\
4.14	-9.78498352616701\\
4.15	-9.78500762628964\\
4.16	-9.78503169326644\\
4.17	-9.78505573072346\\
4.18	-9.78507974222341\\
4.19	-9.78510373122366\\
4.2	-9.78512770103663\\
4.21	-9.78515165479282\\
4.22	-9.78517559540729\\
4.23	-9.78519952554979\\
4.24	-9.7852234476191\\
4.25	-9.78524736372185\\
4.26	-9.785271275656\\
4.27	-9.78529518489938\\
4.28	-9.78531909260311\\
4.29	-9.78534299959011\\
4.3	-9.78536690635864\\
4.31	-9.78539081309063\\
4.32	-9.78541471966471\\
4.33	-9.78543862567359\\
4.34	-9.78546253044546\\
4.35	-9.78548643306908\\
4.36	-9.78551033242196\\
4.37	-9.78553422720131\\
4.38	-9.785558115957\\
4.39	-9.78558199712623\\
4.4	-9.78560586906904\\
4.41	-9.78562973010432\\
4.42	-9.7856535785455\\
4.43	-9.78567741273552\\
4.44	-9.78570123108032\\
4.45	-9.7857250320805\\
4.46	-9.78574881436048\\
4.47	-9.78577257669479\\
4.48	-9.785796318031\\
4.49	-9.78582003750903\\
4.5	-9.78584373447644\\
4.51	-9.78586740849956\\
4.52	-9.7858910593702\\
4.53	-9.78591468710796\\
4.54	-9.78593829195806\\
4.55	-9.78596187438479\\
4.56	-9.78598543506063\\
4.57	-9.78600897485139\\
4.58	-9.78603249479749\\
4.59	-9.78605599609175\\
4.6	-9.78607948005418\\
4.61	-9.78610294810402\\
4.62	-9.7861264017297\\
4.63	-9.78614984245707\\
4.64	-9.78617327181656\\
4.65	-9.78619669130971\\
4.66	-9.78622010237582\\
4.67	-9.78624350635903\\
4.68	-9.78626690447664\\
4.69	-9.78629029778903\\
4.7	-9.78631368717177\\
4.71	-9.78633707329042\\
4.72	-9.78636045657831\\
4.73	-9.78638383721798\\
4.74	-9.7864072151262\\
4.75	-9.78643058994327\\
4.76	-9.78645396102648\\
4.77	-9.78647732744798\\
4.78	-9.78650068799726\\
4.79	-9.78652404118796\\
4.8	-9.7865473852692\\
4.81	-9.78657071824117\\
4.82	-9.78659403787472\\
4.83	-9.78661734173478\\
4.84	-9.78664062720729\\
4.85	-9.78666389152907\\
4.86	-9.78668713182044\\
4.87	-9.78671034511993\\
4.88	-9.7867335284207\\
4.89	-9.78675667870798\\
4.9	-9.7867797929972\\
4.91	-9.78680286837194\\
4.92	-9.78682590202151\\
4.93	-9.78684889127723\\
4.94	-9.7868718336472\\
4.95	-9.7868947268488\\
4.96	-9.78691756883857\\
4.97	-9.78694035783903\\
4.98	-9.78696309236198\\
4.99	-9.786985771228\\
5	-9.78700839358184\\
5.01	-9.78703095890352\\
5.02	-9.78705346701492\\
5.03	-9.78707591808181\\
5.04	-9.78709831261134\\
5.05	-9.78712065144494\\
5.06	-9.7871429357469\\
5.07	-9.7871651669886\\
5.08	-9.7871873469289\\
5.09	-9.78720947759071\\
5.1	-9.78723156123435\\
5.11	-9.78725360032792\\
5.12	-9.78727559751527\\
5.13	-9.7872975555819\\
5.14	-9.78731947741947\\
5.15	-9.78734136598928\\
5.16	-9.78736322428535\\
5.17	-9.78738505529768\\
5.18	-9.7874068619761\\
5.19	-9.78742864719533\\
5.2	-9.78745041372171\\
5.21	-9.78747216418208\\
5.22	-9.78749390103522\\
5.23	-9.7875156265463\\
5.24	-9.78753734276459\\
5.25	-9.7875590515048\\
5.26	-9.78758075433221\\
5.27	-9.78760245255186\\
5.28	-9.78762414720176\\
5.29	-9.78764583905029\\
5.3	-9.78766752859775\\
5.31	-9.78768921608184\\
5.32	-9.78771090148711\\
5.33	-9.78773258455811\\
5.34	-9.78775426481582\\
5.35	-9.78777594157735\\
5.36	-9.78779761397822\\
5.37	-9.78781928099709\\
5.38	-9.78784094148235\\
5.39	-9.78786259418015\\
5.4	-9.78788423776339\\
5.41	-9.78790587086125\\
5.42	-9.7879274920886\\
5.43	-9.787949100075\\
5.44	-9.78797069349268\\
5.45	-9.78799227108308\\
5.46	-9.78801383168153\\
5.47	-9.7880353742397\\
5.48	-9.78805689784539\\
5.49	-9.7880784017394\\
5.5	-9.78809988532924\\
5.51	-9.78812134819944\\
5.52	-9.78814279011829\\
5.53	-9.78816421104094\\
5.54	-9.78818561110887\\
5.55	-9.78820699064571\\
5.56	-9.78822835014949\\
5.57	-9.78824969028147\\
5.58	-9.78827101185185\\
5.59	-9.78829231580245\\
5.6	-9.78831360318676\\
5.61	-9.78833487514773\\
5.62	-9.7883561328936\\
5.63	-9.78837737767224\\
5.64	-9.78839861074444\\
5.65	-9.78841983335652\\
5.66	-9.78844104671284\\
5.67	-9.78846225194862\\
5.68	-9.78848345010351\\
5.69	-9.78850464209639\\
5.7	-9.78852582870182\\
5.71	-9.78854701052853\\
5.72	-9.78856818800038\\
5.73	-9.78858936133995\\
5.74	-9.78861053055536\\
5.75	-9.78863169543016\\
5.76	-9.78865285551684\\
5.77	-9.78867401013389\\
5.78	-9.78869515836649\\
5.79	-9.78871629907095\\
5.8	-9.78873743088278\\
5.81	-9.78875855222834\\
5.82	-9.7887796613398\\
5.83	-9.78880075627342\\
5.84	-9.78882183493068\\
5.85	-9.78884289508207\\
5.86	-9.78886393439313\\
5.87	-9.7888849504525\\
5.88	-9.78890594080133\\
5.89	-9.78892690296387\\
5.9	-9.78894783447857\\
5.91	-9.78896873292947\\
5.92	-9.7889895959771\\
5.93	-9.78901042138878\\
5.94	-9.78903120706762\\
5.95	-9.78905195107984\\
5.96	-9.78907265168018\\
5.97	-9.7890933073347\\
5.98	-9.789113916741\\
5.99	-9.78913447884531\\
6	-9.78915499285627\\
6.01	-9.78917545825524\\
6.02	-9.78919587480298\\
6.03	-9.78921624254259\\
6.04	-9.78923656179871\\
6.05	-9.78925683317291\\
6.06	-9.78927705753551\\
6.07	-9.78929723601383\\
6.08	-9.78931736997706\\
6.09	-9.78933746101808\\
6.1	-9.78935751093243\\
6.11	-9.78937752169474\\
6.12	-9.78939749543303\\
6.13	-9.78941743440127\\
6.14	-9.78943734095051\\
6.15	-9.78945721749913\\
6.16	-9.78947706650263\\
6.17	-9.78949689042327\\
6.18	-9.78951669170022\\
6.19	-9.78953647272044\\
6.2	-9.78955623579089\\
6.21	-9.7895759831123\\
6.22	-9.78959571675499\\
6.23	-9.78961543863699\\
6.24	-9.78963515050483\\
6.25	-9.78965485391718\\
6.26	-9.78967455023155\\
6.27	-9.78969424059431\\
6.28	-9.78971392593399\\
6.29	-9.78973360695798\\
6.3	-9.78975328415265\\
6.31	-9.78977295778681\\
6.32	-9.78979262791838\\
6.33	-9.78981229440417\\
6.34	-9.78983195691253\\
6.35	-9.78985161493864\\
6.36	-9.78987126782215\\
6.37	-9.78989091476685\\
6.38	-9.78991055486203\\
6.39	-9.78993018710514\\
6.4	-9.78994981042542\\
6.41	-9.789969423708\\
6.42	-9.7899890258182\\
6.43	-9.79000861562544\\
6.44	-9.7900281920266\\
6.45	-9.79004775396821\\
6.46	-9.79006730046732\\
6.47	-9.79008683063056\\
6.48	-9.79010634367115\\
6.49	-9.79012583892362\\
6.5	-9.79014531585598\\
6.51	-9.79016477407905\\
6.52	-9.79018421335311\\
6.53	-9.79020363359138\\
6.54	-9.79022303486061\\
6.55	-9.79024241737866\\
6.56	-9.79026178150906\\
6.57	-9.79028112775286\\
6.58	-9.79030045673761\\
6.59	-9.79031976920405\\
6.6	-9.79033906599035\\
6.61	-9.79035834801451\\
6.62	-9.79037761625497\\
6.63	-9.79039687172992\\
6.64	-9.79041611547563\\
6.65	-9.79043534852404\\
6.66	-9.79045457188025\\
6.67	-9.79047378650001\\
6.68	-9.79049299326775\\
6.69	-9.79051219297554\\
6.7	-9.79053138630326\\
6.71	-9.79055057380027\\
6.72	-9.79056975586911\\
6.73	-9.79058893275128\\
6.74	-9.79060810451542\\
6.75	-9.79062727104824\\
6.76	-9.79064643204809\\
6.77	-9.79066558702159\\
6.78	-9.79068473528321\\
6.79	-9.79070387595785\\
6.8	-9.79072300798647\\
6.81	-9.79074213013472\\
6.82	-9.79076124100427\\
6.83	-9.79078033904702\\
6.84	-9.79079942258159\\
6.85	-9.79081848981222\\
6.86	-9.79083753884959\\
6.87	-9.79085656773327\\
6.88	-9.7908755744556\\
6.89	-9.79089455698651\\
6.9	-9.79091351329897\\
6.91	-9.79093244139474\\
6.92	-9.79095133932991\\
6.93	-9.79097020524004\\
6.94	-9.79098903736433\\
6.95	-9.79100783406859\\
6.96	-9.79102659386668\\
6.97	-9.79104531543996\\
6.98	-9.79106399765469\\
6.99	-9.79108263957691\\
7	-9.79110124048472\\
7.01	-9.79111979987771\\
7.02	-9.79113831748348\\
7.03	-9.79115679326106\\
7.04	-9.79117522740132\\
7.05	-9.7911936203242\\
7.06	-9.79121197267298\\
7.07	-9.79123028530562\\
7.08	-9.7912485592832\\
7.09	-9.79126679585586\\
7.1	-9.79128499644621\\
7.11	-9.79130316263073\\
7.12	-9.79132129611915\\
7.13	-9.79133939873235\\
7.14	-9.79135747237905\\
7.15	-9.79137551903155\\
7.16	-9.79139354070097\\
7.17	-9.79141153941237\\
7.18	-9.79142951718\\
7.19	-9.7914474759832\\
7.2	-9.7914654177431\\
7.21	-9.79148334430064\\
7.22	-9.79150125739612\\
7.23	-9.79151915865056\\
7.24	-9.79153704954915\\
7.25	-9.791554931427\\
7.26	-9.79157280545738\\
7.27	-9.79159067264258\\
7.28	-9.79160853380747\\
7.29	-9.7916263895959\\
7.3	-9.79164424046982\\
7.31	-9.7916620867112\\
7.32	-9.79167992842669\\
7.33	-9.79169776555474\\
7.34	-9.79171559787528\\
7.35	-9.79173342502154\\
7.36	-9.79175124649394\\
7.37	-9.79176906167567\\
7.38	-9.79178686984988\\
7.39	-9.7918046702179\\
7.4	-9.79182246191846\\
7.41	-9.79184024404738\\
7.42	-9.79185801567748\\
7.43	-9.79187577587843\\
7.44	-9.79189352373605\\
7.45	-9.79191125837097\\
7.46	-9.7919289789561\\
7.47	-9.79194668473285\\
7.48	-9.79196437502569\\
7.49	-9.79198204925491\\
7.5	-9.79199970694728\\
7.51	-9.79201734774459\\
7.52	-9.79203497140981\\
7.53	-9.79205257783085\\
7.54	-9.79207016702184\\
7.55	-9.79208773912195\\
7.56	-9.79210529439181\\
7.57	-9.79212283320746\\
7.58	-9.79214035605217\\
7.59	-9.79215786350602\\
7.6	-9.79217535623372\\
7.61	-9.79219283497054\\
7.62	-9.79221030050688\\
7.63	-9.79222775367164\\
7.64	-9.7922451953146\\
7.65	-9.79226262628822\\
7.66	-9.79228004742914\\
7.67	-9.79229745953962\\
7.68	-9.79231486336938\\
7.69	-9.79233225959794\\
7.7	-9.79234964881806\\
7.71	-9.79236703152014\\
7.72	-9.79238440807835\\
7.73	-9.79240177873825\\
7.74	-9.79241914360647\\
7.75	-9.79243650264244\\
7.76	-9.7924538556524\\
7.77	-9.79247120228576\\
7.78	-9.79248854203392\\
7.79	-9.79250587423156\\
7.8	-9.7925231980605\\
7.81	-9.79254051255585\\
7.82	-9.79255781661468\\
7.83	-9.79257510900685\\
7.84	-9.79259238838799\\
7.85	-9.79260965331436\\
7.86	-9.79262690225944\\
7.87	-9.79264413363198\\
7.88	-9.79266134579521\\
7.89	-9.79267853708703\\
7.9	-9.79269570584079\\
7.91	-9.79271285040633\\
7.92	-9.79272996917115\\
7.93	-9.79274706058108\\
7.94	-9.79276412316056\\
7.95	-9.79278115553179\\
7.96	-9.7927981564328\\
7.97	-9.79281512473406\\
7.98	-9.79283205945331\\
7.99	-9.7928489597685\\
8	-9.79286582502864\\
8.01	-9.79288265476234\\
8.02	-9.79289944868395\\
8.03	-9.79291620669729\\
8.04	-9.79293292889675\\
8.05	-9.79294961556595\\
8.06	-9.79296626717381\\
8.07	-9.79298288436823\\
8.08	-9.79299946796741\\
8.09	-9.79301601894894\\
8.1	-9.79303253843689\\
8.11	-9.793049027687\\
8.12	-9.79306548807035\\
8.13	-9.79308192105552\\
8.14	-9.79309832818976\\
8.15	-9.79311471107933\\
8.16	-9.79313107136923\\
8.17	-9.79314741072277\\
8.18	-9.79316373080116\\
8.19	-9.79318003324355\\
8.2	-9.79319631964756\\
8.21	-9.79321259155092\\
8.22	-9.79322885041415\\
8.23	-9.79324509760477\\
8.24	-9.7932613343831\\
8.25	-9.79327756188986\\
8.26	-9.7932937811359\\
8.27	-9.79330999299384\\
8.28	-9.79332619819213\\
8.29	-9.79334239731126\\
8.3	-9.79335859078228\\
8.31	-9.79337477888764\\
8.32	-9.79339096176424\\
8.33	-9.79340713940864\\
8.34	-9.79342331168429\\
8.35	-9.79343947833069\\
8.36	-9.79345563897425\\
8.37	-9.79347179314063\\
8.38	-9.79348794026846\\
8.39	-9.79350407972403\\
8.4	-9.79352021081685\\
8.41	-9.7935363328157\\
8.42	-9.793552444965\\
8.43	-9.79356854650109\\
8.44	-9.79358463666834\\
8.45	-9.79360071473462\\
8.46	-9.79361678000605\\
8.47	-9.79363283184064\\
8.48	-9.79364886966082\\
8.49	-9.79366489296432\\
8.5	-9.79368090133362\\
8.51	-9.79369689444352\\
8.52	-9.79371287206686\\
8.53	-9.79372883407829\\
8.54	-9.79374478045603\\
8.55	-9.79376071128153\\
8.56	-9.79377662673722\\
8.57	-9.79379252710222\\
8.58	-9.79380841274616\\
8.59	-9.79382428412128\\
8.6	-9.79384014175283\\
8.61	-9.79385598622806\\
8.62	-9.79387181818389\\
8.63	-9.79388763829355\\
8.64	-9.79390344725232\\
8.65	-9.79391924576281\\
8.66	-9.7939350345197\\
8.67	-9.79395081419456\\
8.68	-9.79396658542078\\
8.69	-9.79398234877889\\
8.7	-9.7939981047826\\
8.71	-9.79401385386578\\
8.72	-9.79402959637046\\
8.73	-9.79404533253631\\
8.74	-9.79406106249153\\
8.75	-9.79407678624551\\
8.76	-9.79409250368323\\
8.77	-9.79410821456164\\
8.78	-9.79412391850801\\
8.79	-9.79413961502027\\
8.8	-9.79415530346954\\
8.81	-9.79417098310445\\
8.82	-9.79418665305769\\
8.83	-9.79420231235425\\
8.84	-9.79421795992152\\
8.85	-9.79423359460099\\
8.86	-9.79424921516143\\
8.87	-9.79426482031334\\
8.88	-9.79428040872443\\
8.89	-9.79429597903596\\
8.9	-9.79431152987964\\
8.91	-9.79432705989494\\
8.92	-9.7943425677464\\
8.93	-9.79435805214084\\
8.94	-9.79437351184409\\
8.95	-9.7943889456971\\
8.96	-9.7944043526311\\
8.97	-9.79441973168168\\
8.98	-9.79443508200143\\
8.99	-9.7944504028712\\
9	-9.79446569370951\\
9.01	-9.79448095408028\\
9.02	-9.79449618369853\\
9.03	-9.79451138243411\\
9.04	-9.79452655031332\\
9.05	-9.79454168751851\\
9.06	-9.79455679438555\\
9.07	-9.79457187139926\\
9.08	-9.79458691918699\\
9.09	-9.79460193851019\\
9.1	-9.79461693025437\\
9.11	-9.79463189541747\\
9.12	-9.79464683509676\\
9.13	-9.79466175047468\\
9.14	-9.79467664280357\\
9.15	-9.7946915133897\\
9.16	-9.79470636357679\\
9.17	-9.79472119472928\\
9.18	-9.79473600821546\\
9.19	-9.79475080539098\\
9.2	-9.79476558758267\\
9.21	-9.79478035607312\\
9.22	-9.79479511208618\\
9.23	-9.79480985677351\\
9.24	-9.79482459120247\\
9.25	-9.79483931634545\\
9.26	-9.79485403307077\\
9.27	-9.79486874213534\\
9.28	-9.79488344417909\\
9.29	-9.79489813972123\\
9.3	-9.79491282915841\\
9.31	-9.7949275127648\\
9.32	-9.79494219069397\\
9.33	-9.79495686298257\\
9.34	-9.79497152955576\\
9.35	-9.79498619023417\\
9.36	-9.7950008447424\\
9.37	-9.79501549271877\\
9.38	-9.79503013372621\\
9.39	-9.79504476726409\\
9.4	-9.79505939278078\\
9.41	-9.79507400968674\\
9.42	-9.79508861736789\\
9.43	-9.79510321519902\\
9.44	-9.79511780255711\\
9.45	-9.79513237883424\\
9.46	-9.79514694344992\\
9.47	-9.79516149586263\\
9.48	-9.7951760355804\\
9.49	-9.79519056217027\\
9.5	-9.79520507526644\\
9.51	-9.79521957457701\\
9.52	-9.7952340598892\\
9.53	-9.79524853107301\\
9.54	-9.7952629880832\\
9.55	-9.79527743095961\\
9.56	-9.79529185982581\\
9.57	-9.79530627488615\\
9.58	-9.79532067642123\\
9.59	-9.7953350647818\\
9.6	-9.79534944038139\\
9.61	-9.79536380368762\\
9.62	-9.7953781552124\\
9.63	-9.79539249550127\\
9.64	-9.79540682512186\\
9.65	-9.79542114465193\\
9.66	-9.79543545466696\\
9.67	-9.79544975572764\\
9.68	-9.79546404836741\\
9.69	-9.79547833308031\\
9.7	-9.79549261030929\\
9.71	-9.79550688043517\\
9.72	-9.79552114376657\\
9.73	-9.7955354005308\\
9.74	-9.79554965086597\\
9.75	-9.79556389481447\\
9.76	-9.79557813231786\\
9.77	-9.79559236321337\\
9.78	-9.79560658723196\\
9.79	-9.79562080399811\\
9.8	-9.7956350130312\\
9.81	-9.79564921374865\\
9.82	-9.79566340547067\\
9.83	-9.79567758742652\\
9.84	-9.79569175876238\\
9.85	-9.7957059185505\\
9.86	-9.79572006579961\\
9.87	-9.79573419946649\\
9.88	-9.79574831846841\\
9.89	-9.79576242169633\\
9.9	-9.79577650802867\\
9.91	-9.79579057634542\\
9.92	-9.79580462554237\\
9.93	-9.79581865454533\\
9.94	-9.79583266232401\\
9.95	-9.79584664790544\\
9.96	-9.79586061038671\\
9.97	-9.79587454894688\\
9.98	-9.79588846285773\\
9.99	-9.79590235149343\\
10	-9.79591621433883\\
10.01	-9.79593005099621\\
10.02	-9.7959438611906\\
10.03	-9.79595764477335\\
10.04	-9.79597140172405\\
10.05	-9.7959851321508\\
10.06	-9.79599883628866\\
10.07	-9.79601251449651\\
10.08	-9.79602616725221\\
10.09	-9.79603979514631\\
10.1	-9.79605339887414\\
10.11	-9.79606697922675\\
10.12	-9.79608053708051\\
10.13	-9.79609407338575\\
10.14	-9.7961075891545\\
10.15	-9.79612108544758\\
10.16	-9.79613456336113\\
10.17	-9.79614802401296\\
10.18	-9.79616146852866\\
10.19	-9.79617489802798\\
10.2	-9.79618831361138\\
10.21	-9.79620171634721\\
10.22	-9.79621510725943\\
10.23	-9.79622848731633\\
10.24	-9.79624185742017\\
10.25	-9.79625521839798\\
10.26	-9.79626857099368\\
10.27	-9.79628191586157\\
10.28	-9.79629525356118\\
10.29	-9.79630858455377\\
10.3	-9.79632190920027\\
10.31	-9.79633522776079\\
10.32	-9.79634854039571\\
10.33	-9.79636184716817\\
10.34	-9.79637514804811\\
10.35	-9.79638844291753\\
10.36	-9.7964017315771\\
10.37	-9.79641501375378\\
10.38	-9.79642828910954\\
10.39	-9.79644155725077\\
10.4	-9.79645481773848\\
10.41	-9.79646807009886\\
10.42	-9.79648131383425\\
10.43	-9.79649454843415\\
10.44	-9.7965077733862\\
10.45	-9.79652098818694\\
10.46	-9.79653419235207\\
10.47	-9.7965473854262\\
10.48	-9.79656056699185\\
10.49	-9.79657373667749\\
10.5	-9.7965868941646\\
10.51	-9.79660003919367\\
10.52	-9.79661317156886\\
10.53	-9.79662629116148\\
10.54	-9.79663939791204\\
10.55	-9.79665249183101\\
10.56	-9.79666557299817\\
10.57	-9.7966786415606\\
10.58	-9.7966916977294\\
10.59	-9.79670474177508\\
10.6	-9.79671777402193\\
10.61	-9.79673079484112\\
10.62	-9.79674380464303\\
10.63	-9.79675680386866\\
10.64	-9.79676979298041\\
10.65	-9.79678277245235\\
10.66	-9.79679574276011\\
10.67	-9.79680870437063\\
10.68	-9.7968216577319\\
10.69	-9.79683460326286\\
10.7	-9.79684754134364\\
10.71	-9.79686047230634\\
10.72	-9.79687339642645\\
10.73	-9.79688631391504\\
10.74	-9.79689922491201\\
10.75	-9.7969121294803\\
10.76	-9.79692502760129\\
10.77	-9.79693791917149\\
10.78	-9.79695080400051\\
10.79	-9.7969636818104\\
10.8	-9.79697655223632\\
10.81	-9.7969894148287\\
10.82	-9.79700226905663\\
10.83	-9.79701511431264\\
10.84	-9.79702794991871\\
10.85	-9.79704077513344\\
10.86	-9.79705358916028\\
10.87	-9.79706639115672\\
10.88	-9.79707918024427\\
10.89	-9.79709195551914\\
10.9	-9.79710471606338\\
10.91	-9.79711746095643\\
10.92	-9.79713018928675\\
10.93	-9.79714290016358\\
10.94	-9.79715559272838\\
10.95	-9.79716826616608\\
10.96	-9.79718091971569\\
10.97	-9.79719355268035\\
10.98	-9.7972061644365\\
10.99	-9.79721875444213\\
11	-9.79723132224396\\
11.01	-9.7972438674835\\
11.02	-9.79725638990169\\
11.03	-9.79726888934245\\
11.04	-9.79728136575465\\
11.05	-9.79729381919278\\
11.06	-9.79730624981621\\
11.07	-9.79731865788705\\
11.08	-9.7973310437666\\
11.09	-9.7973434079106\\
11.1	-9.79735575086318\\
11.11	-9.7973680732497\\
11.12	-9.79738037576857\\
11.13	-9.79739265918217\\
11.14	-9.79740492430705\\
11.15	-9.79741717200339\\
11.16	-9.79742940316419\\
11.17	-9.797441618704\\
11.18	-9.7974538195476\\
11.19	-9.79746600661872\\
11.2	-9.79747818082892\\
11.21	-9.79749034306684\\
11.22	-9.797502494188\\
11.23	-9.79751463500522\\
11.24	-9.79752676627982\\
11.25	-9.79753888871378\\
11.26	-9.79755100294285\\
11.27	-9.79756310953082\\
11.28	-9.79757520896495\\
11.29	-9.79758730165257\\
11.3	-9.79759938791906\\
11.31	-9.79761146800696\\
11.32	-9.79762354207645\\
11.33	-9.79763561020702\\
11.34	-9.79764767240038\\
11.35	-9.7976597285844\\
11.36	-9.79767177861824\\
11.37	-9.79768382229831\\
11.38	-9.79769585936515\\
11.39	-9.79770788951096\\
11.4	-9.79771991238783\\
11.41	-9.79773192761627\\
11.42	-9.79774393479418\\
11.43	-9.79775593350584\\
11.44	-9.79776792333101\\
11.45	-9.79777990385381\\
11.46	-9.79779187467132\\
11.47	-9.79780383540172\\
11.48	-9.79781578569194\\
11.49	-9.79782772522449\\
11.5	-9.79783965372361\\
11.51	-9.79785157096043\\
11.52	-9.79786347675723\\
11.53	-9.79787537099057\\
11.54	-9.79788725359345\\
11.55	-9.79789912455622\\
11.56	-9.79791098392648\\
11.57	-9.79792283180776\\
11.58	-9.79793466835722\\
11.59	-9.7979464937822\\
11.6	-9.79795830833587\\
11.61	-9.79797011231189\\
11.62	-9.79798190603834\\
11.63	-9.79799368987087\\
11.64	-9.79800546418532\\
11.65	-9.79801722936985\\
11.66	-9.79802898581668\\
11.67	-9.79804073391377\\
11.68	-9.7980524740363\\
11.69	-9.79806420653839\\
11.7	-9.79807593174491\\
11.71	-9.79808764994378\\
11.72	-9.7980993613787\\
11.73	-9.79811106624256\\
11.74	-9.79812276467156\\
11.75	-9.79813445674014\\
11.76	-9.79814614245696\\
11.77	-9.79815782176173\\
11.78	-9.79816949452322\\
11.79	-9.79818116053834\\
11.8	-9.79819281953232\\
11.81	-9.79820447116007\\
11.82	-9.79821611500861\\
11.83	-9.79822775060067\\
11.84	-9.79823937739922\\
11.85	-9.79825099481311\\
11.86	-9.79826260220353\\
11.87	-9.79827419889127\\
11.88	-9.79828578416478\\
11.89	-9.79829735728871\\
11.9	-9.79830891751294\\
11.91	-9.79832046408202\\
11.92	-9.7983319962447\\
11.93	-9.79834351326351\\
11.94	-9.79835501442436\\
11.95	-9.79836649904577\\
11.96	-9.79837796648783\\
11.97	-9.79838941616055\\
11.98	-9.79840084753176\\
11.99	-9.79841226013404\\
12	-9.79842365357102\\
12.01	-9.79843502752256\\
12.02	-9.7984463817491\\
12.03	-9.79845771609476\\
12.04	-9.79846903048948\\
12.05	-9.79848032494988\\
12.06	-9.79849159957909\\
12.07	-9.79850285456529\\
12.08	-9.7985140901793\\
12.09	-9.79852530677089\\
12.1	-9.79853650476425\\
12.11	-9.79854768465233\\
12.12	-9.79855884699045\\
12.13	-9.79856999238905\\
12.14	-9.79858112150584\\
12.15	-9.79859223503731\\
12.16	-9.79860333370994\\
12.17	-9.79861441827102\\
12.18	-9.7986254894794\\
12.19	-9.79863654809614\\
12.2	-9.79864759487536\\
12.21	-9.79865863055527\\
12.22	-9.79866965584966\\
12.23	-9.79868067143978\\
12.24	-9.79869167796698\\
12.25	-9.79870267602588\\
12.26	-9.79871366615856\\
12.27	-9.79872464884944\\
12.28	-9.79873562452122\\
12.29	-9.7987465935318\\
12.3	-9.79875755617217\\
12.31	-9.79876851266543\\
12.32	-9.79877946316677\\
12.33	-9.79879040776454\\
12.34	-9.79880134648227\\
12.35	-9.79881227928168\\
12.36	-9.79882320606652\\
12.37	-9.79883412668726\\
12.38	-9.79884504094655\\
12.39	-9.79885594860515\\
12.4	-9.79886684938859\\
12.41	-9.79887774299405\\
12.42	-9.79888862909764\\
12.43	-9.79889950736181\\
12.44	-9.79891037744279\\
12.45	-9.79892123899793\\
12.46	-9.79893209169292\\
12.47	-9.79894293520858\\
12.48	-9.79895376924728\\
12.49	-9.79896459353878\\
12.5	-9.79897540784553\\
12.51	-9.79898621196714\\
12.52	-9.79899700574413\\
12.53	-9.79900778906088\\
12.54	-9.79901856184761\\
12.55	-9.79902932408148\\
12.56	-9.79904007578683\\
12.57	-9.7990508170344\\
12.58	-9.79906154793971\\
12.59	-9.79907226866055\\
12.6	-9.79908297939362\\
12.61	-9.79909368037045\\
12.62	-9.79910437185257\\
12.63	-9.79911505412613\\
12.64	-9.79912572749592\\
12.65	-9.79913639227907\\
12.66	-9.79914704879836\\
12.67	-9.79915769737534\\
12.68	-9.79916833832348\\
12.69	-9.79917897194114\\
12.7	-9.79918959850494\\
12.71	-9.79920021826322\\
12.72	-9.79921083142995\\
12.73	-9.79922143817913\\
12.74	-9.79923203863972\\
12.75	-9.79924263289131\\
12.76	-9.79925322096043\\
12.77	-9.79926380281778\\
12.78	-9.79927437837625\\
12.79	-9.79928494748983\\
12.8	-9.7992955099535\\
12.81	-9.79930606550398\\
12.82	-9.79931661382147\\
12.83	-9.79932715453225\\
12.84	-9.79933768721219\\
12.85	-9.79934821139105\\
12.86	-9.79935872655758\\
12.87	-9.79936923216527\\
12.88	-9.79937972763876\\
12.89	-9.79939021238073\\
12.9	-9.79940068577921\\
12.91	-9.79941114721522\\
12.92	-9.79942159607056\\
12.93	-9.79943203173575\\
12.94	-9.79944245361784\\
12.95	-9.79945286114817\\
12.96	-9.79946325378976\\
12.97	-9.79947363104443\\
12.98	-9.79948399245932\\
12.99	-9.79949433763296\\
13	-9.79950466622053\\
13.01	-9.79951497793855\\
13.02	-9.79952527256857\\
13.03	-9.79953554996014\\
13.04	-9.79954581003281\\
13.05	-9.7995560527772\\
13.06	-9.79956627825513\\
13.07	-9.79957648659882\\
13.08	-9.79958667800909\\
13.09	-9.79959685275275\\
13.1	-9.79960701115908\\
13.11	-9.79961715361547\\
13.12	-9.7996272805624\\
13.13	-9.79963739248768\\
13.14	-9.79964748992017\\
13.15	-9.79965757342296\\
13.16	-9.79966764358624\\
13.17	-9.79967770101985\\
13.18	-9.79968774634569\\
13.19	-9.79969778019004\\
13.2	-9.79970780317597\\
13.21	-9.79971781591597\\
13.22	-9.79972781900473\\
13.23	-9.79973781301245\\
13.24	-9.79974779847847\\
13.25	-9.79975777590562\\
13.26	-9.79976774575505\\
13.27	-9.79977770844189\\
13.28	-9.79978766433162\\
13.29	-9.79979761373724\\
13.3	-9.79980755691727\\
13.31	-9.79981749407465\\
13.32	-9.79982742535643\\
13.33	-9.7998373508544\\
13.34	-9.79984727060643\\
13.35	-9.79985718459873\\
13.36	-9.79986709276873\\
13.37	-9.79987699500875\\
13.38	-9.79988689117024\\
13.39	-9.79989678106859\\
13.4	-9.79990666448833\\
13.41	-9.7999165411888\\
13.42	-9.79992641090999\\
13.43	-9.79993627337862\\
13.44	-9.79994612831423\\
13.45	-9.79995597543532\\
13.46	-9.7999658144652\\
13.47	-9.79997564513783\\
13.48	-9.79998546720308\\
13.49	-9.79999528043183\\
13.5	-9.80000508462036\\
13.51	-9.80001487959431\\
13.52	-9.80002466521199\\
13.53	-9.80003444136696\\
13.54	-9.80004420799001\\
13.55	-9.80005396505024\\
13.56	-9.80006371255551\\
13.57	-9.80007345055213\\
13.58	-9.80008317912368\\
13.59	-9.80009289838925\\
13.6	-9.80010260850088\\
13.61	-9.80011230964043\\
13.62	-9.80012200201577\\
13.63	-9.80013168585653\\
13.64	-9.80014136140938\\
13.65	-9.80015102893286\\
13.66	-9.80016068869201\\
13.67	-9.80017034095283\\
13.68	-9.8001799859765\\
13.69	-9.80018962401377\\
13.7	-9.80019925529934\\
13.71	-9.80020888004643\\
13.72	-9.80021849844169\\
13.73	-9.80022811064043\\
13.74	-9.8002377167623\\
13.75	-9.80024731688754\\
13.76	-9.80025691105375\\
13.77	-9.80026649925335\\
13.78	-9.80027608143176\\
13.79	-9.80028565748621\\
13.8	-9.80029522726542\\
13.81	-9.80030479056996\\
13.82	-9.80031434715341\\
13.83	-9.80032389672426\\
13.84	-9.80033343894854\\
13.85	-9.80034297345316\\
13.86	-9.80035249982984\\
13.87	-9.80036201763973\\
13.88	-9.80037152641842\\
13.89	-9.80038102568154\\
13.9	-9.80039051493063\\
13.91	-9.80039999365934\\
13.92	-9.80040946135979\\
13.93	-9.80041891752907\\
13.94	-9.80042836167572\\
13.95	-9.80043779332606\\
13.96	-9.80044721203047\\
13.97	-9.80045661736925\\
13.98	-9.80046600895818\\
13.99	-9.80047538645362\\
14	-9.80048474955709\\
14.01	-9.80049409801927\\
14.02	-9.80050343164331\\
14.03	-9.80051275028746\\
14.04	-9.80052205386702\\
14.05	-9.80053134235545\\
14.06	-9.80054061578475\\
14.07	-9.80054987424504\\
14.08	-9.80055911788338\\
14.09	-9.80056834690182\\
14.1	-9.8005775615548\\
14.11	-9.80058676214573\\
14.12	-9.8005959490231\\
14.13	-9.8006051225759\\
14.14	-9.80061428322865\\
14.15	-9.80062343143589\\
14.16	-9.80063256767642\\
14.17	-9.80064169244725\\
14.18	-9.8006508062574\\
14.19	-9.80065990962162\\
14.2	-9.80066900305409\\
14.21	-9.80067808706232\\
14.22	-9.80068716214116\\
14.23	-9.80069622876715\\
14.24	-9.8007052873932\\
14.25	-9.8007143384437\\
14.26	-9.80072338231021\\
14.27	-9.80073241934761\\
14.28	-9.80074144987091\\
14.29	-9.80075047415273\\
14.3	-9.80075949242141\\
14.31	-9.80076850485986\\
14.32	-9.8007775116051\\
14.33	-9.80078651274849\\
14.34	-9.80079550833667\\
14.35	-9.80080449837314\\
14.36	-9.80081348282048\\
14.37	-9.80082246160315\\
14.38	-9.8008314346108\\
14.39	-9.80084040170211\\
14.4	-9.8008493627089\\
14.41	-9.80085831744073\\
14.42	-9.80086726568964\\
14.43	-9.80087620723507\\
14.44	-9.80088514184888\\
14.45	-9.80089406930039\\
14.46	-9.80090298936127\\
14.47	-9.80091190181034\\
14.48	-9.80092080643811\\
14.49	-9.80092970305093\\
14.5	-9.80093859147488\\
14.51	-9.80094747155912\\
14.52	-9.80095634317876\\
14.53	-9.80096520623725\\
14.54	-9.80097406066812\\
14.55	-9.80098290643613\\
14.56	-9.80099174353785\\
14.57	-9.80100057200156\\
14.58	-9.80100939188656\\
14.59	-9.80101820328185\\
14.6	-9.8010270063043\\
14.61	-9.80103580109614\\
14.62	-9.8010445878221\\
14.63	-9.80105336666602\\
14.64	-9.80106213782704\\
14.65	-9.8010709015155\\
14.66	-9.80107965794862\\
14.67	-9.80108840734592\\
14.68	-9.80109714992459\\
14.69	-9.80110588589477\\
14.7	-9.80111461545494\\
14.71	-9.80112333878747\\
14.72	-9.80113205605423\\
14.73	-9.80114076739263\\
14.74	-9.80114947291197\\
14.75	-9.80115817269013\\
14.76	-9.80116686677081\\
14.77	-9.80117555516124\\
14.78	-9.80118423783047\\
14.79	-9.80119291470825\\
14.8	-9.80120158568449\\
14.81	-9.80121025060937\\
14.82	-9.80121890929409\\
14.83	-9.80122756151223\\
14.84	-9.80123620700169\\
14.85	-9.80124484546725\\
14.86	-9.80125347658365\\
14.87	-9.80126209999918\\
14.88	-9.80127071533973\\
14.89	-9.8012793222132\\
14.9	-9.80128792021426\\
14.91	-9.80129650892935\\
14.92	-9.80130508794193\\
14.93	-9.8013136568377\\
14.94	-9.80132221520998\\
14.95	-9.80133076266499\\
14.96	-9.80133929882698\\
14.97	-9.80134782334319\\
14.98	-9.80135633588849\\
14.99	-9.80136483616975\\
15	-9.80137332392971\\
15.01	-9.80138179895043\\
15.02	-9.8013902610562\\
15.03	-9.80139871011595\\
15.04	-9.80140714604499\\
15.05	-9.80141556880614\\
15.06	-9.80142397841029\\
15.07	-9.80143237491626\\
15.08	-9.80144075843001\\
15.09	-9.80144912910332\\
15.1	-9.80145748713173\\
15.11	-9.80146583275205\\
15.12	-9.80147416623922\\
15.13	-9.80148248790278\\
15.14	-9.80149079808284\\
15.15	-9.80149909714574\\
15.16	-9.80150738547938\\
15.17	-9.80151566348827\\
15.18	-9.80152393158854\\
15.19	-9.80153219020273\\
15.2	-9.80154043975467\\
15.21	-9.80154868066436\\
15.22	-9.80155691334305\\
15.23	-9.80156513818847\\
15.24	-9.80157335558036\\
15.25	-9.80158156587637\\
15.26	-9.80158976940829\\
15.27	-9.8015979664788\\
15.28	-9.80160615735866\\
15.29	-9.80161434228445\\
15.3	-9.8016225214568\\
15.31	-9.80163069503933\\
15.32	-9.80163886315799\\
15.33	-9.80164702590111\\
15.34	-9.80165518331996\\
15.35	-9.80166333542989\\
15.36	-9.80167148221195\\
15.37	-9.80167962361509\\
15.38	-9.80168775955867\\
15.39	-9.80169588993551\\
15.4	-9.80170401461524\\
15.41	-9.80171213344788\\
15.42	-9.80172024626774\\
15.43	-9.80172835289739\\
15.44	-9.80173645315181\\
15.45	-9.80174454684249\\
15.46	-9.80175263378153\\
15.47	-9.80176071378559\\
15.48	-9.80176878667967\\
15.49	-9.80177685230073\\
15.5	-9.80178491050084\\
15.51	-9.80179296115018\\
15.52	-9.80180100413948\\
15.53	-9.80180903938217\\
15.54	-9.80181706681592\\
15.55	-9.80182508640382\\
15.56	-9.80183309813497\\
15.57	-9.80184110202461\\
15.58	-9.80184909811369\\
15.59	-9.801857086468\\
15.6	-9.80186506717674\\
15.61	-9.80187304035071\\
15.62	-9.80188100612003\\
15.63	-9.80188896463146\\
15.64	-9.80189691604543\\
15.65	-9.80190486053272\\
15.66	-9.80191279827099\\
15.67	-9.80192072944106\\
15.68	-9.80192865422309\\
15.69	-9.80193657279278\\
15.7	-9.80194448531753\\
15.71	-9.80195239195268\\
15.72	-9.80196029283795\\
15.73	-9.80196818809402\\
15.74	-9.80197607781945\\
15.75	-9.80198396208788\\
15.76	-9.80199184094558\\
15.77	-9.80199971440944\\
15.78	-9.80200758246542\\
15.79	-9.80201544506746\\
15.8	-9.80202330213683\\
15.81	-9.80203115356212\\
15.82	-9.80203899919962\\
15.83	-9.80204683887428\\
15.84	-9.80205467238116\\
15.85	-9.80206249948735\\
15.86	-9.80207031993439\\
15.87	-9.80207813344102\\
15.88	-9.80208593970645\\
15.89	-9.80209373841386\\
15.9	-9.8021015292342\\
15.91	-9.8021093118303\\
15.92	-9.80211708586103\\
15.93	-9.8021248509857\\
15.94	-9.80213260686835\\
15.95	-9.8021403531822\\
15.96	-9.80214808961386\\
15.97	-9.80215581586748\\
15.98	-9.80216353166866\\
15.99	-9.8021712367681\\
16	-9.80217893094494\\
16.01	-9.80218661400971\\
16.02	-9.80219428580688\\
16.03	-9.80220194621696\\
16.04	-9.80220959515814\\
16.05	-9.80221723258739\\
16.06	-9.80222485850104\\
16.07	-9.80223247293491\\
16.08	-9.80224007596381\\
16.09	-9.8022476677006\\
16.1	-9.80225524829472\\
16.11	-9.80226281793022\\
16.12	-9.80227037682341\\
16.13	-9.80227792522002\\
16.14	-9.80228546339205\\
16.15	-9.80229299163427\\
16.16	-9.80230051026046\\
16.17	-9.80230801959943\\
16.18	-9.80231551999094\\
16.19	-9.80232301178144\\
16.2	-9.80233049531985\\
16.21	-9.80233797095335\\
16.22	-9.80234543902329\\
16.23	-9.80235289986118\\
16.24	-9.80236035378496\\
16.25	-9.80236780109556\\
16.26	-9.80237524207361\\
16.27	-9.80238267697675\\
16.28	-9.80239010603708\\
16.29	-9.80239752945919\\
16.3	-9.80240494741864\\
16.31	-9.80241236006077\\
16.32	-9.80241976750013\\
16.33	-9.80242716982029\\
16.34	-9.8024345670742\\
16.35	-9.8024419592849\\
16.36	-9.80244934644681\\
16.37	-9.80245672852729\\
16.38	-9.80246410546867\\
16.39	-9.80247147719064\\
16.4	-9.80247884359284\\
16.41	-9.80248620455782\\
16.42	-9.80249355995411\\
16.43	-9.80250090963952\\
16.44	-9.80250825346445\\
16.45	-9.8025155912753\\
16.46	-9.80252292291787\\
16.47	-9.80253024824059\\
16.48	-9.80253756709776\\
16.49	-9.8025448793525\\
16.5	-9.80255218487951\\
16.51	-9.80255948356756\\
16.52	-9.80256677532168\\
16.53	-9.80257406006498\\
16.54	-9.80258133774011\\
16.55	-9.80258860831031\\
16.56	-9.80259587176008\\
16.57	-9.80260312809537\\
16.58	-9.80261037734344\\
16.59	-9.80261761955225\\
16.6	-9.80262485478938\\
16.61	-9.80263208314075\\
16.62	-9.80263930470876\\
16.63	-9.80264651961028\\
16.64	-9.80265372797425\\
16.65	-9.80266092993907\\
16.66	-9.80266812564973\\
16.67	-9.80267531525487\\
16.68	-9.80268249890366\\
16.69	-9.80268967674266\\
16.7	-9.80269684891264\\
16.71	-9.8027040155455\\
16.72	-9.8027111767613\\
16.73	-9.80271833266536\\
16.74	-9.80272548334565\\
16.75	-9.80273262887042\\
16.76	-9.80273976928608\\
16.77	-9.80274690461542\\
16.78	-9.80275403485621\\
16.79	-9.80276115998017\\
16.8	-9.80276827993233\\
16.81	-9.80277539463081\\
16.82	-9.80278250396707\\
16.83	-9.8027896078065\\
16.84	-9.80279670598949\\
16.85	-9.80280379833291\\
16.86	-9.80281088463189\\
16.87	-9.80281796466207\\
16.88	-9.80282503818209\\
16.89	-9.80283210493646\\
16.9	-9.80283916465852\\
16.91	-9.80284621707384\\
16.92	-9.80285326190354\\
16.93	-9.80286029886788\\
16.94	-9.80286732768984\\
16.95	-9.8028743480987\\
16.96	-9.8028813598336\\
16.97	-9.80288836264694\\
16.98	-9.8028953563077\\
16.99	-9.80290234060447\\
17	-9.80290931534833\\
17.01	-9.80291628037536\\
17.02	-9.80292323554885\\
17.03	-9.80293018076118\\
17.04	-9.80293711593522\\
17.05	-9.80294404102548\\
17.06	-9.80295095601867\\
17.07	-9.80295786093395\\
17.08	-9.80296475582271\\
17.09	-9.80297164076786\\
17.1	-9.80297851588284\\
17.11	-9.8029853813101\\
17.12	-9.80299223721927\\
17.13	-9.80299908380496\\
17.14	-9.80300592128427\\
17.15	-9.80301274989397\\
17.16	-9.80301956988753\\
17.17	-9.80302638153188\\
17.18	-9.80303318510407\\
17.19	-9.80303998088782\\
17.2	-9.80304676917006\\
17.21	-9.80305355023745\\
17.22	-9.80306032437294\\
17.23	-9.80306709185252\\
17.24	-9.80307385294204\\
17.25	-9.80308060789429\\
17.26	-9.80308735694627\\
17.27	-9.8030941003168\\
17.28	-9.80310083820438\\
17.29	-9.80310757078544\\
17.3	-9.80311429821291\\
17.31	-9.8031210206152\\
17.32	-9.80312773809556\\
17.33	-9.80313445073176\\
17.34	-9.80314115857632\\
17.35	-9.80314786165692\\
17.36	-9.80315455997736\\
17.37	-9.80316125351872\\
17.38	-9.80316794224096\\
17.39	-9.80317462608473\\
17.4	-9.80318130497347\\
17.41	-9.80318797881576\\
17.42	-9.80319464750779\\
17.43	-9.80320131093602\\
17.44	-9.8032079689799\\
17.45	-9.80321462151468\\
17.46	-9.80322126841414\\
17.47	-9.80322790955336\\
17.48	-9.80323454481138\\
17.49	-9.80324117407364\\
17.5	-9.8032477972344\\
17.51	-9.80325441419878\\
17.52	-9.80326102488472\\
17.53	-9.80326762922451\\
17.54	-9.80327422716613\\
17.55	-9.8032808186742\\
17.56	-9.8032874037307\\
17.57	-9.80329398233518\\
17.58	-9.80330055450477\\
17.59	-9.80330712027383\\
17.6	-9.80331367969315\\
17.61	-9.80332023282898\\
17.62	-9.80332677976165\\
17.63	-9.80333332058395\\
17.64	-9.80333985539928\\
17.65	-9.80334638431954\\
17.66	-9.80335290746287\\
17.67	-9.80335942495122\\
17.68	-9.80336593690785\\
17.69	-9.80337244345473\\
17.7	-9.80337894470996\\
17.71	-9.8033854407852\\
17.72	-9.80339193178318\\
17.73	-9.80339841779531\\
17.74	-9.80340489889947\\
17.75	-9.80341137515793\\
17.76	-9.80341784661562\\
17.77	-9.80342431329849\\
17.78	-9.80343077521232\\
17.79	-9.80343723234172\\
17.8	-9.80344368464952\\
17.81	-9.8034501320765\\
17.82	-9.80345657454138\\
17.83	-9.80346301194129\\
17.84	-9.80346944415251\\
17.85	-9.80347587103151\\
17.86	-9.80348229241643\\
17.87	-9.80348870812871\\
17.88	-9.80349511797512\\
17.89	-9.80350152175001\\
17.9	-9.80350791923772\\
17.91	-9.80351431021524\\
17.92	-9.80352069445498\\
17.93	-9.80352707172764\\
17.94	-9.80353344180519\\
17.95	-9.80353980446375\\
17.96	-9.80354615948659\\
17.97	-9.80355250666693\\
17.98	-9.80355884581071\\
17.99	-9.80356517673917\\
18	-9.80357149929124\\
18.01	-9.80357781332574\\
18.02	-9.80358411872323\\
18.03	-9.80359041538771\\
18.04	-9.80359670324785\\
18.05	-9.80360298225805\\
18.06	-9.80360925239902\\
18.07	-9.80361551367812\\
18.08	-9.80362176612926\\
18.09	-9.80362800981248\\
18.1	-9.80363424481323\\
18.11	-9.80364047124121\\
18.12	-9.803646689229\\
18.13	-9.80365289893034\\
18.14	-9.80365910051813\\
18.15	-9.80366529418228\\
18.16	-9.80367148012721\\
18.17	-9.80367765856931\\
18.18	-9.80368382973424\\
18.19	-9.80368999385407\\
18.2	-9.80369615116448\\
18.21	-9.80370230190186\\
18.22	-9.80370844630053\\
18.23	-9.80371458458996\\
18.24	-9.80372071699214\\
18.25	-9.80372684371912\\
18.26	-9.80373296497074\\
18.27	-9.8037390809325\\
18.28	-9.80374519177379\\
18.29	-9.80375129764631\\
18.3	-9.80375739868284\\
18.31	-9.80376349499624\\
18.32	-9.80376958667886\\
18.33	-9.80377567380216\\
18.34	-9.80378175641678\\
18.35	-9.80378783455279\\
18.36	-9.80379390822035\\
18.37	-9.80379997741062\\
18.38	-9.80380604209694\\
18.39	-9.80381210223623\\
18.4	-9.80381815777072\\
18.41	-9.80382420862973\\
18.42	-9.80383025473173\\
18.43	-9.80383629598645\\
18.44	-9.80384233229708\\
18.45	-9.80384836356261\\
18.46	-9.80385438968006\\
18.47	-9.80386041054677\\
18.48	-9.80386642606257\\
18.49	-9.80387243613195\\
18.5	-9.80387844066593\\
18.51	-9.80388443958396\\
18.52	-9.80389043281547\\
18.53	-9.80389642030128\\
18.54	-9.80390240199478\\
18.55	-9.80390837786277\\
18.56	-9.80391434788617\\
18.57	-9.80392031206026\\
18.58	-9.80392627039484\\
18.59	-9.80393222291395\\
18.6	-9.8039381696554\\
18.61	-9.80394411066998\\
18.62	-9.80395004602045\\
18.63	-9.8039559757803\\
18.64	-9.80396190003223\\
18.65	-9.80396781886657\\
18.66	-9.80397373237935\\
18.67	-9.80397964067048\\
18.68	-9.80398554384158\\
18.69	-9.80399144199397\\
18.7	-9.8039973352265\\
18.71	-9.80400322363342\\
18.72	-9.80400910730233\\
18.73	-9.80401498631219\\
18.74	-9.80402086073141\\
18.75	-9.80402673061617\\
18.76	-9.80403259600883\\
18.77	-9.80403845693661\\
18.78	-9.80404431341045\\
18.79	-9.80405016542415\\
18.8	-9.80405601295377\\
18.81	-9.80406185595727\\
18.82	-9.8040676943745\\
18.83	-9.80407352812739\\
18.84	-9.80407935712051\\
18.85	-9.80408518124183\\
18.86	-9.80409100036386\\
18.87	-9.80409681434489\\
18.88	-9.80410262303057\\
18.89	-9.80410842625572\\
18.9	-9.80411422384625\\
18.91	-9.80412001562128\\
18.92	-9.8041258013954\\
18.93	-9.80413158098099\\
18.94	-9.80413735419063\\
18.95	-9.80414312083954\\
18.96	-9.80414888074798\\
18.97	-9.80415463374361\\
18.98	-9.80416037966379\\
18.99	-9.80416611835776\\
19	-9.80417184968863\\
19.01	-9.80417757353529\\
19.02	-9.80418328979397\\
19.03	-9.80418899837973\\
19.04	-9.80419469922758\\
19.05	-9.8042003922934\\
19.06	-9.80420607755455\\
19.07	-9.80421175501022\\
19.08	-9.80421742468148\\
19.09	-9.80422308661098\\
19.1	-9.80422874086251\\
19.11	-9.8042343875201\\
19.12	-9.80424002668698\\
19.13	-9.80424565848428\\
19.14	-9.80425128304943\\
19.15	-9.80425690053445\\
19.16	-9.80426251110396\\
19.17	-9.80426811493317\\
19.18	-9.80427371220563\\
19.19	-9.80427930311096\\
19.2	-9.80428488784251\\
19.21	-9.80429046659502\\
19.22	-9.80429603956229\\
19.23	-9.8043016069349\\
19.24	-9.80430716889798\\
19.25	-9.80431272562919\\
19.26	-9.80431827729674\\
19.27	-9.80432382405762\\
19.28	-9.80432936605609\\
19.29	-9.80433490342226\\
19.3	-9.80434043627101\\
19.31	-9.80434596470113\\
19.32	-9.80435148879467\\
19.33	-9.8043570086166\\
19.34	-9.80436252421476\\
19.35	-9.80436803561996\\
19.36	-9.80437354284649\\
19.37	-9.80437904589271\\
19.38	-9.80438454474204\\
19.39	-9.80439003936397\\
19.4	-9.80439552971543\\
19.41	-9.80440101574226\\
19.42	-9.80440649738079\\
19.43	-9.80441197455956\\
19.44	-9.80441744720116\\
19.45	-9.80442291522407\\
19.46	-9.80442837854452\\
19.47	-9.80443383707838\\
19.48	-9.804439290743\\
19.49	-9.80444473945893\\
19.5	-9.8044501831516\\
19.51	-9.80445562175286\\
19.52	-9.80446105520234\\
19.53	-9.80446648344866\\
19.54	-9.80447190645046\\
19.55	-9.8044773241772\\
19.56	-9.80448273660972\\
19.57	-9.80448814374069\\
19.58	-9.80449354557462\\
19.59	-9.80449894212786\\
19.6	-9.80450433342819\\
19.61	-9.80450971951434\\
19.62	-9.80451510043513\\
19.63	-9.80452047624858\\
19.64	-9.8045258470207\\
19.65	-9.80453121282423\\
19.66	-9.80453657373712\\
19.67	-9.80454192984104\\
19.68	-9.80454728121966\\
19.69	-9.80455262795695\\
19.7	-9.80455797013548\\
19.71	-9.8045633078346\\
19.72	-9.80456864112878\\
19.73	-9.80457397008589\\
19.74	-9.80457929476571\\
19.75	-9.80458461521837\\
19.76	-9.8045899314831\\
19.77	-9.80459524358703\\
19.78	-9.8046005515442\\
19.79	-9.80460585535481\\
19.8	-9.80461115500459\\
19.81	-9.80461645046453\\
19.82	-9.80462174169065\\
19.83	-9.80462702862425\\
19.84	-9.80463231119212\\
19.85	-9.80463758930723\\
19.86	-9.80464286286949\\
19.87	-9.80464813176675\\
19.88	-9.80465339587606\\
19.89	-9.80465865506503\\
19.9	-9.80466390919339\\
19.91	-9.80466915811474\\
19.92	-9.80467440167827\\
19.93	-9.80467963973074\\
19.94	-9.80468487211841\\
19.95	-9.80469009868902\\
19.96	-9.80469531929379\\
19.97	-9.80470053378936\\
19.98	-9.80470574203974\\
19.99	-9.80471094391809\\
20	-9.80471613930843\\
20.01	-9.80472132810722\\
20.02	-9.80472651022478\\
20.03	-9.80473168558649\\
20.04	-9.80473685413382\\
20.05	-9.80474201582518\\
20.06	-9.80474717063645\\
20.07	-9.80475231856139\\
20.08	-9.80475745961175\\
20.09	-9.8047625938171\\
20.1	-9.80476772122452\\
20.11	-9.804772841898\\
20.12	-9.8047779559176\\
20.13	-9.80478306337847\\
20.14	-9.80478816438962\\
20.15	-9.80479325907249\\
20.16	-9.8047983475595\\
20.17	-9.80480342999228\\
20.18	-9.80480850651997\\
20.19	-9.80481357729729\\
20.2	-9.80481864248273\\
20.21	-9.80482370223653\\
20.22	-9.8048287567188\\
20.23	-9.80483380608763\\
20.24	-9.80483885049724\\
20.25	-9.80484389009627\\
20.26	-9.8048489250261\\
20.27	-9.80485395541941\\
20.28	-9.80485898139877\\
20.29	-9.80486400307555\\
20.3	-9.80486902054884\\
20.31	-9.80487403390476\\
20.32	-9.80487904321581\\
20.33	-9.80488404854055\\
20.34	-9.80488904992345\\
20.35	-9.80489404739494\\
20.36	-9.80489904097172\\
20.37	-9.80490403065722\\
20.38	-9.80490901644229\\
20.39	-9.80491399830606\\
20.4	-9.80491897621695\\
20.41	-9.80492395013383\\
20.42	-9.80492892000733\\
20.43	-9.80493388578119\\
20.44	-9.80493884739374\\
20.45	-9.80494380477944\\
20.46	-9.80494875787037\\
20.47	-9.80495370659781\\
20.48	-9.80495865089374\\
20.49	-9.80496359069231\\
20.5	-9.80496852593122\\
20.51	-9.80497345655304\\
20.52	-9.80497838250634\\
20.53	-9.80498330374679\\
20.54	-9.80498822023802\\
20.55	-9.80499313195232\\
20.56	-9.80499803887123\\
20.57	-9.80500294098587\\
20.58	-9.80500783829711\\
20.59	-9.80501273081559\\
20.6	-9.80501761856146\\
20.61	-9.80502250156399\\
20.62	-9.80502737986102\\
20.63	-9.80503225349822\\
20.64	-9.80503712252814\\
20.65	-9.80504198700924\\
20.66	-9.80504684700469\\
20.67	-9.80505170258114\\
20.68	-9.80505655380734\\
20.69	-9.80506140075279\\
20.7	-9.80506624348628\\
20.71	-9.80507108207448\\
20.72	-9.80507591658049\\
20.73	-9.80508074706248\\
20.74	-9.80508557357234\\
20.75	-9.80509039615448\\
20.76	-9.80509521484469\\
20.77	-9.80510002966911\\
20.78	-9.80510484064341\\
20.79	-9.80510964777207\\
20.8	-9.80511445104784\\
20.81	-9.8051192504514\\
20.82	-9.8051240459512\\
20.83	-9.8051288375035\\
20.84	-9.80513362505254\\
20.85	-9.80513840853105\\
20.86	-9.80514318786073\\
20.87	-9.80514796295315\\
20.88	-9.80515273371058\\
20.89	-9.8051575000272\\
20.9	-9.80516226179023\\
20.91	-9.80516701888138\\
20.92	-9.80517177117826\\
20.93	-9.80517651855591\\
20.94	-9.80518126088844\\
20.95	-9.80518599805061\\
20.96	-9.8051907299195\\
20.97	-9.8051954563761\\
20.98	-9.80520017730695\\
20.99	-9.80520489260558\\
21	-9.80520960217402\\
21.01	-9.80521430592406\\
21.02	-9.80521900377852\\
21.03	-9.80522369567226\\
21.04	-9.80522838155311\\
21.05	-9.80523306138259\\
21.06	-9.80523773513644\\
21.07	-9.80524240280501\\
21.08	-9.80524706439338\\
21.09	-9.80525171992139\\
21.1	-9.80525636942333\\
21.11	-9.80526101294757\\
21.12	-9.80526565055595\\
21.13	-9.80527028232297\\
21.14	-9.80527490833489\\
21.15	-9.80527952868858\\
21.16	-9.80528414349034\\
21.17	-9.80528875285454\\
21.18	-9.80529335690218\\
21.19	-9.80529795575941\\
21.2	-9.80530254955596\\
21.21	-9.80530713842355\\
21.22	-9.80531172249433\\
21.23	-9.80531630189929\\
21.24	-9.80532087676675\\
21.25	-9.8053254472209\\
21.26	-9.80533001338045\\
21.27	-9.80533457535732\\
21.28	-9.8053391332555\\
21.29	-9.80534368717006\\
21.3	-9.80534823718629\\
21.31	-9.80535278337898\\
21.32	-9.80535732581188\\
21.33	-9.80536186453738\\
21.34	-9.80536639959632\\
21.35	-9.80537093101794\\
21.36	-9.80537545882014\\
21.37	-9.80537998300974\\
21.38	-9.80538450358305\\
21.39	-9.80538902052647\\
21.4	-9.80539353381733\\
21.41	-9.80539804342478\\
21.42	-9.80540254931082\\
21.43	-9.80540705143141\\
21.44	-9.80541154973768\\
21.45	-9.8054160441771\\
21.46	-9.80542053469483\\
21.47	-9.80542502123489\\
21.48	-9.80542950374149\\
21.49	-9.80543398216021\\
21.5	-9.80543845643919\\
21.51	-9.80544292653019\\
21.52	-9.80544739238962\\
21.53	-9.80545185397943\\
21.54	-9.80545631126786\\
21.55	-9.80546076423011\\
21.56	-9.80546521284882\\
21.57	-9.80546965711443\\
21.58	-9.80547409702537\\
21.59	-9.80547853258808\\
21.6	-9.80548296381691\\
21.61	-9.80548739073383\\
21.62	-9.80549181336804\\
21.63	-9.80549623175533\\
21.64	-9.80550064593748\\
21.65	-9.80550505596136\\
21.66	-9.80550946187807\\
21.67	-9.8055138637419\\
21.68	-9.80551826160928\\
21.69	-9.80552265553761\\
21.7	-9.80552704558412\\
21.71	-9.80553143180467\\
21.72	-9.80553581425262\\
21.73	-9.80554019297761\\
21.74	-9.80554456802451\\
21.75	-9.80554893943238\\
21.76	-9.80555330723347\\
21.77	-9.8055576714524\\
21.78	-9.8055620321054\\
21.79	-9.80556638919969\\
21.8	-9.80557074273297\\
21.81	-9.80557509269313\\
21.82	-9.80557943905803\\
21.83	-9.80558378179547\\
21.84	-9.80558812086335\\
21.85	-9.80559245620992\\
21.86	-9.80559678777424\\
21.87	-9.80560111548678\\
21.88	-9.80560543927014\\
21.89	-9.80560975903989\\
21.9	-9.8056140747056\\
21.91	-9.8056183861719\\
21.92	-9.80562269333961\\
21.93	-9.80562699610706\\
21.94	-9.80563129437134\\
21.95	-9.80563558802963\\
21.96	-9.80563987698056\\
21.97	-9.80564416112555\\
21.98	-9.8056484403701\\
21.99	-9.80565271462507\\
22	-9.80565698380792\\
22.01	-9.80566124784375\\
22.02	-9.80566550666641\\
22.03	-9.80566976021936\\
22.04	-9.80567400845649\\
22.05	-9.80567825134271\\
22.06	-9.80568248885453\\
22.07	-9.80568672098034\\
22.08	-9.80569094772063\\
22.09	-9.805695169088\\
22.1	-9.80569938510705\\
22.11	-9.80570359581406\\
22.12	-9.80570780125653\\
22.13	-9.80571200149263\\
22.14	-9.80571619659043\\
22.15	-9.80572038662707\\
22.16	-9.80572457168774\\
22.17	-9.80572875186467\\
22.18	-9.80573292725594\\
22.19	-9.80573709796425\\
22.2	-9.80574126409569\\
22.21	-9.80574542575843\\
22.22	-9.80574958306141\\
22.23	-9.80575373611308\\
22.24	-9.80575788502008\\
22.25	-9.80576202988606\\
22.26	-9.80576617081054\\
22.27	-9.80577030788781\\
22.28	-9.80577444120594\\
22.29	-9.80577857084595\\
22.3	-9.80578269688104\\
22.31	-9.80578681937595\\
22.32	-9.80579093838653\\
22.33	-9.80579505395933\\
22.34	-9.80579916613146\\
22.35	-9.80580327493051\\
22.36	-9.80580738037466\\
22.37	-9.8058114824729\\
22.38	-9.80581558122541\\
22.39	-9.80581967662407\\
22.4	-9.80582376865304\\
22.41	-9.80582785728952\\
22.42	-9.80583194250455\\
22.43	-9.80583602426389\\
22.44	-9.80584010252896\\
22.45	-9.80584417725791\\
22.46	-9.80584824840657\\
22.47	-9.80585231592951\\
22.48	-9.80585637978113\\
22.49	-9.8058604399166\\
22.5	-9.80586449629288\\
22.51	-9.8058685488696\\
22.52	-9.80587259760998\\
22.53	-9.8058766424815\\
22.54	-9.80588068345664\\
22.55	-9.80588472051343\\
22.56	-9.80588875363588\\
22.57	-9.80589278281432\\
22.58	-9.80589680804561\\
22.59	-9.8059008293332\\
22.6	-9.80590484668706\\
22.61	-9.80590886012355\\
22.62	-9.80591286966504\\
22.63	-9.80591687533954\\
22.64	-9.80592087718015\\
22.65	-9.80592487522439\\
22.66	-9.80592886951353\\
22.67	-9.80593286009173\\
22.68	-9.80593684700521\\
22.69	-9.80594083030131\\
22.7	-9.80594481002756\\
22.71	-9.80594878623069\\
22.72	-9.80595275895568\\
22.73	-9.80595672824479\\
22.74	-9.80596069413665\\
22.75	-9.8059646566654\\
22.76	-9.80596861585984\\
22.77	-9.80597257174274\\
22.78	-9.80597652433016\\
22.79	-9.80598047363091\\
22.8	-9.80598441964614\\
22.81	-9.805988362369\\
22.82	-9.80599230178441\\
22.83	-9.80599623786906\\
22.84	-9.80600017059144\\
22.85	-9.80600409991202\\
22.86	-9.8060080257836\\
22.87	-9.80601194815172\\
22.88	-9.80601586695525\\
22.89	-9.80601978212707\\
22.9	-9.8060236935948\\
22.91	-9.80602760128171\\
22.92	-9.80603150510763\\
22.93	-9.80603540498998\\
22.94	-9.8060393008448\\
22.95	-9.80604319258783\\
22.96	-9.80604708013564\\
22.97	-9.80605096340671\\
22.98	-9.80605484232252\\
22.99	-9.80605871680861\\
23	-9.8060625867956\\
23.01	-9.80606645222011\\
23.02	-9.80607031302566\\
23.03	-9.80607416916341\\
23.04	-9.80607802059289\\
23.05	-9.80608186728253\\
23.06	-9.80608570921014\\
23.07	-9.80608954636323\\
23.08	-9.8060933787392\\
23.09	-9.80609720634541\\
23.1	-9.80610102919911\\
23.11	-9.80610484732725\\
23.12	-9.80610866076615\\
23.13	-9.80611246956105\\
23.14	-9.80611627376554\\
23.15	-9.80612007344092\\
23.16	-9.80612386865537\\
23.17	-9.80612765948314\\
23.18	-9.80613144600364\\
23.19	-9.8061352283004\\
23.2	-9.80613900646009\\
23.21	-9.80614278057145\\
23.22	-9.8061465507242\\
23.23	-9.80615031700802\\
23.24	-9.80615407951144\\
23.25	-9.8061578383209\\
23.26	-9.8061615935197\\
23.27	-9.80616534518718\\
23.28	-9.80616909339781\\
23.29	-9.8061728382205\\
23.3	-9.80617657971795\\
23.31	-9.80618031794604\\
23.32	-9.80618405295349\\
23.33	-9.80618778478145\\
23.34	-9.80619151346336\\
23.35	-9.80619523902485\\
23.36	-9.80619896148376\\
23.37	-9.80620268085033\\
23.38	-9.80620639712743\\
23.39	-9.80621011031096\\
23.4	-9.80621382039032\\
23.41	-9.80621752734897\\
23.42	-9.80622123116508\\
23.43	-9.80622493181221\\
23.44	-9.80622862926014\\
23.45	-9.80623232347563\\
23.46	-9.80623601442326\\
23.47	-9.80623970206633\\
23.48	-9.80624338636765\\
23.49	-9.80624706729043\\
23.5	-9.80625074479906\\
23.51	-9.80625441885991\\
23.52	-9.80625808944202\\
23.53	-9.80626175651778\\
23.54	-9.80626542006347\\
23.55	-9.80626908005981\\
23.56	-9.80627273649229\\
23.57	-9.80627638935154\\
23.58	-9.8062800386335\\
23.59	-9.80628368433949\\
23.6	-9.80628732647626\\
23.61	-9.80629096505583\\
23.62	-9.80629460009528\\
23.63	-9.80629823161643\\
23.64	-9.80630185964544\\
23.65	-9.80630548421231\\
23.66	-9.80630910535028\\
23.67	-9.80631272309524\\
23.68	-9.806316337485\\
23.69	-9.80631994855854\\
23.7	-9.80632355635526\\
23.71	-9.80632716091419\\
23.72	-9.80633076227315\\
23.73	-9.80633436046802\\
23.74	-9.80633795553195\\
23.75	-9.80634154749462\\
23.76	-9.80634513638155\\
23.77	-9.80634872221349\\
23.78	-9.80635230500587\\
23.79	-9.80635588476829\\
23.8	-9.80635946150417\\
23.81	-9.80636303521042\\
23.82	-9.80636660587728\\
23.83	-9.8063701734882\\
23.84	-9.80637373801984\\
23.85	-9.80637729944227\\
23.86	-9.80638085771912\\
23.87	-9.80638441280791\\
23.88	-9.80638796466057\\
23.89	-9.80639151322384\\
23.9	-9.80639505843998\\
23.91	-9.8063986002474\\
23.92	-9.80640213858143\\
23.93	-9.80640567337512\\
23.94	-9.80640920456012\\
23.95	-9.8064127320675\\
23.96	-9.80641625582874\\
23.97	-9.80641977577655\\
23.98	-9.80642329184583\\
23.99	-9.80642680397453\\
24	-9.80643031210448\\
24.01	-9.80643381618219\\
24.02	-9.80643731615961\\
24.03	-9.80644081199476\\
24.04	-9.80644430365234\\
24.05	-9.80644779110423\\
24.06	-9.80645127432991\\
24.07	-9.80645475331671\\
24.08	-9.80645822806007\\
24.09	-9.80646169856361\\
24.1	-9.80646516483908\\
24.11	-9.80646862690629\\
24.12	-9.80647208479283\\
24.13	-9.80647553853375\\
24.14	-9.80647898817114\\
24.15	-9.80648243375359\\
24.16	-9.80648587533559\\
24.17	-9.80648931297686\\
24.18	-9.80649274674158\\
24.19	-9.80649617669764\\
24.2	-9.80649960291577\\
24.21	-9.80650302546868\\
24.22	-9.80650644443021\\
24.23	-9.80650985987444\\
24.24	-9.80651327187481\\
24.25	-9.8065166805033\\
24.26	-9.80652008582958\\
24.27	-9.80652348792031\\
24.28	-9.80652688683837\\
24.29	-9.80653028264229\\
24.3	-9.80653367538562\\
24.31	-9.80653706511651\\
24.32	-9.80654045187728\\
24.33	-9.80654383570417\\
24.34	-9.80654721662708\\
24.35	-9.80655059466952\\
24.36	-9.80655396984861\\
24.37	-9.80655734217512\\
24.38	-9.80656071165373\\
24.39	-9.80656407828325\\
24.4	-9.80656744205704\\
24.41	-9.80657080296339\\
24.42	-9.80657416098606\\
24.43	-9.80657751610487\\
24.44	-9.80658086829625\\
24.45	-9.80658421753395\\
24.46	-9.80658756378972\\
24.47	-9.80659090703398\\
24.48	-9.80659424723654\\
24.49	-9.8065975843673\\
24.5	-9.80660091839692\\
24.51	-9.80660424929745\\
24.52	-9.80660757704298\\
24.53	-9.80661090161012\\
24.54	-9.80661422297859\\
24.55	-9.80661754113157\\
24.56	-9.80662085605609\\
24.57	-9.80662416774333\\
24.58	-9.80662747618876\\
24.59	-9.80663078139228\\
24.6	-9.80663408335823\\
24.61	-9.80663738209535\\
24.62	-9.80664067761659\\
24.63	-9.8066439699389\\
24.64	-9.8066472590829\\
24.65	-9.80665054507252\\
24.66	-9.80665382793453\\
24.67	-9.80665710769802\\
24.68	-9.80666038439389\\
24.69	-9.8066636580542\\
24.7	-9.80666692871155\\
24.71	-9.80667019639847\\
24.72	-9.80667346114675\\
24.73	-9.80667672298674\\
24.74	-9.80667998194679\\
24.75	-9.80668323805259\\
24.76	-9.80668649132662\\
24.77	-9.80668974178757\\
24.78	-9.80669298944992\\
24.79	-9.80669623432352\\
24.8	-9.80669947641319\\
24.81	-9.80670271571851\\
24.82	-9.80670595223361\\
24.83	-9.80670918594705\\
24.84	-9.80671241684183\\
24.85	-9.80671564489547\\
24.86	-9.80671887008011\\
24.87	-9.80672209236282\\
24.88	-9.8067253117059\\
24.89	-9.80672852806728\\
24.9	-9.80673174140098\\
24.91	-9.80673495165772\\
24.92	-9.80673815878543\\
24.93	-9.80674136272997\\
24.94	-9.80674456343578\\
24.95	-9.80674776084661\\
24.96	-9.80675095490628\\
24.97	-9.80675414555938\\
24.98	-9.80675733275204\\
24.99	-9.80676051643266\\
25	-9.80676369655258\\
25.01	-9.80676687306681\\
25.02	-9.80677004593459\\
25.03	-9.80677321511999\\
25.04	-9.80677638059242\\
25.05	-9.80677954232704\\
25.06	-9.80678270030513\\
25.07	-9.80678585451439\\
25.08	-9.80678900494912\\
25.09	-9.8067921516103\\
25.1	-9.80679529450563\\
25.11	-9.80679843364946\\
25.12	-9.80680156906261\\
25.13	-9.80680470077213\\
25.14	-9.80680782881097\\
25.15	-9.8068109532176\\
25.16	-9.80681407403547\\
25.17	-9.80681719131255\\
25.18	-9.80682030510066\\
25.19	-9.80682341545491\\
25.2	-9.80682652243297\\
25.21	-9.80682962609437\\
25.22	-9.80683272649982\\
25.23	-9.80683582371046\\
25.24	-9.80683891778714\\
25.25	-9.80684200878974\\
25.26	-9.80684509677649\\
25.27	-9.80684818180331\\
25.28	-9.80685126392324\\
25.29	-9.80685434318587\\
25.3	-9.80685741963691\\
25.31	-9.8068604933177\\
25.32	-9.80686356426494\\
25.33	-9.80686663251035\\
25.34	-9.80686969808054\\
25.35	-9.80687276099687\\
25.36	-9.80687582127542\\
25.37	-9.80687887892708\\
25.38	-9.80688193395761\\
25.39	-9.80688498636791\\
25.4	-9.80688803615426\\
25.41	-9.80689108330868\\
25.42	-9.8068941278193\\
25.43	-9.80689716967084\\
25.44	-9.80690020884511\\
25.45	-9.80690324532151\\
25.46	-9.80690627907757\\
25.47	-9.80690931008958\\
25.48	-9.80691233833313\\
25.49	-9.80691536378368\\
25.5	-9.80691838641712\\
25.51	-9.80692140621033\\
25.52	-9.80692442314168\\
25.53	-9.80692743719149\\
25.54	-9.80693044834248\\
25.55	-9.8069334565801\\
25.56	-9.80693646189289\\
25.57	-9.8069394642727\\
25.58	-9.80694246371488\\
25.59	-9.8069454602184\\
25.6	-9.80694845378586\\
25.61	-9.80695144442351\\
25.62	-9.80695443214112\\
25.63	-9.80695741695179\\
25.64	-9.80696039887174\\
25.65	-9.80696337792004\\
25.66	-9.80696635411819\\
25.67	-9.80696932748977\\
25.68	-9.80697229805998\\
25.69	-9.80697526585515\\
25.7	-9.80697823090225\\
25.71	-9.80698119322833\\
25.72	-9.80698415286004\\
25.73	-9.80698710982303\\
25.74	-9.80699006414147\\
25.75	-9.8069930158375\\
25.76	-9.8069959649308\\
25.77	-9.80699891143808\\
25.78	-9.80700185537271\\
25.79	-9.80700479674433\\
25.8	-9.80700773555861\\
25.81	-9.80701067181692\\
25.82	-9.80701360551622\\
25.83	-9.80701653664893\\
25.84	-9.80701946520288\\
25.85	-9.80702239116137\\
25.86	-9.80702531450324\\
25.87	-9.80702823520309\\
25.88	-9.80703115323148\\
25.89	-9.8070340685553\\
25.9	-9.80703698113807\\
25.91	-9.80703989094043\\
25.92	-9.80704279792059\\
25.93	-9.80704570203484\\
25.94	-9.80704860323815\\
25.95	-9.8070515014847\\
25.96	-9.8070543967285\\
25.97	-9.80705728892402\\
25.98	-9.80706017802679\\
25.99	-9.80706306399399\\
26	-9.80706594678503\\
26.01	-9.80706882636216\\
26.02	-9.80707170269091\\
26.03	-9.80707457574067\\
26.04	-9.80707744548504\\
26.05	-9.80708031190225\\
26.06	-9.80708317497547\\
26.07	-9.80708603469306\\
26.08	-9.80708889104876\\
26.09	-9.80709174404178\\
26.1	-9.80709459367685\\
26.11	-9.80709743996419\\
26.12	-9.80710028291939\\
26.13	-9.80710312256324\\
26.14	-9.80710595892146\\
26.15	-9.80710879202441\\
26.16	-9.80711162190672\\
26.17	-9.80711444860685\\
26.18	-9.80711727216661\\
26.19	-9.80712009263069\\
26.2	-9.80712291004604\\
26.21	-9.80712572446138\\
26.22	-9.80712853592658\\
26.23	-9.80713134449204\\
26.24	-9.80713415020816\\
26.25	-9.8071369531247\\
26.26	-9.80713975329025\\
26.27	-9.8071425507517\\
26.28	-9.80714534555372\\
26.29	-9.80714813773828\\
26.3	-9.8071509273443\\
26.31	-9.80715371440722\\
26.32	-9.80715649895876\\
26.33	-9.8071592810266\\
26.34	-9.8071620606343\\
26.35	-9.80716483780108\\
26.36	-9.80716761254189\\
26.37	-9.80717038486731\\
26.38	-9.80717315478375\\
26.39	-9.80717592229351\\
26.4	-9.80717868739503\\
26.41	-9.80718145008314\\
26.42	-9.80718421034938\\
26.43	-9.80718696818236\\
26.44	-9.80718972356812\\
26.45	-9.80719247649061\\
26.46	-9.80719522693209\\
26.47	-9.80719797487363\\
26.48	-9.80720072029556\\
26.49	-9.80720346317795\\
26.5	-9.80720620350108\\
26.51	-9.80720894124588\\
26.52	-9.80721167639438\\
26.53	-9.80721440893008\\
26.54	-9.80721713883833\\
26.55	-9.80721986610664\\
26.56	-9.80722259072498\\
26.57	-9.80722531268594\\
26.58	-9.80722803198499\\
26.59	-9.80723074862051\\
26.6	-9.80723346259388\\
26.61	-9.80723617390949\\
26.62	-9.80723888257462\\
26.63	-9.8072415885994\\
26.64	-9.80724429199656\\
26.65	-9.80724699278124\\
26.66	-9.80724969097071\\
26.67	-9.80725238658404\\
26.68	-9.80725507964178\\
26.69	-9.80725777016553\\
26.7	-9.80726045817755\\
26.71	-9.80726314370033\\
26.72	-9.80726582675616\\
26.73	-9.80726850736668\\
26.74	-9.80727118555244\\
26.75	-9.80727386133247\\
26.76	-9.8072765347239\\
26.77	-9.80727920574155\\
26.78	-9.8072818743976\\
26.79	-9.80728454070128\\
26.8	-9.80728720465859\\
26.81	-9.80728986627212\\
26.82	-9.80729252554083\\
26.83	-9.80729518246\\
26.84	-9.80729783702113\\
26.85	-9.807300489212\\
26.86	-9.80730313901667\\
26.87	-9.80730578641568\\
26.88	-9.80730843138617\\
26.89	-9.80731107390216\\
26.9	-9.80731371393482\\
26.91	-9.80731635145281\\
26.92	-9.80731898642267\\
26.93	-9.80732161880922\\
26.94	-9.80732424857601\\
26.95	-9.8073268756858\\
26.96	-9.80732950010105\\
26.97	-9.80733212178442\\
26.98	-9.80733474069927\\
26.99	-9.80733735681014\\
27	-9.80733997008327\\
27.01	-9.80734258048704\\
27.02	-9.80734518799244\\
27.03	-9.80734779257343\\
27.04	-9.80735039420739\\
27.05	-9.80735299287533\\
27.06	-9.8073555885623\\
27.07	-9.80735818125751\\
27.08	-9.80736077095454\\
27.09	-9.80736335765147\\
27.1	-9.8073659413509\\
27.11	-9.80736852205994\\
27.12	-9.80737109979015\\
27.13	-9.80737367455744\\
27.14	-9.80737624638182\\
27.15	-9.80737881528723\\
27.16	-9.80738138130119\\
27.17	-9.8073839444545\\
27.18	-9.80738650478086\\
27.19	-9.80738906231647\\
27.2	-9.80739161709953\\
27.21	-9.80739416916989\\
27.22	-9.80739671856845\\
27.23	-9.80739926533675\\
27.24	-9.80740180951648\\
27.25	-9.80740435114897\\
27.26	-9.80740689027473\\
27.27	-9.80740942693302\\
27.28	-9.80741196116142\\
27.29	-9.80741449299541\\
27.3	-9.80741702246808\\
27.31	-9.80741954960975\\
27.32	-9.80742207444775\\
27.33	-9.80742459700619\\
27.34	-9.80742711730578\\
27.35	-9.80742963536376\\
27.36	-9.80743215119381\\
27.37	-9.80743466480604\\
27.38	-9.8074371762071\\
27.39	-9.8074396854002\\
27.4	-9.80744219238534\\
27.41	-9.80744469715948\\
27.42	-9.80744719971676\\
27.43	-9.80744970004883\\
27.44	-9.80745219814512\\
27.45	-9.80745469399321\\
27.46	-9.80745718757918\\
27.47	-9.80745967888798\\
27.48	-9.80746216790385\\
27.49	-9.80746465461066\\
27.5	-9.80746713899235\\
27.51	-9.80746962103326\\
27.52	-9.8074721007185\\
27.53	-9.80747457803431\\
27.54	-9.80747705296834\\
27.55	-9.80747952550992\\
27.56	-9.80748199565034\\
27.57	-9.80748446338299\\
27.58	-9.80748692870357\\
27.59	-9.80748939161016\\
27.6	-9.80749185210328\\
27.61	-9.80749431018593\\
27.62	-9.80749676586352\\
27.63	-9.80749921914384\\
27.64	-9.80750167003683\\
27.65	-9.80750411855454\\
27.66	-9.8075065647108\\
27.67	-9.80750900852103\\
27.68	-9.80751145000195\\
27.69	-9.80751388917127\\
27.7	-9.80751632604736\\
27.71	-9.80751876064889\\
27.72	-9.80752119299446\\
27.73	-9.80752362310232\\
27.74	-9.80752605098987\\
27.75	-9.80752847667345\\
27.76	-9.80753090016788\\
27.77	-9.80753332148625\\
27.78	-9.80753574063953\\
27.79	-9.80753815763635\\
27.8	-9.80754057248279\\
27.81	-9.80754298518213\\
27.82	-9.80754539573476\\
27.83	-9.80754780413805\\
27.84	-9.80755021038627\\
27.85	-9.80755261447062\\
27.86	-9.80755501637925\\
27.87	-9.80755741609734\\
27.88	-9.80755981360726\\
27.89	-9.80756220888872\\
27.9	-9.807564601919\\
27.91	-9.80756699267325\\
27.92	-9.80756938112474\\
27.93	-9.80757176724522\\
27.94	-9.80757415100528\\
27.95	-9.80757653237472\\
27.96	-9.80757891132296\\
27.97	-9.80758128781943\\
27.98	-9.80758366183402\\
27.99	-9.80758603333744\\
28	-9.80758840230166\\
28.01	-9.8075907687003\\
28.02	-9.80759313250896\\
28.03	-9.80759549370564\\
28.04	-9.80759785227097\\
28.05	-9.80760020818855\\
28.06	-9.80760256144517\\
28.07	-9.807604912031\\
28.08	-9.80760725993978\\
28.09	-9.80760960516889\\
28.1	-9.80761194771941\\
28.11	-9.80761428759617\\
28.12	-9.80761662480765\\
28.13	-9.80761895936595\\
28.14	-9.80762129128659\\
28.15	-9.80762362058839\\
28.16	-9.8076259472932\\
28.17	-9.80762827142567\\
28.18	-9.80763059301292\\
28.19	-9.80763291208423\\
28.2	-9.8076352286707\\
28.21	-9.80763754280484\\
28.22	-9.8076398545202\\
28.23	-9.807642163851\\
28.24	-9.80764447083169\\
28.25	-9.80764677549655\\
28.26	-9.80764907787936\\
28.27	-9.80765137801293\\
28.28	-9.80765367592886\\
28.29	-9.8076559716571\\
28.3	-9.80765826522574\\
28.31	-9.80766055666067\\
28.32	-9.80766284598539\\
28.33	-9.80766513322083\\
28.34	-9.80766741838515\\
28.35	-9.80766970149369\\
28.36	-9.80767198255886\\
28.37	-9.80767426159017\\
28.38	-9.80767653859423\\
28.39	-9.80767881357482\\
28.4	-9.80768108653301\\
28.41	-9.80768335746734\\
28.42	-9.80768562637394\\
28.43	-9.80768789324682\\
28.44	-9.80769015807809\\
28.45	-9.80769242085823\\
28.46	-9.80769468157638\\
28.47	-9.80769694022067\\
28.48	-9.80769919677852\\
28.49	-9.80770145123694\\
28.5	-9.8077037035829\\
28.51	-9.80770595380359\\
28.52	-9.80770820188673\\
28.53	-9.8077104478209\\
28.54	-9.80771269159571\\
28.55	-9.80771493320214\\
28.56	-9.80771717263267\\
28.57	-9.80771940988147\\
28.58	-9.80772164494456\\
28.59	-9.8077238778199\\
28.6	-9.80772610850745\\
28.61	-9.80772833700918\\
28.62	-9.80773056332909\\
28.63	-9.80773278747312\\
28.64	-9.80773500944906\\
28.65	-9.80773722926644\\
28.66	-9.80773944693634\\
28.67	-9.80774166247124\\
28.68	-9.80774387588471\\
28.69	-9.80774608719128\\
28.7	-9.80774829640607\\
28.71	-9.80775050354456\\
28.72	-9.80775270862229\\
28.73	-9.80775491165457\\
28.74	-9.80775711265617\\
28.75	-9.80775931164102\\
28.76	-9.80776150862196\\
28.77	-9.80776370361043\\
28.78	-9.80776589661625\\
28.79	-9.80776808764737\\
28.8	-9.80777027670972\\
28.81	-9.80777246380697\\
28.82	-9.80777464894045\\
28.83	-9.80777683210903\\
28.84	-9.80777901330907\\
28.85	-9.80778119253438\\
28.86	-9.80778336977625\\
28.87	-9.80778554502353\\
28.88	-9.80778771826268\\
28.89	-9.80778988947795\\
28.9	-9.80779205865152\\
28.91	-9.80779422576374\\
28.92	-9.80779639079334\\
28.93	-9.80779855371769\\
28.94	-9.80780071451312\\
28.95	-9.80780287315519\\
28.96	-9.80780502961907\\
28.97	-9.80780718387978\\
28.98	-9.80780933591262\\
28.99	-9.80781148569347\\
29	-9.80781363319911\\
29.01	-9.80781577840757\\
29.02	-9.80781792129844\\
29.03	-9.80782006185313\\
29.04	-9.80782220005517\\
29.05	-9.80782433589045\\
29.06	-9.80782646934741\\
29.07	-9.8078286004172\\
29.08	-9.8078307290939\\
29.09	-9.80783285537452\\
29.1	-9.80783497925911\\
29.11	-9.80783710075081\\
29.12	-9.80783921985577\\
29.13	-9.80784133658312\\
29.14	-9.80784345094488\\
29.15	-9.80784556295581\\
29.16	-9.80784767263322\\
29.17	-9.80784977999682\\
29.18	-9.8078518850684\\
29.19	-9.80785398787166\\
29.2	-9.80785608843184\\
29.21	-9.8078581867755\\
29.22	-9.80786028293012\\
29.23	-9.80786237692387\\
29.24	-9.80786446878518\\
29.25	-9.8078665585425\\
29.26	-9.80786864622393\\
29.27	-9.80787073185691\\
29.28	-9.80787281546796\\
29.29	-9.80787489708233\\
29.3	-9.80787697672384\\
29.31	-9.80787905441456\\
29.32	-9.80788113017468\\
29.33	-9.80788320402228\\
29.34	-9.80788527597325\\
29.35	-9.80788734604116\\
29.36	-9.80788941423721\\
29.37	-9.80789148057018\\
29.38	-9.80789354504651\\
29.39	-9.80789560767024\\
29.4	-9.80789766844318\\
29.41	-9.80789972736499\\
29.42	-9.80790178443333\\
29.43	-9.80790383964404\\
29.44	-9.80790589299131\\
29.45	-9.80790794446793\\
29.46	-9.80790999406551\\
29.47	-9.80791204177474\\
29.48	-9.80791408758562\\
29.49	-9.80791613148778\\
29.5	-9.80791817347066\\
29.51	-9.80792021352383\\
29.52	-9.80792225163723\\
29.53	-9.80792428780139\\
29.54	-9.80792632200766\\
29.55	-9.80792835424842\\
29.56	-9.80793038451722\\
29.57	-9.807932412809\\
29.58	-9.80793443912015\\
29.59	-9.80793646344861\\
29.6	-9.807938485794\\
29.61	-9.80794050615756\\
29.62	-9.80794252454219\\
29.63	-9.80794454095242\\
29.64	-9.80794655539435\\
29.65	-9.8079485678755\\
29.66	-9.80795057840477\\
29.67	-9.80795258699219\\
29.68	-9.80795459364883\\
29.69	-9.80795659838656\\
29.7	-9.80795860121783\\
29.71	-9.80796060215548\\
29.72	-9.80796260121247\\
29.73	-9.80796459840163\\
29.74	-9.80796659373544\\
29.75	-9.80796858722581\\
29.76	-9.80797057888376\\
29.77	-9.8079725687193\\
29.78	-9.80797455674113\\
29.79	-9.8079765429565\\
29.8	-9.80797852737104\\
29.81	-9.80798050998857\\
29.82	-9.80798249081103\\
29.83	-9.80798446983834\\
29.84	-9.8079864470684\\
29.85	-9.807988422497\\
29.86	-9.80799039611786\\
29.87	-9.80799236792265\\
29.88	-9.80799433790109\\
29.89	-9.80799630604102\\
29.9	-9.80799827232853\\
29.91	-9.80800023674813\\
29.92	-9.80800219928297\\
29.93	-9.80800415991497\\
29.94	-9.80800611862513\\
29.95	-9.80800807539374\\
29.96	-9.80801003020065\\
29.97	-9.80801198302553\\
29.98	-9.80801393384817\\
29.99	-9.80801588264873\\
};
\addlegendentry{EI}

\addplot [color=blue]
  table[row sep=crcr]{%
0	-9.77247004781056\\
0.01	-9.77246999181814\\
0.02	-9.77246984234811\\
0.03	-9.77246960077469\\
0.04	-9.77246926936801\\
0.05	-9.7724688512639\\
0.06	-9.7724683504223\\
0.07	-9.77246777157478\\
0.08	-9.77246712016207\\
0.09	-9.77246640226248\\
0.1	-9.7724656245123\\
0.11	-9.77246479401931\\
0.12	-9.77246391827076\\
0.13	-9.77246300503719\\
0.14	-9.77246206227336\\
0.15	-9.77246109801804\\
0.16	-9.77246012029383\\
0.17	-9.77245913700865\\
0.18	-9.77245815586037\\
0.19	-9.77245718424578\\
0.2	-9.77245622917547\\
0.21	-9.7724552971955\\
0.22	-9.77245439431739\\
0.23	-9.77245352595694\\
0.24	-9.77245269688309\\
0.25	-9.77245191117728\\
0.26	-9.77245117220385\\
0.27	-9.77245048259186\\
0.28	-9.77244984422844\\
0.29	-9.77244925826366\\
0.3	-9.77244872512667\\
0.31	-9.77244824455276\\
0.32	-9.77244781562083\\
0.33	-9.77244743680038\\
0.34	-9.77244710600741\\
0.35	-9.77244682066797\\
0.36	-9.77244657778848\\
0.37	-9.77244637403136\\
0.38	-9.7724462057949\\
0.39	-9.77244606929585\\
0.4	-9.77244596065342\\
0.41	-9.77244587597319\\
0.42	-9.77244581142963\\
0.43	-9.77244576334576\\
0.44	-9.77244572826871\\
0.45	-9.77244570303978\\
0.46	-9.77244568485804\\
0.47	-9.77244567133624\\
0.48	-9.77244566054818\\
0.49	-9.77244565106678\\
0.5	-9.77244564199216\\
0.51	-9.77244563296935\\
0.52	-9.77244562419522\\
0.53	-9.77244561641463\\
0.54	-9.77244561090581\\
0.55	-9.77244560945524\\
0.56	-9.77244561432232\\
0.57	-9.77244562819472\\
0.58	-9.77244565413467\\
0.59	-9.77244569551764\\
0.6	-9.77244575596387\\
0.61	-9.77244583926436\\
0.62	-9.77244594930221\\
0.63	-9.7724460899708\\
0.64	-9.77244626509013\\
0.65	-9.77244647832268\\
0.66	-9.7724467330903\\
0.67	-9.7724470324934\\
0.68	-9.77244737923397\\
0.69	-9.77244777554359\\
0.7	-9.77244822311783\\
0.71	-9.77244872305804\\
0.72	-9.77244927582171\\
0.73	-9.77244988118215\\
0.74	-9.77245053819845\\
0.75	-9.77245124519606\\
0.76	-9.77245199975873\\
0.77	-9.77245279873181\\
0.78	-9.7724536382371\\
0.79	-9.77245451369926\\
0.8	-9.77245541988326\\
0.81	-9.77245635094262\\
0.82	-9.77245730047771\\
0.83	-9.77245826160343\\
0.84	-9.77245922702517\\
0.85	-9.77246018912212\\
0.86	-9.77246114003676\\
0.87	-9.77246207176912\\
0.88	-9.77246297627452\\
0.89	-9.77246384556343\\
0.9	-9.77246467180184\\
0.91	-9.7724654474108\\
0.92	-9.77246616516357\\
0.93	-9.77246681827903\\
0.94	-9.77246740050983\\
0.95	-9.77246790622415\\
0.96	-9.77246833047967\\
0.97	-9.77246866908881\\
0.98	-9.77246891867412\\
0.99	-9.77246907671307\\
1	-9.77246914157164\\
1.01	-9.772469112526\\
1.02	-9.77246898977221\\
1.03	-9.77246877442355\\
1.04	-9.77246846849576\\
1.05	-9.77246807488016\\
1.06	-9.77246759730531\\
1.07	-9.77246704028755\\
1.08	-9.77246640907132\\
1.09	-9.77246570956008\\
1.1	-9.77246494823889\\
1.11	-9.77246413208977\\
1.12	-9.77246326850114\\
1.13	-9.77246236517263\\
1.14	-9.77246143001678\\
1.15	-9.7724604710589\\
1.16	-9.77245949633681\\
1.17	-9.77245851380163\\
1.18	-9.77245753122141\\
1.19	-9.77245655608869\\
1.2	-9.77245559553354\\
1.21	-9.77245465624319\\
1.22	-9.77245374438951\\
1.23	-9.77245286556519\\
1.24	-9.77245202472965\\
1.25	-9.77245122616532\\
1.26	-9.77245047344486\\
1.27	-9.77244976940962\\
1.28	-9.77244911615974\\
1.29	-9.77244851505568\\
1.3	-9.77244796673118\\
1.31	-9.77244747111726\\
1.32	-9.7724470274767\\
1.33	-9.77244663444846\\
1.34	-9.77244629010108\\
1.35	-9.77244599199417\\
1.36	-9.77244573724688\\
1.37	-9.77244552261218\\
1.38	-9.77244534455562\\
1.39	-9.77244519933731\\
1.4	-9.77244508309564\\
1.41	-9.77244499193138\\
1.42	-9.77244492199075\\
1.43	-9.77244486954604\\
1.44	-9.77244483107246\\
1.45	-9.77244480331993\\
1.46	-9.77244478337866\\
1.47	-9.77244476873733\\
1.48	-9.77244475733303\\
1.49	-9.77244474759206\\
1.5	-9.77244473846095\\
1.51	-9.77244472942715\\
1.52	-9.77244472052912\\
1.53	-9.77244471235552\\
1.54	-9.77244470603371\\
1.55	-9.77244470320758\\
1.56	-9.77244470600513\\
1.57	-9.77244471699634\\
1.58	-9.77244473914209\\
1.59	-9.77244477573483\\
1.6	-9.77244483033208\\
1.61	-9.7724449066839\\
1.62	-9.7724450086554\\
1.63	-9.77244514014571\\
1.64	-9.77244530500467\\
1.65	-9.77244550694874\\
1.66	-9.77244574947735\\
1.67	-9.77244603579138\\
1.68	-9.7724463687149\\
1.69	-9.77244675062158\\
1.7	-9.77244718336718\\
1.71	-9.77244766822911\\
1.72	-9.77244820585418\\
1.73	-9.77244879621557\\
1.74	-9.77244943857971\\
1.75	-9.77245013148384\\
1.76	-9.77245087272459\\
1.77	-9.77245165935802\\
1.78	-9.7724524877112\\
1.79	-9.77245335340524\\
1.8	-9.77245425138969\\
1.81	-9.77245517598771\\
1.82	-9.77245612095157\\
1.83	-9.77245707952772\\
1.84	-9.77245804453048\\
1.85	-9.77245900842337\\
1.86	-9.77245996340696\\
1.87	-9.7724609015119\\
1.88	-9.77246181469579\\
1.89	-9.77246269494263\\
1.9	-9.77246353436315\\
1.91	-9.77246432529477\\
1.92	-9.77246506039959\\
1.93	-9.772465732759\\
1.94	-9.77246633596353\\
1.95	-9.77246686419652\\
1.96	-9.77246731231048\\
1.97	-9.77246767589494\\
1.98	-9.77246795133473\\
1.99	-9.77246813585796\\
2	-9.77246822757281\\
2.01	-9.77246822549272\\
2.02	-9.7724681295495\\
2.03	-9.77246794059429\\
2.04	-9.77246766038616\\
2.05	-9.77246729156876\\
2.06	-9.77246683763516\\
2.07	-9.77246630288153\\
2.08	-9.77246569235031\\
2.09	-9.7724650117637\\
2.1	-9.77246426744856\\
2.11	-9.77246346625366\\
2.12	-9.77246261546071\\
2.13	-9.77246172269038\\
2.14	-9.77246079580467\\
2.15	-9.77245984280721\\
2.16	-9.77245887174282\\
2.17	-9.77245789059792\\
2.18	-9.7724569072031\\
2.19	-9.77245592913951\\
2.2	-9.77245496365008\\
2.21	-9.77245401755724\\
2.22	-9.77245309718798\\
2.23	-9.7724522083074\\
2.24	-9.77245135606186\\
2.25	-9.77245054493208\\
2.26	-9.77244977869728\\
2.27	-9.77244906041031\\
2.28	-9.77244839238447\\
2.29	-9.77244777619171\\
2.3	-9.77244721267235\\
2.31	-9.77244670195597\\
2.32	-9.77244624349296\\
2.33	-9.77244583609613\\
2.34	-9.77244547799167\\
2.35	-9.77244516687837\\
2.36	-9.77244489999429\\
2.37	-9.77244467418942\\
2.38	-9.77244448600334\\
2.39	-9.77244433174643\\
2.4	-9.7724442075832\\
2.41	-9.77244410961646\\
2.42	-9.77244403397082\\
2.43	-9.77244397687416\\
2.44	-9.7724439347357\\
2.45	-9.77244390421937\\
2.46	-9.7724438823113\\
2.47	-9.77244386638021\\
2.48	-9.77244385422984\\
2.49	-9.7724438441424\\
2.5	-9.77244383491253\\
2.51	-9.77244382587092\\
2.52	-9.77244381689751\\
2.53	-9.77244380842383\\
2.54	-9.77244380142452\\
2.55	-9.7724437973982\\
2.56	-9.77244379833793\\
2.57	-9.77244380669185\\
2.58	-9.77244382531455\\
2.59	-9.77244385740997\\
2.6	-9.77244390646695\\
2.61	-9.77244397618819\\
2.62	-9.77244407041414\\
2.63	-9.77244419304281\\
2.64	-9.77244434794696\\
2.65	-9.7724445388901\\
2.66	-9.77244476944258\\
2.67	-9.77244504289928\\
2.68	-9.77244536220028\\
2.69	-9.77244572985581\\
2.7	-9.77244614787686\\
2.71	-9.77244661771259\\
2.72	-9.7724471401956\\
2.73	-9.77244771549621\\
2.74	-9.77244834308641\\
2.75	-9.77244902171427\\
2.76	-9.77244974938934\\
2.77	-9.77245052337937\\
2.78	-9.77245134021857\\
2.79	-9.77245219572741\\
2.8	-9.77245308504374\\
2.81	-9.77245400266495\\
2.82	-9.77245494250059\\
2.83	-9.77245589793474\\
2.84	-9.77245686189737\\
2.85	-9.77245782694358\\
2.86	-9.77245878533971\\
2.87	-9.772459729155\\
2.88	-9.77246065035756\\
2.89	-9.77246154091316\\
2.9	-9.7724623928855\\
2.91	-9.77246319853646\\
2.92	-9.77246395042475\\
2.93	-9.7724646415017\\
2.94	-9.77246526520256\\
2.95	-9.77246581553212\\
2.96	-9.77246628714329\\
2.97	-9.77246667540749\\
2.98	-9.7724669764758\\
2.99	-9.77246718732994\\
3	-9.77246730582235\\
3.01	-9.77246733070478\\
3.02	-9.77246726164488\\
3.03	-9.77246709923064\\
3.04	-9.77246684496256\\
3.05	-9.77246650123366\\
3.06	-9.7724660712977\\
3.07	-9.77246555922596\\
3.08	-9.77246496985338\\
3.09	-9.77246430871471\\
3.1	-9.77246358197174\\
3.11	-9.77246279633267\\
3.12	-9.77246195896475\\
3.13	-9.7724610774016\\
3.14	-9.77246015944648\\
3.15	-9.77245921307305\\
3.16	-9.77245824632494\\
3.17	-9.77245726721581\\
3.18	-9.77245628363112\\
3.19	-9.77245530323326\\
3.2	-9.77245433337131\\
3.21	-9.77245338099671\\
3.22	-9.77245245258609\\
3.23	-9.77245155407234\\
3.24	-9.77245069078486\\
3.25	-9.77244986739976\\
3.26	-9.77244908790077\\
3.27	-9.77244835555119\\
3.28	-9.77244767287734\\
3.29	-9.77244704166348\\
3.3	-9.77244646295826\\
3.31	-9.77244593709245\\
3.32	-9.77244546370749\\
3.33	-9.77244504179426\\
3.34	-9.77244466974148\\
3.35	-9.77244434539267\\
3.36	-9.77244406611076\\
3.37	-9.7724438288492\\
3.38	-9.7724436302283\\
3.39	-9.77244346661557\\
3.4	-9.7724433342086\\
3.41	-9.77244322911911\\
3.42	-9.77244314745682\\
3.43	-9.77244308541164\\
3.44	-9.77244303933282\\
3.45	-9.77244300580389\\
3.46	-9.77244298171184\\
3.47	-9.77244296430976\\
3.48	-9.77244295127158\\
3.49	-9.77244294073824\\
3.5	-9.7724429313544\\
3.51	-9.77244292229508\\
3.52	-9.77244291328188\\
3.53	-9.77244290458846\\
3.54	-9.77244289703516\\
3.55	-9.77244289197288\\
3.56	-9.77244289125651\\
3.57	-9.7724428972083\\
3.58	-9.77244291257181\\
3.59	-9.77244294045721\\
3.6	-9.77244298427874\\
3.61	-9.77244304768549\\
3.62	-9.77244313448663\\
3.63	-9.77244324857221\\
3.64	-9.772443393831\\
3.65	-9.77244357406666\\
3.66	-9.77244379291365\\
3.67	-9.77244405375433\\
3.68	-9.77244435963862\\
3.69	-9.77244471320761\\
3.7	-9.77244511662236\\
3.71	-9.77244557149927\\
3.72	-9.77244607885301\\
3.73	-9.77244663904808\\
3.74	-9.77244725175992\\
3.75	-9.77244791594626\\
3.76	-9.77244862982931\\
3.77	-9.77244939088922\\
3.78	-9.77245019586899\\
3.79	-9.77245104079103\\
3.8	-9.77245192098499\\
3.81	-9.77245283112685\\
3.82	-9.77245376528856\\
3.83	-9.77245471699776\\
3.84	-9.77245567930661\\
3.85	-9.77245664486888\\
3.86	-9.77245760602426\\
3.87	-9.77245855488849\\
3.88	-9.77245948344831\\
3.89	-9.77246038365955\\
3.9	-9.77246124754715\\
3.91	-9.77246206730555\\
3.92	-9.77246283539797\\
3.93	-9.77246354465318\\
3.94	-9.7724641883582\\
3.95	-9.77246476034574\\
3.96	-9.77246525507489\\
3.97	-9.77246566770399\\
3.98	-9.77246599415451\\
3.99	-9.77246623116508\\
4	-9.77246637633475\\
4.01	-9.77246642815491\\
4.02	-9.77246638602939\\
4.03	-9.77246625028236\\
4.04	-9.77246602215407\\
4.05	-9.77246570378428\\
4.06	-9.77246529818383\\
4.07	-9.77246480919469\\
4.08	-9.77246424143899\\
4.09	-9.77246360025799\\
4.1	-9.7724628916418\\
4.11	-9.7724621221508\\
4.12	-9.77246129883013\\
4.13	-9.77246042911839\\
4.14	-9.77245952075191\\
4.15	-9.77245858166609\\
4.16	-9.77245761989514\\
4.17	-9.77245664347186\\
4.18	-9.77245566032876\\
4.19	-9.77245467820207\\
4.2	-9.77245370454005\\
4.21	-9.7724527464168\\
4.22	-9.77245181045294\\
4.23	-9.77245090274423\\
4.24	-9.77245002879902\\
4.25	-9.77244919348561\\
4.26	-9.77244840098996\\
4.27	-9.77244765478447\\
4.28	-9.77244695760807\\
4.29	-9.77244631145789\\
4.3	-9.7724457175924\\
4.31	-9.77244517654594\\
4.32	-9.77244468815418\\
4.33	-9.77244425159012\\
4.34	-9.77244386540971\\
4.35	-9.77244352760652\\
4.36	-9.77244323567426\\
4.37	-9.77244298667611\\
4.38	-9.77244277731976\\
4.39	-9.7724426040367\\
4.4	-9.7724424630645\\
4.41	-9.77244235053073\\
4.42	-9.77244226253699\\
4.43	-9.77244219524173\\
4.44	-9.77244214494044\\
4.45	-9.77244210814188\\
4.46	-9.77244208163912\\
4.47	-9.77244206257408\\
4.48	-9.77244204849471\\
4.49	-9.77244203740367\\
4.5	-9.77244202779781\\
4.51	-9.77244201869781\\
4.52	-9.77244200966738\\
4.53	-9.77244200082186\\
4.54	-9.77244199282591\\
4.55	-9.77244198688054\\
4.56	-9.7724419846994\\
4.57	-9.77244198847507\\
4.58	-9.77244200083558\\
4.59	-9.77244202479211\\
4.6	-9.77244206367862\\
4.61	-9.77244212108442\\
4.62	-9.77244220078085\\
4.63	-9.77244230664325\\
4.64	-9.77244244256947\\
4.65	-9.77244261239639\\
4.66	-9.77244281981577\\
4.67	-9.77244306829083\\
4.68	-9.77244336097504\\
4.69	-9.77244370063447\\
4.7	-9.77244408957496\\
4.71	-9.77244452957546\\
4.72	-9.77244502182873\\
4.73	-9.77244556689026\\
4.74	-9.7724461646366\\
4.75	-9.77244681423368\\
4.76	-9.77244751411586\\
4.77	-9.77244826197611\\
4.78	-9.77244905476763\\
4.79	-9.77244988871704\\
4.8	-9.77245075934905\\
4.81	-9.77245166152234\\
4.82	-9.77245258947621\\
4.83	-9.77245353688753\\
4.84	-9.77245449693702\\
4.85	-9.77245546238409\\
4.86	-9.77245642564918\\
4.87	-9.77245737890241\\
4.88	-9.77245831415722\\
4.89	-9.77245922336773\\
4.9	-9.7724600985284\\
4.91	-9.77246093177442\\
4.92	-9.77246171548147\\
4.93	-9.7724624423634\\
4.94	-9.77246310556619\\
4.95	-9.77246369875711\\
4.96	-9.77246421620741\\
4.97	-9.77246465286757\\
4.98	-9.77246500443394\\
4.99	-9.77246526740557\\
5	-9.77246543913073\\
5.01	-9.77246551784209\\
5.02	-9.77246550268026\\
5.03	-9.77246539370524\\
5.04	-9.77246519189563\\
5.05	-9.77246489913556\\
5.06	-9.77246451818965\\
5.07	-9.7724640526663\\
5.08	-9.7724635069698\\
5.09	-9.77246288624214\\
5.1	-9.77246219629519\\
5.11	-9.77246144353449\\
5.12	-9.77246063487552\\
5.13	-9.77245977765396\\
5.14	-9.77245887953108\\
5.15	-9.77245794839574\\
5.16	-9.77245699226447\\
5.17	-9.77245601918108\\
5.18	-9.77245503711719\\
5.19	-9.77245405387534\\
5.2	-9.77245307699585\\
5.21	-9.77245211366899\\
5.22	-9.77245117065349\\
5.23	-9.77245025420281\\
5.24	-9.77244936999998\\
5.25	-9.77244852310197\\
5.26	-9.77244771789445\\
5.27	-9.77244695805728\\
5.28	-9.77244624654136\\
5.29	-9.77244558555696\\
5.3	-9.77244497657357\\
5.31	-9.77244442033123\\
5.32	-9.77244391686286\\
5.33	-9.7724434655273\\
5.34	-9.7724430650523\\
5.35	-9.77244271358663\\
5.36	-9.7724424087605\\
5.37	-9.77244214775305\\
5.38	-9.77244192736586\\
5.39	-9.7724417441012\\
5.4	-9.77244159424356\\
5.41	-9.77244147394328\\
5.42	-9.77244137930066\\
5.43	-9.77244130644931\\
5.44	-9.77244125163721\\
5.45	-9.77244121130426\\
5.46	-9.77244118215486\\
5.47	-9.7724411612245\\
5.48	-9.77244114593912\\
5.49	-9.77244113416636\\
5.5	-9.77244112425771\\
5.51	-9.77244111508105\\
5.52	-9.7724411060429\\
5.53	-9.77244109710006\\
5.54	-9.77244108876047\\
5.55	-9.77244108207323\\
5.56	-9.772441078608\\
5.57	-9.77244108042407\\
5.58	-9.7724410900296\\
5.59	-9.77244111033183\\
5.6	-9.77244114457884\\
5.61	-9.77244119629413\\
5.62	-9.77244126920478\\
5.63	-9.77244136716468\\
5.64	-9.77244149407389\\
5.65	-9.77244165379558\\
5.66	-9.77244185007191\\
5.67	-9.7724420864403\\
5.68	-9.77244236615137\\
5.69	-9.77244269209018\\
5.7	-9.77244306670186\\
5.71	-9.77244349192306\\
5.72	-9.7724439691204\\
5.73	-9.77244449903696\\
5.74	-9.7724450817478\\
5.75	-9.77244571662536\\
5.76	-9.77244640231536\\
5.77	-9.77244713672372\\
5.78	-9.77244791701494\\
5.79	-9.77244873962195\\
5.8	-9.77244960026746\\
5.81	-9.77245049399671\\
5.82	-9.77245141522108\\
5.83	-9.77245235777213\\
5.84	-9.77245331496534\\
5.85	-9.77245427967253\\
5.86	-9.77245524440225\\
5.87	-9.77245620138667\\
5.88	-9.77245714267398\\
5.89	-9.77245806022483\\
5.9	-9.77245894601144\\
5.91	-9.77245979211798\\
5.92	-9.77246059084063\\
5.93	-9.77246133478605\\
5.94	-9.77246201696654\\
5.95	-9.7724626308907\\
5.96	-9.77246317064813\\
5.97	-9.77246363098697\\
5.98	-9.77246400738305\\
5.99	-9.77246429609969\\
6	-9.77246449423722\\
6.01	-9.77246459977154\\
6.02	-9.77246461158095\\
6.03	-9.77246452946119\\
6.04	-9.7724643541281\\
6.05	-9.77246408720808\\
6.06	-9.77246373121651\\
6.07	-9.77246328952423\\
6.08	-9.77246276631288\\
6.09	-9.77246216651955\\
6.1	-9.77246149577161\\
6.11	-9.7724607603128\\
6.12	-9.77245996692157\\
6.13	-9.77245912282285\\
6.14	-9.77245823559475\\
6.15	-9.77245731307137\\
6.16	-9.77245636324329\\
6.17	-9.77245539415709\\
6.18	-9.77245441381557\\
6.19	-9.77245343007987\\
6.2	-9.77245245057518\\
6.21	-9.77245148260118\\
6.22	-9.77245053304867\\
6.23	-9.77244960832348\\
6.24	-9.7724487142787\\
6.25	-9.77244785615632\\
6.26	-9.77244703853885\\
6.27	-9.77244626531173\\
6.28	-9.7724455396369\\
6.29	-9.77244486393777\\
6.3	-9.77244423989583\\
6.31	-9.77244366845865\\
6.32	-9.77244314985914\\
6.33	-9.77244268364557\\
6.34	-9.77244226872178\\
6.35	-9.77244190339676\\
6.36	-9.77244158544279\\
6.37	-9.77244131216102\\
6.38	-9.77244108045343\\
6.39	-9.77244088689974\\
6.4	-9.7724407278383\\
6.41	-9.77244059944913\\
6.42	-9.77244049783812\\
6.43	-9.77244041912071\\
6.44	-9.77244035950378\\
6.45	-9.77244031536431\\
6.46	-9.77244028332354\\
6.47	-9.77244026031541\\
6.48	-9.77244024364807\\
6.49	-9.77244023105758\\
6.5	-9.77244022075275\\
6.51	-9.77244021145048\\
6.52	-9.77244020240105\\
6.53	-9.77244019340277\\
6.54	-9.77244018480602\\
6.55	-9.77244017750628\\
6.56	-9.77244017292665\\
6.57	-9.77244017298976\\
6.58	-9.7724401800798\\
6.59	-9.77244019699523\\
6.6	-9.77244022689288\\
6.61	-9.77244027322448\\
6.62	-9.77244033966652\\
6.63	-9.77244043004482\\
6.64	-9.77244054825476\\
6.65	-9.7724406981789\\
6.66	-9.7724408836029\\
6.67	-9.7724411081316\\
6.68	-9.7724413751063\\
6.69	-9.77244168752494\\
6.7	-9.77244204796628\\
6.71	-9.7724424585196\\
6.72	-9.77244292072103\\
6.73	-9.77244343549755\\
6.74	-9.77244400311993\\
6.75	-9.77244462316512\\
6.76	-9.77244529448914\\
6.77	-9.77244601521078\\
6.78	-9.77244678270666\\
6.79	-9.77244759361776\\
6.8	-9.77244844386756\\
6.81	-9.77244932869145\\
6.82	-9.77245024267735\\
6.83	-9.77245117981681\\
6.84	-9.77245213356601\\
6.85	-9.77245309691586\\
6.86	-9.77245406247016\\
6.87	-9.77245502253077\\
6.88	-9.77245596918859\\
6.89	-9.77245689441894\\
6.9	-9.7724577901801\\
6.91	-9.77245864851344\\
6.92	-9.77245946164378\\
6.93	-9.77246022207841\\
6.94	-9.77246092270338\\
6.95	-9.77246155687563\\
6.96	-9.77246211850949\\
6.97	-9.77246260215638\\
6.98	-9.77246300307658\\
6.99	-9.7724633173017\\
7	-9.77246354168733\\
7.01	-9.77246367395473\\
7.02	-9.77246371272115\\
7.03	-9.77246365751824\\
7.04	-9.77246350879828\\
7.05	-9.77246326792815\\
7.06	-9.77246293717115\\
7.07	-9.77246251965691\\
7.08	-9.77246201933981\\
7.09	-9.77246144094665\\
7.1	-9.77246078991419\\
7.11	-9.77246007231765\\
7.12	-9.77245929479111\\
7.13	-9.77245846444109\\
7.14	-9.77245758875451\\
7.15	-9.77245667550251\\
7.16	-9.77245573264139\\
7.17	-9.77245476821235\\
7.18	-9.77245379024124\\
7.19	-9.77245280664007\\
7.2	-9.77245182511149\\
7.21	-9.77245085305778\\
7.22	-9.77244989749552\\
7.23	-9.77244896497733\\
7.24	-9.77244806152159\\
7.25	-9.77244719255131\\
7.26	-9.77244636284279\\
7.27	-9.77244557648489\\
7.28	-9.77244483684934\\
7.29	-9.77244414657247\\
7.3	-9.77244350754841\\
7.31	-9.77244292093391\\
7.32	-9.77244238716436\\
7.33	-9.77244190598079\\
7.34	-9.77244147646722\\
7.35	-9.77244109709766\\
7.36	-9.77244076579191\\
7.37	-9.77244047997907\\
7.38	-9.77244023666782\\
7.39	-9.7724400325221\\
7.4	-9.7724398639409\\
7.41	-9.77243972714089\\
7.42	-9.77243961824045\\
7.43	-9.7724395333436\\
7.44	-9.77243946862259\\
7.45	-9.77243942039759\\
7.46	-9.77243938521233\\
7.47	-9.77243935990423\\
7.48	-9.7724393416681\\
7.49	-9.7724393281121\\
7.5	-9.77243931730524\\
7.51	-9.77243930781553\\
7.52	-9.77243929873818\\
7.53	-9.77243928971336\\
7.54	-9.77243928093325\\
7.55	-9.77243927313832\\
7.56	-9.77243926760273\\
7.57	-9.77243926610932\\
7.58	-9.7724392709144\\
7.59	-9.77243928470303\\
7.6	-9.77243931053556\\
7.61	-9.77243935178615\\
7.62	-9.77243941207449\\
7.63	-9.7724394951917\\
7.64	-9.77243960502177\\
7.65	-9.77243974545965\\
7.66	-9.77243992032761\\
7.67	-9.77244013329111\\
7.68	-9.77244038777554\\
7.69	-9.77244068688547\\
7.7	-9.77244103332753\\
7.71	-9.77244142933841\\
7.72	-9.77244187661909\\
7.73	-9.77244237627665\\
7.74	-9.77244292877444\\
7.75	-9.77244353389174\\
7.76	-9.77244419069353\\
7.77	-9.77244489751107\\
7.78	-9.77244565193367\\
7.79	-9.7724464508119\\
7.8	-9.77244729027241\\
7.81	-9.77244816574413\\
7.82	-9.77244907199574\\
7.83	-9.77245000318382\\
7.84	-9.77245095291107\\
7.85	-9.77245191429391\\
7.86	-9.77245288003843\\
7.87	-9.77245384252369\\
7.88	-9.77245479389116\\
7.89	-9.77245572613897\\
7.9	-9.77245663121969\\
7.91	-9.77245750114018\\
7.92	-9.77245832806203\\
7.93	-9.77245910440112\\
7.94	-9.77245982292483\\
7.95	-9.77246047684553\\
7.96	-9.77246105990883\\
7.97	-9.77246156647537\\
7.98	-9.77246199159496\\
7.99	-9.77246233107185\\
8	-9.77246258152029\\
8.01	-9.77246274040939\\
8.02	-9.77246280609681\\
8.03	-9.77246277785063\\
8.04	-9.77246265585902\\
8.05	-9.77246244122784\\
8.06	-9.7724621359658\\
8.07	-9.77246174295787\\
8.08	-9.77246126592683\\
8.09	-9.77246070938403\\
8.1	-9.77246007856969\\
8.11	-9.77245937938392\\
8.12	-9.77245861830934\\
8.13	-9.77245780232642\\
8.14	-9.77245693882298\\
8.15	-9.772456035499\\
8.16	-9.77245510026828\\
8.17	-9.7724541411583\\
8.18	-9.77245316620991\\
8.19	-9.77245218337809\\
8.2	-9.77245120043548\\
8.21	-9.7724502248799\\
8.22	-9.77244926384729\\
8.23	-9.77244832403129\\
8.24	-9.77244741161054\\
8.25	-9.77244653218485\\
8.26	-9.77244569072097\\
8.27	-9.77244489150876\\
8.28	-9.77244413812827\\
8.29	-9.77244343342818\\
8.3	-9.77244277951572\\
8.31	-9.77244217775812\\
8.32	-9.7724416287955\\
8.33	-9.77244113256475\\
8.34	-9.77244068833399\\
8.35	-9.77244029474686\\
8.36	-9.77243994987591\\
8.37	-9.772439651284\\
8.38	-9.7724393960928\\
8.39	-9.772439181057\\
8.4	-9.77243900264306\\
8.41	-9.77243885711127\\
8.42	-9.77243874059937\\
8.43	-9.77243864920681\\
8.44	-9.77243857907775\\
8.45	-9.7724385264818\\
8.46	-9.77243848789094\\
8.47	-9.77243846005133\\
8.48	-9.77243844004901\\
8.49	-9.77243842536816\\
8.5	-9.77243841394113\\
8.51	-9.77243840418936\\
8.52	-9.77243839505441\\
8.53	-9.77243838601889\\
8.54	-9.77243837711648\\
8.55	-9.77243836893137\\
8.56	-9.77243836258675\\
8.57	-9.77243835972274\\
8.58	-9.77243836246406\\
8.59	-9.77243837337798\\
8.6	-9.77243839542325\\
8.61	-9.77243843189085\\
8.62	-9.77243848633753\\
8.63	-9.7724385625133\\
8.64	-9.77243866428391\\
8.65	-9.77243879554987\\
8.66	-9.77243896016312\\
8.67	-9.77243916184285\\
8.68	-9.77243940409191\\
8.69	-9.77243969011517\\
8.7	-9.77244002274119\\
8.71	-9.77244040434863\\
8.72	-9.77244083679861\\
8.73	-9.77244132137415\\
8.74	-9.77244185872791\\
8.75	-9.772442448839\\
8.76	-9.77244309097981\\
8.77	-9.7724437836934\\
8.78	-9.77244452478205\\
8.79	-9.77244531130714\\
8.8	-9.77244613960069\\
8.81	-9.77244700528829\\
8.82	-9.77244790332336\\
8.83	-9.77244882803231\\
8.84	-9.77244977316994\\
8.85	-9.77245073198447\\
8.86	-9.77245169729115\\
8.87	-9.77245266155364\\
8.88	-9.77245361697172\\
8.89	-9.77245455557441\\
8.9	-9.7724554693168\\
8.91	-9.77245635017953\\
8.92	-9.77245719026913\\
8.93	-9.77245798191809\\
8.94	-9.77245871778281\\
8.95	-9.77245939093836\\
8.96	-9.77245999496836\\
8.97	-9.77246052404874\\
8.98	-9.77246097302424\\
8.99	-9.77246133747628\\
9	-9.77246161378143\\
9.01	-9.77246179915942\\
9.02	-9.77246189171013\\
9.03	-9.77246189043878\\
9.04	-9.77246179526926\\
9.05	-9.77246160704512\\
9.06	-9.7724613275183\\
9.07	-9.77246095932588\\
9.08	-9.77246050595499\\
9.09	-9.77245997169656\\
9.1	-9.7724593615886\\
9.11	-9.7724586813497\\
9.12	-9.77245793730401\\
9.13	-9.77245713629849\\
9.14	-9.77245628561399\\
9.15	-9.77245539287125\\
9.16	-9.77245446593327\\
9.17	-9.77245351280559\\
9.18	-9.77245254153583\\
9.19	-9.77245156011402\\
9.2	-9.77245057637515\\
9.21	-9.77244959790545\\
9.22	-9.77244863195357\\
9.23	-9.7724476853482\\
9.24	-9.772446764423\\
9.25	-9.77244587495014\\
9.26	-9.7724450220832\\
9.27	-9.77244421031036\\
9.28	-9.77244344341823\\
9.29	-9.77244272446704\\
9.3	-9.77244205577722\\
9.31	-9.77244143892762\\
9.32	-9.77244087476505\\
9.33	-9.77244036342514\\
9.34	-9.77243990436375\\
9.35	-9.77243949639859\\
9.36	-9.77243913776002\\
9.37	-9.77243882615035\\
9.38	-9.77243855881033\\
9.39	-9.77243833259191\\
9.4	-9.77243814403585\\
9.41	-9.77243798945288\\
9.42	-9.7724378650071\\
9.43	-9.7724377668002\\
9.44	-9.77243769095492\\
9.45	-9.77243763369666\\
9.46	-9.77243759143152\\
9.47	-9.77243756081986\\
9.48	-9.77243753884371\\
9.49	-9.7724375228674\\
9.5	-9.77243751068998\\
9.51	-9.77243750058882\\
9.52	-9.77243749135361\\
9.53	-9.77243748231016\\
9.54	-9.77243747333361\\
9.55	-9.77243746485091\\
9.56	-9.77243745783244\\
9.57	-9.77243745377291\\
9.58	-9.77243745466198\\
9.59	-9.7724374629449\\
9.6	-9.77243748147394\\
9.61	-9.77243751345141\\
9.62	-9.77243756236514\\
9.63	-9.7724376319176\\
9.64	-9.77243772594967\\
9.65	-9.77243784836053\\
9.66	-9.77243800302483\\
9.67	-9.77243819370864\\
9.68	-9.77243842398554\\
9.69	-9.77243869715423\\
9.7	-9.77243901615916\\
9.71	-9.77243938351541\\
9.72	-9.77243980123922\\
9.73	-9.7724402707853\\
9.74	-9.77244079299206\\
9.75	-9.77244136803573\\
9.76	-9.77244199539424\\
9.77	-9.77244267382156\\
9.78	-9.77244340133292\\
9.79	-9.77244417520152\\
9.8	-9.77244499196664\\
9.81	-9.77244584745334\\
9.82	-9.77244673680356\\
9.83	-9.77244765451811\\
9.84	-9.77244859450928\\
9.85	-9.77244955016315\\
9.86	-9.77245051441086\\
9.87	-9.77245147980788\\
9.88	-9.77245243862003\\
9.89	-9.77245338291516\\
9.9	-9.77245430465912\\
9.91	-9.77245519581456\\
9.92	-9.77245604844123\\
9.93	-9.77245685479626\\
9.94	-9.7724576074329\\
9.95	-9.77245829929631\\
9.96	-9.772458923815\\
9.97	-9.77245947498648\\
9.98	-9.77245994745599\\
9.99	-9.77246033658693\\
10	-9.77246063852213\\
10.01	-9.77246085023496\\
10.02	-9.77246096956954\\
10.03	-9.77246099526936\\
10.04	-9.77246092699405\\
10.05	-9.7724607653239\\
10.06	-9.77246051175214\\
10.07	-9.77246016866505\\
10.08	-9.77245973931026\\
10.09	-9.77245922775361\\
10.1	-9.77245863882536\\
10.11	-9.77245797805641\\
10.12	-9.77245725160559\\
10.13	-9.77245646617902\\
10.14	-9.77245562894281\\
10.15	-9.77245474743039\\
10.16	-9.77245382944576\\
10.17	-9.77245288296422\\
10.18	-9.77245191603196\\
10.19	-9.77245093666602\\
10.2	-9.77244995275603\\
10.21	-9.77244897196929\\
10.22	-9.77244800166043\\
10.23	-9.77244704878697\\
10.24	-9.77244611983214\\
10.25	-9.77244522073577\\
10.26	-9.77244435683446\\
10.27	-9.77244353281173\\
10.28	-9.77244275265871\\
10.29	-9.77244201964611\\
10.3	-9.77244133630746\\
10.31	-9.772440704434\\
10.32	-9.77244012508099\\
10.33	-9.77243959858539\\
10.34	-9.77243912459427\\
10.35	-9.7724387021036\\
10.36	-9.77243832950648\\
10.37	-9.77243800465013\\
10.38	-9.77243772490042\\
10.39	-9.77243748721298\\
10.4	-9.77243728820951\\
10.41	-9.7724371242581\\
10.42	-9.77243699155617\\
10.43	-9.7724368862145\\
10.44	-9.77243680434117\\
10.45	-9.77243674212374\\
10.46	-9.77243669590858\\
10.47	-9.77243666227571\\
10.48	-9.77243663810819\\
10.49	-9.7724366206548\\
10.5	-9.77243660758485\\
10.51	-9.77243659703447\\
10.52	-9.77243658764339\\
10.53	-9.77243657858171\\
10.54	-9.77243656956622\\
10.55	-9.77243656086593\\
10.56	-9.77243655329679\\
10.57	-9.7724365482057\\
10.58	-9.77243654744399\\
10.59	-9.77243655333088\\
10.6	-9.77243656860743\\
10.61	-9.77243659638197\\
10.62	-9.77243664006757\\
10.63	-9.77243670331285\\
10.64	-9.77243678992721\\
10.65	-9.77243690380167\\
10.66	-9.77243704882669\\
10.67	-9.7724372288083\\
10.68	-9.77243744738399\\
10.69	-9.77243770793979\\
10.7	-9.77243801352984\\
10.71	-9.77243836679993\\
10.72	-9.77243876991628\\
10.73	-9.77243922450076\\
10.74	-9.77243973157379\\
10.75	-9.77244029150576\\
10.76	-9.77244090397805\\
10.77	-9.7724415679543\\
10.78	-9.7724422816625\\
10.79	-9.7724430425883\\
10.8	-9.77244384747991\\
10.81	-9.77244469236447\\
10.82	-9.77244557257586\\
10.83	-9.77244648279368\\
10.84	-9.77244741709285\\
10.85	-9.7724483690032\\
10.86	-9.77244933157832\\
10.87	-9.77245029747258\\
10.88	-9.77245125902538\\
10.89	-9.77245220835136\\
10.9	-9.77245313743522\\
10.91	-9.77245403822992\\
10.92	-9.7724549027567\\
10.93	-9.77245572320546\\
10.94	-9.77245649203417\\
10.95	-9.7724572020656\\
10.96	-9.77245784658021\\
10.97	-9.77245841940358\\
10.98	-9.77245891498721\\
10.99	-9.7724593284815\\
11	-9.77245965579976\\
11.01	-9.77245989367228\\
11.02	-9.77246003968975\\
11.03	-9.77246009233528\\
11.04	-9.77246005100459\\
11.05	-9.77245991601408\\
11.06	-9.77245968859657\\
11.07	-9.77245937088492\\
11.08	-9.77245896588366\\
11.09	-9.7724584774291\\
11.1	-9.77245791013849\\
11.11	-9.77245726934899\\
11.12	-9.77245656104747\\
11.13	-9.77245579179199\\
11.14	-9.77245496862628\\
11.15	-9.77245409898848\\
11.16	-9.77245319061538\\
11.17	-9.77245225144378\\
11.18	-9.77245128951018\\
11.19	-9.77245031285054\\
11.2	-9.7724493294013\\
11.21	-9.77244834690343\\
11.22	-9.77244737281051\\
11.23	-9.77244641420264\\
11.24	-9.77244547770686\\
11.25	-9.77244456942579\\
11.26	-9.77244369487494\\
11.27	-9.77244285892993\\
11.28	-9.77244206578417\\
11.29	-9.77244131891745\\
11.3	-9.772440621076\\
11.31	-9.772439974264\\
11.32	-9.77243937974666\\
11.33	-9.7724388380646\\
11.34	-9.77243834905934\\
11.35	-9.77243791190908\\
11.36	-9.7724375251744\\
11.37	-9.77243718685275\\
11.38	-9.77243689444101\\
11.39	-9.77243664500475\\
11.4	-9.77243643525329\\
11.41	-9.77243626161892\\
11.42	-9.77243612033926\\
11.43	-9.77243600754116\\
11.44	-9.77243591932477\\
11.45	-9.77243585184641\\
11.46	-9.77243580139879\\
11.47	-9.77243576448738\\
11.48	-9.7724357379014\\
11.49	-9.77243571877858\\
11.5	-9.77243570466231\\
11.51	-9.7724356935505\\
11.52	-9.77243568393509\\
11.53	-9.77243567483184\\
11.54	-9.77243566579959\\
11.55	-9.77243565694896\\
11.56	-9.77243564894018\\
11.57	-9.7724356429701\\
11.58	-9.77243564074871\\
11.59	-9.77243564446537\\
11.6	-9.77243565674546\\
11.61	-9.77243568059813\\
11.62	-9.77243571935597\\
11.63	-9.77243577660763\\
11.64	-9.77243585612447\\
11.65	-9.77243596178258\\
11.66	-9.77243609748131\\
11.67	-9.77243626705971\\
11.68	-9.77243647421238\\
11.69	-9.77243672240603\\
11.7	-9.77243701479821\\
11.71	-9.77243735415955\\
11.72	-9.77243774280093\\
11.73	-9.77243818250671\\
11.74	-9.77243867447529\\
11.75	-9.77243921926804\\
11.76	-9.77243981676745\\
11.77	-9.77244046614539\\
11.78	-9.77244116584202\\
11.79	-9.77244191355591\\
11.8	-9.77244270624558\\
11.81	-9.7724435401425\\
11.82	-9.77244441077579\\
11.83	-9.77244531300792\\
11.84	-9.77244624108135\\
11.85	-9.77244718867537\\
11.86	-9.77244814897236\\
11.87	-9.77244911473255\\
11.88	-9.7724500783764\\
11.89	-9.7724510320731\\
11.9	-9.77245196783436\\
11.91	-9.77245287761166\\
11.92	-9.77245375339598\\
11.93	-9.77245458731821\\
11.94	-9.77245537174899\\
11.95	-9.77245609939636\\
11.96	-9.77245676339992\\
11.97	-9.77245735741993\\
11.98	-9.77245787572018\\
11.99	-9.77245831324334\\
12	-9.77245866567761\\
12.01	-9.77245892951375\\
12.02	-9.77245910209166\\
12.03	-9.7724591816357\\
12.04	-9.77245916727832\\
12.05	-9.77245905907164\\
12.06	-9.77245885798669\\
12.07	-9.77245856590057\\
12.08	-9.77245818557141\\
12.09	-9.77245772060174\\
12.1	-9.77245717539075\\
12.11	-9.77245655507608\\
12.12	-9.77245586546615\\
12.13	-9.77245511296387\\
12.14	-9.77245430448308\\
12.15	-9.77245344735871\\
12.16	-9.77245254925222\\
12.17	-9.77245161805361\\
12.18	-9.77245066178147\\
12.19	-9.77244968848248\\
12.2	-9.77244870613201\\
12.21	-9.7724477225371\\
12.22	-9.77244674524326\\
12.23	-9.77244578144652\\
12.24	-9.77244483791196\\
12.25	-9.77244392089978\\
12.26	-9.77244303610008\\
12.27	-9.77244218857714\\
12.28	-9.77244138272402\\
12.29	-9.77244062222803\\
12.3	-9.77243991004739\\
12.31	-9.77243924839951\\
12.32	-9.77243863876071\\
12.33	-9.77243808187746\\
12.34	-9.77243757778865\\
12.35	-9.77243712585853\\
12.36	-9.77243672481964\\
12.37	-9.77243637282485\\
12.38	-9.77243606750775\\
12.39	-9.7724358060501\\
12.4	-9.77243558525533\\
12.41	-9.77243540162673\\
12.42	-9.77243525144908\\
12.43	-9.7724351308722\\
12.44	-9.77243503599515\\
12.45	-9.77243496294959\\
12.46	-9.77243490798093\\
12.47	-9.77243486752588\\
12.48	-9.77243483828516\\
12.49	-9.77243481729017\\
12.5	-9.77243480196238\\
12.51	-9.77243479016472\\
12.52	-9.7724347802438\\
12.53	-9.77243477106259\\
12.54	-9.7724347620227\\
12.55	-9.77243475307617\\
12.56	-9.77243474472639\\
12.57	-9.77243473801826\\
12.58	-9.77243473451759\\
12.59	-9.77243473628031\\
12.6	-9.7724347458118\\
12.61	-9.77243476601705\\
12.62	-9.77243480014262\\
12.63	-9.77243485171109\\
12.64	-9.7724349244494\\
12.65	-9.77243502221197\\
12.66	-9.77243514890014\\
12.67	-9.77243530837906\\
12.68	-9.77243550439357\\
12.69	-9.77243574048436\\
12.7	-9.77243601990599\\
12.71	-9.77243634554794\\
12.72	-9.77243671986025\\
12.73	-9.77243714478488\\
12.74	-9.77243762169405\\
12.75	-9.77243815133665\\
12.76	-9.77243873379367\\
12.77	-9.7724393684435\\
12.78	-9.77244005393771\\
12.79	-9.77244078818791\\
12.8	-9.77244156836398\\
12.81	-9.77244239090386\\
12.82	-9.77244325153481\\
12.83	-9.77244414530603\\
12.84	-9.77244506663226\\
12.85	-9.7724460093477\\
12.86	-9.77244696676969\\
12.87	-9.77244793177116\\
12.88	-9.77244889686087\\
12.89	-9.77244985427035\\
12.9	-9.7724507960463\\
12.91	-9.77245171414697\\
12.92	-9.77245260054135\\
12.93	-9.77245344730953\\
12.94	-9.77245424674287\\
12.95	-9.77245499144244\\
12.96	-9.77245567441429\\
12.97	-9.77245628916019\\
12.98	-9.77245682976242\\
12.99	-9.77245729096138\\
13	-9.77245766822483\\
13.01	-9.77245795780784\\
13.02	-9.7724581568024\\
13.03	-9.77245826317604\\
13.04	-9.77245827579887\\
13.05	-9.77245819445863\\
13.06	-9.77245801986351\\
13.07	-9.77245775363273\\
13.08	-9.77245739827497\\
13.09	-9.77245695715505\\
13.1	-9.77245643444928\\
13.11	-9.77245583509016\\
13.12	-9.77245516470136\\
13.13	-9.77245442952376\\
13.14	-9.77245363633386\\
13.15	-9.77245279235559\\
13.16	-9.77245190516699\\
13.17	-9.77245098260304\\
13.18	-9.77245003265607\\
13.19	-9.77244906337537\\
13.2	-9.7724480827672\\
13.21	-9.772447098697\\
13.22	-9.77244611879495\\
13.23	-9.77244515036636\\
13.24	-9.77244420030821\\
13.25	-9.77244327503295\\
13.26	-9.77244238040069\\
13.27	-9.77244152166064\\
13.28	-9.77244070340271\\
13.29	-9.77243992951978\\
13.3	-9.77243920318115\\
13.31	-9.77243852681744\\
13.32	-9.77243790211706\\
13.33	-9.77243733003414\\
13.34	-9.77243681080771\\
13.35	-9.77243634399161\\
13.36	-9.77243592849466\\
13.37	-9.77243556263014\\
13.38	-9.77243524417392\\
13.39	-9.77243497043001\\
13.4	-9.77243473830244\\
13.41	-9.77243454437222\\
13.42	-9.77243438497816\\
13.43	-9.77243425630001\\
13.44	-9.77243415444263\\
13.45	-9.7724340755197\\
13.46	-9.77243401573569\\
13.47	-9.77243397146458\\
13.48	-9.77243393932407\\
13.49	-9.77243391624408\\
13.5	-9.77243389952843\\
13.51	-9.7724338869085\\
13.52	-9.77243387658828\\
13.53	-9.77243386727976\\
13.54	-9.77243385822827\\
13.55	-9.77243384922734\\
13.56	-9.77243384062271\\
13.57	-9.77243383330556\\
13.58	-9.77243382869502\\
13.59	-9.77243382871023\\
13.6	-9.77243383573242\\
13.61	-9.77243385255764\\
13.62	-9.77243388234099\\
13.63	-9.77243392853309\\
13.64	-9.77243399481008\\
13.65	-9.77243408499811\\
13.66	-9.77243420299364\\
13.67	-9.77243435268099\\
13.68	-9.77243453784831\\
13.69	-9.77243476210355\\
13.7	-9.77243502879176\\
13.71	-9.77243534091516\\
13.72	-9.77243570105732\\
13.73	-9.7724361113127\\
13.74	-9.77243657322294\\
13.75	-9.77243708772081\\
13.76	-9.77243765508296\\
13.77	-9.77243827489232\\
13.78	-9.77243894601079\\
13.79	-9.77243966656289\\
13.8	-9.77244043393074\\
13.81	-9.77244124476046\\
13.82	-9.77244209498017\\
13.83	-9.77244297982944\\
13.84	-9.77244389389972\\
13.85	-9.7724448311854\\
13.86	-9.77244578514478\\
13.87	-9.77244674877006\\
13.88	-9.77244771466554\\
13.89	-9.77244867513268\\
13.9	-9.77244962226112\\
13.91	-9.77245054802405\\
13.92	-9.77245144437676\\
13.93	-9.77245230335678\\
13.94	-9.77245311718431\\
13.95	-9.77245387836125\\
13.96	-9.77245457976762\\
13.97	-9.77245521475367\\
13.98	-9.77245577722652\\
13.99	-9.77245626173\\
14	-9.77245666351636\\
14.01	-9.77245697860904\\
14.02	-9.77245720385528\\
14.03	-9.77245733696797\\
14.04	-9.77245737655611\\
14.05	-9.77245732214328\\
14.06	-9.77245717417401\\
14.07	-9.77245693400783\\
14.08	-9.77245660390123\\
14.09	-9.77245618697759\\
14.1	-9.77245568718576\\
14.11	-9.77245510924774\\
14.12	-9.77245445859631\\
14.13	-9.77245374130359\\
14.14	-9.77245296400146\\
14.15	-9.77245213379515\\
14.16	-9.77245125817125\\
14.17	-9.77245034490151\\
14.18	-9.77244940194375\\
14.19	-9.77244843734156\\
14.2	-9.77244745912409\\
14.21	-9.77244647520739\\
14.22	-9.77244549329892\\
14.23	-9.77244452080641\\
14.24	-9.77244356475247\\
14.25	-9.77244263169624\\
14.26	-9.77244172766299\\
14.27	-9.77244085808292\\
14.28	-9.77244002773969\\
14.29	-9.77243924072959\\
14.3	-9.77243850043176\\
14.31	-9.77243780948978\\
14.32	-9.7724371698048\\
14.33	-9.77243658254024\\
14.34	-9.77243604813774\\
14.35	-9.77243556634407\\
14.36	-9.77243513624841\\
14.37	-9.77243475632927\\
14.38	-9.77243442451023\\
14.39	-9.77243413822347\\
14.4	-9.77243389447997\\
14.41	-9.77243368994514\\
14.42	-9.7724335210187\\
14.43	-9.77243338391725\\
14.44	-9.77243327475835\\
14.45	-9.77243318964446\\
14.46	-9.77243312474562\\
14.47	-9.77243307637915\\
14.48	-9.77243304108537\\
14.49	-9.77243301569785\\
14.5	-9.77243299740712\\
14.51	-9.77243298381674\\
14.52	-9.77243297299096\\
14.53	-9.77243296349288\\
14.54	-9.77243295441277\\
14.55	-9.77243294538595\\
14.56	-9.77243293659992\\
14.57	-9.77243292879073\\
14.58	-9.77243292322849\\
14.59	-9.77243292169238\\
14.6	-9.7724329264356\\
14.61	-9.77243294014065\\
14.62	-9.77243296586592\\
14.63	-9.77243300698431\\
14.64	-9.77243306711491\\
14.65	-9.77243315004901\\
14.66	-9.77243325967143\\
14.67	-9.77243339987871\\
14.68	-9.77243357449539\\
14.69	-9.77243378718987\\
14.7	-9.77243404139112\\
14.71	-9.77243434020784\\
14.72	-9.77243468635133\\
14.73	-9.77243508206333\\
14.74	-9.7724355290503\\
14.75	-9.772436028425\\
14.76	-9.77243658065666\\
14.77	-9.7724371855305\\
14.78	-9.77243784211744\\
14.79	-9.77243854875453\\
14.8	-9.77243930303665\\
14.81	-9.7724401018196\\
14.82	-9.77244094123484\\
14.83	-9.77244181671564\\
14.84	-9.77244272303439\\
14.85	-9.7724436543507\\
14.86	-9.77244460426961\\
14.87	-9.77244556590909\\
14.88	-9.77244653197593\\
14.89	-9.77244749484908\\
14.9	-9.77244844666899\\
14.91	-9.77244937943189\\
14.92	-9.77245028508762\\
14.93	-9.77245115563945\\
14.94	-9.77245198324455\\
14.95	-9.77245276031363\\
14.96	-9.77245347960821\\
14.97	-9.77245413433417\\
14.98	-9.77245471823004\\
14.99	-9.77245522564895\\
15	-9.77245565163283\\
15.01	-9.77245599197778\\
15.02	-9.77245624328971\\
15.03	-9.77245640302939\\
15.04	-9.77245646954618\\
15.05	-9.7724564421\\
15.06	-9.77245632087119\\
15.07	-9.77245610695813\\
15.08	-9.77245580236256\\
15.09	-9.77245540996302\\
15.1	-9.77245493347654\\
15.11	-9.77245437740946\\
15.12	-9.7724537469978\\
15.13	-9.77245304813829\\
15.14	-9.77245228731108\\
15.15	-9.77245147149512\\
15.16	-9.77245060807759\\
15.17	-9.77244970475883\\
15.18	-9.77244876945389\\
15.19	-9.77244781019242\\
15.2	-9.77244683501828\\
15.21	-9.77244585189031\\
15.22	-9.77244486858573\\
15.23	-9.77244389260763\\
15.24	-9.77244293109785\\
15.25	-9.77244199075639\\
15.26	-9.7724410777687\\
15.27	-9.77244019774171\\
15.28	-9.77243935564949\\
15.29	-9.7724385557893\\
15.3	-9.77243780174864\\
15.31	-9.7724370963835\\
15.32	-9.77243644180817\\
15.33	-9.7724358393967\\
15.34	-9.77243528979559\\
15.35	-9.7724347929476\\
15.36	-9.7724343481262\\
15.37	-9.77243395397971\\
15.38	-9.77243360858468\\
15.39	-9.77243330950727\\
15.4	-9.77243305387165\\
15.41	-9.77243283843423\\
15.42	-9.77243265966242\\
15.43	-9.77243251381666\\
15.44	-9.77243239703408\\
15.45	-9.77243230541278\\
15.46	-9.77243223509489\\
15.47	-9.77243218234736\\
15.48	-9.7724321436389\\
15.49	-9.77243211571193\\
15.5	-9.77243209564834\\
15.51	-9.7724320809278\\
15.52	-9.77243206947792\\
15.53	-9.77243205971524\\
15.54	-9.77243205057643\\
15.55	-9.77243204153917\\
15.56	-9.77243203263243\\
15.57	-9.77243202443589\\
15.58	-9.77243201806857\\
15.59	-9.77243201516681\\
15.6	-9.77243201785205\\
15.61	-9.77243202868884\\
15.62	-9.7724320506338\\
15.63	-9.77243208697644\\
15.64	-9.77243214127273\\
15.65	-9.7724322172726\\
15.66	-9.77243231884246\\
15.67	-9.77243244988419\\
15.68	-9.77243261425181\\
15.69	-9.77243281566726\\
15.7	-9.77243305763679\\
15.71	-9.77243334336925\\
15.72	-9.77243367569772\\
15.73	-9.77243405700582\\
15.74	-9.77243448916001\\
15.75	-9.77243497344897\\
15.76	-9.77243551053118\\
15.77	-9.77243610039169\\
15.78	-9.77243674230882\\
15.79	-9.7724374348315\\
15.8	-9.77243817576763\\
15.81	-9.77243896218396\\
15.82	-9.77243979041743\\
15.83	-9.77244065609811\\
15.84	-9.77244155418331\\
15.85	-9.77244247900269\\
15.86	-9.7724434243136\\
15.87	-9.77244438336602\\
15.88	-9.77244534897618\\
15.89	-9.77244631360782\\
15.9	-9.77244726945999\\
15.91	-9.77244820856006\\
15.92	-9.77244912286064\\
15.93	-9.77245000433898\\
15.94	-9.77245084509746\\
15.95	-9.7724516374636\\
15.96	-9.77245237408815\\
15.97	-9.77245304803983\\
15.98	-9.77245365289535\\
15.99	-9.77245418282326\\
16	-9.77245463266049\\
16.01	-9.77245499798039\\
16.02	-9.77245527515124\\
16.03	-9.77245546138442\\
16.04	-9.77245555477145\\
16.05	-9.77245555430939\\
16.06	-9.77245545991417\\
16.07	-9.77245527242174\\
16.08	-9.77245499357693\\
16.09	-9.77245462601021\\
16.1	-9.77245417320277\\
16.11	-9.77245363944031\\
16.12	-9.77245302975638\\
16.13	-9.77245234986598\\
16.14	-9.77245160609051\\
16.15	-9.77245080527513\\
16.16	-9.77244995469982\\
16.17	-9.77244906198536\\
16.18	-9.77244813499577\\
16.19	-9.77244718173849\\
16.2	-9.77244621026392\\
16.21	-9.77244522856571\\
16.22	-9.77244424448323\\
16.23	-9.77244326560779\\
16.24	-9.77244229919377\\
16.25	-9.7724413520761\\
16.26	-9.77244043059512\\
16.27	-9.77243954053004\\
16.28	-9.77243868704173\\
16.29	-9.77243787462576\\
16.3	-9.77243710707617\\
16.31	-9.77243638746055\\
16.32	-9.7724357181065\\
16.33	-9.77243510059975\\
16.34	-9.77243453579362\\
16.35	-9.77243402382977\\
16.36	-9.77243356416956\\
16.37	-9.77243315563559\\
16.38	-9.77243279646243\\
16.39	-9.77243248435587\\
16.4	-9.77243221655941\\
16.41	-9.77243198992695\\
16.42	-9.7724318010004\\
16.43	-9.77243164609087\\
16.44	-9.77243152136209\\
16.45	-9.77243142291458\\
16.46	-9.77243134686926\\
16.47	-9.77243128944902\\
16.48	-9.77243124705689\\
16.49	-9.7724312163496\\
16.5	-9.77243119430514\\
16.51	-9.77243117828342\\
16.52	-9.7724311660788\\
16.53	-9.77243115596382\\
16.54	-9.77243114672322\\
16.55	-9.77243113767792\\
16.56	-9.77243112869826\\
16.57	-9.77243112020661\\
16.58	-9.77243111316906\\
16.59	-9.77243110907646\\
16.6	-9.77243110991502\\
16.61	-9.77243111812709\\
16.62	-9.77243113656264\\
16.63	-9.77243116842233\\
16.64	-9.77243121719301\\
16.65	-9.77243128657685\\
16.66	-9.77243138041518\\
16.67	-9.77243150260835\\
16.68	-9.7724316570329\\
16.69	-9.77243184745745\\
16.7	-9.7724320774588\\
16.71	-9.77243235033947\\
16.72	-9.77243266904828\\
16.73	-9.77243303610516\\
16.74	-9.77243345353158\\
16.75	-9.77243392278783\\
16.76	-9.7724344447181\\
16.77	-9.77243501950455\\
16.78	-9.77243564663106\\
16.79	-9.77243632485746\\
16.8	-9.77243705220472\\
16.81	-9.77243782595147\\
16.82	-9.77243864264205\\
16.83	-9.77243949810616\\
16.84	-9.77244038748978\\
16.85	-9.77244130529717\\
16.86	-9.77244224544338\\
16.87	-9.77244320131648\\
16.88	-9.77244416584881\\
16.89	-9.77244513159619\\
16.9	-9.77244609082395\\
16.91	-9.77244703559855\\
16.92	-9.77244795788357\\
16.93	-9.77244884963856\\
16.94	-9.77244970291933\\
16.95	-9.77245050997827\\
16.96	-9.77245126336316\\
16.97	-9.772451956013\\
16.98	-9.77245258134954\\
16.99	-9.77245313336307\\
17	-9.77245360669109\\
17.01	-9.77245399668901\\
17.02	-9.77245429949143\\
17.03	-9.77245451206336\\
17.04	-9.77245463224057\\
17.05	-9.7724546587583\\
17.06	-9.77245459126816\\
17.07	-9.77245443034273\\
17.08	-9.77245417746797\\
17.09	-9.77245383502341\\
17.1	-9.77245340625052\\
17.11	-9.77245289520972\\
17.12	-9.77245230672658\\
17.13	-9.77245164632814\\
17.14	-9.77245092017024\\
17.15	-9.77245013495693\\
17.16	-9.77244929785318\\
17.17	-9.7724484163922\\
17.18	-9.77244749837872\\
17.19	-9.77244655178971\\
17.2	-9.77244558467388\\
17.21	-9.77244460505165\\
17.22	-9.77244362081683\\
17.23	-9.77244263964161\\
17.24	-9.77244166888614\\
17.25	-9.77244071551408\\
17.26	-9.77243978601522\\
17.27	-9.77243888633632\\
17.28	-9.77243802182123\\
17.29	-9.77243719716081\\
17.3	-9.77243641635366\\
17.31	-9.77243568267786\\
17.32	-9.77243499867417\\
17.33	-9.77243436614081\\
17.34	-9.77243378613966\\
17.35	-9.77243325901387\\
17.36	-9.77243278441618\\
17.37	-9.77243236134759\\
17.38	-9.77243198820565\\
17.39	-9.77243166284126\\
17.4	-9.77243138262325\\
17.41	-9.77243114450942\\
17.42	-9.77243094512287\\
17.43	-9.77243078083229\\
17.44	-9.77243064783492\\
17.45	-9.77243054224063\\
17.46	-9.77243046015583\\
17.47	-9.77243039776579\\
17.48	-9.77243035141394\\
17.49	-9.77243031767683\\
17.5	-9.7724302934336\\
17.51	-9.77243027592868\\
17.52	-9.7724302628268\\
17.53	-9.77243025225926\\
17.54	-9.77243024286088\\
17.55	-9.77243023379684\\
17.56	-9.77243022477908\\
17.57	-9.77243021607196\\
17.58	-9.77243020848707\\
17.59	-9.7724302033673\\
17.6	-9.7724302025604\\
17.61	-9.77243020838253\\
17.62	-9.77243022357229\\
17.63	-9.77243025123615\\
17.64	-9.77243029478599\\
17.65	-9.77243035786995\\
17.66	-9.77243044429769\\
17.67	-9.77243055796117\\
17.68	-9.77243070275253\\
17.69	-9.77243088248015\\
17.7	-9.77243110078459\\
17.71	-9.77243136105552\\
17.72	-9.77243166635128\\
17.73	-9.77243201932239\\
17.74	-9.77243242214021\\
17.75	-9.77243287643211\\
17.76	-9.7724333832242\\
17.77	-9.7724339428928\\
17.78	-9.77243455512523\\
17.79	-9.77243521889105\\
17.8	-9.772435932424\\
17.81	-9.77243669321529\\
17.82	-9.77243749801829\\
17.83	-9.7724383428649\\
17.84	-9.77243922309324\\
17.85	-9.77244013338654\\
17.86	-9.77244106782267\\
17.87	-9.77244201993368\\
17.88	-9.77244298277459\\
17.89	-9.77244394900038\\
17.9	-9.77244491095019\\
17.91	-9.77244586073752\\
17.92	-9.77244679034508\\
17.93	-9.77244769172295\\
17.94	-9.77244855688866\\
17.95	-9.7724493780276\\
17.96	-9.77245014759247\\
17.97	-9.77245085840007\\
17.98	-9.77245150372429\\
17.99	-9.77245207738357\\
18	-9.77245257382183\\
18.01	-9.77245298818156\\
18.02	-9.77245331636786\\
18.03	-9.7724535551027\\
18.04	-9.77245370196842\\
18.05	-9.77245375543982\\
18.06	-9.77245371490455\\
18.07	-9.77245358067116\\
18.08	-9.77245335396509\\
18.09	-9.77245303691229\\
18.1	-9.77245263251096\\
18.11	-9.77245214459174\\
18.12	-9.77245157776698\\
18.13	-9.77245093736977\\
18.14	-9.7724502293837\\
18.15	-9.77244946036451\\
18.16	-9.77244863735451\\
18.17	-9.77244776779136\\
18.18	-9.77244685941233\\
18.19	-9.77244592015559\\
18.2	-9.77244495805994\\
18.21	-9.77244398116447\\
18.22	-9.77244299740959\\
18.23	-9.77244201454093\\
18.24	-9.77244104001747\\
18.25	-9.77244008092524\\
18.26	-9.77243914389774\\
18.27	-9.77243823504443\\
18.28	-9.77243735988798\\
18.29	-9.77243652331136\\
18.3	-9.77243572951539\\
18.31	-9.7724349819873\\
18.32	-9.77243428348058\\
18.33	-9.77243363600648\\
18.34	-9.77243304083693\\
18.35	-9.77243249851891\\
18.36	-9.77243200889971\\
18.37	-9.77243157116282\\
18.38	-9.77243118387337\\
18.39	-9.77243084503277\\
18.4	-9.77243055214104\\
18.41	-9.77243030226617\\
18.42	-9.77243009211909\\
18.43	-9.77242991813291\\
18.44	-9.77242977654532\\
18.45	-9.77242966348244\\
18.46	-9.77242957504297\\
18.47	-9.77242950738111\\
18.48	-9.77242945678682\\
18.49	-9.77242941976224\\
18.5	-9.77242939309278\\
18.51	-9.77242937391195\\
18.52	-9.77242935975862\\
18.53	-9.7724293486259\\
18.54	-9.77242933900088\\
18.55	-9.77242932989434\\
18.56	-9.77242932086028\\
18.57	-9.77242931200459\\
18.58	-9.77242930398304\\
18.59	-9.77242929798838\\
18.6	-9.77242929572687\\
18.61	-9.77242929938466\\
18.62	-9.7724293115845\\
18.63	-9.77242933533349\\
18.64	-9.77242937396285\\
18.65	-9.77242943106052\\
18.66	-9.77242951039792\\
18.67	-9.77242961585189\\
18.68	-9.77242975132324\\
18.69	-9.7724299206532\\
18.7	-9.77243012753921\\
18.71	-9.77243037545147\\
18.72	-9.77243066755159\\
18.73	-9.77243100661478\\
18.74	-9.77243139495694\\
18.75	-9.77243183436783\\
18.76	-9.77243232605152\\
18.77	-9.77243287057524\\
18.78	-9.77243346782742\\
18.79	-9.77243411698587\\
18.8	-9.77243481649658\\
18.81	-9.77243556406371\\
18.82	-9.77243635665108\\
18.83	-9.77243719049507\\
18.84	-9.77243806112915\\
18.85	-9.77243896341963\\
18.86	-9.77243989161212\\
18.87	-9.77244083938836\\
18.88	-9.77244179993236\\
18.89	-9.77244276600523\\
18.9	-9.77244373002739\\
18.91	-9.77244468416717\\
18.92	-9.77244562043451\\
18.93	-9.77244653077829\\
18.94	-9.772447407186\\
18.95	-9.77244824178424\\
18.96	-9.7724490269386\\
18.97	-9.77244975535135\\
18.98	-9.77245042015567\\
18.99	-9.77245101500483\\
19	-9.77245153415519\\
19.01	-9.77245197254156\\
19.02	-9.77245232584401\\
19.03	-9.77245259054501\\
19.04	-9.7724527639761\\
19.05	-9.77245284435332\\
19.06	-9.77245283080094\\
19.07	-9.77245272336318\\
19.08	-9.77245252300355\\
19.09	-9.77245223159208\\
19.1	-9.77245185188038\\
19.11	-9.77245138746515\\
19.12	-9.77245084274044\\
19.13	-9.77245022283955\\
19.14	-9.77244953356741\\
19.15	-9.77244878132434\\
19.16	-9.77244797302248\\
19.17	-9.77244711599602\\
19.18	-9.77244621790662\\
19.19	-9.77244528664541\\
19.2	-9.77244433023298\\
19.21	-9.77244335671896\\
19.22	-9.77244237408241\\
19.23	-9.77244139013487\\
19.24	-9.77244041242703\\
19.25	-9.77243944816071\\
19.26	-9.77243850410732\\
19.27	-9.77243758653378\\
19.28	-9.77243670113728\\
19.29	-9.7724358529894\\
19.3	-9.77243504649062\\
19.31	-9.77243428533567\\
19.32	-9.77243357249011\\
19.33	-9.77243291017848\\
19.34	-9.77243229988394\\
19.35	-9.77243174235942\\
19.36	-9.77243123764976\\
19.37	-9.77243078512464\\
19.38	-9.77243038352138\\
19.39	-9.77243003099698\\
19.4	-9.77242972518841\\
19.41	-9.77242946328005\\
19.42	-9.77242924207718\\
19.43	-9.77242905808416\\
19.44	-9.77242890758601\\
19.45	-9.77242878673206\\
19.46	-9.77242869162014\\
19.47	-9.77242861837997\\
19.48	-9.77242856325438\\
19.49	-9.77242852267691\\
19.5	-9.77242849334461\\
19.51	-9.77242847228475\\
19.52	-9.77242845691437\\
19.53	-9.77242844509166\\
19.54	-9.77242843515839\\
19.55	-9.77242842597258\\
19.56	-9.77242841693094\\
19.57	-9.77242840798075\\
19.58	-9.77242839962086\\
19.59	-9.77242839289193\\
19.6	-9.77242838935594\\
19.61	-9.77242839106548\\
19.62	-9.77242840052308\\
19.63	-9.77242842063155\\
19.64	-9.77242845463584\\
19.65	-9.77242850605767\\
19.66	-9.77242857862379\\
19.67	-9.77242867618918\\
19.68	-9.77242880265643\\
19.69	-9.77242896189268\\
19.7	-9.77242915764541\\
19.71	-9.77242939345864\\
19.72	-9.77242967259081\\
19.73	-9.77242999793584\\
19.74	-9.77243037194869\\
19.75	-9.7724307965766\\
19.76	-9.77243127319741\\
19.77	-9.7724318025658\\
19.78	-9.77243238476872\\
19.79	-9.77243301919049\\
19.8	-9.77243370448855\\
19.81	-9.77243443858018\\
19.82	-9.77243521864068\\
19.83	-9.77243604111298\\
19.84	-9.77243690172889\\
19.85	-9.7724377955416\\
19.86	-9.77243871696919\\
19.87	-9.77243965984852\\
19.88	-9.77244061749883\\
19.89	-9.77244158279411\\
19.9	-9.77244254824336\\
19.91	-9.7724435060775\\
19.92	-9.77244444834171\\
19.93	-9.77244536699188\\
19.94	-9.77244625399374\\
19.95	-9.77244710142335\\
19.96	-9.77244790156722\\
19.97	-9.77244864702083\\
19.98	-9.77244933078402\\
19.99	-9.77244994635172\\
20	-9.77245048779887\\
20.01	-9.77245094985814\\
20.02	-9.77245132798923\\
20.03	-9.77245161843896\\
20.04	-9.77245181829094\\
20.05	-9.77245192550438\\
20.06	-9.77245193894116\\
20.07	-9.77245185838102\\
20.08	-9.77245168452451\\
20.09	-9.77245141898364\\
20.1	-9.77245106426042\\
20.11	-9.77245062371361\\
20.12	-9.77245010151418\\
20.13	-9.77244950259006\\
20.14	-9.77244883256118\\
20.15	-9.77244809766556\\
20.16	-9.77244730467774\\
20.17	-9.77244646082066\\
20.18	-9.77244557367227\\
20.19	-9.77244465106841\\
20.2	-9.77244370100319\\
20.21	-9.77244273152854\\
20.22	-9.77244175065424\\
20.23	-9.77244076624999\\
20.24	-9.77243978595096\\
20.25	-9.7724388170681\\
20.26	-9.77243786650454\\
20.27	-9.7724369406794\\
20.28	-9.77243604545974\\
20.29	-9.77243518610204\\
20.3	-9.7724343672036\\
20.31	-9.77243359266474\\
20.32	-9.77243286566213\\
20.33	-9.77243218863359\\
20.34	-9.77243156327446\\
20.35	-9.77243099054545\\
20.36	-9.7724304706917\\
20.37	-9.77243000327264\\
20.38	-9.77242958720205\\
20.39	-9.77242922079754\\
20.4	-9.77242890183858\\
20.41	-9.77242862763202\\
20.42	-9.77242839508396\\
20.43	-9.77242820077669\\
20.44	-9.77242804104953\\
20.45	-9.77242791208194\\
20.46	-9.77242780997774\\
20.47	-9.77242773084888\\
20.48	-9.77242767089742\\
20.49	-9.77242762649434\\
20.5	-9.77242759425379\\
20.51	-9.77242757110171\\
20.52	-9.77242755433753\\
20.53	-9.77242754168802\\
20.54	-9.77242753135231\\
20.55	-9.77242752203746\\
20.56	-9.7724275129839\\
20.57	-9.77242750398034\\
20.58	-9.77242749536789\\
20.59	-9.77242748803343\\
20.6	-9.77242748339211\\
20.61	-9.77242748335957\\
20.62	-9.77242749031406\\
20.63	-9.77242750704925\\
20.64	-9.77242753671846\\
20.65	-9.77242758277124\\
20.66	-9.77242764888335\\
20.67	-9.77242773888125\\
20.68	-9.7724278566625\\
20.69	-9.77242800611317\\
20.7	-9.77242819102387\\
20.71	-9.77242841500569\\
20.72	-9.77242868140741\\
20.73	-9.77242899323554\\
20.74	-9.77242935307842\\
20.75	-9.77242976303572\\
20.76	-9.77243022465459\\
20.77	-9.77243073887358\\
20.78	-9.77243130597524\\
20.79	-9.77243192554845\\
20.8	-9.77243259646101\\
20.81	-9.77243331684319\\
20.82	-9.77243408408259\\
20.83	-9.77243489483046\\
20.84	-9.77243574501962\\
20.85	-9.77243662989378\\
20.86	-9.77243754404795\\
20.87	-9.77243848147936\\
20.88	-9.77243943564843\\
20.89	-9.77244039954869\\
20.9	-9.77244136578488\\
20.91	-9.77244232665814\\
20.92	-9.77244327425682\\
20.93	-9.77244420055199\\
20.94	-9.77244509749595\\
20.95	-9.77244595712242\\
20.96	-9.77244677164695\\
20.97	-9.77244753356611\\
20.98	-9.77244823575383\\
20.99	-9.7724488715537\\
21	-9.77244943486565\\
21.01	-9.77244992022587\\
21.02	-9.77245032287865\\
21.03	-9.77245063883921\\
21.04	-9.77245086494643\\
21.05	-9.77245099890487\\
21.06	-9.77245103931525\\
21.07	-9.77245098569307\\
21.08	-9.77245083847511\\
21.09	-9.77245059901357\\
21.1	-9.77245026955807\\
21.11	-9.77244985322577\\
21.12	-9.77244935395997\\
21.13	-9.77244877647786\\
21.14	-9.77244812620824\\
21.15	-9.77244740922012\\
21.16	-9.77244663214316\\
21.17	-9.7724458020813\\
21.18	-9.77244492652081\\
21.19	-9.772444013234\\
21.2	-9.77244307018023\\
21.21	-9.77244210540549\\
21.22	-9.77244112694218\\
21.23	-9.77244014271043\\
21.24	-9.77243916042247\\
21.25	-9.7724381874915\\
21.26	-9.77243723094614\\
21.27	-9.77243629735205\\
21.28	-9.77243539274141\\
21.29	-9.77243452255159\\
21.3	-9.77243369157362\\
21.31	-9.77243290391122\\
21.32	-9.77243216295094\\
21.33	-9.77243147134362\\
21.34	-9.77243083099743\\
21.35	-9.77243024308247\\
21.36	-9.77242970804664\\
21.37	-9.77242922564245\\
21.38	-9.77242879496423\\
21.39	-9.77242841449502\\
21.4	-9.7724280821622\\
21.41	-9.77242779540102\\
21.42	-9.77242755122474\\
21.43	-9.7724273463003\\
21.44	-9.77242717702813\\
21.45	-9.77242703962477\\
21.46	-9.77242693020695\\
21.47	-9.77242684487561\\
21.48	-9.77242677979855\\
21.49	-9.77242673129025\\
21.5	-9.77242669588767\\
21.51	-9.77242667042046\\
21.52	-9.7724266520749\\
21.53	-9.77242663845\\
21.54	-9.77242662760518\\
21.55	-9.77242661809865\\
21.56	-9.77242660901575\\
21.57	-9.77242659998694\\
21.58	-9.77242659119504\\
21.59	-9.77242658337168\\
21.6	-9.77242657778289\\
21.61	-9.77242657620424\\
21.62	-9.77242658088575\\
21.63	-9.77242659450738\\
21.64	-9.77242662012557\\
21.65	-9.77242666111192\\
21.66	-9.77242672108497\\
21.67	-9.7724268038361\\
21.68	-9.77242691325101\\
21.69	-9.7724270532278\\
21.7	-9.77242722759329\\
21.71	-9.7724274400188\\
21.72	-9.77242769393687\\
21.73	-9.77242799246036\\
21.74	-9.77242833830519\\
21.75	-9.7724287337182\\
21.76	-9.77242918041128\\
21.77	-9.7724296795029\\
21.78	-9.77243023146817\\
21.79	-9.77243083609826\\
21.8	-9.77243149247001\\
21.81	-9.77243219892627\\
21.82	-9.77243295306748\\
21.83	-9.77243375175471\\
21.84	-9.77243459112423\\
21.85	-9.77243546661363\\
21.86	-9.77243637299901\\
21.87	-9.77243730444307\\
21.88	-9.77243825455316\\
21.89	-9.77243921644879\\
21.9	-9.77244018283751\\
21.91	-9.77244114609812\\
21.92	-9.77244209837006\\
21.93	-9.77244303164771\\
21.94	-9.77244393787815\\
21.95	-9.77244480906105\\
21.96	-9.7724456373492\\
21.97	-9.77244641514814\\
21.98	-9.77244713521354\\
21.99	-9.77244779074475\\
22	-9.77244837547327\\
22.01	-9.77244888374472\\
22.02	-9.77244931059311\\
22.03	-9.77244965180641\\
22.04	-9.7724499039822\\
22.05	-9.7724500645729\\
22.06	-9.77245013191954\\
22.07	-9.77245010527392\\
22.08	-9.77244998480853\\
22.09	-9.77244977161425\\
22.1	-9.77244946768584\\
22.11	-9.77244907589541\\
22.12	-9.77244859995426\\
22.13	-9.77244804436368\\
22.14	-9.77244741435546\\
22.15	-9.772446715823\\
22.16	-9.77244595524394\\
22.17	-9.77244513959567\\
22.18	-9.77244427626479\\
22.19	-9.77244337295194\\
22.2	-9.77244243757343\\
22.21	-9.77244147816107\\
22.22	-9.77244050276173\\
22.23	-9.77243951933809\\
22.24	-9.77243853567196\\
22.25	-9.77243755927171\\
22.26	-9.77243659728503\\
22.27	-9.77243565641833\\
22.28	-9.77243474286384\\
22.29	-9.7724338622356\\
22.3	-9.77243301951502\\
22.31	-9.77243221900678\\
22.32	-9.77243146430581\\
22.33	-9.77243075827538\\
22.34	-9.77243010303693\\
22.35	-9.77242949997127\\
22.36	-9.77242894973128\\
22.37	-9.77242845226565\\
22.38	-9.77242800685314\\
22.39	-9.77242761214681\\
22.4	-9.77242726622722\\
22.41	-9.7724269666638\\
22.42	-9.77242671058325\\
22.43	-9.7724264947437\\
22.44	-9.77242631561353\\
22.45	-9.77242616945331\\
22.46	-9.7724260523996\\
22.47	-9.77242596054913\\
22.48	-9.77242589004202\\
22.49	-9.77242583714254\\
22.5	-9.77242579831616\\
22.51	-9.77242577030157\\
22.52	-9.77242575017647\\
22.53	-9.77242573541606\\
22.54	-9.7724257239432\\
22.55	-9.77242571416954\\
22.56	-9.77242570502681\\
22.57	-9.77242569598782\\
22.58	-9.77242568707679\\
22.59	-9.7724256788689\\
22.6	-9.77242567247897\\
22.61	-9.77242566953959\\
22.62	-9.77242567216892\\
22.63	-9.77242568292874\\
22.64	-9.77242570477356\\
22.65	-9.77242574099141\\
22.66	-9.77242579513748\\
22.67	-9.77242587096161\\
22.68	-9.77242597233087\\
22.69	-9.77242610314849\\
22.7	-9.77242626727056\\
22.71	-9.7724264684218\\
22.72	-9.77242671011182\\
22.73	-9.77242699555347\\
22.74	-9.77242732758431\\
22.75	-9.77242770859296\\
22.76	-9.77242814045122\\
22.77	-9.77242862445339\\
22.78	-9.77242916126379\\
22.79	-9.77242975087343\\
22.8	-9.77243039256657\\
22.81	-9.77243108489797\\
22.82	-9.77243182568118\\
22.83	-9.7724326119883\\
22.84	-9.77243344016123\\
22.85	-9.77243430583451\\
22.86	-9.77243520396935\\
22.87	-9.77243612889868\\
22.88	-9.77243707438238\\
22.89	-9.77243803367222\\
22.9	-9.77243899958538\\
22.91	-9.77243996458574\\
22.92	-9.7724409208716\\
22.93	-9.77244186046868\\
22.94	-9.77244277532711\\
22.95	-9.7724436574208\\
22.96	-9.77244449884796\\
22.97	-9.77244529193113\\
22.98	-9.77244602931541\\
22.99	-9.7724467040632\\
23	-9.7724473097443\\
23.01	-9.77244784051987\\
23.02	-9.77244829121905\\
23.03	-9.77244865740708\\
23.04	-9.77244893544398\\
23.05	-9.77244912253276\\
23.06	-9.7724492167566\\
23.07	-9.77244921710436\\
23.08	-9.77244912348403\\
23.09	-9.77244893672397\\
23.1	-9.77244865856184\\
23.11	-9.77244829162155\\
23.12	-9.77244783937831\\
23.13	-9.77244730611259\\
23.14	-9.77244669685348\\
23.15	-9.77244601731235\\
23.16	-9.77244527380788\\
23.17	-9.7724444731834\\
23.18	-9.772443622718\\
23.19	-9.77244273003253\\
23.2	-9.77244180299199\\
23.21	-9.77244084960573\\
23.22	-9.77243987792692\\
23.23	-9.7724388959528\\
23.24	-9.77243791152715\\
23.25	-9.77243693224635\\
23.26	-9.77243596537049\\
23.27	-9.77243501774074\\
23.28	-9.77243409570415\\
23.29	-9.77243320504694\\
23.3	-9.77243235093727\\
23.31	-9.7724315378781\\
23.32	-9.77243076967092\\
23.33	-9.77243004939068\\
23.34	-9.77242937937215\\
23.35	-9.77242876120792\\
23.36	-9.77242819575787\\
23.37	-9.77242768316968\\
23.38	-9.77242722291022\\
23.39	-9.77242681380697\\
23.4	-9.77242645409873\\
23.41	-9.77242614149479\\
23.42	-9.77242587324137\\
23.43	-9.77242564619435\\
23.44	-9.7724254568968\\
23.45	-9.77242530166023\\
23.46	-9.77242517664797\\
23.47	-9.77242507795942\\
23.48	-9.77242500171366\\
23.49	-9.77242494413107\\
23.5	-9.77242490161162\\
23.51	-9.77242487080841\\
23.52	-9.7724248486954\\
23.53	-9.77242483262807\\
23.54	-9.77242482039612\\
23.55	-9.77242481026721\\
23.56	-9.77242480102119\\
23.57	-9.77242479197402\\
23.58	-9.77242478299127\\
23.59	-9.77242477449074\\
23.6	-9.77242476743422\\
23.61	-9.77242476330867\\
23.62	-9.77242476409686\\
23.63	-9.77242477223825\\
23.64	-9.77242479058049\\
23.65	-9.77242482232257\\
23.66	-9.77242487095035\\
23.67	-9.77242494016573\\
23.68	-9.77242503381047\\
23.69	-9.77242515578609\\
23.7	-9.77242530997098\\
23.71	-9.77242550013634\\
23.72	-9.77242572986218\\
23.73	-9.77242600245483\\
23.74	-9.77242632086746\\
23.75	-9.77242668762486\\
23.76	-9.77242710475376\\
23.77	-9.77242757371999\\
23.78	-9.77242809537355\\
23.79	-9.77242866990249\\
23.8	-9.77242929679668\\
23.81	-9.77242997482182\\
23.82	-9.77243070200458\\
23.83	-9.77243147562905\\
23.84	-9.77243229224465\\
23.85	-9.77243314768568\\
23.86	-9.77243403710221\\
23.87	-9.77243495500196\\
23.88	-9.77243589530271\\
23.89	-9.77243685139455\\
23.9	-9.77243781621104\\
23.91	-9.77243878230834\\
23.92	-9.77243974195128\\
23.93	-9.77244068720496\\
23.94	-9.77244161003072\\
23.95	-9.77244250238498\\
23.96	-9.77244335631963\\
23.97	-9.77244416408231\\
23.98	-9.77244491821533\\
23.99	-9.77244561165156\\
24	-9.77244623780602\\
24.01	-9.77244679066168\\
24.02	-9.77244726484841\\
24.03	-9.77244765571357\\
24.04	-9.77244795938353\\
24.05	-9.77244817281496\\
24.06	-9.77244829383525\\
24.07	-9.77244832117142\\
24.08	-9.77244825446702\\
24.09	-9.77244809428695\\
24.1	-9.77244784210987\\
24.11	-9.77244750030854\\
24.12	-9.77244707211831\\
24.13	-9.77244656159412\\
24.14	-9.77244597355686\\
24.15	-9.77244531352971\\
24.16	-9.77244458766551\\
24.17	-9.77244380266622\\
24.18	-9.77244296569566\\
24.19	-9.77244208428684\\
24.2	-9.7724411662452\\
24.21	-9.77244021954934\\
24.22	-9.77243925225052\\
24.23	-9.77243827237252\\
24.24	-9.77243728781333\\
24.25	-9.77243630625001\\
24.26	-9.77243533504827\\
24.27	-9.77243438117787\\
24.28	-9.77243345113514\\
24.29	-9.77243255087383\\
24.3	-9.77243168574498\\
24.31	-9.77243086044683\\
24.32	-9.77243007898544\\
24.33	-9.77242934464626\\
24.34	-9.77242865997729\\
24.35	-9.77242802678373\\
24.36	-9.77242744613411\\
24.37	-9.77242691837774\\
24.38	-9.77242644317303\\
24.39	-9.77242601952608\\
24.4	-9.77242564583883\\
24.41	-9.77242531996589\\
24.42	-9.77242503927907\\
24.43	-9.77242480073834\\
24.44	-9.7724246009682\\
24.45	-9.77242443633796\\
24.46	-9.77242430304469\\
24.47	-9.77242419719732\\
24.48	-9.77242411490064\\
24.49	-9.77242405233761\\
24.5	-9.77242400584873\\
24.51	-9.77242397200708\\
24.52	-9.77242394768788\\
24.53	-9.77242393013124\\
24.54	-9.77242391699727\\
24.55	-9.77242390641246\\
24.56	-9.77242389700672\\
24.57	-9.77242388794028\\
24.58	-9.77242387892027\\
24.59	-9.77242387020638\\
24.6	-9.77242386260583\\
24.61	-9.77242385745749\\
24.62	-9.77242385660554\\
24.63	-9.77242386236308\\
24.64	-9.77242387746622\\
24.65	-9.77242390501956\\
24.66	-9.77242394843382\\
24.67	-9.77242401135664\\
24.68	-9.7724240975979\\
24.69	-9.77242421105053\\
24.7	-9.77242435560833\\
24.71	-9.77242453508206\\
24.72	-9.77242475311529\\
24.73	-9.77242501310134\\
24.74	-9.77242531810278\\
24.75	-9.77242567077483\\
24.76	-9.77242607329399\\
24.77	-9.77242652729312\\
24.78	-9.77242703380406\\
24.79	-9.77242759320902\\
24.8	-9.77242820520129\\
24.81	-9.77242886875632\\
24.82	-9.77242958211365\\
24.83	-9.77243034276999\\
24.84	-9.77243114748394\\
24.85	-9.77243199229214\\
24.86	-9.77243287253695\\
24.87	-9.77243378290524\\
24.88	-9.77243471747783\\
24.89	-9.77243566978902\\
24.9	-9.77243663289525\\
24.91	-9.77243759945211\\
24.92	-9.77243856179851\\
24.93	-9.7724395120468\\
24.94	-9.77244044217771\\
24.95	-9.77244134413845\\
24.96	-9.77244220994284\\
24.97	-9.77244303177178\\
24.98	-9.77244380207268\\
24.99	-9.77244451365641\\
25	-9.77244515979025\\
25.01	-9.77244573428552\\
25.02	-9.77244623157858\\
25.03	-9.77244664680397\\
25.04	-9.77244697585863\\
25.05	-9.77244721545617\\
25.06	-9.77244736317057\\
25.07	-9.77244741746837\\
25.08	-9.77244737772906\\
25.09	-9.77244724425342\\
25.1	-9.77244701825946\\
25.11	-9.7724467018662\\
25.12	-9.77244629806553\\
25.13	-9.77244581068242\\
25.14	-9.77244524432429\\
25.15	-9.77244460432013\\
25.16	-9.77244389665029\\
25.17	-9.77244312786816\\
25.18	-9.77244230501462\\
25.19	-9.77244143552685\\
25.2	-9.77244052714257\\
25.21	-9.77243958780134\\
25.22	-9.77243862554423\\
25.23	-9.77243764841347\\
25.24	-9.77243666435341\\
25.25	-9.77243568111439\\
25.26	-9.77243470616074\\
25.27	-9.77243374658441\\
25.28	-9.7724328090254\\
25.29	-9.77243189959998\\
25.3	-9.77243102383799\\
25.31	-9.77243018662972\\
25.32	-9.77242939218346\\
25.33	-9.77242864399383\\
25.34	-9.77242794482162\\
25.35	-9.77242729668514\\
25.36	-9.77242670086306\\
25.37	-9.77242615790868\\
25.38	-9.77242566767514\\
25.39	-9.77242522935114\\
25.4	-9.77242484150647\\
25.41	-9.77242450214641\\
25.42	-9.77242420877418\\
25.43	-9.77242395846019\\
25.44	-9.77242374791698\\
25.45	-9.7724235735785\\
25.46	-9.7724234316825\\
25.47	-9.77242331835438\\
25.48	-9.77242322969141\\
25.49	-9.77242316184565\\
25.5	-9.77242311110434\\
25.51	-9.77242307396627\\
25.52	-9.77242304721308\\
25.53	-9.77242302797402\\
25.54	-9.77242301378345\\
25.55	-9.77242300262971\\
25.56	-9.77242299299498\\
25.57	-9.77242298388513\\
25.58	-9.77242297484926\\
25.59	-9.77242296598856\\
25.6	-9.77242295795433\\
25.61	-9.77242295193519\\
25.62	-9.77242294963369\\
25.63	-9.77242295323278\\
25.64	-9.77242296535253\\
25.65	-9.77242298899801\\
25.66	-9.77242302749905\\
25.67	-9.77242308444291\\
25.68	-9.77242316360104\\
25.69	-9.772423268851\\
25.7	-9.77242340409508\\
25.71	-9.77242357317668\\
25.72	-9.77242377979609\\
25.73	-9.77242402742698\\
25.74	-9.77242431923501\\
25.75	-9.77242465799998\\
25.76	-9.77242504604281\\
25.77	-9.77242548515864\\
25.78	-9.77242597655724\\
25.79	-9.77242652081168\\
25.8	-9.77242711781634\\
25.81	-9.77242776675496\\
25.82	-9.77242846607936\\
25.83	-9.77242921349934\\
25.84	-9.77243000598397\\
25.85	-9.77243083977455\\
25.86	-9.77243171040896\\
25.87	-9.7724326127573\\
25.88	-9.77243354106836\\
25.89	-9.77243448902631\\
25.9	-9.77243544981686\\
25.91	-9.77243641620194\\
25.92	-9.77243738060197\\
25.93	-9.77243833518444\\
25.94	-9.77243927195751\\
25.95	-9.77244018286745\\
25.96	-9.77244105989828\\
25.97	-9.77244189517233\\
25.98	-9.77244268105013\\
25.99	-9.77244341022818\\
26	-9.77244407583325\\
26.01	-9.77244467151165\\
26.02	-9.77244519151225\\
26.03	-9.77244563076203\\
26.04	-9.77244598493295\\
26.05	-9.77244625049918\\
26.06	-9.77244642478387\\
26.07	-9.77244650599475\\
26.08	-9.77244649324793\\
26.09	-9.77244638657969\\
26.1	-9.77244618694604\\
26.11	-9.7724458962099\\
26.12	-9.7724455171164\\
26.13	-9.77244505325636\\
26.14	-9.77244450901869\\
26.15	-9.77244388953237\\
26.16	-9.7724432005988\\
26.17	-9.7724424486157\\
26.18	-9.77244164049352\\
26.19	-9.7724407835657\\
26.2	-9.77243988549409\\
26.21	-9.77243895417093\\
26.22	-9.77243799761884\\
26.23	-9.77243702389033\\
26.24	-9.77243604096821\\
26.25	-9.77243505666847\\
26.26	-9.77243407854699\\
26.27	-9.77243311381139\\
26.28	-9.77243216923937\\
26.29	-9.77243125110463\\
26.3	-9.7724303651114\\
26.31	-9.77242951633858\\
26.32	-9.77242870919407\\
26.33	-9.77242794738002\\
26.34	-9.77242723386934\\
26.35	-9.77242657089369\\
26.36	-9.77242595994311\\
26.37	-9.77242540177693\\
26.38	-9.77242489644602\\
26.39	-9.77242444332546\\
26.4	-9.77242404115736\\
26.41	-9.77242368810286\\
26.42	-9.7724233818023\\
26.43	-9.77242311944273\\
26.44	-9.77242289783125\\
26.45	-9.77242271347329\\
26.46	-9.77242256265415\\
26.47	-9.77242244152268\\
26.48	-9.77242234617546\\
26.49	-9.77242227274029\\
26.5	-9.7724222174574\\
26.51	-9.77242217675719\\
26.52	-9.77242214733302\\
26.53	-9.77242212620806\\
26.54	-9.77242211079489\\
26.55	-9.77242209894701\\
26.56	-9.77242208900129\\
26.57	-9.77242207981083\\
26.58	-9.77242207076745\\
26.59	-9.77242206181363\\
26.6	-9.77242205344371\\
26.61	-9.77242204669407\\
26.62	-9.7724220431229\\
26.63	-9.7724220447794\\
26.64	-9.77242205416328\\
26.65	-9.77242207417513\\
26.66	-9.77242210805831\\
26.67	-9.77242215933367\\
26.68	-9.77242223172777\\
26.69	-9.77242232909612\\
26.7	-9.77242245534258\\
26.71	-9.77242261433625\\
26.72	-9.77242280982728\\
26.73	-9.77242304536296\\
26.74	-9.77242332420565\\
26.75	-9.77242364925372\\
26.76	-9.772424022967\\
26.77	-9.77242444729805\\
26.78	-9.77242492363029\\
26.79	-9.77242545272425\\
26.8	-9.77242603467278\\
26.81	-9.77242666886616\\
26.82	-9.7724273539677\\
26.83	-9.77242808790038\\
26.84	-9.77242886784487\\
26.85	-9.77242969024913\\
26.86	-9.77243055084952\\
26.87	-9.77243144470323\\
26.88	-9.77243236623169\\
26.89	-9.77243330927443\\
26.9	-9.77243426715257\\
26.91	-9.77243523274119\\
26.92	-9.77243619854955\\
26.93	-9.77243715680794\\
26.94	-9.77243809956003\\
26.95	-9.77243901875936\\
26.96	-9.77243990636844\\
26.97	-9.77244075445926\\
26.98	-9.7724415553135\\
26.99	-9.77244230152107\\
27	-9.77244298607556\\
27.01	-9.7724436024651\\
27.02	-9.7724441447573\\
27.03	-9.77244460767706\\
27.04	-9.77244498667606\\
27.05	-9.77244527799284\\
27.06	-9.77244547870266\\
27.07	-9.77244558675637\\
27.08	-9.77244560100762\\
27.09	-9.77244552122818\\
27.1	-9.77244534811092\\
27.11	-9.77244508326064\\
27.12	-9.77244472917267\\
27.13	-9.77244428919971\\
27.14	-9.77244376750739\\
27.15	-9.77244316901904\\
27.16	-9.77244249935086\\
27.17	-9.772441764738\\
27.18	-9.77244097195304\\
27.19	-9.77244012821788\\
27.2	-9.77243924111041\\
27.21	-9.7724383184673\\
27.22	-9.77243736828446\\
27.23	-9.77243639861644\\
27.24	-9.77243541747652\\
27.25	-9.77243443273867\\
27.26	-9.77243345204305\\
27.27	-9.77243248270622\\
27.28	-9.7724315316375\\
27.29	-9.77243060526263\\
27.3	-9.77242970945567\\
27.31	-9.77242884948034\\
27.32	-9.77242802994134\\
27.33	-9.77242725474643\\
27.34	-9.77242652707965\\
27.35	-9.77242584938603\\
27.36	-9.77242522336785\\
27.37	-9.7724246499924\\
27.38	-9.77242412951094\\
27.39	-9.77242366148847\\
27.4	-9.77242324484378\\
27.41	-9.7724228778988\\
27.42	-9.77242255843659\\
27.43	-9.77242228376688\\
27.44	-9.77242205079781\\
27.45	-9.772421856113\\
27.46	-9.77242169605225\\
27.47	-9.77242156679474\\
27.48	-9.77242146444327\\
27.49	-9.77242138510807\\
27.5	-9.77242132498879\\
27.51	-9.7724212804534\\
27.52	-9.77242124811252\\
27.53	-9.7724212248881\\
27.54	-9.7724212080752\\
27.55	-9.77242119539595\\
27.56	-9.77242118504464\\
27.57	-9.77242117572342\\
27.58	-9.77242116666778\\
27.59	-9.7724211576616\\
27.6	-9.77242114904141\\
27.61	-9.77242114168971\\
27.62	-9.77242113701773\\
27.63	-9.77242113693758\\
27.64	-9.77242114382454\\
27.65	-9.77242116046986\\
27.66	-9.77242119002511\\
27.67	-9.77242123593875\\
27.68	-9.77242130188613\\
27.69	-9.77242139169407\\
27.7	-9.77242150926116\\
27.71	-9.77242165847524\\
27.72	-9.77242184312941\\
27.73	-9.77242206683784\\
27.74	-9.77242233295306\\
27.75	-9.77242264448588\\
27.76	-9.77242300402939\\
27.77	-9.77242341368848\\
27.78	-9.77242387501581\\
27.79	-9.77242438895569\\
27.8	-9.77242495579659\\
27.81	-9.77242557513333\\
27.82	-9.77242624583962\\
27.83	-9.7724269660515\\
27.84	-9.77242773316202\\
27.85	-9.7724285438276\\
27.86	-9.77242939398574\\
27.87	-9.77243027888431\\
27.88	-9.77243119312187\\
27.89	-9.77243213069853\\
27.9	-9.7724330850768\\
27.91	-9.77243404925155\\
27.92	-9.77243501582803\\
27.93	-9.77243597710697\\
27.94	-9.7724369251755\\
27.95	-9.77243785200258\\
27.96	-9.7724387495375\\
27.97	-9.77243960981018\\
27.98	-9.77244042503155\\
27.99	-9.77244118769279\\
28	-9.77244189066182\\
28.01	-9.77244252727554\\
28.02	-9.77244309142669\\
28.03	-9.77244357764383\\
28.04	-9.77244398116327\\
28.05	-9.77244429799201\\
28.06	-9.77244452496062\\
28.07	-9.77244465976526\\
28.08	-9.77244470099838\\
28.09	-9.77244464816743\\
28.1	-9.77244450170141\\
28.11	-9.77244426294516\\
28.12	-9.77244393414147\\
28.13	-9.77244351840125\\
28.14	-9.77244301966224\\
28.15	-9.77244244263681\\
28.16	-9.77244179274979\\
28.17	-9.77244107606705\\
28.18	-9.77244029921602\\
28.19	-9.77243946929937\\
28.2	-9.772438593803\\
28.21	-9.7724376804998\\
28.22	-9.77243673735066\\
28.23	-9.77243577240395\\
28.24	-9.77243479369534\\
28.25	-9.772433809149\\
28.26	-9.77243282648196\\
28.27	-9.77243185311285\\
28.28	-9.77243089607634\\
28.29	-9.77242996194459\\
28.3	-9.77242905675667\\
28.31	-9.77242818595714\\
28.32	-9.77242735434439\\
28.33	-9.7724265660296\\
28.34	-9.7724258244067\\
28.35	-9.7724251321338\\
28.36	-9.77242449112611\\
28.37	-9.77242390256041\\
28.38	-9.77242336689087\\
28.39	-9.77242288387573\\
28.4	-9.77242245261451\\
28.41	-9.77242207159474\\
28.42	-9.77242173874765\\
28.43	-9.77242145151153\\
28.44	-9.77242120690197\\
28.45	-9.77242100158741\\
28.46	-9.77242083196901\\
28.47	-9.77242069426326\\
28.48	-9.77242058458608\\
28.49	-9.77242049903686\\
28.5	-9.77242043378119\\
28.51	-9.77242038513073\\
28.52	-9.77242034961902\\
28.53	-9.77242032407187\\
28.54	-9.77242030567128\\
28.55	-9.77242029201168\\
28.56	-9.77242028114771\\
28.57	-9.77242027163266\\
28.58	-9.77242026254694\\
28.59	-9.77242025351615\\
28.6	-9.77242024471843\\
28.61	-9.77242023688099\\
28.62	-9.77242023126577\\
28.63	-9.7724202296447\\
28.64	-9.77242023426467\\
28.65	-9.77242024780304\\
28.66	-9.77242027331432\\
28.67	-9.77242031416882\\
28.68	-9.77242037398447\\
28.69	-9.7724204565528\\
28.7	-9.77242056576033\\
28.71	-9.77242070550673\\
28.72	-9.77242087962111\\
28.73	-9.77242109177769\\
28.74	-9.77242134541261\\
28.75	-9.77242164364283\\
28.76	-9.77242198918893\\
28.77	-9.77242238430283\\
28.78	-9.77242283070186\\
28.79	-9.77242332951019\\
28.8	-9.77242388120881\\
28.81	-9.77242448559484\\
28.82	-9.77242514175105\\
28.83	-9.7724258480261\\
28.84	-9.772426602026\\
28.85	-9.77242740061707\\
28.86	-9.7724282399404\\
28.87	-9.77242911543791\\
28.88	-9.77243002188944\\
28.89	-9.77243095346075\\
28.9	-9.77243190376153\\
28.91	-9.77243286591285\\
28.92	-9.77243383262298\\
28.93	-9.77243479627062\\
28.94	-9.7724357489942\\
28.95	-9.77243668278624\\
28.96	-9.77243758959107\\
28.97	-9.7724384614048\\
28.98	-9.77243929037581\\
28.99	-9.77244006890449\\
29	-9.77244078974063\\
29.01	-9.77244144607711\\
29.02	-9.77244203163836\\
29.03	-9.77244254076248\\
29.04	-9.77244296847561\\
29.05	-9.77244331055755\\
29.06	-9.77244356359757\\
29.07	-9.77244372503972\\
29.08	-9.77244379321671\\
29.09	-9.77244376737222\\
29.1	-9.77244364767085\\
29.11	-9.772443435196\\
29.12	-9.77244313193545\\
29.13	-9.77244274075487\\
29.14	-9.7724422653598\\
29.15	-9.77244171024648\\
29.16	-9.7724410806425\\
29.17	-9.77244038243785\\
29.18	-9.77243962210771\\
29.19	-9.77243880662789\\
29.2	-9.77243794338439\\
29.21	-9.77243704007814\\
29.22	-9.77243610462671\\
29.23	-9.77243514506405\\
29.24	-9.77243416944007\\
29.25	-9.77243318572124\\
29.26	-9.77243220169399\\
29.27	-9.77243122487194\\
29.28	-9.77243026240866\\
29.29	-9.77242932101691\\
29.3	-9.77242840689576\\
29.31	-9.77242752566634\\
29.32	-9.77242668231739\\
29.33	-9.77242588116102\\
29.34	-9.77242512579958\\
29.35	-9.77242441910366\\
29.36	-9.77242376320181\\
29.37	-9.77242315948163\\
29.38	-9.7724226086024\\
29.39	-9.7724221105187\\
29.4	-9.77242166451465\\
29.41	-9.77242126924801\\
29.42	-9.77242092280336\\
29.43	-9.77242062275342\\
29.44	-9.77242036622741\\
29.45	-9.77242014998523\\
29.46	-9.7724199704962\\
29.47	-9.77241982402104\\
29.48	-9.77241970669574\\
29.49	-9.77241961461567\\
29.5	-9.77241954391895\\
29.51	-9.77241949086716\\
29.52	-9.77241945192254\\
29.53	-9.77241942382006\\
29.54	-9.77241940363323\\
29.55	-9.77241938883276\\
29.56	-9.77241937733679\\
29.57	-9.77241936755206\\
29.58	-9.77241935840538\\
29.59	-9.77241934936466\\
29.6	-9.77241934044938\\
29.61	-9.77241933223021\\
29.62	-9.77241932581779\\
29.63	-9.77241932284092\\
29.64	-9.77241932541448\\
29.65	-9.7724193360975\\
29.66	-9.77241935784235\\
29.67	-9.77241939393559\\
29.68	-9.77241944793161\\
29.69	-9.77241952358016\\
29.7	-9.77241962474895\\
29.71	-9.77241975534258\\
29.72	-9.7724199192192\\
29.73	-9.77242012010627\\
29.74	-9.77242036151681\\
29.75	-9.77242064666762\\
29.76	-9.77242097840079\\
29.77	-9.77242135910988\\
29.78	-9.77242179067204\\
29.79	-9.77242227438722\\
29.8	-9.77242281092558\\
29.81	-9.77242340028406\\
29.82	-9.77242404175287\\
29.83	-9.77242473389261\\
29.84	-9.7724254745225\\
29.85	-9.77242626071998\\
29.86	-9.77242708883192\\
29.87	-9.77242795449732\\
29.88	-9.77242885268132\\
29.89	-9.77242977772009\\
29.9	-9.77243072337609\\
29.91	-9.77243168290287\\
29.92	-9.77243264911858\\
29.93	-9.77243361448722\\
29.94	-9.77243457120635\\
29.95	-9.77243551130009\\
29.96	-9.77243642671605\\
29.97	-9.77243730942482\\
29.98	-9.77243815152043\\
29.99	-9.77243894532049\\
};
\addlegendentry{MP}

\addplot [color=mycolor1]
  table[row sep=crcr]{%
0	-9.77247004781056\\
0.01	-9.7724698604565\\
0.02	-9.77246962798229\\
0.03	-9.77246930570321\\
0.04	-9.77246889675322\\
0.05	-9.77246840508984\\
0.06	-9.77246783544175\\
0.07	-9.7724671932464\\
0.08	-9.77246648457847\\
0.09	-9.77246571607034\\
0.1	-9.7724648948257\\
0.11	-9.7724640283276\\
0.12	-9.77246312434233\\
0.13	-9.77246219082052\\
0.14	-9.7724612357969\\
0.15	-9.77246026729024\\
0.16	-9.77245929320499\\
0.17	-9.77245832123586\\
0.18	-9.77245735877696\\
0.19	-9.7724564128366\\
0.2	-9.7724554899592\\
0.21	-9.7724545961551\\
0.22	-9.77245373683954\\
0.23	-9.77245291678148\\
0.24	-9.77245214006296\\
0.25	-9.77245141004948\\
0.26	-9.77245072937179\\
0.27	-9.77245009991921\\
0.28	-9.77244952284444\\
0.29	-9.77244899857963\\
0.3	-9.77244852686341\\
0.31	-9.7724481067782\\
0.32	-9.77244773679721\\
0.33	-9.7724474148402\\
0.34	-9.77244713833696\\
0.35	-9.7724469042975\\
0.36	-9.77244670938763\\
0.37	-9.77244655000877\\
0.38	-9.77244642238039\\
0.39	-9.77244632262396\\
0.4	-9.77244624684684\\
0.41	-9.77244619122469\\
0.42	-9.77244615208111\\
0.43	-9.77244612596316\\
0.44	-9.77244610971143\\
0.45	-9.77244610052357\\
0.46	-9.7724460960103\\
0.47	-9.77244609424273\\
0.48	-9.77244609379053\\
0.49	-9.77244609375001\\
0.5	-9.77244609376194\\
0.51	-9.77244609401852\\
0.52	-9.77244609525968\\
0.53	-9.77244609875851\\
0.54	-9.77244610629624\\
0.55	-9.77244612012711\\
0.56	-9.77244614293369\\
0.57	-9.77244617777347\\
0.58	-9.77244622801749\\
0.59	-9.77244629728211\\
0.6	-9.77244638935504\\
0.61	-9.77244650811683\\
0.62	-9.7724466574591\\
0.63	-9.77244684120099\\
0.64	-9.7724470630051\\
0.65	-9.7724473262944\\
0.66	-9.77244763417147\\
0.67	-9.77244798934155\\
0.68	-9.77244839404056\\
0.69	-9.77244884996942\\
0.7	-9.7724493582359\\
0.71	-9.77244991930482\\
0.72	-9.77245053295773\\
0.73	-9.77245119826272\\
0.74	-9.77245191355494\\
0.75	-9.77245267642836\\
0.76	-9.77245348373895\\
0.77	-9.77245433161943\\
0.78	-9.7724552155054\\
0.79	-9.77245613017273\\
0.8	-9.77245706978557\\
0.81	-9.7724580279545\\
0.82	-9.77245899780402\\
0.83	-9.77245997204834\\
0.84	-9.77246094307456\\
0.85	-9.77246190303197\\
0.86	-9.77246284392618\\
0.87	-9.77246375771677\\
0.88	-9.772464636417\\
0.89	-9.77246547219413\\
0.9	-9.77246625746887\\
0.91	-9.77246698501253\\
0.92	-9.77246764804035\\
0.93	-9.77246824029974\\
0.94	-9.77246875615202\\
0.95	-9.77246919064646\\
0.96	-9.77246953958563\\
0.97	-9.77246979958084\\
0.98	-9.77246996809717\\
0.99	-9.77247004348709\\
1	-9.77247002501241\\
1.01	-9.77246991285408\\
1.02	-9.77246970810983\\
1.03	-9.7724694127795\\
1.04	-9.77246902973847\\
1.05	-9.7724685626995\\
1.06	-9.77246801616352\\
1.07	-9.77246739536013\\
1.08	-9.77246670617879\\
1.09	-9.77246595509154\\
1.1	-9.77246514906858\\
1.11	-9.77246429548789\\
1.12	-9.7724634020402\\
1.13	-9.77246247663078\\
1.14	-9.77246152727944\\
1.15	-9.77246056202032\\
1.16	-9.77245958880277\\
1.17	-9.77245861539497\\
1.18	-9.77245764929148\\
1.19	-9.77245669762626\\
1.2	-9.77245576709223\\
1.21	-9.77245486386861\\
1.22	-9.77245399355706\\
1.23	-9.77245316112734\\
1.24	-9.77245237087352\\
1.25	-9.77245162638087\\
1.26	-9.77245093050428\\
1.27	-9.77245028535799\\
1.28	-9.77244969231694\\
1.29	-9.77244915202949\\
1.3	-9.77244866444116\\
1.31	-9.77244822882887\\
1.32	-9.77244784384514\\
1.33	-9.77244750757121\\
1.34	-9.77244721757838\\
1.35	-9.77244697099622\\
1.36	-9.7724467645867\\
1.37	-9.77244659482281\\
1.38	-9.77244645797037\\
1.39	-9.77244635017171\\
1.4	-9.77244626752965\\
1.41	-9.77244620619057\\
1.42	-9.772446162425\\
1.43	-9.77244613270453\\
1.44	-9.77244611377369\\
1.45	-9.77244610271554\\
1.46	-9.77244609701008\\
1.47	-9.7724460945843\\
1.48	-9.7724460938532\\
1.49	-9.77244609375099\\
1.5	-9.77244609375203\\
1.51	-9.77244609388118\\
1.52	-9.77244609471325\\
1.53	-9.77244609736177\\
1.54	-9.77244610345711\\
1.55	-9.7724461151143\\
1.56	-9.77244613489127\\
1.57	-9.77244616573791\\
1.58	-9.77244621093705\\
1.59	-9.77244627403824\\
1.6	-9.7724463587853\\
1.61	-9.77244646903915\\
1.62	-9.77244660869682\\
1.63	-9.77244678160832\\
1.64	-9.77244699149259\\
1.65	-9.77244724185395\\
1.66	-9.7724475359005\\
1.67	-9.77244787646587\\
1.68	-9.77244826593558\\
1.69	-9.77244870617937\\
1.7	-9.77244919849056\\
1.71	-9.77244974353374\\
1.72	-9.77245034130137\\
1.73	-9.77245099108053\\
1.74	-9.77245169143008\\
1.75	-9.77245244016902\\
1.76	-9.77245323437613\\
1.77	-9.77245407040129\\
1.78	-9.77245494388812\\
1.79	-9.772455849808\\
1.8	-9.77245678250495\\
1.81	-9.77245773575076\\
1.82	-9.77245870280971\\
1.83	-9.77245967651199\\
1.84	-9.77246064933476\\
1.85	-9.7724616134897\\
1.86	-9.7724625610158\\
1.87	-9.77246348387615\\
1.88	-9.77246437405712\\
1.89	-9.77246522366867\\
1.9	-9.77246602504425\\
1.91	-9.77246677083875\\
1.92	-9.77246745412315\\
1.93	-9.77246806847451\\
1.94	-9.77246860805969\\
1.95	-9.772469067712\\
1.96	-9.77246944299914\\
1.97	-9.77246973028188\\
1.98	-9.77246992676221\\
1.99	-9.77247003052059\\
2	-9.77247004054137\\
2.01	-9.7724699567264\\
2.02	-9.7724697798963\\
2.03	-9.77246951177954\\
2.04	-9.77246915498944\\
2.05	-9.77246871298944\\
2.06	-9.77246819004723\\
2.07	-9.77246759117825\\
2.08	-9.77246692207958\\
2.09	-9.77246618905517\\
2.1	-9.77246539893342\\
2.11	-9.77246455897846\\
2.12	-9.77246367679643\\
2.13	-9.772462760238\\
2.14	-9.77246181729888\\
2.15	-9.77246085601939\\
2.16	-9.77245988438495\\
2.17	-9.77245891022867\\
2.18	-9.77245794113756\\
2.19	-9.77245698436381\\
2.2	-9.77245604674216\\
2.21	-9.77245513461483\\
2.22	-9.77245425376486\\
2.23	-9.77245340935882\\
2.24	-9.77245260589972\\
2.25	-9.77245184719067\\
2.26	-9.77245113630971\\
2.27	-9.77245047559612\\
2.28	-9.77244986664829\\
2.29	-9.77244931033294\\
2.3	-9.77244880680565\\
2.31	-9.77244835554194\\
2.32	-9.77244795537848\\
2.33	-9.77244760456369\\
2.34	-9.77244730081664\\
2.35	-9.7724470413933\\
2.36	-9.77244682315903\\
2.37	-9.77244664266593\\
2.38	-9.77244649623391\\
2.39	-9.77244638003385\\
2.4	-9.77244629017171\\
2.41	-9.77244622277201\\
2.42	-9.77244617405928\\
2.43	-9.77244614043631\\
2.44	-9.77244611855756\\
2.45	-9.77244610539689\\
2.46	-9.77244609830821\\
2.47	-9.77244609507817\\
2.48	-9.77244609396997\\
2.49	-9.77244609375762\\
2.5	-9.77244609375008\\
2.51	-9.77244609380475\\
2.52	-9.77244609433034\\
2.53	-9.7724460962788\\
2.54	-9.77244610112664\\
2.55	-9.77244611084588\\
2.56	-9.77244612786506\\
2.57	-9.77244615502108\\
2.58	-9.77244619550246\\
2.59	-9.77244625278522\\
2.6	-9.77244633056223\\
2.61	-9.77244643266729\\
2.62	-9.77244656299519\\
2.63	-9.77244672541907\\
2.64	-9.7724469237065\\
2.65	-9.77244716143566\\
2.66	-9.77244744191293\\
2.67	-9.77244776809355\\
2.68	-9.77244814250643\\
2.69	-9.7724485671845\\
2.7	-9.77244904360186\\
2.71	-9.77244957261884\\
2.72	-9.77245015443576\\
2.73	-9.77245078855657\\
2.74	-9.77245147376277\\
2.75	-9.77245220809835\\
2.76	-9.77245298886598\\
2.77	-9.77245381263475\\
2.78	-9.77245467525942\\
2.79	-9.77245557191106\\
2.8	-9.77245649711861\\
2.81	-9.77245744482101\\
2.82	-9.77245840842912\\
2.83	-9.77245938089657\\
2.84	-9.77246035479863\\
2.85	-9.77246132241791\\
2.86	-9.77246227583572\\
2.87	-9.77246320702779\\
2.88	-9.77246410796282\\
2.89	-9.77246497070271\\
2.9	-9.77246578750263\\
2.91	-9.77246655090978\\
2.92	-9.77246725385921\\
2.93	-9.77246788976528\\
2.94	-9.7724684526075\\
2.95	-9.77246893700943\\
2.96	-9.77246933830937\\
2.97	-9.77246965262191\\
2.98	-9.77246987688946\\
2.99	-9.77247000892274\\
3	-9.77247004742995\\
3.01	-9.77246999203392\\
3.02	-9.77246984327718\\
3.03	-9.77246960261469\\
3.04	-9.7724692723945\\
3.05	-9.77246885582655\\
3.06	-9.77246835694005\\
3.07	-9.77246778053019\\
3.08	-9.77246713209481\\
3.09	-9.77246641776215\\
3.1	-9.77246564421064\\
3.11	-9.77246481858195\\
3.12	-9.77246394838862\\
3.13	-9.77246304141759\\
3.14	-9.77246210563118\\
3.15	-9.77246114906675\\
3.16	-9.77246017973681\\
3.17	-9.77245920553076\\
3.18	-9.7724582341199\\
3.19	-9.77245727286701\\
3.2	-9.77245632874176\\
3.21	-9.77245540824322\\
3.22	-9.77245451733055\\
3.23	-9.77245366136276\\
3.24	-9.7724528450485\\
3.25	-9.77245207240636\\
3.26	-9.77245134673628\\
3.27	-9.7724506706024\\
3.28	-9.77245004582739\\
3.29	-9.77244947349824\\
3.3	-9.77244895398346\\
3.31	-9.77244848696092\\
3.32	-9.77244807145616\\
3.33	-9.77244770589018\\
3.34	-9.77244738813592\\
3.35	-9.77244711558244\\
3.36	-9.7724468852056\\
3.37	-9.77244669364411\\
3.38	-9.77244653727959\\
3.39	-9.77244641231925\\
3.4	-9.77244631487988\\
3.41	-9.77244624107162\\
3.42	-9.77244618708025\\
3.43	-9.77244614924646\\
3.44	-9.77244612414094\\
3.45	-9.77244610863391\\
3.46	-9.77244609995806\\
3.47	-9.77244609576371\\
3.48	-9.77244609416543\\
3.49	-9.77244609377927\\
3.5	-9.77244609375\\
3.51	-9.77244609376787\\
3.52	-9.77244609407469\\
3.53	-9.77244609545914\\
3.54	-9.7724460992412\\
3.55	-9.77244610724628\\
3.56	-9.77244612176911\\
3.57	-9.77244614552833\\
3.58	-9.77244618161226\\
3.59	-9.77244623341692\\
3.6	-9.77244630457719\\
3.61	-9.7724463988924\\
3.62	-9.77244652024749\\
3.63	-9.77244667253099\\
3.64	-9.7724468595514\\
3.65	-9.77244708495313\\
3.66	-9.77244735213356\\
3.67	-9.77244766416262\\
3.68	-9.77244802370615\\
3.69	-9.77244843295452\\
3.7	-9.77244889355754\\
3.71	-9.772449406567\\
3.72	-9.77244997238774\\
3.73	-9.77245059073819\\
3.74	-9.7724512606211\\
3.75	-9.77245198030509\\
3.76	-9.77245274731741\\
3.77	-9.77245355844816\\
3.78	-9.77245440976608\\
3.79	-9.77245529664571\\
3.8	-9.77245621380573\\
3.81	-9.77245715535795\\
3.82	-9.77245811486633\\
3.83	-9.77245908541519\\
3.84	-9.77246005968582\\
3.85	-9.77246103004017\\
3.86	-9.77246198861068\\
3.87	-9.77246292739478\\
3.88	-9.77246383835278\\
3.89	-9.7724647135077\\
3.9	-9.77246554504555\\
3.91	-9.77246632541467\\
3.92	-9.77246704742256\\
3.93	-9.77246770432884\\
3.94	-9.77246828993292\\
3.95	-9.77246879865512\\
3.96	-9.77246922560998\\
3.97	-9.77246956667072\\
3.98	-9.77246981852383\\
3.99	-9.77246997871304\\
4	-9.77247004567193\\
4.01	-9.77247001874482\\
4.02	-9.77246989819544\\
4.03	-9.77246968520349\\
4.04	-9.77246938184887\\
4.05	-9.77246899108406\\
4.06	-9.77246851669493\\
4.07	-9.77246796325056\\
4.08	-9.77246733604289\\
4.09	-9.77246664101709\\
4.1	-9.77246588469358\\
4.11	-9.77246507408307\\
4.12	-9.77246421659561\\
4.13	-9.77246331994531\\
4.14	-9.77246239205183\\
4.15	-9.7724614409404\\
4.16	-9.77246047464153\\
4.17	-9.77245950109219\\
4.18	-9.77245852803963\\
4.19	-9.77245756294944\\
4.2	-9.77245661291906\\
4.21	-9.77245568459804\\
4.22	-9.77245478411611\\
4.23	-9.7724539170201\\
4.24	-9.77245308822064\\
4.25	-9.77245230194915\\
4.26	-9.77245156172593\\
4.27	-9.77245087033943\\
4.28	-9.77245022983713\\
4.29	-9.7724496415279\\
4.3	-9.77244910599571\\
4.31	-9.77244862312435\\
4.32	-9.77244819213261\\
4.33	-9.77244781161927\\
4.34	-9.77244747961708\\
4.35	-9.77244719365467\\
4.36	-9.77244695082542\\
4.37	-9.77244674786203\\
4.38	-9.77244658121545\\
4.39	-9.77244644713694\\
4.4	-9.77244634176185\\
4.41	-9.77244626119353\\
4.42	-9.77244620158628\\
4.43	-9.77244615922566\\
4.44	-9.77244613060496\\
4.45	-9.77244611249657\\
4.46	-9.77244610201702\\
4.47	-9.77244609668462\\
4.48	-9.77244609446873\\
4.49	-9.77244609382995\\
4.5	-9.77244609375041\\
4.51	-9.77244609375375\\
4.52	-9.77244609391449\\
4.53	-9.77244609485656\\
4.54	-9.77244609774106\\
4.55	-9.77244610424343\\
4.56	-9.77244611652042\\
4.57	-9.77244613716745\\
4.58	-9.77244616916689\\
4.59	-9.77244621582841\\
4.6	-9.77244628072207\\
4.61	-9.77244636760547\\
4.62	-9.77244648034612\\
4.63	-9.77244662284026\\
4.64	-9.77244679892955\\
4.65	-9.77244701231701\\
4.66	-9.7724472664836\\
4.67	-9.77244756460688\\
4.68	-9.77244790948306\\
4.69	-9.77244830345394\\
4.7	-9.77244874833972\\
4.71	-9.7724492453792\\
4.72	-9.77244979517806\\
4.73	-9.77245039766652\\
4.74	-9.77245105206688\\
4.75	-9.77245175687173\\
4.76	-9.77245250983328\\
4.77	-9.7724533079641\\
4.78	-9.7724541475494\\
4.79	-9.77245502417074\\
4.8	-9.77245593274104\\
4.81	-9.77245686755042\\
4.82	-9.77245782232221\\
4.83	-9.77245879027849\\
4.84	-9.77245976421426\\
4.85	-9.77246073657911\\
4.86	-9.77246169956531\\
4.87	-9.77246264520102\\
4.88	-9.77246356544732\\
4.89	-9.77246445229759\\
4.9	-9.77246529787786\\
4.91	-9.7724660945465\\
4.92	-9.77246683499205\\
4.93	-9.77246751232745\\
4.94	-9.77246812017947\\
4.95	-9.77246865277197\\
4.96	-9.77246910500167\\
4.97	-9.77246947250542\\
4.98	-9.77246975171786\\
4.99	-9.77246993991872\\
5	-9.77247003526891\\
5.01	-9.77247003683499\\
5.02	-9.77246994460164\\
5.03	-9.77246975947179\\
5.04	-9.77246948325467\\
5.05	-9.77246911864171\\
5.06	-9.77246866917077\\
5.07	-9.77246813917929\\
5.08	-9.77246753374691\\
5.09	-9.77246685862851\\
5.1	-9.77246612017871\\
5.11	-9.77246532526885\\
5.12	-9.77246448119774\\
5.13	-9.77246359559756\\
5.14	-9.77246267633613\\
5.15	-9.77246173141728\\
5.16	-9.77246076888046\\
5.17	-9.7724597967013\\
5.18	-9.77245882269456\\
5.19	-9.77245785442067\\
5.2	-9.7724568990975\\
5.21	-9.77245596351845\\
5.22	-9.772455053978\\
5.23	-9.77245417620591\\
5.24	-9.7724533353108\\
5.25	-9.77245253573402\\
5.26	-9.77245178121429\\
5.27	-9.77245107476354\\
5.28	-9.77245041865429\\
5.29	-9.77244981441853\\
5.3	-9.772449262858\\
5.31	-9.77244876406563\\
5.32	-9.7724483174576\\
5.33	-9.77244792181546\\
5.34	-9.77244757533746\\
5.35	-9.77244727569816\\
5.36	-9.77244702011533\\
5.37	-9.77244680542291\\
5.38	-9.77244662814875\\
5.39	-9.7724464845959\\
5.4	-9.77244637092597\\
5.41	-9.77244628324318\\
5.42	-9.77244621767776\\
5.43	-9.77244617046715\\
5.44	-9.77244613803383\\
5.45	-9.77244611705837\\
5.46	-9.77244610454643\\
5.47	-9.77244609788885\\
5.48	-9.7724460949135\\
5.49	-9.77244609392831\\
5.5	-9.7724460937546\\
5.51	-9.77244609375028\\
5.52	-9.77244609382243\\
5.53	-9.77244609442912\\
5.54	-9.77244609657044\\
5.55	-9.77244610176885\\
5.56	-9.77244611203917\\
5.57	-9.77244612984883\\
5.58	-9.77244615806878\\
5.59	-9.77244619991616\\
5.6	-9.77244625888949\\
5.61	-9.7724463386975\\
5.62	-9.7724464431828\\
5.63	-9.77244657624164\\
5.64	-9.77244674174111\\
5.65	-9.77244694343516\\
5.66	-9.77244718488083\\
5.67	-9.77244746935611\\
5.68	-9.77244779978089\\
5.69	-9.77244817864217\\
5.7	-9.77244860792507\\
5.71	-9.77244908905058\\
5.72	-9.77244962282136\\
5.73	-9.77245020937643\\
5.74	-9.77245084815561\\
5.75	-9.77245153787449\\
5.76	-9.77245227651032\\
5.77	-9.77245306129925\\
5.78	-9.77245388874513\\
5.79	-9.77245475463978\\
5.8	-9.77245565409461\\
5.81	-9.77245658158316\\
5.82	-9.77245753099419\\
5.83	-9.77245849569438\\
5.84	-9.77245946859996\\
5.85	-9.77246044225627\\
5.86	-9.77246140892397\\
5.87	-9.77246236067091\\
5.88	-9.7724632894681\\
5.89	-9.77246418728855\\
5.9	-9.77246504620743\\
5.91	-9.7724658585022\\
5.92	-9.7724666167511\\
5.93	-9.77246731392862\\
5.94	-9.77246794349658\\
5.95	-9.77246849948935\\
5.96	-9.77246897659203\\
5.97	-9.7724693702104\\
5.98	-9.77246967653163\\
5.99	-9.77246989257476\\
6	-9.77247001623026\\
6.01	-9.77247004628814\\
6.02	-9.77246998245396\\
6.03	-9.77246982535286\\
6.04	-9.77246957652114\\
6.05	-9.77246923838586\\
6.06	-9.77246881423261\\
6.07	-9.77246830816189\\
6.08	-9.77246772503483\\
6.09	-9.77246707040911\\
6.1	-9.77246635046587\\
6.11	-9.7724655719289\\
6.12	-9.77246474197707\\
6.13	-9.77246386815163\\
6.14	-9.7724629582594\\
6.15	-9.77246202027355\\
6.16	-9.77246106223329\\
6.17	-9.77246009214403\\
6.18	-9.77245911787937\\
6.19	-9.77245814708647\\
6.2	-9.77245718709611\\
6.21	-9.77245624483865\\
6.22	-9.77245532676729\\
6.23	-9.77245443878948\\
6.24	-9.77245358620763\\
6.25	-9.77245277366973\\
6.26	-9.7724520051307\\
6.27	-9.77245128382481\\
6.28	-9.7724506122495\\
6.29	-9.77244999216084\\
6.3	-9.77244942458029\\
6.31	-9.7724489098129\\
6.32	-9.77244844747611\\
6.33	-9.77244803653898\\
6.34	-9.77244767537073\\
6.35	-9.77244736179803\\
6.36	-9.77244709316976\\
6.37	-9.77244686642826\\
6.38	-9.77244667818579\\
6.39	-9.77244652480483\\
6.4	-9.77244640248103\\
6.41	-9.77244630732714\\
6.42	-9.7724462354568\\
6.43	-9.77244618306652\\
6.44	-9.77244614651469\\
6.45	-9.77244612239621\\
6.46	-9.77244610761145\\
6.47	-9.77244609942854\\
6.48	-9.77244609553781\\
6.49	-9.77244609409759\\
6.5	-9.77244609377054\\
6.51	-9.77244609375\\
6.52	-9.77244609377577\\
6.53	-9.77244609413924\\
6.54	-9.77244609567767\\
6.55	-9.77244609975778\\
6.56	-9.77244610824882\\
6.57	-9.77244612348567\\
6.58	-9.77244614822258\\
6.59	-9.77244618557823\\
6.6	-9.77244623897301\\
6.61	-9.77244631205975\\
6.62	-9.77244640864877\\
6.63	-9.77244653262871\\
6.64	-9.77244668788435\\
6.65	-9.77244687821286\\
6.66	-9.77244710723977\\
6.67	-9.77244737833625\\
6.68	-9.77244769453893\\
6.69	-9.77244805847371\\
6.7	-9.77244847228484\\
6.71	-9.77244893757056\\
6.72	-9.77244945532631\\
6.73	-9.77245002589662\\
6.74	-9.77245064893654\\
6.75	-9.77245132338335\\
6.76	-9.77245204743908\\
6.77	-9.77245281856429\\
6.78	-9.77245363348334\\
6.79	-9.7724544882012\\
6.8	-9.77245537803168\\
6.81	-9.77245629763666\\
6.82	-9.7724572410761\\
6.83	-9.77245820186785\\
6.84	-9.77245917305672\\
6.85	-9.77246014729173\\
6.86	-9.77246111691042\\
6.87	-9.77246207402915\\
6.88	-9.77246301063796\\
6.89	-9.7724639186987\\
6.9	-9.77246479024495\\
6.91	-9.77246561748229\\
6.92	-9.77246639288747\\
6.93	-9.77246710930494\\
6.94	-9.77246776003939\\
6.95	-9.77246833894291\\
6.96	-9.77246884049542\\
6.97	-9.77246925987724\\
6.98	-9.77246959303269\\
6.99	-9.77246983672379\\
7	-9.77246998857317\\
7.01	-9.77247004709571\\
7.02	-9.7724700117183\\
7.03	-9.77246988278744\\
7.04	-9.77246966156465\\
7.05	-9.77246935020969\\
7.06	-9.7724689517518\\
7.07	-9.77246847004955\\
7.08	-9.7724679097397\\
7.09	-9.77246727617596\\
7.1	-9.77246657535857\\
7.11	-9.77246581385566\\
7.12	-9.77246499871768\\
7.13	-9.77246413738599\\
7.14	-9.77246323759728\\
7.15	-9.77246230728482\\
7.16	-9.77246135447841\\
7.17	-9.77246038720415\\
7.18	-9.77245941338583\\
7.19	-9.77245844074903\\
7.2	-9.77245747672968\\
7.21	-9.77245652838806\\
7.22	-9.77245560232974\\
7.23	-9.77245470463445\\
7.24	-9.7724538407938\\
7.25	-9.77245301565887\\
7.26	-9.77245223339826\\
7.27	-9.77245149746704\\
7.28	-9.77245081058721\\
7.29	-9.77245017473949\\
7.3	-9.7724495911668\\
7.31	-9.77244906038891\\
7.32	-9.77244858222817\\
7.33	-9.77244815584558\\
7.34	-9.77244777978667\\
7.35	-9.77244745203614\\
7.36	-9.77244717008056\\
7.37	-9.77244693097772\\
7.38	-9.77244673143165\\
7.39	-9.77244656787195\\
7.4	-9.77244643653602\\
7.41	-9.77244633355287\\
7.42	-9.77244625502709\\
7.43	-9.77244619712155\\
7.44	-9.77244615613743\\
7.45	-9.7724461285903\\
7.46	-9.77244611128098\\
7.47	-9.7724461013599\\
7.48	-9.77244609638407\\
7.49	-9.77244609436558\\
7.5	-9.77244609381087\\
7.51	-9.77244609375013\\
7.52	-9.77244609375637\\
7.53	-9.77244609395377\\
7.54	-9.77244609501526\\
7.55	-9.77244609814933\\
7.56	-9.77244610507623\\
7.57	-9.77244611799408\\
7.58	-9.77244613953535\\
7.59	-9.77244617271446\\
7.6	-9.77244622086731\\
7.61	-9.77244628758393\\
7.62	-9.77244637663505\\
7.63	-9.77244649189412\\
7.64	-9.77244663725591\\
7.65	-9.77244681655303\\
7.66	-9.77244703347193\\
7.67	-9.77244729146955\\
7.68	-9.77244759369224\\
7.69	-9.77244794289825\\
7.7	-9.77244834138507\\
7.71	-9.77244879092293\\
7.72	-9.77244929269561\\
7.73	-9.77244984724962\\
7.74	-9.77245045445262\\
7.75	-9.77245111346198\\
7.76	-9.77245182270392\\
7.77	-9.77245257986388\\
7.78	-9.77245338188832\\
7.79	-9.77245422499794\\
7.8	-9.77245510471255\\
7.81	-9.77245601588699\\
7.82	-9.77245695275792\\
7.83	-9.77245790900074\\
7.84	-9.77245887779594\\
7.85	-9.77245985190399\\
7.86	-9.77246082374762\\
7.87	-9.77246178550044\\
7.88	-9.77246272918057\\
7.89	-9.7724636467479\\
7.9	-9.77246453020365\\
7.91	-9.77246537169066\\
7.92	-9.77246616359309\\
7.93	-9.77246689863384\\
7.94	-9.77246756996855\\
7.95	-9.7724681712744\\
7.96	-9.77246869683283\\
7.97	-9.77246914160455\\
7.98	-9.772469501296\\
7.99	-9.7724697724161\\
8	-9.77246995232257\\
8.01	-9.772470039257\\
8.02	-9.77247003236826\\
8.03	-9.77246993172384\\
8.04	-9.77246973830891\\
8.05	-9.77246945401326\\
8.06	-9.77246908160612\\
8.07	-9.77246862469941\\
8.08	-9.77246808769986\\
8.09	-9.77246747575074\\
8.1	-9.77246679466413\\
8.11	-9.77246605084466\\
8.12	-9.77246525120587\\
8.13	-9.77246440308049\\
8.14	-9.77246351412595\\
8.15	-9.77246259222642\\
8.16	-9.77246164539303\\
8.17	-9.77246068166353\\
8.18	-9.77245970900303\\
8.19	-9.77245873520712\\
8.2	-9.77245776780895\\
8.21	-9.77245681399145\\
8.22	-9.77245588050603\\
8.23	-9.77245497359891\\
8.24	-9.7724540989461\\
8.25	-9.77245326159792\\
8.26	-9.77245246593376\\
8.27	-9.77245171562774\\
8.28	-9.7724510136256\\
8.29	-9.7724503621331\\
8.3	-9.7724497626159\\
8.31	-9.77244921581092\\
8.32	-9.7724487217487\\
8.33	-9.77244827978646\\
8.34	-9.77244788865101\\
8.35	-9.77244754649091\\
8.36	-9.77244725093677\\
8.37	-9.77244699916876\\
8.38	-9.77244678799001\\
8.39	-9.77244661390479\\
8.4	-9.77244647319997\\
8.41	-9.77244636202856\\
8.42	-9.77244627649373\\
8.43	-9.77244621273206\\
8.44	-9.77244616699453\\
8.45	-9.77244613572394\\
8.46	-9.77244611562747\\
8.47	-9.77244610374309\\
8.48	-9.77244609749899\\
8.49	-9.77244609476461\\
8.5	-9.77244609389286\\
8.51	-9.77244609375258\\
8.52	-9.77244609375073\\
8.53	-9.77244609384407\\
8.54	-9.77244609453998\\
8.55	-9.77244609688652\\
8.56	-9.77244610245184\\
8.57	-9.77244611329327\\
8.58	-9.77244613191669\\
8.59	-9.77244616122667\\
8.6	-9.77244620446844\\
8.61	-9.77244626516243\\
8.62	-9.77244634703261\\
8.63	-9.77244645392976\\
8.64	-9.77244658975092\\
8.65	-9.77244675835644\\
8.66	-9.77244696348597\\
8.67	-9.77244720867476\\
8.68	-9.77244749717176\\
8.69	-9.77244783186087\\
8.7	-9.77244821518668\\
8.71	-9.77244864908596\\
8.72	-9.77244913492614\\
8.73	-9.77244967345189\\
8.74	-9.77245026474059\\
8.75	-9.77245090816775\\
8.76	-9.77245160238292\\
8.77	-9.77245234529659\\
8.78	-9.7724531340785\\
8.79	-9.77245396516739\\
8.8	-9.7724548342924\\
8.81	-9.77245573650561\\
8.82	-9.77245666622563\\
8.83	-9.77245761729152\\
8.84	-9.77245858302635\\
8.85	-9.77245955630963\\
8.86	-9.77246052965743\\
8.87	-9.77246149530924\\
8.88	-9.77246244532021\\
8.89	-9.77246337165746\\
8.9	-9.77246426629906\\
8.91	-9.77246512133431\\
8.92	-9.77246592906376\\
8.93	-9.77246668209742\\
8.94	-9.77246737345003\\
8.95	-9.77246799663152\\
8.96	-9.77246854573174\\
8.97	-9.77246901549791\\
8.98	-9.7724694014038\\
8.99	-9.77246969970954\\
9	-9.77246990751114\\
9.01	-9.77247002277906\\
9.02	-9.77247004438522\\
9.03	-9.77246997211797\\
9.04	-9.77246980668496\\
9.05	-9.77246954970368\\
9.06	-9.77246920367996\\
9.07	-9.77246877197467\\
9.08	-9.77246825875913\\
9.09	-9.77246766895994\\
9.1	-9.772467008194\\
9.11	-9.77246628269471\\
9.12	-9.77246549923047\\
9.13	-9.77246466501666\\
9.14	-9.77246378762241\\
9.15	-9.77246287487363\\
9.16	-9.77246193475352\\
9.17	-9.77246097530233\\
9.18	-9.77246000451758\\
9.19	-9.77245903025631\\
9.2	-9.77245806014082\\
9.21	-9.77245710146916\\
9.22	-9.7724561611318\\
9.23	-9.7724552455355\\
9.24	-9.77245436053554\\
9.25	-9.7724535113773\\
9.26	-9.77245270264782\\
9.27	-9.77245193823811\\
9.28	-9.77245122131663\\
9.29	-9.77245055431412\\
9.3	-9.77244993892006\\
9.31	-9.77244937609057\\
9.32	-9.77244886606744\\
9.33	-9.77244840840802\\
9.34	-9.77244800202526\\
9.35	-9.7724476452371\\
9.36	-9.77244733582444\\
9.37	-9.77244707109657\\
9.38	-9.77244684796293\\
9.39	-9.77244666300994\\
9.4	-9.77244651258168\\
9.41	-9.77244639286287\\
9.42	-9.77244629996294\\
9.43	-9.77244622999965\\
9.44	-9.77244617918091\\
9.45	-9.77244614388343\\
9.46	-9.77244612072684\\
9.47	-9.77244610664218\\
9.48	-9.77244609893347\\
9.49	-9.77244609533142\\
9.5	-9.7724460940384\\
9.51	-9.77244609376393\\
9.52	-9.77244609375001\\
9.53	-9.77244609378604\\
9.54	-9.77244609421294\\
9.55	-9.77244609591646\\
9.56	-9.77244610030978\\
9.57	-9.77244610930572\\
9.58	-9.77244612527893\\
9.59	-9.77244615101884\\
9.6	-9.77244618967394\\
9.61	-9.77244624468849\\
9.62	-9.77244631973262\\
9.63	-9.77244641862699\\
9.64	-9.77244654526331\\
9.65	-9.77244670352191\\
9.66	-9.77244689718793\\
9.67	-9.77244712986738\\
9.68	-9.77244740490456\\
9.69	-9.77244772530218\\
9.7	-9.7724480936456\\
9.71	-9.77244851203248\\
9.72	-9.772448982009\\
9.73	-9.77244950451386\\
9.74	-9.77245007983096\\
9.75	-9.77245070755176\\
9.76	-9.77245138654791\\
9.77	-9.77245211495481\\
9.78	-9.77245289016636\\
9.79	-9.77245370884133\\
9.8	-9.77245456692116\\
9.81	-9.77245545965921\\
9.82	-9.77245638166101\\
9.83	-9.77245732693511\\
9.84	-9.77245828895385\\
9.85	-9.77245926072314\\
9.86	-9.77246023486041\\
9.87	-9.77246120367952\\
9.88	-9.77246215928152\\
9.89	-9.77246309364987\\
9.9	-9.77246399874875\\
9.91	-9.77246486662313\\
9.92	-9.77246568949896\\
9.93	-9.77246645988218\\
9.94	-9.77246717065494\\
9.95	-9.77246781516772\\
9.96	-9.77246838732592\\
9.97	-9.77246888166966\\
9.98	-9.77246929344555\\
9.99	-9.77246961866946\\
10	-9.77246985417926\\
10.01	-9.77246999767678\\
10.02	-9.77247004775833\\
10.03	-9.77247000393343\\
10.04	-9.77246986663131\\
10.05	-9.7724696371952\\
10.06	-9.77246931786445\\
10.07	-9.77246891174477\\
10.08	-9.772468422767\\
10.09	-9.77246785563508\\
10.1	-9.77246721576392\\
10.11	-9.7724665092082\\
10.12	-9.77246574258306\\
10.13	-9.77246492297793\\
10.14	-9.77246405786475\\
10.15	-9.77246315500196\\
10.16	-9.77246222233563\\
10.17	-9.7724612678993\\
10.18	-9.7724602997139\\
10.19	-9.77245932568923\\
10.2	-9.77245835352849\\
10.21	-9.77245739063721\\
10.22	-9.77245644403789\\
10.23	-9.77245552029156\\
10.24	-9.77245462542743\\
10.25	-9.77245376488145\\
10.26	-9.77245294344485\\
10.27	-9.7724521652231\\
10.28	-9.77245143360595\\
10.29	-9.77245075124883\\
10.3	-9.77245012006575\\
10.31	-9.77244954123379\\
10.32	-9.77244901520873\\
10.33	-9.77244854175178\\
10.34	-9.77244811996654\\
10.35	-9.77244774834568\\
10.36	-9.77244742482641\\
10.37	-9.77244714685378\\
10.38	-9.77244691145059\\
10.39	-9.77244671529287\\
10.4	-9.77244655478954\\
10.41	-9.77244642616476\\
10.42	-9.77244632554196\\
10.43	-9.77244624902758\\
10.44	-9.77244619279373\\
10.45	-9.77244615315784\\
10.46	-9.77244612665833\\
10.47	-9.77244611012495\\
10.48	-9.77244610074252\\
10.49	-9.77244609610715\\
10.5	-9.77244609427394\\
10.51	-9.77244609379543\\
10.52	-9.77244609375003\\
10.53	-9.77244609376017\\
10.54	-9.77244609399968\\
10.55	-9.77244609519041\\
10.56	-9.772446098588\\
10.57	-9.77244610595725\\
10.58	-9.77244611953732\\
10.59	-9.7724461419973\\
10.6	-9.77244617638313\\
10.61	-9.77244622605646\\
10.62	-9.77244629462663\\
10.63	-9.77244638587689\\
10.64	-9.77244650368598\\
10.65	-9.77244665194651\\
10.66	-9.77244683448139\\
10.67	-9.77244705495978\\
10.68	-9.77244731681395\\
10.69	-9.77244762315845\\
10.7	-9.77244797671295\\
10.71	-9.7724483797301\\
10.72	-9.77244883392965\\
10.73	-9.77244934043999\\
10.74	-9.77244989974809\\
10.75	-9.77245051165882\\
10.76	-9.77245117526444\\
10.77	-9.77245188892472\\
10.78	-9.77245265025836\\
10.79	-9.77245345614577\\
10.8	-9.77245430274341\\
10.81	-9.77245518550958\\
10.82	-9.77245609924145\\
10.83	-9.77245703812265\\
10.84	-9.77245799578122\\
10.85	-9.77245896535665\\
10.86	-9.77245993957554\\
10.87	-9.77246091083449\\
10.88	-9.77246187128923\\
10.89	-9.77246281294859\\
10.9	-9.7724637277721\\
10.91	-9.77246460776961\\
10.92	-9.77246544510162\\
10.93	-9.77246623217882\\
10.94	-9.7724669617593\\
10.95	-9.77246762704204\\
10.96	-9.77246822175536\\
10.97	-9.77246874023886\\
10.98	-9.77246917751779\\
10.99	-9.77246952936862\\
11	-9.77246979237495\\
11.01	-9.77246996397277\\
11.02	-9.77247004248454\\
11.03	-9.77247002714153\\
11.04	-9.77246991809403\\
11.05	-9.77246971640934\\
11.06	-9.77246942405761\\
11.07	-9.77246904388559\\
11.08	-9.77246857957884\\
11.09	-9.77246803561291\\
11.1	-9.77246741719418\\
11.11	-9.7724667301913\\
11.12	-9.77246598105821\\
11.13	-9.77246517674994\\
11.14	-9.77246432463239\\
11.15	-9.77246343238741\\
11.16	-9.77246250791473\\
11.17	-9.77246155923197\\
11.18	-9.77246059437438\\
11.19	-9.77245962129573\\
11.2	-9.77245864777173\\
11.21	-9.77245768130753\\
11.22	-9.77245672905042\\
11.23	-9.77245579770926\\
11.24	-9.77245489348148\\
11.25	-9.77245402198889\\
11.26	-9.77245318822314\\
11.27	-9.77245239650137\\
11.28	-9.77245165043293\\
11.29	-9.77245095289728\\
11.3	-9.77245030603338\\
11.31	-9.7724497112407\\
11.32	-9.77244916919148\\
11.33	-9.77244867985419\\
11.34	-9.77244824252739\\
11.35	-9.77244785588359\\
11.36	-9.77244751802215\\
11.37	-9.77244722653029\\
11.38	-9.77244697855115\\
11.39	-9.77244677085771\\
11.4	-9.77244659993129\\
11.41	-9.77244646204328\\
11.42	-9.77244635333877\\
11.43	-9.77244626992054\\
11.44	-9.77244620793221\\
11.45	-9.77244616363891\\
11.46	-9.77244613350434\\
11.47	-9.77244611426282\\
11.48	-9.77244610298515\\
11.49	-9.77244609713723\\
11.5	-9.77244609463046\\
11.51	-9.77244609386298\\
11.52	-9.77244609375131\\
11.53	-9.77244609375158\\
11.54	-9.77244609387022\\
11.55	-9.77244609466386\\
11.56	-9.77244609722834\\
11.57	-9.77244610317726\\
11.58	-9.77244611461013\\
11.59	-9.77244613407087\\
11.6	-9.77244616449721\\
11.61	-9.77244620916193\\
11.62	-9.7724462716068\\
11.63	-9.7724463555704\\
11.64	-9.77244646491102\\
11.65	-9.77244660352582\\
11.66	-9.77244677526775\\
11.67	-9.77244698386144\\
11.68	-9.77244723281972\\
11.69	-9.77244752536185\\
11.7	-9.77244786433515\\
11.71	-9.7724482521412\\
11.72	-9.77244869066796\\
11.73	-9.77244918122889\\
11.74	-9.77244972451025\\
11.75	-9.77245032052753\\
11.76	-9.77245096859174\\
11.77	-9.77245166728628\\
11.78	-9.77245241445488\\
11.79	-9.77245320720089\\
11.8	-9.77245404189818\\
11.81	-9.77245491421344\\
11.82	-9.77245581913979\\
11.83	-9.77245675104134\\
11.84	-9.77245770370795\\
11.85	-9.77245867041971\\
11.86	-9.77245964402\\
11.87	-9.77246061699638\\
11.88	-9.77246158156789\\
11.89	-9.77246252977776\\
11.9	-9.77246345359001\\
11.91	-9.77246434498862\\
11.92	-9.77246519607783\\
11.93	-9.77246599918202\\
11.94	-9.77246674694383\\
11.95	-9.77246743241889\\
11.96	-9.77246804916599\\
11.97	-9.77246859133106\\
11.98	-9.77246905372402\\
11.99	-9.77246943188712\\
12	-9.77246972215382\\
12.01	-9.77246992169744\\
12.02	-9.77247002856863\\
12.03	-9.77247004172134\\
12.04	-9.77246996102676\\
12.05	-9.77246978727495\\
12.06	-9.77246952216444\\
12.07	-9.77246916827953\\
12.08	-9.77246872905601\\
12.09	-9.7724682087356\\
12.1	-9.77246761230983\\
12.11	-9.77246694545421\\
12.12	-9.77246621445375\\
12.13	-9.77246542612075\\
12.14	-9.77246458770632\\
12.15	-9.77246370680673\\
12.16	-9.77246279126611\\
12.17	-9.77246184907694\\
12.18	-9.77246088827967\\
12.19	-9.77245991686313\\
12.2	-9.77245894266705\\
12.21	-9.77245797328812\\
12.22	-9.77245701599103\\
12.23	-9.77245607762572\\
12.24	-9.77245516455194\\
12.25	-9.77245428257235\\
12.26	-9.7724534368749\\
12.27	-9.77245263198536\\
12.28	-9.77245187173066\\
12.29	-9.77245115921328\\
12.3	-9.77245049679723\\
12.31	-9.77244988610552\\
12.32	-9.77244932802901\\
12.33	-9.77244882274653\\
12.34	-9.77244836975566\\
12.35	-9.77244796791361\\
12.36	-9.77244761548751\\
12.37	-9.77244731021305\\
12.38	-9.77244704936049\\
12.39	-9.772446829807\\
12.4	-9.77244664811384\\
12.41	-9.77244650060731\\
12.42	-9.77244638346193\\
12.43	-9.77244629278445\\
12.44	-9.77244622469745\\
12.45	-9.77244617542086\\
12.46	-9.77244614135029\\
12.47	-9.77244611913069\\
12.48	-9.77244610572427\\
12.49	-9.77244609847135\\
12.5	-9.77244609514339\\
12.51	-9.77244609398711\\
12.52	-9.77244609375905\\
12.53	-9.77244609375004\\
12.54	-9.77244609379911\\
12.55	-9.77244609429663\\
12.56	-9.7724460961767\\
12.57	-9.77244610089875\\
12.58	-9.77244611041885\\
12.59	-9.77244612715106\\
12.6	-9.77244615391951\\
12.61	-9.77244619390201\\
12.62	-9.7724462505661\\
12.63	-9.77244632759862\\
12.64	-9.77244642882993\\
12.65	-9.7724465581541\\
12.66	-9.77244671944637\\
12.67	-9.77244691647916\\
12.68	-9.7724471528383\\
12.69	-9.77244743184055\\
12.7	-9.7724477564541\\
12.71	-9.77244812922321\\
12.72	-9.77244855219839\\
12.73	-9.77244902687335\\
12.74	-9.77244955412963\\
12.75	-9.77245013419023\\
12.76	-9.77245076658278\\
12.77	-9.77245145011318\\
12.78	-9.77245218285013\\
12.79	-9.77245296212095\\
12.8	-9.77245378451895\\
12.81	-9.77245464592228\\
12.82	-9.77245554152417\\
12.83	-9.77245646587421\\
12.84	-9.77245741293006\\
12.85	-9.77245837611908\\
12.86	-9.77245934840894\\
12.87	-9.77246032238615\\
12.88	-9.77246129034164\\
12.89	-9.77246224436192\\
12.9	-9.77246317642465\\
12.91	-9.77246407849717\\
12.92	-9.77246494263665\\
12.93	-9.77246576109021\\
12.94	-9.77246652639375\\
12.95	-9.77246723146789\\
12.96	-9.77246786970958\\
12.97	-9.7724684350782\\
12.98	-9.77246892217462\\
12.99	-9.77246932631227\\
13	-9.772469643579\\
13.01	-9.77246987088887\\
13.02	-9.77247000602313\\
13.03	-9.77247004765971\\
13.04	-9.77246999539081\\
13.05	-9.77246984972833\\
13.06	-9.77246961209706\\
13.07	-9.77246928481571\\
13.08	-9.77246887106611\\
13.09	-9.77246837485096\\
13.1	-9.77246780094087\\
13.11	-9.77246715481138\\
13.12	-9.77246644257097\\
13.13	-9.77246567088108\\
13.14	-9.7724648468694\\
13.15	-9.77246397803763\\
13.16	-9.77246307216518\\
13.17	-9.77246213721011\\
13.18	-9.7724611812089\\
13.19	-9.77246021217649\\
13.2	-9.77245923800793\\
13.21	-9.77245826638328\\
13.22	-9.772457304677\\
13.23	-9.77245635987317\\
13.24	-9.77245543848772\\
13.25	-9.7724545464988\\
13.26	-9.77245368928633\\
13.27	-9.77245287158133\\
13.28	-9.77245209742592\\
13.29	-9.77245137014434\\
13.3	-9.77245069232543\\
13.31	-9.77245006581652\\
13.32	-9.77244949172896\\
13.33	-9.77244897045477\\
13.34	-9.77244850169432\\
13.35	-9.77244808449418\\
13.36	-9.77244771729462\\
13.37	-9.77244739798588\\
13.38	-9.77244712397203\\
13.39	-9.77244689224151\\
13.4	-9.77244669944301\\
13.41	-9.77244654196539\\
13.42	-9.77244641602032\\
13.43	-9.77244631772626\\
13.44	-9.77244624319224\\
13.45	-9.7724461886002\\
13.46	-9.77244615028445\\
13.47	-9.77244612480684\\
13.48	-9.77244610902657\\
13.49	-9.77244610016328\\
13.5	-9.77244609585258\\
13.51	-9.77244609419292\\
13.52	-9.77244609378313\\
13.53	-9.77244609375\\
13.54	-9.77244609376546\\
13.55	-9.77244609405295\\
13.56	-9.77244609538308\\
13.57	-9.77244609905851\\
13.58	-9.7724461068883\\
13.59	-9.77244612115224\\
13.6	-9.77244614455564\\
13.61	-9.77244618017546\\
13.62	-9.77244623139854\\
13.63	-9.77244630185296\\
13.64	-9.77244639533383\\
13.65	-9.77244651572453\\
13.66	-9.77244666691483\\
13.67	-9.77244685271723\\
13.68	-9.77244707678297\\
13.69	-9.77244734251898\\
13.7	-9.77244765300737\\
13.71	-9.77244801092863\\
13.72	-9.77244841849007\\
13.73	-9.77244887736049\\
13.74	-9.77244938861247\\
13.75	-9.7724499526731\\
13.76	-9.77245056928423\\
13.77	-9.77245123747282\\
13.78	-9.77245195553216\\
13.79	-9.7724527210142\\
13.8	-9.77245353073343\\
13.81	-9.77245438078227\\
13.82	-9.77245526655785\\
13.83	-9.77245618279998\\
13.84	-9.7724571236398\\
13.85	-9.77245808265849\\
13.86	-9.77245905295518\\
13.87	-9.77246002722327\\
13.88	-9.77246099783395\\
13.89	-9.77246195692581\\
13.9	-9.77246289649922\\
13.91	-9.77246380851412\\
13.92	-9.77246468498982\\
13.93	-9.7724655181053\\
13.94	-9.77246630029856\\
13.95	-9.7724670243636\\
13.96	-9.77246768354353\\
13.97	-9.77246827161842\\
13.98	-9.77246878298665\\
13.99	-9.77246921273854\\
14	-9.77246955672107\\
14.01	-9.77246981159283\\
14.02	-9.7724699748684\\
14.03	-9.77247004495128\\
14.04	-9.77247002115521\\
14.05	-9.77246990371328\\
14.06	-9.77246969377482\\
14.07	-9.77246939339009\\
14.08	-9.77246900548307\\
14.09	-9.77246853381257\\
14.1	-9.77246798292248\\
14.11	-9.77246735808172\\
14.12	-9.7724666652149\\
14.13	-9.77246591082459\\
14.14	-9.77246510190657\\
14.15	-9.77246424585912\\
14.16	-9.77246335038777\\
14.17	-9.77246242340694\\
14.18	-9.77246147293996\\
14.19	-9.77246050701876\\
14.2	-9.772459533585\\
14.21	-9.77245856039377\\
14.22	-9.77245759492147\\
14.23	-9.77245664427915\\
14.24	-9.77245571513249\\
14.25	-9.77245481362961\\
14.26	-9.77245394533771\\
14.27	-9.77245311518937\\
14.28	-9.77245232743924\\
14.29	-9.77245158563172\\
14.3	-9.77245089257987\\
14.31	-9.77245025035591\\
14.32	-9.77244966029316\\
14.33	-9.77244912299944\\
14.34	-9.77244863838136\\
14.35	-9.77244820567921\\
14.36	-9.77244782351163\\
14.37	-9.77244748992926\\
14.38	-9.77244720247649\\
14.39	-9.77244695826003\\
14.4	-9.77244675402334\\
14.41	-9.77244658622547\\
14.42	-9.772446451123\\
14.43	-9.77244634485376\\
14.44	-9.77244626352086\\
14.45	-9.77244620327556\\
14.46	-9.77244616039779\\
14.47	-9.77244613137274\\
14.48	-9.77244611296241\\
14.49	-9.77244610227088\\
14.5	-9.77244609680221\\
14.51	-9.77244609451005\\
14.52	-9.77244609383805\\
14.53	-9.77244609375058\\
14.54	-9.77244609375302\\
14.55	-9.77244609390149\\
14.56	-9.77244609480173\\
14.57	-9.77244609759724\\
14.58	-9.7724461039468\\
14.59	-9.77244611599175\\
14.6	-9.77244613631365\\
14.61	-9.77244616788291\\
14.62	-9.77244621399931\\
14.63	-9.77244627822538\\
14.64	-9.77244636431371\\
14.65	-9.77244647612942\\
14.66	-9.77244661756913\\
14.67	-9.77244679247768\\
14.68	-9.77244700456405\\
14.69	-9.77244725731794\\
14.7	-9.7724475539283\\
14.71	-9.77244789720528\\
14.72	-9.77244828950692\\
14.73	-9.77244873267185\\
14.74	-9.7724492279591\\
14.75	-9.77244977599622\\
14.76	-9.77245037673652\\
14.77	-9.77245102942632\\
14.78	-9.77245173258277\\
14.79	-9.77245248398281\\
14.8	-9.77245328066355\\
14.81	-9.77245411893412\\
14.82	-9.77245499439904\\
14.83	-9.77245590199285\\
14.84	-9.77245683602558\\
14.85	-9.77245779023844\\
14.86	-9.7724587578691\\
14.87	-9.7724597317255\\
14.88	-9.77246070426737\\
14.89	-9.77246166769406\\
14.9	-9.77246261403768\\
14.91	-9.77246353525994\\
14.92	-9.77246442335155\\
14.93	-9.77246527043247\\
14.94	-9.77246606885176\\
14.95	-9.77246681128539\\
14.96	-9.77246749083069\\
14.97	-9.77246810109592\\
14.98	-9.77246863628377\\
14.99	-9.77246909126738\\
15	-9.77246946165794\\
15.01	-9.77246974386269\\
15.02	-9.77246993513251\\
15.03	-9.77247003359849\\
15.04	-9.77247003829673\\
15.05	-9.77246994918121\\
15.06	-9.77246976712438\\
15.07	-9.77246949390559\\
15.08	-9.77246913218734\\
15.09	-9.77246868548\\
15.1	-9.7724681580952\\
15.11	-9.77246755508886\\
15.12	-9.77246688219452\\
15.13	-9.77246614574812\\
15.14	-9.77246535260516\\
15.15	-9.7724645100517\\
15.16	-9.77246362571036\\
15.17	-9.77246270744272\\
15.18	-9.77246176324967\\
15.19	-9.77246080117109\\
15.2	-9.77245982918634\\
15.21	-9.77245885511703\\
15.22	-9.77245788653356\\
15.23	-9.77245693066658\\
15.24	-9.77245599432486\\
15.25	-9.77245508382064\\
15.26	-9.77245420490347\\
15.27	-9.77245336270349\\
15.28	-9.77245256168492\\
15.29	-9.77245180561035\\
15.3	-9.77245109751622\\
15.31	-9.77245043969979\\
15.32	-9.77244983371762\\
15.33	-9.77244928039553\\
15.34	-9.7724487798496\\
15.35	-9.77244833151797\\
15.36	-9.77244793420256\\
15.37	-9.77244758612015\\
15.38	-9.77244728496172\\
15.39	-9.77244702795913\\
15.4	-9.77244681195788\\
15.41	-9.77244663349473\\
15.42	-9.7724464888789\\
15.43	-9.77244637427535\\
15.44	-9.77244628578888\\
15.45	-9.77244621954749\\
15.46	-9.77244617178381\\
15.47	-9.77244613891293\\
15.48	-9.77244611760568\\
15.49	-9.7724461048559\\
15.5	-9.77244609804069\\
15.51	-9.77244609497261\\
15.52	-9.77244609394298\\
15.53	-9.77244609375559\\
15.54	-9.77244609375018\\
15.55	-9.77244609381545\\
15.56	-9.77244609439117\\
15.57	-9.77244609645962\\
15.58	-9.77244610152627\\
15.59	-9.77244611159012\\
15.6	-9.77244612910424\\
15.61	-9.77244615692702\\
15.62	-9.77244619826505\\
15.63	-9.7724462566086\\
15.64	-9.77244633566058\\
15.65	-9.77244643926041\\
15.66	-9.77244657130389\\
15.67	-9.77244673566043\\
15.68	-9.77244693608911\\
15.69	-9.77244717615485\\
15.7	-9.77244745914627\\
15.71	-9.77244778799639\\
15.72	-9.77244816520783\\
15.73	-9.77244859278346\\
15.74	-9.77244907216402\\
15.75	-9.77244960417357\\
15.76	-9.77245018897385\\
15.77	-9.77245082602849\\
15.78	-9.7724515140775\\
15.79	-9.77245225112285\\
15.8	-9.77245303442534\\
15.81	-9.77245386051295\\
15.82	-9.77245472520082\\
15.83	-9.77245562362238\\
15.84	-9.77245655027168\\
15.85	-9.77245749905598\\
15.86	-9.77245846335829\\
15.87	-9.7724594361086\\
15.88	-9.77246040986325\\
15.89	-9.77246137689096\\
15.9	-9.77246232926448\\
15.91	-9.77246325895646\\
15.92	-9.77246415793823\\
15.93	-9.77246501827994\\
15.94	-9.7724658322507\\
15.95	-9.77246659241717\\
15.96	-9.77246729173914\\
15.97	-9.77246792366076\\
15.98	-9.77246848219601\\
15.99	-9.77246896200712\\
16	-9.77246935847482\\
16.01	-9.77246966775934\\
16.02	-9.77246988685129\\
16.03	-9.77247001361158\\
16.04	-9.77247004679989\\
16.05	-9.77246998609113\\
16.06	-9.77246983207984\\
16.07	-9.7724695862722\\
16.08	-9.77246925106605\\
16.09	-9.77246882971899\\
16.1	-9.77246832630514\\
16.11	-9.77246774566128\\
16.12	-9.772467093323\\
16.13	-9.77246637545191\\
16.14	-9.77246559875506\\
16.15	-9.77246477039765\\
16.16	-9.77246389791038\\
16.17	-9.77246298909279\\
16.18	-9.77246205191414\\
16.19	-9.77246109441304\\
16.2	-9.77246012459762\\
16.21	-9.77245915034743\\
16.22	-9.77245817931867\\
16.23	-9.77245721885402\\
16.24	-9.7724562758985\\
16.25	-9.77245535692237\\
16.26	-9.77245446785228\\
16.27	-9.77245361401167\\
16.28	-9.77245280007103\\
16.29	-9.77245203000889\\
16.3	-9.77245130708386\\
16.31	-9.77245063381812\\
16.32	-9.77245001199237\\
16.33	-9.77244944265236\\
16.34	-9.77244892612659\\
16.35	-9.77244846205489\\
16.36	-9.77244804942717\\
16.37	-9.77244768663179\\
16.38	-9.7724473715125\\
16.39	-9.77244710143298\\
16.4	-9.77244687334794\\
16.41	-9.77244668387935\\
16.42	-9.7724465293967\\
16.43	-9.77244640609985\\
16.44	-9.77244631010297\\
16.45	-9.77244623751832\\
16.46	-9.77244618453835\\
16.47	-9.77244614751485\\
16.48	-9.77244612303366\\
16.49	-9.77244610798393\\
16.5	-9.77244609962061\\
16.51	-9.77244609561916\\
16.52	-9.77244609412167\\
16.53	-9.77244609377351\\
16.54	-9.77244609375\\
16.55	-9.77244609377259\\
16.56	-9.7724460941143\\
16.57	-9.77244609559442\\
16.58	-9.77244609956237\\
16.59	-9.77244610787118\\
16.6	-9.77244612284093\\
16.61	-9.77244614721271\\
16.62	-9.77244618409402\\
16.63	-9.77244623689627\\
16.64	-9.77244630926574\\
16.65	-9.77244640500874\\
16.66	-9.7724465280126\\
16.67	-9.7724466821636\\
16.68	-9.77244687126316\\
16.69	-9.77244709894389\\
16.7	-9.77244736858676\\
16.71	-9.77244768324078\\
16.72	-9.77244804554674\\
16.73	-9.77244845766603\\
16.74	-9.77244892121604\\
16.75	-9.77244943721314\\
16.76	-9.77245000602423\\
16.77	-9.77245062732788\\
16.78	-9.77245130008564\\
16.79	-9.77245202252419\\
16.8	-9.77245279212882\\
16.81	-9.77245360564821\\
16.82	-9.77245445911095\\
16.83	-9.77245534785328\\
16.84	-9.77245626655811\\
16.85	-9.77245720930452\\
16.86	-9.77245816962739\\
16.87	-9.77245914058611\\
16.88	-9.77246011484154\\
16.89	-9.77246108474021\\
16.9	-9.77246204240433\\
16.91	-9.7724629798266\\
16.92	-9.77246388896818\\
16.93	-9.77246476185867\\
16.94	-9.77246559069629\\
16.95	-9.77246636794717\\
16.96	-9.772467086442\\
16.97	-9.77246773946869\\
16.98	-9.77246832085972\\
16.99	-9.77246882507288\\
17	-9.77246924726405\\
17.01	-9.77246958335117\\
17.02	-9.77246983006822\\
17.03	-9.77246998500859\\
17.04	-9.77247004665702\\
17.05	-9.77247001440979\\
17.06	-9.77246988858275\\
17.07	-9.77246967040713\\
17.08	-9.77246936201311\\
17.09	-9.77246896640156\\
17.1	-9.77246848740416\\
17.11	-9.77246792963263\\
17.12	-9.7724672984179\\
17.13	-9.77246659973985\\
17.14	-9.77246584014904\\
17.15	-9.77246502668127\\
17.16	-9.7724641667664\\
17.17	-9.77246326813284\\
17.18	-9.77246233870892\\
17.19	-9.77246138652283\\
17.2	-9.77246041960242\\
17.21	-9.77245944587643\\
17.22	-9.77245847307857\\
17.23	-9.77245750865582\\
17.24	-9.77245655968232\\
17.25	-9.77245563278001\\
17.26	-9.77245473404717\\
17.27	-9.77245386899594\\
17.28	-9.77245304249949\\
17.29	-9.77245225874973\\
17.3	-9.77245152122591\\
17.31	-9.77245083267465\\
17.32	-9.7724501951014\\
17.33	-9.77244960977351\\
17.34	-9.7724490772345\\
17.35	-9.77244859732945\\
17.36	-9.77244816924073\\
17.37	-9.77244779153353\\
17.38	-9.7724474622103\\
17.39	-9.77244717877313\\
17.4	-9.77244693829292\\
17.41	-9.77244673748425\\
17.42	-9.77244657278455\\
17.43	-9.77244644043628\\
17.44	-9.7724463365707\\
17.45	-9.77244625729188\\
17.46	-9.77244619875944\\
17.47	-9.77244615726868\\
17.48	-9.77244612932688\\
17.49	-9.77244611172425\\
17.5	-9.77244610159861\\
17.51	-9.77244609649259\\
17.52	-9.7724460944024\\
17.53	-9.77244609381749\\
17.54	-9.77244609375021\\
17.55	-9.77244609375528\\
17.56	-9.77244609393848\\
17.57	-9.77244609495461\\
17.58	-9.77244609799461\\
17.59	-9.77244610476218\\
17.6	-9.77244611744015\\
17.61	-9.7724461386473\\
17.62	-9.77244617138625\\
17.63	-9.77244621898324\\
17.64	-9.77244628502097\\
17.65	-9.77244637326538\\
17.66	-9.77244648758779\\
17.67	-9.77244663188361\\
17.68	-9.77244680998888\\
17.69	-9.77244702559625\\
17.7	-9.77244728217164\\
17.71	-9.77244758287305\\
17.72	-9.77244793047285\\
17.73	-9.77244832728501\\
17.74	-9.77244877509833\\
17.75	-9.77244927511701\\
17.76	-9.77244982790954\\
17.77	-9.77245043336677\\
17.78	-9.77245109067015\\
17.79	-9.7724517982705\\
17.8	-9.77245255387799\\
17.81	-9.77245335446355\\
17.82	-9.77245419627177\\
17.83	-9.77245507484529\\
17.84	-9.77245598506044\\
17.85	-9.77245692117361\\
17.86	-9.7724578768779\\
17.87	-9.77245884536915\\
17.88	-9.77245981942053\\
17.89	-9.77246079146464\\
17.9	-9.77246175368192\\
17.91	-9.77246269809411\\
17.92	-9.77246361666145\\
17.93	-9.77246450138216\\
17.94	-9.77246534439275\\
17.95	-9.77246613806776\\
17.96	-9.77246687511725\\
17.97	-9.77246754868095\\
17.98	-9.77246815241731\\
17.99	-9.77246868058635\\
18	-9.77246912812503\\
18.01	-9.77246949071393\\
18.02	-9.77246976483443\\
18.03	-9.77246994781531\\
18.04	-9.77247003786825\\
18.05	-9.77247003411164\\
18.06	-9.77246993658226\\
18.07	-9.77246974623484\\
18.08	-9.77246946492936\\
18.09	-9.77246909540624\\
18.1	-9.77246864125003\\
18.11	-9.77246810684183\\
18.12	-9.77246749730141\\
18.13	-9.77246681841974\\
18.14	-9.77246607658298\\
18.15	-9.77246527868913\\
18.16	-9.77246443205846\\
18.17	-9.77246354433912\\
18.18	-9.77246262340932\\
18.19	-9.77246167727758\\
18.2	-9.77246071398238\\
18.21	-9.77245974149282\\
18.22	-9.77245876761168\\
18.23	-9.77245779988228\\
18.24	-9.7724568455006\\
18.25	-9.77245591123365\\
18.26	-9.77245500334561\\
18.27	-9.77245412753244\\
18.28	-9.77245328886611\\
18.29	-9.77245249174899\\
18.3	-9.77245173987916\\
18.31	-9.77245103622689\\
18.32	-9.77245038302267\\
18.33	-9.77244978175673\\
18.34	-9.77244923318996\\
18.35	-9.77244873737601\\
18.36	-9.77244829369389\\
18.37	-9.77244790089064\\
18.38	-9.77244755713316\\
18.39	-9.7724472600683\\
18.4	-9.77244700689008\\
18.41	-9.77244679441296\\
18.42	-9.77244661914988\\
18.43	-9.77244647739362\\
18.44	-9.7724463653003\\
18.45	-9.77244627897341\\
18.46	-9.77244621454708\\
18.47	-9.77244616826721\\
18.48	-9.77244613656901\\
18.49	-9.77244611614972\\
18.5	-9.77244610403531\\
18.51	-9.77244609764007\\
18.52	-9.772446094818\\
18.53	-9.77244609390531\\
18.54	-9.77244609375323\\
18.55	-9.77244609375052\\
18.56	-9.77244609383555\\
18.57	-9.77244609449744\\
18.58	-9.7724460967665\\
18.59	-9.77244610219396\\
18.6	-9.77244611282147\\
18.61	-9.77244613114069\\
18.62	-9.7724461600438\\
18.63	-9.77244620276569\\
18.64	-9.77244626281873\\
18.65	-9.77244634392133\\
18.66	-9.7724464499213\\
18.67	-9.77244658471548\\
18.68	-9.7724467521668\\
18.69	-9.77244695602028\\
18.7	-9.77244719981932\\
18.71	-9.7724474868237\\
18.72	-9.77244781993073\\
18.73	-9.77244820160076\\
18.74	-9.77244863378855\\
18.75	-9.77244911788142\\
18.76	-9.77244965464555\\
18.77	-9.7724502441812\\
18.78	-9.77245088588774\\
18.79	-9.77245157843917\\
18.8	-9.77245231977073\\
18.81	-9.77245310707674\\
18.82	-9.77245393682007\\
18.83	-9.77245480475302\\
18.84	-9.77245570594962\\
18.85	-9.77245663484878\\
18.86	-9.77245758530791\\
18.87	-9.77245855066616\\
18.88	-9.77245952381658\\
18.89	-9.772460497286\\
18.9	-9.77246146332166\\
18.91	-9.77246241398333\\
18.92	-9.77246334123946\\
18.93	-9.77246423706618\\
18.94	-9.77246509354744\\
18.95	-9.77246590297514\\
18.96	-9.77246665794745\\
18.97	-9.77246735146409\\
18.98	-9.77246797701711\\
18.99	-9.77246852867571\\
19	-9.77246900116406\\
19.01	-9.77246938993066\\
19.02	-9.77246969120858\\
19.03	-9.77246990206526\\
19.04	-9.77247002044152\\
19.05	-9.77247004517891\\
19.06	-9.77246997603512\\
19.07	-9.77246981368723\\
19.08	-9.77246955972267\\
19.09	-9.77246921661814\\
19.1	-9.77246878770664\\
19.11	-9.77246827713332\\
19.12	-9.77246768980058\\
19.13	-9.77246703130346\\
19.14	-9.77246630785607\\
19.15	-9.77246552621036\\
19.16	-9.77246469356829\\
19.17	-9.77246381748874\\
19.18	-9.77246290579064\\
19.19	-9.77246196645359\\
19.2	-9.77246100751753\\
19.21	-9.77246003698299\\
19.22	-9.77245906271324\\
19.23	-9.77245809233989\\
19.24	-9.77245713317317\\
19.25	-9.77245619211842\\
19.26	-9.77245527559966\\
19.27	-9.77245438949155\\
19.28	-9.77245353906065\\
19.29	-9.77245272891661\\
19.3	-9.77245196297415\\
19.31	-9.7724512444261\\
19.32	-9.77245057572794\\
19.33	-9.7724499585938\\
19.34	-9.772449394004\\
19.35	-9.77244888222371\\
19.36	-9.77244842283254\\
19.37	-9.77244801476415\\
19.38	-9.77244765635544\\
19.39	-9.77244734540419\\
19.4	-9.7724470792343\\
19.41	-9.77244685476732\\
19.42	-9.77244666859917\\
19.43	-9.77244651708067\\
19.44	-9.7724463964005\\
19.45	-9.77244630266925\\
19.46	-9.77244623200307\\
19.47	-9.77244618060558\\
19.48	-9.77244614484663\\
19.49	-9.77244612133662\\
19.5	-9.77244610699517\\
19.51	-9.77244609911295\\
19.52	-9.77244609540568\\
19.53	-9.77244609405937\\
19.54	-9.77244609376616\\
19.55	-9.77244609375\\
19.56	-9.77244609378193\\
19.57	-9.77244609418452\\
19.58	-9.77244609582556\\
19.59	-9.77244610010106\\
19.6	-9.77244610890772\\
19.61	-9.77244612460552\\
19.62	-9.7724461499709\\
19.63	-9.77244618814136\\
19.64	-9.77244624255239\\
19.65	-9.77244631686778\\
19.66	-9.77244641490444\\
19.67	-9.77244654055301\\
19.68	-9.77244669769555\\
19.69	-9.77244689012175\\
19.7	-9.77244712144491\\
19.71	-9.77244739501938\\
19.72	-9.77244771386049\\
19.73	-9.77244808056869\\
19.74	-9.77244849725896\\
19.75	-9.77244896549683\\
19.76	-9.77244948624205\\
19.77	-9.77245005980102\\
19.78	-9.77245068578879\\
19.79	-9.77245136310134\\
19.8	-9.77245208989875\\
19.81	-9.77245286359961\\
19.82	-9.77245368088699\\
19.83	-9.77245453772582\\
19.84	-9.77245542939181\\
19.85	-9.77245635051133\\
19.86	-9.77245729511192\\
19.87	-9.7724582566827\\
19.88	-9.77245922824394\\
19.89	-9.77246020242467\\
19.9	-9.77246117154746\\
19.91	-9.77246212771892\\
19.92	-9.77246306292486\\
19.93	-9.77246396912851\\
19.94	-9.77246483837052\\
19.95	-9.7724656628692\\
19.96	-9.77246643511959\\
19.97	-9.77246714798977\\
19.98	-9.7724677948132\\
19.99	-9.77246836947545\\
20	-9.77246886649425\\
20.01	-9.77246928109161\\
20.02	-9.77246960925684\\
20.03	-9.77246984779966\\
20.04	-9.77246999439254\\
20.05	-9.77247004760162\\
20.06	-9.77247000690578\\
20.07	-9.77246987270362\\
20.08	-9.77246964630812\\
20.09	-9.77246932992915\\
20.1	-9.77246892664415\\
20.11	-9.77246844035723\\
20.12	-9.77246787574749\\
20.13	-9.77246723820726\\
20.14	-9.77246653377111\\
20.15	-9.77246576903685\\
20.16	-9.77246495107956\\
20.17	-9.77246408735993\\
20.18	-9.77246318562845\\
20.19	-9.77246225382653\\
20.2	-9.77246129998643\\
20.21	-9.77246033213109\\
20.22	-9.77245935817557\\
20.23	-9.77245838583145\\
20.24	-9.77245742251561\\
20.25	-9.77245647526462\\
20.26	-9.77245555065608\\
20.27	-9.77245465473796\\
20.28	-9.77245379296689\\
20.29	-9.77245297015632\\
20.3	-9.77245219043511\\
20.31	-9.77245145721725\\
20.32	-9.77245077318282\\
20.33	-9.77245014027056\\
20.34	-9.7724495596819\\
20.35	-9.77244903189632\\
20.36	-9.77244855669765\\
20.37	-9.77244813321069\\
20.38	-9.77244775994765\\
20.39	-9.77244743486329\\
20.4	-9.77244715541795\\
20.41	-9.77244691864733\\
20.42	-9.77244672123777\\
20.43	-9.77244655960574\\
20.44	-9.77244642998027\\
20.45	-9.77244632848674\\
20.46	-9.77244625123086\\
20.47	-9.77244619438121\\
20.48	-9.77244615424913\\
20.49	-9.77244612736454\\
20.5	-9.77244611054638\\
20.51	-9.7724461009667\\
20.52	-9.77244609620706\\
20.53	-9.77244609430661\\
20.54	-9.77244609380076\\
20.55	-9.77244609375005\\
20.56	-9.7724460937586\\
20.57	-9.77244609398186\\
20.58	-9.77244609512354\\
20.59	-9.77244609842185\\
20.6	-9.77244610562514\\
20.61	-9.77244611895737\\
20.62	-9.77244614107414\\
20.63	-9.77244617500976\\
20.64	-9.77244622411641\\
20.65	-9.77244629199635\\
20.66	-9.77244638242825\\
20.67	-9.77244649928899\\
20.68	-9.77244664647203\\
20.69	-9.77244682780399\\
20.7	-9.77244704696048\\
20.71	-9.772447307383\\
20.72	-9.77244761219796\\
20.73	-9.77244796413937\\
20.74	-9.77244836547658\\
20.75	-9.77244881794808\\
20.76	-9.77244932270284\\
20.77	-9.7724498802499\\
20.78	-9.77245049041745\\
20.79	-9.77245115232187\\
20.8	-9.77245186434757\\
20.81	-9.77245262413796\\
20.82	-9.7724534285979\\
20.83	-9.77245427390764\\
20.84	-9.77245515554824\\
20.85	-9.77245606833816\\
20.86	-9.77245700648065\\
20.87	-9.77245796362121\\
20.88	-9.77245893291446\\
20.89	-9.77245990709947\\
20.9	-9.77246087858243\\
20.91	-9.7724618395256\\
20.92	-9.77246278194118\\
20.93	-9.77246369778871\\
20.94	-9.77246457907476\\
20.95	-9.77246541795321\\
20.96	-9.77246620682481\\
20.97	-9.77246693843454\\
20.98	-9.77246760596523\\
20.99	-9.77246820312618\\
21	-9.77246872423536\\
21.01	-9.77246916429408\\
21.02	-9.77246951905278\\
21.03	-9.77246978506738\\
21.04	-9.77246995974481\\
21.05	-9.77247004137758\\
21.06	-9.77247002916641\\
21.07	-9.77246992323091\\
21.08	-9.77246972460799\\
21.09	-9.77246943523803\\
21.1	-9.77246905793911\\
21.11	-9.77246859636956\\
21.12	-9.77246805497947\\
21.13	-9.77246743895193\\
21.14	-9.7724667541347\\
21.15	-9.77246600696352\\
21.16	-9.77246520437814\\
21.17	-9.77246435373226\\
21.18	-9.77246346269879\\
21.19	-9.77246253917177\\
21.2	-9.77246159116652\\
21.21	-9.77246062671932\\
21.22	-9.77245965378821\\
21.23	-9.77245868015638\\
21.24	-9.77245771333941\\
21.25	-9.77245676049788\\
21.26	-9.77245582835648\\
21.27	-9.77245492313078\\
21.28	-9.77245405046273\\
21.29	-9.77245321536573\\
21.3	-9.77245242218004\\
21.31	-9.77245167453901\\
21.32	-9.77245097534667\\
21.33	-9.77245032676672\\
21.34	-9.77244973022315\\
21.35	-9.77244918641213\\
21.36	-9.7724486953251\\
21.37	-9.7724482562823\\
21.38	-9.77244786797631\\
21.39	-9.77244752852467\\
21.4	-9.7724472355306\\
21.41	-9.77244698615089\\
21.42	-9.77244677716961\\
21.43	-9.77244660507651\\
21.44	-9.77244646614864\\
21.45	-9.77244635653393\\
21.46	-9.77244627233525\\
21.47	-9.77244620969351\\
21.48	-9.77244616486854\\
21.49	-9.77244613431623\\
21.5	-9.77244611476076\\
21.51	-9.77244610326075\\
21.52	-9.77244609726806\\
21.53	-9.7724460946785\\
21.54	-9.77244609387344\\
21.55	-9.77244609375171\\
21.56	-9.77244609375121\\
21.57	-9.77244609385997\\
21.58	-9.77244609461639\\
21.59	-9.77244609709862\\
21.6	-9.77244610290347\\
21.61	-9.77244611411485\\
21.62	-9.77244613326263\\
21.63	-9.77244616327232\\
21.64	-9.77244620740656\\
21.65	-9.77244626919927\\
21.66	-9.7724463523837\\
21.67	-9.77244646081543\\
21.68	-9.77244659839166\\
21.69	-9.77244676896815\\
21.7	-9.77244697627519\\
21.71	-9.77244722383398\\
21.72	-9.77244751487484\\
21.73	-9.77244785225874\\
21.74	-9.77244823840325\\
21.75	-9.77244867521446\\
21.76	-9.77244916402588\\
21.77	-9.77244970554544\\
21.78	-9.77245029981161\\
21.79	-9.7724509461593\\
21.8	-9.77245164319644\\
21.81	-9.77245238879147\\
21.82	-9.77245318007235\\
21.83	-9.77245401343696\\
21.84	-9.77245488457507\\
21.85	-9.77245578850162\\
21.86	-9.77245671960086\\
21.87	-9.77245767168081\\
21.88	-9.7724586380374\\
21.89	-9.77245961152735\\
21.9	-9.77246058464866\\
21.91	-9.7724615496279\\
21.92	-9.7724624985126\\
21.93	-9.77246342326783\\
21.94	-9.7724643158753\\
21.95	-9.77246516843362\\
21.96	-9.77246597325824\\
21.97	-9.77246672297963\\
21.98	-9.77246741063818\\
21.99	-9.7724680297745\\
22	-9.77246857451368\\
22.01	-9.77246903964235\\
22.02	-9.77246942067731\\
22.03	-9.77246971392484\\
22.04	-9.77246991652956\\
22.05	-9.77247002651241\\
22.06	-9.77247004279692\\
22.07	-9.77246996522358\\
22.08	-9.77246979455197\\
22.09	-9.77246953245058\\
22.1	-9.77246918147468\\
22.11	-9.77246874503236\\
22.12	-9.77246822733931\\
22.13	-9.77246763336306\\
22.14	-9.77246696875749\\
22.15	-9.77246623978855\\
22.16	-9.77246545325238\\
22.17	-9.77246461638693\\
22.18	-9.7724637367785\\
22.19	-9.77246282226459\\
22.2	-9.77246188083432\\
22.21	-9.77246092052816\\
22.22	-9.77245994933825\\
22.23	-9.77245897511083\\
22.24	-9.77245800545215\\
22.25	-9.77245704763935\\
22.26	-9.77245610853744\\
22.27	-9.77245519452367\\
22.28	-9.77245431142024\\
22.29	-9.7724534644364\\
22.3	-9.77245265812069\\
22.31	-9.77245189632378\\
22.32	-9.7724511821726\\
22.33	-9.77245051805591\\
22.34	-9.77244990562131\\
22.35	-9.77244934578383\\
22.36	-9.7724488387456\\
22.37	-9.77244838402629\\
22.38	-9.77244798050372\\
22.39	-9.77244762646379\\
22.4	-9.77244731965887\\
22.41	-9.77244705737362\\
22.42	-9.77244683649707\\
22.43	-9.77244665359976\\
22.44	-9.77244650501447\\
22.45	-9.77244638691942\\
22.46	-9.7724462954223\\
22.47	-9.77244622664379\\
22.48	-9.77244617679932\\
22.49	-9.77244614227742\\
22.5	-9.77244611971359\\
22.51	-9.77244610605844\\
22.52	-9.7724460986388\\
22.53	-9.77244609521098\\
22.54	-9.77244609400524\\
22.55	-9.77244609376068\\
22.56	-9.77244609375002\\
22.57	-9.77244609379391\\
22.58	-9.77244609426441\\
22.59	-9.77244609607771\\
22.6	-9.77244610067615\\
22.61	-9.77244610999981\\
22.62	-9.77244612644818\\
22.63	-9.77244615283261\\
22.64	-9.7724461923201\\
22.65	-9.77244624836962\\
22.66	-9.7724463246619\\
22.67	-9.7724464250238\\
22.68	-9.77244655334857\\
22.69	-9.77244671351341\\
22.7	-9.77244690929556\\
22.71	-9.77244714428838\\
22.72	-9.77244742181892\\
22.73	-9.77244774486822\\
22.74	-9.77244811599584\\
22.75	-9.77244853726982\\
22.76	-9.77244901020334\\
22.77	-9.77244953569918\\
22.78	-9.77245011400294\\
22.79	-9.7724507446659\\
22.8	-9.77245142651836\\
22.81	-9.7724521576537\\
22.82	-9.77245293542393\\
22.83	-9.77245375644657\\
22.84	-9.77245461662321\\
22.85	-9.7724555111693\\
22.86	-9.7724564346551\\
22.87	-9.77245738105708\\
22.88	-9.7724583438192\\
22.89	-9.7724593159232\\
22.9	-9.77246028996698\\
22.91	-9.77246125824989\\
22.92	-9.77246221286371\\
22.93	-9.77246314578817\\
22.94	-9.77246404898934\\
22.95	-9.77246491451978\\
22.96	-9.77246573461867\\
22.97	-9.77246650181073\\
22.98	-9.77246720900222\\
22.99	-9.77246784957282\\
23	-9.77246841746182\\
23.01	-9.77246890724752\\
23.02	-9.77246931421854\\
23.03	-9.77246963443602\\
23.04	-9.77246986478574\\
23.05	-9.77247000301951\\
23.06	-9.77247004778501\\
23.07	-9.7724699986438\\
23.08	-9.77246985607716\\
23.09	-9.77246962147969\\
23.1	-9.77246929714072\\
23.11	-9.77246888621393\\
23.12	-9.77246839267543\\
23.13	-9.77246782127121\\
23.14	-9.77246717745441\\
23.15	-9.77246646731365\\
23.16	-9.77246569749331\\
23.17	-9.77246487510699\\
23.18	-9.77246400764546\\
23.19	-9.77246310288044\\
23.2	-9.77246216876565\\
23.21	-9.77246121333658\\
23.22	-9.77246024461048\\
23.23	-9.77245927048795\\
23.24	-9.77245829865772\\
23.25	-9.77245733650583\\
23.26	-9.77245639103067\\
23.27	-9.77245546876494\\
23.28	-9.77245457570577\\
23.29	-9.77245371725388\\
23.3	-9.77245289816263\\
23.31	-9.77245212249765\\
23.32	-9.77245139360746\\
23.33	-9.77245071410556\\
23.34	-9.77245008586399\\
23.35	-9.77244951001844\\
23.36	-9.77244898698452\\
23.37	-9.7724485164851\\
23.38	-9.77244809758782\\
23.39	-9.77244772875234\\
23.4	-9.77244740788622\\
23.41	-9.77244713240865\\
23.42	-9.77244689932072\\
23.43	-9.77244670528118\\
23.44	-9.77244654668624\\
23.45	-9.77244641975214\\
23.46	-9.77244632059906\\
23.47	-9.77244624533501\\
23.48	-9.77244619013823\\
23.49	-9.77244615133668\\
23.5	-9.77244612548349\\
23.51	-9.77244610942687\\
23.52	-9.77244610037352\\
23.53	-9.77244609594436\\
23.54	-9.77244609422175\\
23.55	-9.77244609378735\\
23.56	-9.77244609375001\\
23.57	-9.77244609376331\\
23.58	-9.77244609403231\\
23.59	-9.77244609530959\\
23.6	-9.7724460988804\\
23.61	-9.77244610653744\\
23.62	-9.77244612054549\\
23.63	-9.7724461435965\\
23.64	-9.77244617875598\\
23.65	-9.77244622940152\\
23.66	-9.77244629915432\\
23.67	-9.77244639180518\\
23.68	-9.77244651123582\\
23.69	-9.77244666133715\\
23.7	-9.77244684592561\\
23.71	-9.77244706865916\\
23.72	-9.77244733295419\\
23.73	-9.77244764190488\\
23.74	-9.77244799820634\\
23.75	-9.77244840408271\\
23.76	-9.77244886122174\\
23.77	-9.77244937071671\\
23.78	-9.77244993301695\\
23.79	-9.77245054788767\\
23.8	-9.77245121438005\\
23.81	-9.77245193081201\\
23.82	-9.77245269476022\\
23.83	-9.77245350306359\\
23.84	-9.77245435183823\\
23.85	-9.77245523650391\\
23.86	-9.77245615182161\\
23.87	-9.77245709194191\\
23.88	-9.77245805046323\\
23.89	-9.77245902049962\\
23.9	-9.77245999475669\\
23.91	-9.77246096561495\\
23.92	-9.77246192521925\\
23.93	-9.77246286557302\\
23.94	-9.77246377863594\\
23.95	-9.77246465642372\\
23.96	-9.77246549110841\\
23.97	-9.77246627511778\\
23.98	-9.77246700123247\\
23.99	-9.77246766267913\\
24	-9.77246825321859\\
24.01	-9.7724687672274\\
24.02	-9.77246919977169\\
24.03	-9.77246954667228\\
24.04	-9.77246980455994\\
24.05	-9.77246997092008\\
24.06	-9.77247004412619\\
24.07	-9.77247002346143\\
24.08	-9.77246990912822\\
24.09	-9.77246970224553\\
24.1	-9.77246940483395\\
24.11	-9.77246901978889\\
24.12	-9.77246855084207\\
24.13	-9.77246800251213\\
24.14	-9.77246738004489\\
24.15	-9.77246668934427\\
24.16	-9.77246593689494\\
24.17	-9.77246512967766\\
24.18	-9.77246427507879\\
24.19	-9.77246338079519\\
24.2	-9.77246245473595\\
24.21	-9.77246150492234\\
24.22	-9.77246053938765\\
24.23	-9.7724595660781\\
24.24	-9.77245859275651\\
24.25	-9.77245762691002\\
24.26	-9.77245667566317\\
24.27	-9.77245574569771\\
24.28	-9.77245484318009\\
24.29	-9.77245397369778\\
24.3	-9.7724531422053\\
24.31	-9.77245235298047\\
24.32	-9.77245160959178\\
24.33	-9.77245091487688\\
24.34	-9.77245027093273\\
24.35	-9.77244967911715\\
24.36	-9.77244914006179\\
24.37	-9.77244865369614\\
24.38	-9.77244821928204\\
24.39	-9.77244783545803\\
24.4	-9.77244750029276\\
24.41	-9.7724472113464\\
24.42	-9.7724469657391\\
24.43	-9.77244676022518\\
24.44	-9.77244659127185\\
24.45	-9.77244645514111\\
24.46	-9.77244634797339\\
24.47	-9.77244626587161\\
24.48	-9.77244620498412\\
24.49	-9.77244616158529\\
24.5	-9.77244613215229\\
24.51	-9.77244611343679\\
24.52	-9.7724461025305\\
24.53	-9.77244609692329\\
24.54	-9.77244609455311\\
24.55	-9.77244609384675\\
24.56	-9.7724460937508\\
24.57	-9.77244609375241\\
24.58	-9.77244609388927\\
24.59	-9.77244609474896\\
24.6	-9.77244609745734\\
24.61	-9.77244610365647\\
24.62	-9.77244611547224\\
24.63	-9.77244613547232\\
24.64	-9.77244616661507\\
24.65	-9.77244621219032\\
24.66	-9.77244627575299\\
24.67	-9.77244636105054\\
24.68	-9.77244647194565\\
24.69	-9.77244661233521\\
24.7	-9.77244678606714\\
24.71	-9.77244699685632\\
24.72	-9.77244724820107\\
24.73	-9.77244754330165\\
24.74	-9.77244788498205\\
24.75	-9.7724482756165\\
24.76	-9.77244871706197\\
24.77	-9.7724492105977\\
24.78	-9.77244975687303\\
24.79	-9.77245035586434\\
24.8	-9.77245100684192\\
24.81	-9.7724517083475\\
24.82	-9.77245245818274\\
24.83	-9.7724532534093\\
24.84	-9.77245409036025\\
24.85	-9.77245496466311\\
24.86	-9.7724558712741\\
24.87	-9.77245680452322\\
24.88	-9.77245775816963\\
24.89	-9.77245872546666\\
24.9	-9.7724596992353\\
24.91	-9.7724606719455\\
24.92	-9.77246163580383\\
24.93	-9.77246258284641\\
24.94	-9.77246350503573\\
24.95	-9.7724643943599\\
24.96	-9.77246524293296\\
24.97	-9.77246604309475\\
24.98	-9.77246678750878\\
24.99	-9.77246746925687\\
25	-9.77246808192886\\
25.01	-9.77246861970634\\
25.02	-9.77246907743898\\
25.03	-9.77246945071236\\
25.04	-9.77246973590635\\
25.05	-9.77246993024306\\
25.06	-9.77247003182376\\
25.07	-9.77247003965409\\
25.08	-9.77246995365738\\
25.09	-9.77246977467556\\
25.1	-9.77246950445805\\
25.11	-9.77246914563843\\
25.12	-9.77246870169948\\
25.13	-9.77246817692698\\
25.14	-9.77246757635306\\
25.15	-9.77246690568984\\
25.16	-9.77246617125446\\
25.17	-9.77246537988651\\
25.18	-9.77246453885919\\
25.19	-9.77246365578545\\
25.2	-9.77246273852049\\
25.21	-9.77246179506219\\
25.22	-9.77246083345073\\
25.23	-9.77245986166908\\
25.24	-9.77245888754564\\
25.25	-9.77245791866063\\
25.26	-9.7724569622574\\
25.27	-9.77245602516004\\
25.28	-9.77245511369845\\
25.29	-9.77245423364191\\
25.3	-9.77245339014202\\
25.31	-9.77245258768584\\
25.32	-9.77245183005981\\
25.33	-9.77245112032486\\
25.34	-9.77245046080297\\
25.35	-9.77244985307532\\
25.36	-9.77244929799178\\
25.37	-9.77244879569167\\
25.38	-9.7724483456351\\
25.39	-9.77244794664444\\
25.4	-9.77244759695505\\
25.41	-9.77244729427441\\
25.42	-9.77244703584854\\
25.43	-9.7724468185346\\
25.44	-9.77244663887836\\
25.45	-9.77244649319527\\
25.46	-9.77244637765377\\
25.47	-9.77244628835929\\
25.48	-9.77244622143775\\
25.49	-9.772446173117\\
25.5	-9.77244613980485\\
25.51	-9.77244611816247\\
25.52	-9.77244610517193\\
25.53	-9.77244609819665\\
25.54	-9.77244609503393\\
25.55	-9.77244609395854\\
25.56	-9.77244609375673\\
25.57	-9.77244609375011\\
25.58	-9.77244609380898\\
25.59	-9.77244609435482\\
25.6	-9.77244609635208\\
25.61	-9.7724461012892\\
25.62	-9.77244611114932\\
25.63	-9.77244612837108\\
25.64	-9.77244615580025\\
25.65	-9.77244619663283\\
25.66	-9.77244625435071\\
25.67	-9.77244633265094\\
25.68	-9.77244643536965\\
25.69	-9.77244656640208\\
25.7	-9.77244672961988\\
25.71	-9.77244692878714\\
25.72	-9.77244716747662\\
25.73	-9.77244744898743\\
25.74	-9.7724477762657\\
25.75	-9.77244815182953\\
25.76	-9.77244857769951\\
25.77	-9.77244905533603\\
25.78	-9.77244958558451\\
25.79	-9.77245016862943\\
25.8	-9.77245080395812\\
25.81	-9.77245149033504\\
25.82	-9.77245222578687\\
25.83	-9.77245300759905\\
25.84	-9.77245383232375\\
25.85	-9.77245469579941\\
25.86	-9.77245559318159\\
25.87	-9.77245651898484\\
25.88	-9.77245746713507\\
25.89	-9.77245843103161\\
25.9	-9.77245940361837\\
25.91	-9.77246037746277\\
25.92	-9.77246134484168\\
25.93	-9.77246229783284\\
25.94	-9.77246322841067\\
25.95	-9.77246412854493\\
25.96	-9.77246499030087\\
25.97	-9.77246580593936\\
25.98	-9.77246656801557\\
25.99	-9.77246726947469\\
26	-9.77246790374331\\
26.01	-9.7724684648151\\
26.02	-9.77246894732951\\
26.03	-9.77246934664225\\
26.04	-9.77246965888672\\
26.05	-9.77246988102511\\
26.06	-9.77247001088881\\
26.07	-9.77247004720718\\
26.08	-9.7724699896245\\
26.09	-9.77246983870469\\
26.1	-9.77246959592382\\
26.11	-9.77246926365042\\
26.12	-9.77246884511406\\
26.13	-9.77246834436248\\
26.14	-9.77246776620799\\
26.15	-9.77246711616399\\
26.16	-9.7724664003725\\
26.17	-9.77246562552376\\
26.18	-9.77246479876915\\
26.19	-9.77246392762872\\
26.2	-9.77246301989465\\
26.21	-9.77246208353214\\
26.22	-9.77246112657911\\
26.23	-9.77246015704631\\
26.24	-9.77245918281911\\
26.25	-9.77245821156264\\
26.26	-9.77245725063144\\
26.27	-9.77245630698506\\
26.28	-9.77245538711075\\
26.29	-9.77245449695431\\
26.3	-9.77245364186013\\
26.31	-9.77245282652117\\
26.32	-9.77245205493954\\
26.33	-9.77245133039819\\
26.34	-9.77245065544398\\
26.35	-9.7724500318823\\
26.36	-9.77244946078319\\
26.37	-9.77244894249866\\
26.38	-9.77244847669091\\
26.39	-9.77244806237081\\
26.4	-9.77244769794589\\
26.41	-9.77244738127707\\
26.42	-9.77244710974292\\
26.43	-9.77244688031057\\
26.44	-9.77244668961181\\
26.45	-9.77244653402325\\
26.46	-9.77244640974903\\
26.47	-9.77244631290482\\
26.48	-9.77244623960161\\
26.49	-9.7724461860279\\
26.5	-9.77244614852892\\
26.51	-9.77244612368155\\
26.52	-9.7724461083638\\
26.53	-9.77244609981748\\
26.54	-9.77244609570324\\
26.55	-9.77244609414698\\
26.56	-9.77244609377679\\
26.57	-9.77244609375\\
26.58	-9.77244609376971\\
26.59	-9.77244609409058\\
26.6	-9.77244609551388\\
26.61	-9.77244609937173\\
26.62	-9.77244610750091\\
26.63	-9.7724461222066\\
26.64	-9.77244614621671\\
26.65	-9.77244618262748\\
26.66	-9.77244623484128\\
26.67	-9.7724463064977\\
26.68	-9.772446401399\\
26.69	-9.77244652343113\\
26.7	-9.77244667648171\\
26.71	-9.77244686435637\\
26.72	-9.77244709069469\\
26.73	-9.77244735888733\\
26.74	-9.77244767199565\\
26.75	-9.7724480326752\\
26.76	-9.77244844310445\\
26.77	-9.77244890491989\\
26.78	-9.77244941915874\\
26.79	-9.77244998621027\\
26.8	-9.77245060577649\\
26.81	-9.77245127684321\\
26.82	-9.7724519976618\\
26.83	-9.77245276574223\\
26.84	-9.77245357785755\\
26.85	-9.77245443005997\\
26.86	-9.77245531770825\\
26.87	-9.77245623550632\\
26.88	-9.77245717755252\\
26.89	-9.77245813739879\\
26.9	-9.77245910811918\\
26.91	-9.77246008238653\\
26.92	-9.77246105255641\\
26.93	-9.77246201075701\\
26.94	-9.77246294898378\\
26.95	-9.77246385919736\\
26.96	-9.77246473342339\\
26.97	-9.77246556385291\\
26.98	-9.77246634294153\\
26.99	-9.77246706350625\\
27	-9.7724677188183\\
27.01	-9.77246830269068\\
27.02	-9.7724688095591\\
27.03	-9.77246923455508\\
27.04	-9.77246957357023\\
27.05	-9.77246982331056\\
27.06	-9.77246998134023\\
27.07	-9.77247004611387\\
27.08	-9.77247001699716\\
27.09	-9.77246989427532\\
27.1	-9.77246967914923\\
27.11	-9.77246937371951\\
27.12	-9.77246898095857\\
27.13	-9.77246850467113\\
27.14	-9.77246794944387\\
27.15	-9.77246732058479\\
27.16	-9.77246662405338\\
27.17	-9.7724658663825\\
27.18	-9.7724650545932\\
27.19	-9.77246419610375\\
27.2	-9.77246329863416\\
27.21	-9.77246237010771\\
27.22	-9.77246141855088\\
27.23	-9.77246045199314\\
27.24	-9.77245947836809\\
27.25	-9.77245850541744\\
27.26	-9.77245754059918\\
27.27	-9.77245659100118\\
27.28	-9.77245566326164\\
27.29	-9.77245476349739\\
27.3	-9.77245389724099\\
27.31	-9.77245306938769\\
27.32	-9.77245228415265\\
27.33	-9.77245154503928\\
27.34	-9.77245085481882\\
27.35	-9.77245021552146\\
27.36	-9.77244962843897\\
27.37	-9.77244909413866\\
27.38	-9.77244861248839\\
27.39	-9.77244818269192\\
27.4	-9.77244780333421\\
27.41	-9.77244747243551\\
27.42	-9.77244718751349\\
27.43	-9.77244694565226\\
27.44	-9.77244674357702\\
27.45	-9.77244657773312\\
27.46	-9.77244644436819\\
27.47	-9.77244633961585\\
27.48	-9.77244625957972\\
27.49	-9.77244620041625\\
27.5	-9.77244615841498\\
27.51	-9.77244613007493\\
27.52	-9.7724461121758\\
27.53	-9.77244610184287\\
27.54	-9.77244609660441\\
27.55	-9.77244609444085\\
27.56	-9.77244609382463\\
27.57	-9.77244609375032\\
27.58	-9.77244609375433\\
27.59	-9.77244609392407\\
27.6	-9.77244609489616\\
27.61	-9.772446097844\\
27.62	-9.77244610445466\\
27.63	-9.77244611689566\\
27.64	-9.77244613777204\\
27.65	-9.77244617007453\\
27.66	-9.77244621711965\\
27.67	-9.77244628248268\\
27.68	-9.77244636992469\\
27.69	-9.7724464833148\\
27.7	-9.77244662654891\\
27.71	-9.77244680346643\\
27.72	-9.77244701776613\\
27.73	-9.77244727292283\\
27.74	-9.77244757210603\\
27.75	-9.7724479181022\\
27.76	-9.77244831324169\\
27.77	-9.77244875933182\\
27.78	-9.77244925759714\\
27.79	-9.77244980862807\\
27.8	-9.77245041233862\\
27.81	-9.7724510679343\\
27.82	-9.77245177389051\\
27.83	-9.77245252794216\\
27.84	-9.77245332708467\\
27.85	-9.77245416758653\\
27.86	-9.77245504501326\\
27.87	-9.77245595426272\\
27.88	-9.77245688961112\\
27.89	-9.77245784476932\\
27.9	-9.77245881294857\\
27.91	-9.77245978693486\\
27.92	-9.77246075917076\\
27.93	-9.77246172184361\\
27.94	-9.77246266697891\\
27.95	-9.77246358653735\\
27.96	-9.77246447251428\\
27.97	-9.77246531703998\\
27.98	-9.77246611247944\\
27.99	-9.77246685153003\\
28	-9.77246752731568\\
28.01	-9.77246813347615\\
28.02	-9.77246866425018\\
28.03	-9.77246911455099\\
28.04	-9.77246948003343\\
28.05	-9.77246975715136\\
28.06	-9.77246994320467\\
28.07	-9.77247003637515\\
28.08	-9.77247003575069\\
28.09	-9.77246994133742\\
28.1	-9.77246975405959\\
28.11	-9.77246947574731\\
28.12	-9.77246910911221\\
28.13	-9.77246865771138\\
28.14	-9.77246812590023\\
28.15	-9.77246751877495\\
28.16	-9.77246684210531\\
28.17	-9.77246610225895\\
28.18	-9.77246530611819\\
28.19	-9.77246446099074\\
28.2	-9.77246357451535\\
28.21	-9.77246265456421\\
28.22	-9.77246170914307\\
28.23	-9.77246074629104\\
28.24	-9.7724597739811\\
28.25	-9.7724588000231\\
28.26	-9.7724578319705\\
28.27	-9.77245687703215\\
28.28	-9.77245594199065\\
28.29	-9.77245503312802\\
28.3	-9.77245415616013\\
28.31	-9.77245331618055\\
28.32	-9.77245251761459\\
28.33	-9.77245176418424\\
28.34	-9.77245105888432\\
28.35	-9.77245040397004\\
28.36	-9.77244980095619\\
28.37	-9.77244925062771\\
28.38	-9.77244875306133\\
28.39	-9.77244830765793\\
28.4	-9.77244791318483\\
28.41	-9.77244756782736\\
28.42	-9.77244726924865\\
28.43	-9.77244701465666\\
28.44	-9.7724468008773\\
28.45	-9.77244662443223\\
28.46	-9.77244648162026\\
28.47	-9.77244636860069\\
28.48	-9.77244628147742\\
28.49	-9.77244621638225\\
28.5	-9.77244616955607\\
28.51	-9.77244613742658\\
28.52	-9.77244611668117\\
28.53	-9.77244610433385\\
28.54	-9.77244609778507\\
28.55	-9.77244609487346\\
28.56	-9.77244609391856\\
28.57	-9.77244609375399\\
28.58	-9.77244609375037\\
28.59	-9.77244609382763\\
28.6	-9.77244609445664\\
28.61	-9.77244609664993\\
28.62	-9.77244610194182\\
28.63	-9.77244611235818\\
28.64	-9.77244613037642\\
28.65	-9.77244615887627\\
28.66	-9.77244620108218\\
28.67	-9.77244626049842\\
28.68	-9.77244634083771\\
28.69	-9.77244644594485\\
28.7	-9.77244657971636\\
28.71	-9.77244674601765\\
28.72	-9.77244694859902\\
28.73	-9.77244719101193\\
28.74	-9.77244747652693\\
28.75	-9.77244780805461\\
28.76	-9.77244818807105\\
28.77	-9.77244861854889\\
28.78	-9.7724491008953\\
28.79	-9.77244963589794\\
28.8	-9.77245022367988\\
28.81	-9.77245086366431\\
28.82	-9.77245155454971\\
28.83	-9.77245229429603\\
28.84	-9.77245308012223\\
28.85	-9.77245390851526\\
28.86	-9.77245477525067\\
28.87	-9.77245567542447\\
28.88	-9.77245660349594\\
28.89	-9.7724575533409\\
28.9	-9.77245851831467\\
28.91	-9.77245949132392\\
28.92	-9.77246046490633\\
28.93	-9.772461431317\\
28.94	-9.77246238262043\\
28.95	-9.77246331078653\\
28.96	-9.77246420778954\\
28.97	-9.77246506570823\\
28.98	-9.77246587682596\\
28.99	-9.77246663372912\\
29	-9.77246732940256\\
29.01	-9.7724679573205\\
29.02	-9.77246851153162\\
29.03	-9.77246898673707\\
29.04	-9.77246937836019\\
29.05	-9.77246968260702\\
29.06	-9.77246989651649\\
29.07	-9.77247001799981\\
29.08	-9.77247004586816\\
29.09	-9.77246997984859\\
29.1	-9.77246982058758\\
29.11	-9.7724695696425\\
29.12	-9.77246922946087\\
29.13	-9.77246880334778\\
29.14	-9.77246829542213\\
29.15	-9.77246771056207\\
29.16	-9.77246705434067\\
29.17	-9.7724663329527\\
29.18	-9.77246555313354\\
29.19	-9.77246472207161\\
29.2	-9.77246384731547\\
29.21	-9.77246293667694\\
29.22	-9.77246199813187\\
29.23	-9.77246103971983\\
29.24	-9.77246006944426\\
29.25	-9.77245909517453\\
29.26	-9.77245812455145\\
29.27	-9.77245716489736\\
29.28	-9.77245622313237\\
29.29	-9.77245530569768\\
29.3	-9.77245441848728\\
29.31	-9.77245356678885\\
29.32	-9.7724527552346\\
29.33	-9.77245198776293\\
29.34	-9.77245126759105\\
29.35	-9.77245059719915\\
29.36	-9.77244997832602\\
29.37	-9.77244941197619\\
29.38	-9.77244889843828\\
29.39	-9.77244843731416\\
29.4	-9.77244802755829\\
29.41	-9.77244766752657\\
29.42	-9.77244735503376\\
29.43	-9.77244708741842\\
29.44	-9.77244686161431\\
29.45	-9.77244667422694\\
29.46	-9.77244652161395\\
29.47	-9.77244639996809\\
29.48	-9.7724463054012\\
29.49	-9.7724462340279\\
29.5	-9.77244618204762\\
29.51	-9.77244614582342\\
29.52	-9.77244612195655\\
29.53	-9.7724461073553\\
29.54	-9.77244609929703\\
29.55	-9.77244609548251\\
29.56	-9.77244609408145\\
29.57	-9.77244609376864\\
29.58	-9.77244609375\\
29.59	-9.77244609377818\\
29.6	-9.77244609415742\\
29.61	-9.77244609573755\\
29.62	-9.77244609989734\\
29.63	-9.77244610851735\\
29.64	-9.77244612394283\\
29.65	-9.77244614893717\\
29.66	-9.77244618662681\\
29.67	-9.7724462404384\\
29.68	-9.7724463140293\\
29.69	-9.77244641121258\\
29.7	-9.77244653587772\\
29.71	-9.77244669190844\\
29.72	-9.77244688309883\\
29.73	-9.77244711306945\\
29.74	-9.77244738518455\\
29.75	-9.77244770247205\\
29.76	-9.77244806754738\\
29.77	-9.7724484825428\\
29.78	-9.77244894904308\\
29.79	-9.772449468029\\
29.8	-9.77245003982942\\
29.81	-9.77245066408295\\
29.82	-9.77245133970985\\
29.83	-9.7724520648949\\
29.84	-9.77245283708139\\
29.85	-9.77245365297667\\
29.86	-9.77245450856924\\
29.87	-9.7724553991572\\
29.88	-9.7724563193878\\
29.89	-9.77245726330763\\
29.9	-9.7724582244227\\
29.91	-9.77245919576769\\
29.92	-9.77246016998334\\
29.93	-9.77246113940101\\
29.94	-9.77246209613302\\
29.95	-9.77246303216761\\
29.96	-9.77246393946717\\
29.97	-9.77246481006815\\
29.98	-9.77246563618133\\
29.99	-9.77246641029094\\
};
\addlegendentry{ANA}

\end{axis}
\end{tikzpicture}%
    \caption{Course of the system´s total energy over time as a function of the integration scheme chosen. The studied time interval was $30\,s$ at a time step of $0.01\,s$ for all integrators. EE: explicit Euler. EI: implicit Euler. MP: middle-point rule. ANA: analytical solution}
    \label{fig: EnergyPlot}
\end{figure}

For the case of the midpoint rule, the course of energy over time was further analysed in the form of the instantaneous energy difference, which was defined as:

\begin{equation}
    \Delta E (t) = E(t) - E(t-\Delta t) \quad ,\,t \in [t_o + \Delta t, T].
\end{equation}


The results are shown in Figure \ref{fig: EnergyDiffPlot}. As can be seen, the instantaneous energy difference oscillates in a regular fashion throughout the course of the simulation. The amplitude of this oscillation is nearly $10^{-6}\,J$ and the pattern has a period of about $2\,s$.


\begin{figure}[h]
    \centering
    \setlength{\figH}{0.3\textheight}
    \setlength{\figW}{0.6\textwidth}
    % This file was created by matlab2tikz.
%
\definecolor{mycolor1}{rgb}{0.00000,0.44700,0.74100}%
%
\begin{tikzpicture}

\begin{axis}[%
width=0.951\figW,
height=\figH,
at={(0\figW,0\figH)},
scale only axis,
xmin=0,
xmax=30,
xlabel style={font=\color{white!15!black}},
xlabel={Time [s]},
ymin=-2e-06,
ymax=2e-06,
ylabel style={font=\color{white!15!black}},
ylabel={Instantaneous Energy difference [J]},
axis background/.style={fill=white}
]
\addplot [color=mycolor1, forget plot]
  table[row sep=crcr]{%
0.01	5.599e-08\\
0.03	2.416e-07\\
0.04	3.314e-07\\
0.05	4.181e-07\\
0.06	5.008e-07\\
0.07	5.788e-07\\
0.08	6.514e-07\\
0.09	7.179e-07\\
0.1	7.778e-07\\
0.11	8.305e-07\\
0.12	8.757e-07\\
0.13	9.132e-07\\
0.14	9.428e-07\\
0.15	9.643e-07\\
0.16	9.777e-07\\
0.17	9.833e-07\\
0.18	9.811e-07\\
0.19	9.716e-07\\
0.2	9.551e-07\\
0.21	9.32e-07\\
0.22	9.029e-07\\
0.23	8.684e-07\\
0.24	8.291e-07\\
0.25	7.857e-07\\
0.26	7.39e-07\\
0.27	6.896e-07\\
0.29	5.86e-07\\
0.32	4.289e-07\\
0.33	3.788e-07\\
0.34	3.308e-07\\
0.35	2.853e-07\\
0.36	2.429e-07\\
0.37	2.038e-07\\
0.38	1.682e-07\\
0.39	1.365e-07\\
0.4	1.086e-07\\
0.41	8.468e-08\\
0.42	6.454e-08\\
0.43	4.808e-08\\
0.44	3.508e-08\\
0.45	2.523e-08\\
0.46	1.818e-08\\
0.47	1.352e-08\\
0.49	9.481e-09\\
0.53	7.781e-09\\
0.54	5.509e-09\\
0.55	1.451e-09\\
0.56	-4.867e-09\\
0.57	-1.387e-08\\
0.58	-2.594e-08\\
0.59	-4.138e-08\\
0.6	-6.045e-08\\
0.61	-8.33e-08\\
0.62	-1.1e-07\\
0.63	-1.407e-07\\
0.64	-1.751e-07\\
0.65	-2.132e-07\\
0.66	-2.548e-07\\
0.67	-2.994e-07\\
0.68	-3.467e-07\\
0.69	-3.963e-07\\
0.71	-4.999e-07\\
0.74	-6.57e-07\\
0.75	-7.07e-07\\
0.76	-7.546e-07\\
0.77	-7.99e-07\\
0.78	-8.395e-07\\
0.79	-8.755e-07\\
0.8	-9.062e-07\\
0.81	-9.311e-07\\
0.82	-9.495e-07\\
0.83	-9.611e-07\\
0.84	-9.654e-07\\
0.85	-9.621e-07\\
0.86	-9.509e-07\\
0.87	-9.317e-07\\
0.88	-9.045e-07\\
0.89	-8.693e-07\\
0.9	-8.262e-07\\
0.91	-7.756e-07\\
0.92	-7.178e-07\\
0.93	-6.531e-07\\
0.94	-5.822e-07\\
0.95	-5.057e-07\\
0.96	-4.243e-07\\
0.97	-3.386e-07\\
0.98	-2.496e-07\\
0.99	-1.58e-07\\
1.03	2.153e-07\\
1.04	3.059e-07\\
1.05	3.936e-07\\
1.06	4.776e-07\\
1.07	5.57e-07\\
1.08	6.312e-07\\
1.09	6.995e-07\\
1.1	7.613e-07\\
1.11	8.161e-07\\
1.12	8.636e-07\\
1.13	9.033e-07\\
1.14	9.352e-07\\
1.15	9.59e-07\\
1.16	9.747e-07\\
1.17	9.825e-07\\
1.18	9.826e-07\\
1.19	9.751e-07\\
1.2	9.606e-07\\
1.21	9.393e-07\\
1.22	9.119e-07\\
1.23	8.788e-07\\
1.24	8.408e-07\\
1.25	7.986e-07\\
1.26	7.527e-07\\
1.27	7.04e-07\\
1.29	6.011e-07\\
1.32	4.436e-07\\
1.33	3.93e-07\\
1.34	3.443e-07\\
1.35	2.981e-07\\
1.36	2.547e-07\\
1.37	2.146e-07\\
1.38	1.781e-07\\
1.39	1.452e-07\\
1.4	1.162e-07\\
1.41	9.116e-08\\
1.42	6.994e-08\\
1.43	5.244e-08\\
1.44	3.847e-08\\
1.45	2.775e-08\\
1.46	1.994e-08\\
1.47	1.464e-08\\
1.48	1.14e-08\\
1.5	9.131e-09\\
1.53	8.174e-09\\
1.54	6.322e-09\\
1.55	2.826e-09\\
1.56	-2.798e-09\\
1.57	-1.099e-08\\
1.58	-2.215e-08\\
1.59	-3.659e-08\\
1.6	-5.46e-08\\
1.61	-7.635e-08\\
1.62	-1.02e-07\\
1.63	-1.315e-07\\
1.64	-1.649e-07\\
1.65	-2.019e-07\\
1.66	-2.425e-07\\
1.67	-2.863e-07\\
1.68	-3.329e-07\\
1.69	-3.819e-07\\
1.71	-4.849e-07\\
1.74	-6.424e-07\\
1.75	-6.929e-07\\
1.76	-7.412e-07\\
1.77	-7.866e-07\\
1.78	-8.284e-07\\
1.79	-8.657e-07\\
1.8	-8.98e-07\\
1.81	-9.246e-07\\
1.82	-9.45e-07\\
1.83	-9.586e-07\\
1.84	-9.65e-07\\
1.85	-9.639e-07\\
1.86	-9.55e-07\\
1.87	-9.381e-07\\
1.88	-9.132e-07\\
1.89	-8.802e-07\\
1.9	-8.394e-07\\
1.91	-7.909e-07\\
1.92	-7.351e-07\\
1.93	-6.724e-07\\
1.94	-6.032e-07\\
1.95	-5.282e-07\\
1.96	-4.481e-07\\
1.97	-3.636e-07\\
1.98	-2.754e-07\\
1.99	-1.845e-07\\
2.01	2.08e-09\\
2.03	1.89e-07\\
2.04	2.802e-07\\
2.05	3.688e-07\\
2.06	4.539e-07\\
2.07	5.348e-07\\
2.08	6.105e-07\\
2.09	6.806e-07\\
2.1	7.443e-07\\
2.11	8.012e-07\\
2.12	8.508e-07\\
2.13	8.928e-07\\
2.14	9.269e-07\\
2.15	9.53e-07\\
2.16	9.711e-07\\
2.17	9.811e-07\\
2.18	9.834e-07\\
2.19	9.781e-07\\
2.2	9.655e-07\\
2.21	9.461e-07\\
2.22	9.204e-07\\
2.23	8.889e-07\\
2.24	8.522e-07\\
2.25	8.111e-07\\
2.26	7.662e-07\\
2.27	7.183e-07\\
2.28	6.68e-07\\
2.3	5.635e-07\\
2.32	4.585e-07\\
2.33	4.074e-07\\
2.34	3.581e-07\\
2.35	3.111e-07\\
2.36	2.669e-07\\
2.37	2.258e-07\\
2.38	1.882e-07\\
2.39	1.543e-07\\
2.4	1.242e-07\\
2.41	9.797e-08\\
2.42	7.565e-08\\
2.43	5.71e-08\\
2.44	4.214e-08\\
2.45	3.052e-08\\
2.46	2.191e-08\\
2.47	1.593e-08\\
2.48	1.215e-08\\
2.5	9.23e-09\\
2.54	6.999e-09\\
2.55	4.026e-09\\
2.56	-9.397e-10\\
2.57	-8.354e-09\\
2.58	-1.862e-08\\
2.59	-3.21e-08\\
2.6	-4.906e-08\\
2.61	-6.972e-08\\
2.62	-9.423e-08\\
2.63	-1.226e-07\\
2.64	-1.549e-07\\
2.65	-1.909e-07\\
2.66	-2.306e-07\\
2.67	-2.735e-07\\
2.68	-3.193e-07\\
2.69	-3.677e-07\\
2.71	-4.698e-07\\
2.74	-6.276e-07\\
2.75	-6.786e-07\\
2.76	-7.277e-07\\
2.77	-7.74e-07\\
2.78	-8.168e-07\\
2.79	-8.555e-07\\
2.8	-8.893e-07\\
2.81	-9.176e-07\\
2.82	-9.398e-07\\
2.83	-9.554e-07\\
2.84	-9.64e-07\\
2.85	-9.65e-07\\
2.86	-9.584e-07\\
2.87	-9.438e-07\\
2.88	-9.212e-07\\
2.89	-8.906e-07\\
2.9	-8.52e-07\\
2.91	-8.057e-07\\
2.92	-7.519e-07\\
2.93	-6.911e-07\\
2.94	-6.237e-07\\
2.95	-5.503e-07\\
2.96	-4.716e-07\\
2.97	-3.883e-07\\
2.98	-3.011e-07\\
2.99	-2.109e-07\\
3.01	-2.488e-08\\
3.03	1.624e-07\\
3.04	2.543e-07\\
3.05	3.437e-07\\
3.06	4.299e-07\\
3.07	5.121e-07\\
3.08	5.894e-07\\
3.09	6.611e-07\\
3.1	7.267e-07\\
3.11	7.856e-07\\
3.12	8.374e-07\\
3.13	8.816e-07\\
3.14	9.18e-07\\
3.15	9.464e-07\\
3.16	9.667e-07\\
3.17	9.791e-07\\
3.18	9.836e-07\\
3.19	9.804e-07\\
3.2	9.699e-07\\
3.21	9.524e-07\\
3.22	9.284e-07\\
3.23	8.985e-07\\
3.24	8.633e-07\\
3.25	8.234e-07\\
3.26	7.795e-07\\
3.27	7.323e-07\\
3.28	6.827e-07\\
3.3	5.787e-07\\
3.32	4.734e-07\\
3.33	4.219e-07\\
3.34	3.721e-07\\
3.35	3.243e-07\\
3.36	2.793e-07\\
3.37	2.373e-07\\
3.38	1.986e-07\\
3.39	1.636e-07\\
3.4	1.324e-07\\
3.41	1.051e-07\\
3.42	8.166e-08\\
3.43	6.205e-08\\
3.44	4.608e-08\\
3.45	3.353e-08\\
3.46	2.409e-08\\
3.47	1.74e-08\\
3.48	1.304e-08\\
3.5	9.384e-09\\
3.54	7.553e-09\\
3.55	5.062e-09\\
3.56	7.164e-10\\
3.57	-5.952e-09\\
3.58	-1.536e-08\\
3.59	-2.789e-08\\
3.6	-4.382e-08\\
3.61	-6.341e-08\\
3.62	-8.68e-08\\
3.63	-1.141e-07\\
3.64	-1.453e-07\\
3.65	-1.802e-07\\
3.66	-2.188e-07\\
3.67	-2.608e-07\\
3.68	-3.059e-07\\
3.69	-3.536e-07\\
3.7	-4.034e-07\\
3.72	-5.074e-07\\
3.74	-6.127e-07\\
3.75	-6.642e-07\\
3.76	-7.139e-07\\
3.77	-7.611e-07\\
3.78	-8.05e-07\\
3.79	-8.449e-07\\
3.8	-8.802e-07\\
3.81	-9.101e-07\\
3.82	-9.342e-07\\
3.83	-9.517e-07\\
3.84	-9.623e-07\\
3.85	-9.656e-07\\
3.86	-9.612e-07\\
3.87	-9.489e-07\\
3.88	-9.286e-07\\
3.89	-9.002e-07\\
3.9	-8.639e-07\\
3.91	-8.198e-07\\
3.92	-7.681e-07\\
3.93	-7.093e-07\\
3.94	-6.437e-07\\
3.95	-5.72e-07\\
3.96	-4.947e-07\\
3.97	-4.126e-07\\
3.98	-3.265e-07\\
3.99	-2.37e-07\\
4.01	-5.182e-08\\
4.03	1.357e-07\\
4.04	2.281e-07\\
4.05	3.184e-07\\
4.06	4.056e-07\\
4.07	4.89e-07\\
4.08	5.678e-07\\
4.09	6.412e-07\\
4.1	7.086e-07\\
4.11	7.695e-07\\
4.12	8.233e-07\\
4.13	8.697e-07\\
4.14	9.084e-07\\
4.15	9.391e-07\\
4.16	9.618e-07\\
4.17	9.764e-07\\
4.18	9.831e-07\\
4.19	9.821e-07\\
4.2	9.737e-07\\
4.21	9.581e-07\\
4.22	9.36e-07\\
4.23	9.077e-07\\
4.24	8.739e-07\\
4.25	8.353e-07\\
4.26	7.925e-07\\
4.27	7.462e-07\\
4.28	6.972e-07\\
4.3	5.939e-07\\
4.33	4.366e-07\\
4.34	3.862e-07\\
4.35	3.378e-07\\
4.36	2.919e-07\\
4.37	2.49e-07\\
4.38	2.094e-07\\
4.39	1.733e-07\\
4.4	1.41e-07\\
4.41	1.125e-07\\
4.42	8.799e-08\\
4.43	6.73e-08\\
4.44	5.03e-08\\
4.45	3.68e-08\\
4.46	2.65e-08\\
4.47	1.907e-08\\
4.48	1.408e-08\\
4.5	9.606e-09\\
4.55	5.945e-09\\
4.56	2.181e-09\\
4.57	-3.776e-09\\
4.58	-1.236e-08\\
4.59	-2.396e-08\\
4.6	-3.889e-08\\
4.61	-5.741e-08\\
4.62	-7.97e-08\\
4.63	-1.059e-07\\
4.64	-1.359e-07\\
4.65	-1.698e-07\\
4.66	-2.074e-07\\
4.67	-2.485e-07\\
4.68	-2.927e-07\\
4.69	-3.397e-07\\
4.7	-3.889e-07\\
4.72	-4.923e-07\\
4.75	-6.496e-07\\
4.76	-6.999e-07\\
4.77	-7.479e-07\\
4.78	-7.928e-07\\
4.79	-8.339e-07\\
4.8	-8.706e-07\\
4.81	-9.022e-07\\
4.82	-9.28e-07\\
4.83	-9.474e-07\\
4.84	-9.6e-07\\
4.85	-9.654e-07\\
4.86	-9.633e-07\\
4.87	-9.533e-07\\
4.88	-9.353e-07\\
4.89	-9.092e-07\\
4.9	-8.752e-07\\
4.91	-8.332e-07\\
4.92	-7.837e-07\\
4.93	-7.269e-07\\
4.94	-6.632e-07\\
4.95	-5.932e-07\\
4.96	-5.175e-07\\
4.97	-4.367e-07\\
4.98	-3.516e-07\\
4.99	-2.63e-07\\
5	-1.717e-07\\
5.02	1.516e-08\\
5.04	2.018e-07\\
5.05	2.928e-07\\
5.06	3.809e-07\\
5.07	4.655e-07\\
5.08	5.457e-07\\
5.09	6.207e-07\\
5.1	6.899e-07\\
5.11	7.528e-07\\
5.12	8.087e-07\\
5.13	8.572e-07\\
5.14	8.981e-07\\
5.15	9.311e-07\\
5.16	9.561e-07\\
5.17	9.731e-07\\
5.18	9.821e-07\\
5.19	9.832e-07\\
5.2	9.769e-07\\
5.21	9.633e-07\\
5.22	9.43e-07\\
5.23	9.165e-07\\
5.24	8.842e-07\\
5.25	8.469e-07\\
5.26	8.052e-07\\
5.27	7.598e-07\\
5.28	7.115e-07\\
5.3	6.09e-07\\
5.33	4.513e-07\\
5.34	4.005e-07\\
5.35	3.515e-07\\
5.36	3.048e-07\\
5.37	2.61e-07\\
5.38	2.204e-07\\
5.39	1.833e-07\\
5.4	1.499e-07\\
5.41	1.203e-07\\
5.42	9.464e-08\\
5.43	7.285e-08\\
5.44	5.481e-08\\
5.45	4.033e-08\\
5.46	2.915e-08\\
5.47	2.093e-08\\
5.48	1.529e-08\\
5.49	1.177e-08\\
5.51	9.177e-09\\
5.54	8.34e-09\\
5.55	6.687e-09\\
5.56	3.465e-09\\
5.57	-1.816e-09\\
5.58	-9.606e-09\\
5.59	-2.03e-08\\
5.6	-3.425e-08\\
5.61	-5.172e-08\\
5.62	-7.291e-08\\
5.63	-9.796e-08\\
5.64	-1.269e-07\\
5.65	-1.597e-07\\
5.66	-1.963e-07\\
5.67	-2.364e-07\\
5.68	-2.797e-07\\
5.69	-3.259e-07\\
5.7	-3.746e-07\\
5.72	-4.772e-07\\
5.75	-6.349e-07\\
5.76	-6.857e-07\\
5.77	-7.344e-07\\
5.78	-7.803e-07\\
5.79	-8.226e-07\\
5.8	-8.606e-07\\
5.81	-8.937e-07\\
5.82	-9.212e-07\\
5.83	-9.426e-07\\
5.84	-9.572e-07\\
5.85	-9.647e-07\\
5.86	-9.647e-07\\
5.87	-9.57e-07\\
5.88	-9.413e-07\\
5.89	-9.176e-07\\
5.9	-8.858e-07\\
5.91	-8.461e-07\\
5.92	-7.987e-07\\
5.93	-7.439e-07\\
5.94	-6.822e-07\\
5.95	-6.139e-07\\
5.96	-5.398e-07\\
5.97	-4.603e-07\\
5.98	-3.764e-07\\
5.99	-2.887e-07\\
6	-1.981e-07\\
6.02	-1.181e-08\\
6.04	1.753e-07\\
6.05	2.669e-07\\
6.06	3.56e-07\\
6.07	4.417e-07\\
6.08	5.232e-07\\
6.09	5.998e-07\\
6.1	6.707e-07\\
6.11	7.355e-07\\
6.12	7.934e-07\\
6.13	8.441e-07\\
6.14	8.872e-07\\
6.15	9.225e-07\\
6.16	9.498e-07\\
6.17	9.691e-07\\
6.18	9.803e-07\\
6.19	9.837e-07\\
6.2	9.795e-07\\
6.21	9.68e-07\\
6.22	9.496e-07\\
6.23	9.247e-07\\
6.24	8.94e-07\\
6.25	8.581e-07\\
6.26	8.176e-07\\
6.27	7.732e-07\\
6.28	7.257e-07\\
6.29	6.757e-07\\
6.31	5.714e-07\\
6.33	4.662e-07\\
6.34	4.149e-07\\
6.35	3.653e-07\\
6.36	3.18e-07\\
6.37	2.733e-07\\
6.38	2.317e-07\\
6.39	1.936e-07\\
6.4	1.591e-07\\
6.41	1.284e-07\\
6.42	1.016e-07\\
6.43	7.872e-08\\
6.44	5.962e-08\\
6.45	4.414e-08\\
6.46	3.204e-08\\
6.47	2.301e-08\\
6.48	1.667e-08\\
6.49	1.259e-08\\
6.51	9.302e-09\\
6.55	7.3e-09\\
6.56	4.58e-09\\
6.57	-6.311e-11\\
6.58	-7.09e-09\\
6.59	-1.692e-08\\
6.6	-2.99e-08\\
6.61	-4.633e-08\\
6.62	-6.644e-08\\
6.63	-9.038e-08\\
6.64	-1.182e-07\\
6.65	-1.499e-07\\
6.66	-1.854e-07\\
6.67	-2.245e-07\\
6.68	-2.67e-07\\
6.69	-3.124e-07\\
6.7	-3.604e-07\\
6.71	-4.106e-07\\
6.73	-5.148e-07\\
6.75	-6.2e-07\\
6.76	-6.713e-07\\
6.77	-7.207e-07\\
6.78	-7.675e-07\\
6.79	-8.109e-07\\
6.8	-8.502e-07\\
6.81	-8.848e-07\\
6.82	-9.14e-07\\
6.83	-9.371e-07\\
6.84	-9.537e-07\\
6.85	-9.633e-07\\
6.86	-9.656e-07\\
6.87	-9.601e-07\\
6.88	-9.467e-07\\
6.89	-9.252e-07\\
6.9	-8.958e-07\\
6.91	-8.583e-07\\
6.92	-8.131e-07\\
6.93	-7.604e-07\\
6.94	-7.006e-07\\
6.95	-6.342e-07\\
6.96	-5.616e-07\\
6.97	-4.836e-07\\
6.98	-4.009e-07\\
6.99	-3.142e-07\\
7	-2.244e-07\\
7.02	-3.877e-08\\
7.04	1.487e-07\\
7.05	2.409e-07\\
7.06	3.308e-07\\
7.07	4.175e-07\\
7.08	5.003e-07\\
7.09	5.784e-07\\
7.1	6.51e-07\\
7.11	7.176e-07\\
7.12	7.775e-07\\
7.13	8.304e-07\\
7.14	8.757e-07\\
7.15	9.133e-07\\
7.16	9.429e-07\\
7.17	9.644e-07\\
7.18	9.78e-07\\
7.19	9.836e-07\\
7.2	9.815e-07\\
7.21	9.721e-07\\
7.22	9.556e-07\\
7.23	9.325e-07\\
7.24	9.035e-07\\
7.25	8.69e-07\\
7.26	8.297e-07\\
7.27	7.864e-07\\
7.28	7.396e-07\\
7.29	6.903e-07\\
7.31	5.866e-07\\
7.34	4.295e-07\\
7.35	3.794e-07\\
7.36	3.313e-07\\
7.37	2.858e-07\\
7.38	2.433e-07\\
7.39	2.041e-07\\
7.4	1.686e-07\\
7.41	1.368e-07\\
7.42	1.089e-07\\
7.43	8.49e-08\\
7.44	6.472e-08\\
7.45	4.822e-08\\
7.46	3.519e-08\\
7.47	2.531e-08\\
7.48	1.824e-08\\
7.49	1.356e-08\\
7.51	9.49e-09\\
7.55	7.795e-09\\
7.56	5.536e-09\\
7.57	1.493e-09\\
7.58	-4.805e-09\\
7.59	-1.379e-08\\
7.6	-2.583e-08\\
7.61	-4.125e-08\\
7.62	-6.029e-08\\
7.63	-8.312e-08\\
7.64	-1.098e-07\\
7.65	-1.404e-07\\
7.66	-1.749e-07\\
7.67	-2.13e-07\\
7.68	-2.545e-07\\
7.69	-2.991e-07\\
7.7	-3.464e-07\\
7.71	-3.96e-07\\
7.73	-4.997e-07\\
7.76	-6.568e-07\\
7.77	-7.068e-07\\
7.78	-7.544e-07\\
7.79	-7.989e-07\\
7.8	-8.395e-07\\
7.81	-8.755e-07\\
7.82	-9.063e-07\\
7.83	-9.312e-07\\
7.84	-9.497e-07\\
7.85	-9.614e-07\\
7.86	-9.657e-07\\
7.87	-9.625e-07\\
7.88	-9.514e-07\\
7.89	-9.322e-07\\
7.9	-9.051e-07\\
7.91	-8.699e-07\\
7.92	-8.269e-07\\
7.93	-7.763e-07\\
7.94	-7.185e-07\\
7.95	-6.539e-07\\
7.96	-5.831e-07\\
7.97	-5.066e-07\\
7.98	-4.251e-07\\
7.99	-3.395e-07\\
8	-2.504e-07\\
8.01	-1.589e-07\\
8.05	2.146e-07\\
8.06	3.053e-07\\
8.07	3.93e-07\\
8.08	4.77e-07\\
8.09	5.565e-07\\
8.1	6.308e-07\\
8.11	6.992e-07\\
8.12	7.611e-07\\
8.13	8.16e-07\\
8.14	8.635e-07\\
8.15	9.033e-07\\
8.16	9.352e-07\\
8.17	9.591e-07\\
8.18	9.749e-07\\
8.19	9.828e-07\\
8.2	9.829e-07\\
8.21	9.756e-07\\
8.22	9.61e-07\\
8.23	9.398e-07\\
8.24	9.124e-07\\
8.25	8.794e-07\\
8.26	8.415e-07\\
8.27	7.992e-07\\
8.28	7.534e-07\\
8.29	7.047e-07\\
8.31	6.018e-07\\
8.34	4.442e-07\\
8.35	3.936e-07\\
8.36	3.449e-07\\
8.37	2.986e-07\\
8.38	2.552e-07\\
8.39	2.15e-07\\
8.4	1.784e-07\\
8.41	1.455e-07\\
8.42	1.165e-07\\
8.43	9.139e-08\\
8.44	7.013e-08\\
8.45	5.26e-08\\
8.46	3.859e-08\\
8.47	2.784e-08\\
8.48	2e-08\\
8.49	1.468e-08\\
8.5	1.143e-08\\
8.52	9.135e-09\\
8.55	8.185e-09\\
8.56	6.345e-09\\
8.57	2.864e-09\\
8.58	-2.741e-09\\
8.59	-1.091e-08\\
8.6	-2.205e-08\\
8.61	-3.647e-08\\
8.62	-5.445e-08\\
8.63	-7.618e-08\\
8.64	-1.018e-07\\
8.65	-1.313e-07\\
8.66	-1.646e-07\\
8.67	-2.017e-07\\
8.68	-2.422e-07\\
8.69	-2.86e-07\\
8.7	-3.326e-07\\
8.71	-3.816e-07\\
8.73	-4.846e-07\\
8.76	-6.421e-07\\
8.77	-6.927e-07\\
8.78	-7.411e-07\\
8.79	-7.865e-07\\
8.8	-8.283e-07\\
8.81	-8.657e-07\\
8.82	-8.98e-07\\
8.83	-9.247e-07\\
8.84	-9.451e-07\\
8.85	-9.588e-07\\
8.86	-9.653e-07\\
8.87	-9.643e-07\\
8.88	-9.554e-07\\
8.89	-9.386e-07\\
8.9	-9.137e-07\\
8.91	-8.809e-07\\
8.92	-8.401e-07\\
8.93	-7.916e-07\\
8.94	-7.359e-07\\
8.95	-6.732e-07\\
8.96	-6.04e-07\\
8.97	-5.291e-07\\
8.98	-4.49e-07\\
8.99	-3.645e-07\\
9	-2.763e-07\\
9.01	-1.854e-07\\
9.03	1.271e-09\\
9.05	1.882e-07\\
9.06	2.795e-07\\
9.07	3.682e-07\\
9.08	4.534e-07\\
9.09	5.343e-07\\
9.1	6.101e-07\\
9.11	6.802e-07\\
9.12	7.44e-07\\
9.13	8.01e-07\\
9.14	8.507e-07\\
9.15	8.927e-07\\
9.16	9.269e-07\\
9.17	9.531e-07\\
9.18	9.713e-07\\
9.19	9.814e-07\\
9.2	9.837e-07\\
9.21	9.785e-07\\
9.22	9.66e-07\\
9.23	9.466e-07\\
9.24	9.209e-07\\
9.25	8.895e-07\\
9.26	8.529e-07\\
9.27	8.118e-07\\
9.28	7.669e-07\\
9.29	7.19e-07\\
9.3	6.687e-07\\
9.32	5.642e-07\\
9.34	4.591e-07\\
9.35	4.08e-07\\
9.36	3.586e-07\\
9.37	3.116e-07\\
9.38	2.673e-07\\
9.39	2.262e-07\\
9.4	1.886e-07\\
9.41	1.546e-07\\
9.42	1.244e-07\\
9.43	9.821e-08\\
9.44	7.585e-08\\
9.45	5.726e-08\\
9.46	4.227e-08\\
9.47	3.061e-08\\
9.48	2.198e-08\\
9.49	1.598e-08\\
9.5	1.218e-08\\
9.52	9.235e-09\\
9.56	7.018e-09\\
9.57	4.06e-09\\
9.58	-8.891e-10\\
9.59	-8.283e-09\\
9.6	-1.853e-08\\
9.61	-3.198e-08\\
9.62	-4.891e-08\\
9.63	-6.955e-08\\
9.64	-9.403e-08\\
9.65	-1.224e-07\\
9.66	-1.547e-07\\
9.67	-1.907e-07\\
9.68	-2.303e-07\\
9.69	-2.732e-07\\
9.7	-3.19e-07\\
9.71	-3.674e-07\\
9.73	-4.695e-07\\
9.76	-6.274e-07\\
9.77	-6.784e-07\\
9.78	-7.275e-07\\
9.79	-7.739e-07\\
9.8	-8.168e-07\\
9.81	-8.555e-07\\
9.82	-8.894e-07\\
9.83	-9.177e-07\\
9.84	-9.4e-07\\
9.85	-9.557e-07\\
9.86	-9.642e-07\\
9.87	-9.654e-07\\
9.88	-9.588e-07\\
9.89	-9.443e-07\\
9.9	-9.217e-07\\
9.91	-8.912e-07\\
9.92	-8.526e-07\\
9.93	-8.064e-07\\
9.94	-7.526e-07\\
9.95	-6.919e-07\\
9.96	-6.245e-07\\
9.97	-5.512e-07\\
9.98	-4.725e-07\\
9.99	-3.891e-07\\
10	-3.019e-07\\
10.01	-2.117e-07\\
10.03	-2.57e-08\\
10.05	1.617e-07\\
10.06	2.536e-07\\
10.07	3.431e-07\\
10.08	4.294e-07\\
10.09	5.116e-07\\
10.1	5.889e-07\\
10.11	6.608e-07\\
10.12	7.265e-07\\
10.13	7.854e-07\\
10.14	8.372e-07\\
10.15	8.815e-07\\
10.16	9.18e-07\\
10.17	9.465e-07\\
10.18	9.669e-07\\
10.19	9.794e-07\\
10.2	9.839e-07\\
10.21	9.808e-07\\
10.22	9.703e-07\\
10.23	9.529e-07\\
10.24	9.29e-07\\
10.25	8.991e-07\\
10.26	8.639e-07\\
10.27	8.24e-07\\
10.28	7.802e-07\\
10.29	7.33e-07\\
10.3	6.833e-07\\
10.32	5.794e-07\\
10.34	4.74e-07\\
10.35	4.225e-07\\
10.36	3.726e-07\\
10.37	3.249e-07\\
10.38	2.797e-07\\
10.39	2.377e-07\\
10.4	1.99e-07\\
10.41	1.64e-07\\
10.42	1.327e-07\\
10.43	1.053e-07\\
10.44	8.187e-08\\
10.45	6.222e-08\\
10.46	4.622e-08\\
10.47	3.363e-08\\
10.48	2.417e-08\\
10.49	1.745e-08\\
10.5	1.307e-08\\
10.52	9.391e-09\\
10.56	7.569e-09\\
10.57	5.091e-09\\
10.58	7.617e-10\\
10.59	-5.887e-09\\
10.6	-1.528e-08\\
10.61	-2.777e-08\\
10.62	-4.369e-08\\
10.63	-6.325e-08\\
10.64	-8.661e-08\\
10.65	-1.139e-07\\
10.66	-1.45e-07\\
10.67	-1.8e-07\\
10.68	-2.186e-07\\
10.69	-2.606e-07\\
10.7	-3.056e-07\\
10.71	-3.533e-07\\
10.72	-4.031e-07\\
10.74	-5.071e-07\\
10.76	-6.125e-07\\
10.77	-6.64e-07\\
10.78	-7.137e-07\\
10.79	-7.609e-07\\
10.8	-8.049e-07\\
10.81	-8.449e-07\\
10.82	-8.802e-07\\
10.83	-9.102e-07\\
10.84	-9.343e-07\\
10.85	-9.519e-07\\
10.86	-9.626e-07\\
10.87	-9.659e-07\\
10.88	-9.616e-07\\
10.89	-9.493e-07\\
10.9	-9.291e-07\\
10.91	-9.008e-07\\
10.92	-8.645e-07\\
10.93	-8.204e-07\\
10.94	-7.688e-07\\
10.95	-7.1e-07\\
10.96	-6.445e-07\\
10.97	-5.728e-07\\
10.98	-4.956e-07\\
10.99	-4.135e-07\\
11	-3.273e-07\\
11.01	-2.379e-07\\
11.03	-5.265e-08\\
11.05	1.35e-07\\
11.06	2.274e-07\\
11.07	3.177e-07\\
11.08	4.05e-07\\
11.09	4.885e-07\\
11.1	5.673e-07\\
11.11	6.408e-07\\
11.12	7.083e-07\\
11.13	7.693e-07\\
11.14	8.232e-07\\
11.15	8.696e-07\\
11.16	9.084e-07\\
11.17	9.392e-07\\
11.18	9.619e-07\\
11.19	9.767e-07\\
11.2	9.834e-07\\
11.21	9.825e-07\\
11.22	9.741e-07\\
11.23	9.586e-07\\
11.24	9.365e-07\\
11.25	9.083e-07\\
11.26	8.746e-07\\
11.27	8.359e-07\\
11.28	7.931e-07\\
11.29	7.469e-07\\
11.3	6.978e-07\\
11.32	5.945e-07\\
11.35	4.372e-07\\
11.36	3.867e-07\\
11.37	3.383e-07\\
11.38	2.924e-07\\
11.39	2.494e-07\\
11.4	2.098e-07\\
11.41	1.736e-07\\
11.42	1.413e-07\\
11.43	1.128e-07\\
11.44	8.822e-08\\
11.45	6.748e-08\\
11.46	5.045e-08\\
11.47	3.691e-08\\
11.48	2.659e-08\\
11.49	1.912e-08\\
11.5	1.412e-08\\
11.52	9.615e-09\\
11.57	5.97e-09\\
11.58	2.221e-09\\
11.59	-3.717e-09\\
11.6	-1.228e-08\\
11.61	-2.385e-08\\
11.62	-3.876e-08\\
11.63	-5.725e-08\\
11.64	-7.952e-08\\
11.65	-1.057e-07\\
11.66	-1.357e-07\\
11.67	-1.696e-07\\
11.68	-2.072e-07\\
11.69	-2.482e-07\\
11.7	-2.924e-07\\
11.71	-3.394e-07\\
11.72	-3.886e-07\\
11.74	-4.92e-07\\
11.77	-6.494e-07\\
11.78	-6.997e-07\\
11.79	-7.477e-07\\
11.8	-7.927e-07\\
11.81	-8.339e-07\\
11.82	-8.706e-07\\
11.83	-9.022e-07\\
11.84	-9.281e-07\\
11.85	-9.476e-07\\
11.86	-9.603e-07\\
11.87	-9.658e-07\\
11.88	-9.636e-07\\
11.89	-9.537e-07\\
11.9	-9.358e-07\\
11.91	-9.098e-07\\
11.92	-8.758e-07\\
11.93	-8.339e-07\\
11.94	-7.844e-07\\
11.95	-7.276e-07\\
11.96	-6.64e-07\\
11.97	-5.94e-07\\
11.98	-5.183e-07\\
11.99	-4.375e-07\\
12	-3.524e-07\\
12.01	-2.638e-07\\
12.02	-1.726e-07\\
12.04	1.436e-08\\
12.06	2.011e-07\\
12.07	2.921e-07\\
12.08	3.803e-07\\
12.09	4.65e-07\\
12.1	5.452e-07\\
12.11	6.203e-07\\
12.12	6.896e-07\\
12.13	7.525e-07\\
12.14	8.085e-07\\
12.15	8.571e-07\\
12.16	8.981e-07\\
12.17	9.312e-07\\
12.18	9.563e-07\\
12.19	9.733e-07\\
12.2	9.824e-07\\
12.21	9.836e-07\\
12.22	9.773e-07\\
12.23	9.638e-07\\
12.24	9.435e-07\\
12.25	9.17e-07\\
12.26	8.848e-07\\
12.27	8.475e-07\\
12.28	8.059e-07\\
12.29	7.605e-07\\
12.3	7.122e-07\\
12.32	6.096e-07\\
12.35	4.519e-07\\
12.36	4.01e-07\\
12.37	3.52e-07\\
12.38	3.053e-07\\
12.39	2.615e-07\\
12.4	2.208e-07\\
12.41	1.836e-07\\
12.42	1.502e-07\\
12.43	1.206e-07\\
12.44	9.488e-08\\
12.45	7.305e-08\\
12.46	5.497e-08\\
12.47	4.046e-08\\
12.48	2.924e-08\\
12.49	2.099e-08\\
12.5	1.533e-08\\
12.51	1.18e-08\\
12.53	9.181e-09\\
12.56	8.35e-09\\
12.57	6.708e-09\\
12.58	3.501e-09\\
12.59	-1.763e-09\\
12.6	-9.531e-09\\
12.61	-2.021e-08\\
12.62	-3.413e-08\\
12.63	-5.157e-08\\
12.64	-7.274e-08\\
12.65	-9.776e-08\\
12.66	-1.267e-07\\
12.67	-1.595e-07\\
12.68	-1.96e-07\\
12.69	-2.361e-07\\
12.7	-2.794e-07\\
12.71	-3.256e-07\\
12.72	-3.743e-07\\
12.74	-4.769e-07\\
12.77	-6.346e-07\\
12.78	-6.855e-07\\
12.79	-7.343e-07\\
12.8	-7.802e-07\\
12.81	-8.225e-07\\
12.82	-8.606e-07\\
12.83	-8.938e-07\\
12.84	-9.213e-07\\
12.85	-9.427e-07\\
12.86	-9.574e-07\\
12.87	-9.65e-07\\
12.88	-9.651e-07\\
12.89	-9.574e-07\\
12.9	-9.418e-07\\
12.91	-9.181e-07\\
12.92	-8.864e-07\\
12.93	-8.468e-07\\
12.94	-7.994e-07\\
12.95	-7.447e-07\\
12.96	-6.83e-07\\
12.97	-6.147e-07\\
12.98	-5.406e-07\\
12.99	-4.612e-07\\
13	-3.773e-07\\
13.01	-2.896e-07\\
13.02	-1.99e-07\\
13.04	-1.262e-08\\
13.06	1.746e-07\\
13.07	2.662e-07\\
13.08	3.554e-07\\
13.09	4.411e-07\\
13.1	5.227e-07\\
13.11	5.994e-07\\
13.12	6.704e-07\\
13.13	7.352e-07\\
13.14	7.932e-07\\
13.15	8.44e-07\\
13.16	8.872e-07\\
13.17	9.226e-07\\
13.18	9.499e-07\\
13.19	9.693e-07\\
13.2	9.806e-07\\
13.21	9.841e-07\\
13.22	9.799e-07\\
13.23	9.684e-07\\
13.24	9.501e-07\\
13.25	9.253e-07\\
13.26	8.946e-07\\
13.27	8.587e-07\\
13.28	8.183e-07\\
13.29	7.739e-07\\
13.3	7.263e-07\\
13.31	6.764e-07\\
13.33	5.721e-07\\
13.35	4.668e-07\\
13.36	4.155e-07\\
13.37	3.659e-07\\
13.38	3.185e-07\\
13.39	2.737e-07\\
13.4	2.321e-07\\
13.41	1.939e-07\\
13.42	1.594e-07\\
13.43	1.287e-07\\
13.44	1.019e-07\\
13.45	7.892e-08\\
13.46	5.978e-08\\
13.47	4.427e-08\\
13.48	3.214e-08\\
13.49	2.308e-08\\
13.5	1.672e-08\\
13.51	1.262e-08\\
13.53	9.309e-09\\
13.57	7.317e-09\\
13.58	4.611e-09\\
13.59	-1.521e-11\\
13.6	-7.022e-09\\
13.61	-1.683e-08\\
13.62	-2.978e-08\\
13.63	-4.619e-08\\
13.64	-6.628e-08\\
13.65	-9.019e-08\\
13.66	-1.18e-07\\
13.67	-1.497e-07\\
13.68	-1.852e-07\\
13.69	-2.243e-07\\
13.7	-2.667e-07\\
13.71	-3.121e-07\\
13.72	-3.601e-07\\
13.73	-4.103e-07\\
13.75	-5.145e-07\\
13.77	-6.198e-07\\
13.78	-6.711e-07\\
13.79	-7.206e-07\\
13.8	-7.674e-07\\
13.81	-8.108e-07\\
13.82	-8.502e-07\\
13.83	-8.848e-07\\
13.84	-9.141e-07\\
13.85	-9.373e-07\\
13.86	-9.54e-07\\
13.87	-9.636e-07\\
13.88	-9.659e-07\\
13.89	-9.605e-07\\
13.9	-9.471e-07\\
13.91	-9.258e-07\\
13.92	-8.964e-07\\
13.93	-8.59e-07\\
13.94	-8.138e-07\\
13.95	-7.612e-07\\
13.96	-7.014e-07\\
13.97	-6.35e-07\\
13.98	-5.625e-07\\
13.99	-4.845e-07\\
14	-4.018e-07\\
14.01	-3.151e-07\\
14.02	-2.252e-07\\
14.04	-3.959e-08\\
14.06	1.48e-07\\
14.07	2.402e-07\\
14.08	3.301e-07\\
14.09	4.169e-07\\
14.1	4.998e-07\\
14.11	5.779e-07\\
14.12	6.507e-07\\
14.13	7.173e-07\\
14.14	7.773e-07\\
14.15	8.302e-07\\
14.16	8.756e-07\\
14.17	9.133e-07\\
14.18	9.43e-07\\
14.19	9.646e-07\\
14.2	9.782e-07\\
14.21	9.839e-07\\
14.22	9.819e-07\\
14.23	9.725e-07\\
14.24	9.561e-07\\
14.25	9.331e-07\\
14.26	9.04e-07\\
14.27	8.696e-07\\
14.28	8.303e-07\\
14.29	7.87e-07\\
14.3	7.403e-07\\
14.31	6.909e-07\\
14.33	5.873e-07\\
14.36	4.301e-07\\
14.37	3.799e-07\\
14.38	3.318e-07\\
14.39	2.863e-07\\
14.4	2.437e-07\\
14.41	2.045e-07\\
14.42	1.689e-07\\
14.43	1.371e-07\\
14.44	1.092e-07\\
14.45	8.511e-08\\
14.46	6.49e-08\\
14.47	4.837e-08\\
14.48	3.529e-08\\
14.49	2.539e-08\\
14.5	1.829e-08\\
14.51	1.359e-08\\
14.53	9.498e-09\\
14.57	7.809e-09\\
14.58	5.562e-09\\
14.59	1.536e-09\\
14.6	-4.743e-09\\
14.61	-1.371e-08\\
14.62	-2.573e-08\\
14.63	-4.112e-08\\
14.64	-6.013e-08\\
14.65	-8.293e-08\\
14.66	-1.096e-07\\
14.67	-1.402e-07\\
14.68	-1.746e-07\\
14.69	-2.127e-07\\
14.7	-2.542e-07\\
14.71	-2.988e-07\\
14.72	-3.461e-07\\
14.73	-3.957e-07\\
14.75	-4.994e-07\\
14.78	-6.566e-07\\
14.79	-7.066e-07\\
14.8	-7.543e-07\\
14.81	-7.988e-07\\
14.82	-8.394e-07\\
14.83	-8.755e-07\\
14.84	-9.063e-07\\
14.85	-9.313e-07\\
14.86	-9.499e-07\\
14.87	-9.616e-07\\
14.88	-9.661e-07\\
14.89	-9.629e-07\\
14.9	-9.518e-07\\
14.91	-9.328e-07\\
14.92	-9.057e-07\\
14.93	-8.706e-07\\
14.94	-8.276e-07\\
14.95	-7.771e-07\\
14.96	-7.193e-07\\
14.97	-6.547e-07\\
14.98	-5.839e-07\\
14.99	-5.074e-07\\
15	-4.26e-07\\
15.01	-3.403e-07\\
15.02	-2.513e-07\\
15.03	-1.597e-07\\
15.07	2.139e-07\\
15.08	3.046e-07\\
15.09	3.924e-07\\
15.1	4.765e-07\\
15.11	5.561e-07\\
15.12	6.304e-07\\
15.13	6.989e-07\\
15.14	7.608e-07\\
15.15	8.158e-07\\
15.16	8.634e-07\\
15.17	9.033e-07\\
15.18	9.353e-07\\
15.19	9.593e-07\\
15.2	9.752e-07\\
15.21	9.831e-07\\
15.22	9.833e-07\\
15.23	9.76e-07\\
15.24	9.615e-07\\
15.25	9.403e-07\\
15.26	9.13e-07\\
15.27	8.8e-07\\
15.28	8.421e-07\\
15.29	7.999e-07\\
15.3	7.54e-07\\
15.31	7.054e-07\\
15.33	6.024e-07\\
15.36	4.448e-07\\
15.37	3.941e-07\\
15.38	3.454e-07\\
15.39	2.991e-07\\
15.4	2.556e-07\\
15.41	2.154e-07\\
15.42	1.788e-07\\
15.43	1.458e-07\\
15.44	1.168e-07\\
15.45	9.162e-08\\
15.46	7.032e-08\\
15.47	5.275e-08\\
15.48	3.871e-08\\
15.49	2.793e-08\\
15.5	2.006e-08\\
15.51	1.472e-08\\
15.52	1.145e-08\\
15.54	9.139e-09\\
15.57	8.197e-09\\
15.58	6.367e-09\\
15.59	2.902e-09\\
15.6	-2.685e-09\\
15.61	-1.084e-08\\
15.62	-2.194e-08\\
15.63	-3.634e-08\\
15.64	-5.43e-08\\
15.65	-7.6e-08\\
15.66	-1.016e-07\\
15.67	-1.31e-07\\
15.68	-1.644e-07\\
15.69	-2.014e-07\\
15.7	-2.42e-07\\
15.71	-2.857e-07\\
15.72	-3.323e-07\\
15.73	-3.813e-07\\
15.75	-4.843e-07\\
15.78	-6.419e-07\\
15.79	-6.925e-07\\
15.8	-7.409e-07\\
15.81	-7.864e-07\\
15.82	-8.282e-07\\
15.83	-8.657e-07\\
15.84	-8.981e-07\\
15.85	-9.248e-07\\
15.86	-9.453e-07\\
15.87	-9.591e-07\\
15.88	-9.656e-07\\
15.89	-9.646e-07\\
15.9	-9.559e-07\\
15.91	-9.391e-07\\
15.92	-9.143e-07\\
15.93	-8.815e-07\\
15.94	-8.408e-07\\
15.95	-7.924e-07\\
15.96	-7.366e-07\\
15.97	-6.74e-07\\
15.98	-6.049e-07\\
15.99	-5.299e-07\\
16	-4.498e-07\\
16.01	-3.653e-07\\
16.02	-2.772e-07\\
16.03	-1.862e-07\\
16.05	4.621e-10\\
16.07	1.875e-07\\
16.08	2.788e-07\\
16.09	3.676e-07\\
16.1	4.528e-07\\
16.11	5.338e-07\\
16.12	6.097e-07\\
16.13	6.799e-07\\
16.14	7.438e-07\\
16.15	8.008e-07\\
16.16	8.506e-07\\
16.17	8.927e-07\\
16.18	9.27e-07\\
16.19	9.533e-07\\
16.2	9.715e-07\\
16.21	9.817e-07\\
16.22	9.841e-07\\
16.23	9.789e-07\\
16.24	9.664e-07\\
16.25	9.471e-07\\
16.26	9.215e-07\\
16.27	8.901e-07\\
16.28	8.535e-07\\
16.29	8.124e-07\\
16.3	7.675e-07\\
16.31	7.196e-07\\
16.32	6.694e-07\\
16.34	5.648e-07\\
16.36	4.597e-07\\
16.37	4.085e-07\\
16.38	3.592e-07\\
16.39	3.121e-07\\
16.4	2.678e-07\\
16.41	2.266e-07\\
16.42	1.889e-07\\
16.43	1.549e-07\\
16.44	1.247e-07\\
16.45	9.845e-08\\
16.46	7.605e-08\\
16.47	5.742e-08\\
16.48	4.239e-08\\
16.49	3.071e-08\\
16.5	2.204e-08\\
16.51	1.602e-08\\
16.52	1.22e-08\\
16.54	9.241e-09\\
16.58	7.038e-09\\
16.59	4.093e-09\\
16.6	-8.386e-10\\
16.61	-8.212e-09\\
16.62	-1.844e-08\\
16.63	-3.186e-08\\
16.64	-4.877e-08\\
16.65	-6.938e-08\\
16.66	-9.384e-08\\
16.67	-1.222e-07\\
16.68	-1.544e-07\\
16.69	-1.904e-07\\
16.7	-2.3e-07\\
16.71	-2.729e-07\\
16.72	-3.187e-07\\
16.73	-3.671e-07\\
16.75	-4.693e-07\\
16.78	-6.271e-07\\
16.79	-6.782e-07\\
16.8	-7.273e-07\\
16.81	-7.737e-07\\
16.82	-8.167e-07\\
16.83	-8.555e-07\\
16.84	-8.894e-07\\
16.85	-9.178e-07\\
16.86	-9.401e-07\\
16.87	-9.559e-07\\
16.88	-9.645e-07\\
16.89	-9.657e-07\\
16.9	-9.592e-07\\
16.91	-9.448e-07\\
16.92	-9.223e-07\\
16.93	-8.918e-07\\
16.94	-8.533e-07\\
16.95	-8.071e-07\\
16.96	-7.534e-07\\
16.97	-6.926e-07\\
16.98	-6.253e-07\\
16.99	-5.52e-07\\
17	-4.733e-07\\
17.01	-3.9e-07\\
17.02	-3.028e-07\\
17.03	-2.126e-07\\
17.05	-2.652e-08\\
17.07	1.609e-07\\
17.08	2.529e-07\\
17.09	3.424e-07\\
17.1	4.288e-07\\
17.11	5.11e-07\\
17.12	5.885e-07\\
17.13	6.604e-07\\
17.14	7.262e-07\\
17.15	7.852e-07\\
17.16	8.371e-07\\
17.17	8.815e-07\\
17.18	9.18e-07\\
17.19	9.466e-07\\
17.2	9.671e-07\\
17.21	9.796e-07\\
17.22	9.842e-07\\
17.23	9.812e-07\\
17.24	9.708e-07\\
17.25	9.534e-07\\
17.26	9.295e-07\\
17.27	8.997e-07\\
17.28	8.645e-07\\
17.29	8.247e-07\\
17.3	7.808e-07\\
17.31	7.337e-07\\
17.32	6.84e-07\\
17.34	5.8e-07\\
17.36	4.746e-07\\
17.37	4.231e-07\\
17.38	3.731e-07\\
17.39	3.254e-07\\
17.4	2.802e-07\\
17.41	2.381e-07\\
17.42	1.994e-07\\
17.43	1.643e-07\\
17.44	1.33e-07\\
17.45	1.056e-07\\
17.46	8.208e-08\\
17.47	6.239e-08\\
17.48	4.635e-08\\
17.49	3.374e-08\\
17.5	2.424e-08\\
17.51	1.75e-08\\
17.52	1.31e-08\\
17.54	9.398e-09\\
17.58	7.585e-09\\
17.59	5.12e-09\\
17.6	8.069e-10\\
17.61	-5.822e-09\\
17.62	-1.519e-08\\
17.63	-2.766e-08\\
17.64	-4.355e-08\\
17.65	-6.308e-08\\
17.66	-8.643e-08\\
17.67	-1.137e-07\\
17.68	-1.448e-07\\
17.69	-1.797e-07\\
17.7	-2.183e-07\\
17.71	-2.603e-07\\
17.72	-3.053e-07\\
17.73	-3.53e-07\\
17.74	-4.028e-07\\
17.76	-5.068e-07\\
17.78	-6.122e-07\\
17.79	-6.638e-07\\
17.8	-7.135e-07\\
17.81	-7.608e-07\\
17.82	-8.048e-07\\
17.83	-8.448e-07\\
17.84	-8.802e-07\\
17.85	-9.103e-07\\
17.86	-9.344e-07\\
17.87	-9.521e-07\\
17.88	-9.628e-07\\
17.89	-9.662e-07\\
17.9	-9.619e-07\\
17.91	-9.498e-07\\
17.92	-9.296e-07\\
17.93	-9.014e-07\\
17.94	-8.652e-07\\
17.95	-8.211e-07\\
17.96	-7.696e-07\\
17.97	-7.108e-07\\
17.98	-6.453e-07\\
17.99	-5.737e-07\\
18	-4.964e-07\\
18.01	-4.144e-07\\
18.02	-3.282e-07\\
18.03	-2.387e-07\\
18.04	-1.469e-07\\
18.08	2.267e-07\\
18.09	3.171e-07\\
18.1	4.044e-07\\
18.11	4.879e-07\\
18.12	5.668e-07\\
18.13	6.404e-07\\
18.14	7.08e-07\\
18.15	7.69e-07\\
18.16	8.23e-07\\
18.17	8.696e-07\\
18.18	9.084e-07\\
18.19	9.393e-07\\
18.2	9.621e-07\\
18.21	9.769e-07\\
18.22	9.838e-07\\
18.23	9.829e-07\\
18.24	9.745e-07\\
18.25	9.591e-07\\
18.26	9.37e-07\\
18.27	9.089e-07\\
18.28	8.752e-07\\
18.29	8.366e-07\\
18.3	7.938e-07\\
18.31	7.475e-07\\
18.32	6.985e-07\\
18.34	5.952e-07\\
18.37	4.377e-07\\
18.38	3.873e-07\\
18.39	3.388e-07\\
18.4	2.929e-07\\
18.41	2.499e-07\\
18.42	2.101e-07\\
18.43	1.74e-07\\
18.44	1.416e-07\\
18.45	1.131e-07\\
18.46	8.844e-08\\
18.47	6.766e-08\\
18.48	5.059e-08\\
18.49	3.702e-08\\
18.5	2.667e-08\\
18.51	1.918e-08\\
18.52	1.415e-08\\
18.54	9.625e-09\\
18.59	5.995e-09\\
18.6	2.262e-09\\
18.61	-3.658e-09\\
18.62	-1.22e-08\\
18.63	-2.375e-08\\
18.64	-3.863e-08\\
18.65	-5.71e-08\\
18.66	-7.934e-08\\
18.67	-1.055e-07\\
18.68	-1.355e-07\\
18.69	-1.693e-07\\
18.7	-2.069e-07\\
18.71	-2.479e-07\\
18.72	-2.921e-07\\
18.73	-3.391e-07\\
18.74	-3.883e-07\\
18.76	-4.917e-07\\
18.79	-6.492e-07\\
18.8	-6.995e-07\\
18.81	-7.476e-07\\
18.82	-7.926e-07\\
18.83	-8.338e-07\\
18.84	-8.706e-07\\
18.85	-9.023e-07\\
18.86	-9.282e-07\\
18.87	-9.478e-07\\
18.88	-9.605e-07\\
18.89	-9.661e-07\\
18.9	-9.64e-07\\
18.91	-9.541e-07\\
18.92	-9.363e-07\\
18.93	-9.103e-07\\
18.94	-8.764e-07\\
18.95	-8.346e-07\\
18.96	-7.852e-07\\
18.97	-7.284e-07\\
18.98	-6.648e-07\\
18.99	-5.948e-07\\
19	-5.192e-07\\
19.01	-4.384e-07\\
19.02	-3.533e-07\\
19.03	-2.647e-07\\
19.04	-1.734e-07\\
19.06	1.355e-08\\
19.08	2.004e-07\\
19.09	2.914e-07\\
19.1	3.797e-07\\
19.11	4.644e-07\\
19.12	5.447e-07\\
19.13	6.199e-07\\
19.14	6.893e-07\\
19.15	7.522e-07\\
19.16	8.083e-07\\
19.17	8.57e-07\\
19.18	8.981e-07\\
19.19	9.313e-07\\
19.2	9.564e-07\\
19.21	9.735e-07\\
19.22	9.826e-07\\
19.23	9.839e-07\\
19.24	9.777e-07\\
19.25	9.643e-07\\
19.26	9.441e-07\\
19.27	9.176e-07\\
19.28	8.854e-07\\
19.29	8.481e-07\\
19.3	8.065e-07\\
19.31	7.612e-07\\
19.32	7.128e-07\\
19.34	6.103e-07\\
19.37	4.525e-07\\
19.38	4.016e-07\\
19.39	3.525e-07\\
19.4	3.058e-07\\
19.41	2.619e-07\\
19.42	2.212e-07\\
19.43	1.84e-07\\
19.44	1.505e-07\\
19.45	1.209e-07\\
19.46	9.511e-08\\
19.47	7.324e-08\\
19.48	5.513e-08\\
19.49	4.058e-08\\
19.5	2.933e-08\\
19.51	2.106e-08\\
19.52	1.537e-08\\
19.53	1.182e-08\\
19.55	9.186e-09\\
19.58	8.36e-09\\
19.59	6.729e-09\\
19.6	3.536e-09\\
19.61	-1.71e-09\\
19.62	-9.458e-09\\
19.63	-2.011e-08\\
19.64	-3.4e-08\\
19.65	-5.142e-08\\
19.66	-7.257e-08\\
19.67	-9.757e-08\\
19.68	-1.265e-07\\
19.69	-1.592e-07\\
19.7	-1.958e-07\\
19.71	-2.358e-07\\
19.72	-2.791e-07\\
19.73	-3.253e-07\\
19.74	-3.74e-07\\
19.76	-4.766e-07\\
19.79	-6.344e-07\\
19.8	-6.853e-07\\
19.81	-7.341e-07\\
19.82	-7.801e-07\\
19.83	-8.225e-07\\
19.84	-8.606e-07\\
19.85	-8.938e-07\\
19.86	-9.214e-07\\
19.87	-9.429e-07\\
19.88	-9.577e-07\\
19.89	-9.653e-07\\
19.9	-9.654e-07\\
19.91	-9.578e-07\\
19.92	-9.423e-07\\
19.93	-9.187e-07\\
19.94	-8.87e-07\\
19.95	-8.474e-07\\
19.96	-8.001e-07\\
19.97	-7.455e-07\\
19.98	-6.838e-07\\
19.99	-6.156e-07\\
20	-5.414e-07\\
20.01	-4.621e-07\\
20.02	-3.781e-07\\
20.03	-2.904e-07\\
20.04	-1.999e-07\\
20.06	-1.344e-08\\
20.08	1.739e-07\\
20.09	2.655e-07\\
20.1	3.547e-07\\
20.11	4.405e-07\\
20.12	5.222e-07\\
20.13	5.989e-07\\
20.14	6.7e-07\\
20.15	7.349e-07\\
20.16	7.93e-07\\
20.17	8.439e-07\\
20.18	8.871e-07\\
20.19	9.226e-07\\
20.2	9.501e-07\\
20.21	9.695e-07\\
20.22	9.809e-07\\
20.23	9.844e-07\\
20.24	9.803e-07\\
20.25	9.689e-07\\
20.26	9.506e-07\\
20.27	9.258e-07\\
20.28	8.952e-07\\
20.29	8.594e-07\\
20.3	8.189e-07\\
20.31	7.745e-07\\
20.32	7.27e-07\\
20.33	6.77e-07\\
20.35	5.727e-07\\
20.37	4.674e-07\\
20.38	4.161e-07\\
20.39	3.664e-07\\
20.4	3.19e-07\\
20.41	2.742e-07\\
20.42	2.325e-07\\
20.43	1.943e-07\\
20.44	1.597e-07\\
20.45	1.29e-07\\
20.46	1.021e-07\\
20.47	7.913e-08\\
20.48	5.995e-08\\
20.49	4.44e-08\\
20.5	3.224e-08\\
20.51	2.315e-08\\
20.52	1.676e-08\\
20.53	1.265e-08\\
20.55	9.315e-09\\
20.59	7.334e-09\\
20.6	4.641e-09\\
20.61	3.254e-11\\
20.62	-6.954e-09\\
20.63	-1.674e-08\\
20.64	-2.967e-08\\
20.65	-4.605e-08\\
20.66	-6.611e-08\\
20.67	-9e-08\\
20.68	-1.178e-07\\
20.69	-1.495e-07\\
20.7	-1.849e-07\\
20.71	-2.24e-07\\
20.72	-2.664e-07\\
20.73	-3.118e-07\\
20.74	-3.598e-07\\
20.75	-4.1e-07\\
20.77	-5.142e-07\\
20.79	-6.196e-07\\
20.8	-6.709e-07\\
20.81	-7.204e-07\\
20.82	-7.672e-07\\
20.83	-8.107e-07\\
20.84	-8.502e-07\\
20.85	-8.849e-07\\
20.86	-9.142e-07\\
20.87	-9.374e-07\\
20.88	-9.542e-07\\
20.89	-9.639e-07\\
20.9	-9.662e-07\\
20.91	-9.609e-07\\
20.92	-9.476e-07\\
20.93	-9.263e-07\\
20.94	-8.969e-07\\
20.95	-8.596e-07\\
20.96	-8.145e-07\\
20.97	-7.619e-07\\
20.98	-7.022e-07\\
20.99	-6.358e-07\\
21	-5.633e-07\\
21.01	-4.854e-07\\
21.02	-4.027e-07\\
21.03	-3.16e-07\\
21.04	-2.261e-07\\
21.06	-4.041e-08\\
21.08	1.472e-07\\
21.09	2.395e-07\\
21.1	3.295e-07\\
21.11	4.163e-07\\
21.12	4.993e-07\\
21.13	5.775e-07\\
21.14	6.503e-07\\
21.15	7.17e-07\\
21.16	7.771e-07\\
21.17	8.301e-07\\
21.18	8.756e-07\\
21.19	9.133e-07\\
21.2	9.431e-07\\
21.21	9.648e-07\\
21.22	9.785e-07\\
21.23	9.842e-07\\
21.24	9.823e-07\\
21.25	9.729e-07\\
21.26	9.565e-07\\
21.27	9.336e-07\\
21.28	9.046e-07\\
21.29	8.702e-07\\
21.3	8.31e-07\\
21.31	7.877e-07\\
21.32	7.41e-07\\
21.33	6.916e-07\\
21.35	5.879e-07\\
21.38	4.307e-07\\
21.39	3.805e-07\\
21.4	3.323e-07\\
21.41	2.868e-07\\
21.42	2.442e-07\\
21.43	2.049e-07\\
21.44	1.693e-07\\
21.45	1.374e-07\\
21.46	1.094e-07\\
21.47	8.533e-08\\
21.48	6.508e-08\\
21.49	4.851e-08\\
21.5	3.54e-08\\
21.51	2.547e-08\\
21.52	1.835e-08\\
21.53	1.362e-08\\
21.55	9.507e-09\\
21.59	7.823e-09\\
21.6	5.589e-09\\
21.61	1.579e-09\\
21.62	-4.682e-09\\
21.63	-1.362e-08\\
21.64	-2.562e-08\\
21.65	-4.099e-08\\
21.66	-5.997e-08\\
21.67	-8.275e-08\\
21.68	-1.094e-07\\
21.69	-1.4e-07\\
21.7	-1.744e-07\\
21.71	-2.124e-07\\
21.72	-2.539e-07\\
21.73	-2.985e-07\\
21.74	-3.458e-07\\
21.75	-3.954e-07\\
21.77	-4.991e-07\\
21.8	-6.564e-07\\
21.81	-7.065e-07\\
21.82	-7.541e-07\\
21.83	-7.987e-07\\
21.84	-8.394e-07\\
21.85	-8.755e-07\\
21.86	-9.064e-07\\
21.87	-9.314e-07\\
21.88	-9.501e-07\\
21.89	-9.619e-07\\
21.9	-9.664e-07\\
21.91	-9.633e-07\\
21.92	-9.523e-07\\
21.93	-9.333e-07\\
21.94	-9.062e-07\\
21.95	-8.712e-07\\
21.96	-8.283e-07\\
21.97	-7.778e-07\\
21.98	-7.201e-07\\
21.99	-6.555e-07\\
22	-5.847e-07\\
22.01	-5.083e-07\\
22.02	-4.268e-07\\
22.03	-3.412e-07\\
22.04	-2.522e-07\\
22.05	-1.606e-07\\
22.09	2.132e-07\\
22.1	3.039e-07\\
22.11	3.918e-07\\
22.12	4.759e-07\\
22.13	5.556e-07\\
22.14	6.3e-07\\
22.15	6.985e-07\\
22.16	7.606e-07\\
22.17	8.156e-07\\
22.18	8.633e-07\\
22.19	9.033e-07\\
22.2	9.354e-07\\
22.21	9.594e-07\\
22.22	9.754e-07\\
22.23	9.834e-07\\
22.24	9.837e-07\\
22.25	9.764e-07\\
22.26	9.62e-07\\
22.27	9.409e-07\\
22.28	9.136e-07\\
22.29	8.806e-07\\
22.3	8.427e-07\\
22.31	8.005e-07\\
22.32	7.547e-07\\
22.33	7.06e-07\\
22.35	6.031e-07\\
22.38	4.454e-07\\
22.39	3.947e-07\\
22.4	3.459e-07\\
22.41	2.996e-07\\
22.42	2.561e-07\\
22.43	2.158e-07\\
22.44	1.791e-07\\
22.45	1.462e-07\\
22.46	1.171e-07\\
22.47	9.185e-08\\
22.48	7.051e-08\\
22.49	5.29e-08\\
22.5	3.883e-08\\
22.51	2.801e-08\\
22.52	2.013e-08\\
22.53	1.476e-08\\
22.54	1.147e-08\\
22.56	9.143e-09\\
22.59	8.208e-09\\
22.6	6.39e-09\\
22.61	2.939e-09\\
22.62	-2.629e-09\\
22.63	-1.076e-08\\
22.64	-2.184e-08\\
22.65	-3.622e-08\\
22.66	-5.415e-08\\
22.67	-7.582e-08\\
22.68	-1.014e-07\\
22.69	-1.308e-07\\
22.7	-1.641e-07\\
22.71	-2.012e-07\\
22.72	-2.417e-07\\
22.73	-2.854e-07\\
22.74	-3.32e-07\\
22.75	-3.81e-07\\
22.77	-4.84e-07\\
22.8	-6.417e-07\\
22.81	-6.923e-07\\
22.82	-7.408e-07\\
22.83	-7.863e-07\\
22.84	-8.282e-07\\
22.85	-8.657e-07\\
22.86	-8.981e-07\\
22.87	-9.249e-07\\
22.88	-9.455e-07\\
22.89	-9.593e-07\\
22.9	-9.659e-07\\
22.91	-9.65e-07\\
22.92	-9.563e-07\\
22.93	-9.396e-07\\
22.94	-9.149e-07\\
22.95	-8.821e-07\\
22.96	-8.414e-07\\
22.97	-7.931e-07\\
22.98	-7.374e-07\\
22.99	-6.747e-07\\
23	-6.057e-07\\
23.01	-5.308e-07\\
23.02	-4.507e-07\\
23.03	-3.662e-07\\
23.04	-2.78e-07\\
23.05	-1.871e-07\\
23.07	-3.478e-10\\
23.09	1.868e-07\\
23.1	2.782e-07\\
23.11	3.669e-07\\
23.12	4.522e-07\\
23.13	5.333e-07\\
23.14	6.093e-07\\
23.15	6.795e-07\\
23.16	7.435e-07\\
23.17	8.006e-07\\
23.18	8.505e-07\\
23.19	8.927e-07\\
23.2	9.27e-07\\
23.21	9.534e-07\\
23.22	9.717e-07\\
23.23	9.82e-07\\
23.24	9.844e-07\\
23.25	9.793e-07\\
23.26	9.669e-07\\
23.27	9.476e-07\\
23.28	9.22e-07\\
23.29	8.907e-07\\
23.3	8.541e-07\\
23.31	8.131e-07\\
23.32	7.682e-07\\
23.33	7.203e-07\\
23.34	6.7e-07\\
23.36	5.655e-07\\
23.38	4.603e-07\\
23.39	4.091e-07\\
23.4	3.597e-07\\
23.41	3.126e-07\\
23.42	2.683e-07\\
23.43	2.27e-07\\
23.44	1.893e-07\\
23.45	1.552e-07\\
23.46	1.25e-07\\
23.47	9.869e-08\\
23.48	7.625e-08\\
23.49	5.758e-08\\
23.5	4.252e-08\\
23.51	3.08e-08\\
23.52	2.211e-08\\
23.53	1.607e-08\\
23.54	1.223e-08\\
23.56	9.246e-09\\
23.6	7.057e-09\\
23.61	4.126e-09\\
23.62	-7.882e-10\\
23.63	-8.141e-09\\
23.64	-1.834e-08\\
23.65	-3.174e-08\\
23.66	-4.863e-08\\
23.67	-6.922e-08\\
23.68	-9.364e-08\\
23.69	-1.22e-07\\
23.7	-1.542e-07\\
23.71	-1.902e-07\\
23.72	-2.297e-07\\
23.73	-2.726e-07\\
23.74	-3.184e-07\\
23.75	-3.668e-07\\
23.77	-4.69e-07\\
23.8	-6.269e-07\\
23.81	-6.78e-07\\
23.82	-7.272e-07\\
23.83	-7.736e-07\\
23.84	-8.166e-07\\
23.85	-8.554e-07\\
23.86	-8.894e-07\\
23.87	-9.179e-07\\
23.88	-9.403e-07\\
23.89	-9.561e-07\\
23.9	-9.648e-07\\
23.91	-9.661e-07\\
23.92	-9.596e-07\\
23.93	-9.453e-07\\
23.94	-9.228e-07\\
23.95	-8.924e-07\\
23.96	-8.539e-07\\
23.97	-8.078e-07\\
23.98	-7.541e-07\\
23.99	-6.934e-07\\
24	-6.262e-07\\
24.01	-5.529e-07\\
24.02	-4.742e-07\\
24.03	-3.909e-07\\
24.04	-3.037e-07\\
24.05	-2.134e-07\\
24.07	-2.734e-08\\
24.09	1.602e-07\\
24.1	2.522e-07\\
24.11	3.418e-07\\
24.12	4.282e-07\\
24.13	5.105e-07\\
24.14	5.88e-07\\
24.15	6.6e-07\\
24.16	7.259e-07\\
24.17	7.85e-07\\
24.18	8.37e-07\\
24.19	8.814e-07\\
24.2	9.18e-07\\
24.21	9.467e-07\\
24.22	9.673e-07\\
24.23	9.799e-07\\
24.24	9.846e-07\\
24.25	9.816e-07\\
24.26	9.712e-07\\
24.27	9.539e-07\\
24.28	9.3e-07\\
24.29	9.003e-07\\
24.3	8.651e-07\\
24.31	8.253e-07\\
24.32	7.815e-07\\
24.33	7.343e-07\\
24.34	6.847e-07\\
24.36	5.806e-07\\
24.38	4.752e-07\\
24.39	4.236e-07\\
24.4	3.737e-07\\
24.41	3.259e-07\\
24.42	2.807e-07\\
24.43	2.385e-07\\
24.44	1.998e-07\\
24.45	1.646e-07\\
24.46	1.333e-07\\
24.47	1.058e-07\\
24.48	8.23e-08\\
24.49	6.256e-08\\
24.5	4.649e-08\\
24.51	3.384e-08\\
24.52	2.432e-08\\
24.53	1.756e-08\\
24.54	1.313e-08\\
24.56	9.406e-09\\
24.6	7.601e-09\\
24.61	5.148e-09\\
24.62	8.52e-10\\
24.63	-5.758e-09\\
24.64	-1.51e-08\\
24.65	-2.755e-08\\
24.66	-4.341e-08\\
24.67	-6.292e-08\\
24.68	-8.624e-08\\
24.69	-1.135e-07\\
24.7	-1.446e-07\\
24.71	-1.795e-07\\
24.72	-2.18e-07\\
24.73	-2.6e-07\\
24.74	-3.05e-07\\
24.75	-3.527e-07\\
24.76	-4.025e-07\\
24.78	-5.065e-07\\
24.8	-6.12e-07\\
24.81	-6.636e-07\\
24.82	-7.134e-07\\
24.83	-7.607e-07\\
24.84	-8.047e-07\\
24.85	-8.448e-07\\
24.86	-8.802e-07\\
24.87	-9.104e-07\\
24.88	-9.346e-07\\
24.89	-9.523e-07\\
24.9	-9.631e-07\\
24.91	-9.666e-07\\
24.92	-9.623e-07\\
24.93	-9.502e-07\\
24.94	-9.301e-07\\
24.95	-9.02e-07\\
24.96	-8.658e-07\\
24.97	-8.218e-07\\
24.98	-7.703e-07\\
24.99	-7.116e-07\\
25	-6.461e-07\\
25.01	-5.745e-07\\
25.02	-4.973e-07\\
25.03	-4.152e-07\\
25.04	-3.291e-07\\
25.05	-2.396e-07\\
25.06	-1.477e-07\\
25.1	2.26e-07\\
25.11	3.164e-07\\
25.12	4.038e-07\\
25.13	4.874e-07\\
25.14	5.664e-07\\
25.15	6.4e-07\\
25.16	7.077e-07\\
25.17	7.688e-07\\
25.18	8.229e-07\\
25.19	8.695e-07\\
25.2	9.084e-07\\
25.21	9.393e-07\\
25.22	9.623e-07\\
25.23	9.771e-07\\
25.24	9.841e-07\\
25.25	9.832e-07\\
25.26	9.75e-07\\
25.27	9.596e-07\\
25.28	9.376e-07\\
25.29	9.094e-07\\
25.3	8.758e-07\\
25.31	8.372e-07\\
25.32	7.944e-07\\
25.33	7.482e-07\\
25.34	6.992e-07\\
25.36	5.958e-07\\
25.39	4.383e-07\\
25.4	3.878e-07\\
25.41	3.394e-07\\
25.42	2.934e-07\\
25.43	2.503e-07\\
25.44	2.105e-07\\
25.45	1.743e-07\\
25.46	1.419e-07\\
25.47	1.133e-07\\
25.48	8.866e-08\\
25.49	6.785e-08\\
25.5	5.074e-08\\
25.51	3.714e-08\\
25.52	2.675e-08\\
25.53	1.924e-08\\
25.54	1.419e-08\\
25.56	9.635e-09\\
25.61	6.019e-09\\
25.62	2.301e-09\\
25.63	-3.599e-09\\
25.64	-1.212e-08\\
25.65	-2.365e-08\\
25.66	-3.85e-08\\
25.67	-5.694e-08\\
25.68	-7.916e-08\\
25.69	-1.052e-07\\
25.7	-1.352e-07\\
25.71	-1.691e-07\\
25.72	-2.066e-07\\
25.73	-2.476e-07\\
25.74	-2.918e-07\\
25.75	-3.388e-07\\
25.76	-3.88e-07\\
25.78	-4.914e-07\\
25.81	-6.489e-07\\
25.82	-6.993e-07\\
25.83	-7.474e-07\\
25.84	-7.925e-07\\
25.85	-8.338e-07\\
25.86	-8.706e-07\\
25.87	-9.023e-07\\
25.88	-9.283e-07\\
25.89	-9.48e-07\\
25.9	-9.608e-07\\
25.91	-9.664e-07\\
25.92	-9.644e-07\\
25.93	-9.546e-07\\
25.94	-9.368e-07\\
25.95	-9.109e-07\\
25.96	-8.77e-07\\
25.97	-8.353e-07\\
25.98	-7.859e-07\\
25.99	-7.292e-07\\
26	-6.656e-07\\
26.01	-5.957e-07\\
26.02	-5.2e-07\\
26.03	-4.392e-07\\
26.04	-3.542e-07\\
26.05	-2.656e-07\\
26.06	-1.743e-07\\
26.08	1.275e-08\\
26.1	1.996e-07\\
26.11	2.907e-07\\
26.12	3.791e-07\\
26.13	4.639e-07\\
26.14	5.442e-07\\
26.15	6.195e-07\\
26.16	6.889e-07\\
26.17	7.52e-07\\
26.18	8.081e-07\\
26.19	8.569e-07\\
26.2	8.981e-07\\
26.21	9.313e-07\\
26.22	9.566e-07\\
26.23	9.737e-07\\
26.24	9.829e-07\\
26.25	9.843e-07\\
26.26	9.781e-07\\
26.27	9.647e-07\\
26.28	9.446e-07\\
26.29	9.181e-07\\
26.3	8.86e-07\\
26.31	8.488e-07\\
26.32	8.071e-07\\
26.33	7.618e-07\\
26.34	7.135e-07\\
26.36	6.11e-07\\
26.39	4.531e-07\\
26.4	4.022e-07\\
26.41	3.531e-07\\
26.42	3.063e-07\\
26.43	2.624e-07\\
26.44	2.216e-07\\
26.45	1.844e-07\\
26.46	1.508e-07\\
26.47	1.211e-07\\
26.48	9.535e-08\\
26.49	7.344e-08\\
26.5	5.528e-08\\
26.51	4.07e-08\\
26.52	2.942e-08\\
26.53	2.112e-08\\
26.54	1.541e-08\\
26.55	1.185e-08\\
26.57	9.19e-09\\
26.6	8.37e-09\\
26.61	6.75e-09\\
26.62	3.571e-09\\
26.63	-1.656e-09\\
26.64	-9.384e-09\\
26.65	-2.001e-08\\
26.66	-3.388e-08\\
26.67	-5.128e-08\\
26.68	-7.239e-08\\
26.69	-9.737e-08\\
26.7	-1.262e-07\\
26.71	-1.59e-07\\
26.72	-1.955e-07\\
26.73	-2.355e-07\\
26.74	-2.788e-07\\
26.75	-3.25e-07\\
26.76	-3.737e-07\\
26.78	-4.763e-07\\
26.81	-6.342e-07\\
26.82	-6.851e-07\\
26.83	-7.339e-07\\
26.84	-7.799e-07\\
26.85	-8.224e-07\\
26.86	-8.606e-07\\
26.87	-8.939e-07\\
26.88	-9.215e-07\\
26.89	-9.43e-07\\
26.9	-9.579e-07\\
26.91	-9.656e-07\\
26.92	-9.658e-07\\
26.93	-9.583e-07\\
26.94	-9.428e-07\\
26.95	-9.192e-07\\
26.96	-8.876e-07\\
26.97	-8.481e-07\\
26.98	-8.009e-07\\
26.99	-7.462e-07\\
27	-6.846e-07\\
27.01	-6.164e-07\\
27.02	-5.423e-07\\
27.03	-4.629e-07\\
27.04	-3.79e-07\\
27.05	-2.913e-07\\
27.06	-2.007e-07\\
27.08	-1.425e-08\\
27.1	1.731e-07\\
27.11	2.649e-07\\
27.12	3.541e-07\\
27.13	4.4e-07\\
27.14	5.217e-07\\
27.15	5.985e-07\\
27.16	6.697e-07\\
27.17	7.346e-07\\
27.18	7.928e-07\\
27.19	8.437e-07\\
27.2	8.871e-07\\
27.21	9.226e-07\\
27.22	9.502e-07\\
27.23	9.697e-07\\
27.24	9.811e-07\\
27.25	9.847e-07\\
27.26	9.807e-07\\
27.27	9.693e-07\\
27.28	9.511e-07\\
27.29	9.264e-07\\
27.3	8.958e-07\\
27.31	8.6e-07\\
27.32	8.195e-07\\
27.33	7.752e-07\\
27.34	7.277e-07\\
27.35	6.777e-07\\
27.37	5.734e-07\\
27.39	4.68e-07\\
27.4	4.166e-07\\
27.41	3.669e-07\\
27.42	3.195e-07\\
27.43	2.747e-07\\
27.44	2.33e-07\\
27.45	1.947e-07\\
27.46	1.601e-07\\
27.47	1.293e-07\\
27.48	1.024e-07\\
27.49	7.934e-08\\
27.5	6.012e-08\\
27.51	4.454e-08\\
27.52	3.234e-08\\
27.53	2.322e-08\\
27.54	1.681e-08\\
27.55	1.268e-08\\
27.57	9.321e-09\\
27.61	7.352e-09\\
27.62	4.672e-09\\
27.63	8.015e-11\\
27.64	-6.887e-09\\
27.65	-1.665e-08\\
27.66	-2.956e-08\\
27.67	-4.591e-08\\
27.68	-6.595e-08\\
27.69	-8.981e-08\\
27.7	-1.176e-07\\
27.71	-1.492e-07\\
27.72	-1.847e-07\\
27.73	-2.237e-07\\
27.74	-2.661e-07\\
27.75	-3.115e-07\\
27.76	-3.595e-07\\
27.77	-4.097e-07\\
27.79	-5.139e-07\\
27.81	-6.193e-07\\
27.82	-6.707e-07\\
27.83	-7.202e-07\\
27.84	-7.671e-07\\
27.85	-8.107e-07\\
27.86	-8.502e-07\\
27.87	-8.849e-07\\
27.88	-9.142e-07\\
27.89	-9.376e-07\\
27.9	-9.544e-07\\
27.91	-9.642e-07\\
27.92	-9.666e-07\\
27.93	-9.613e-07\\
27.94	-9.481e-07\\
27.95	-9.268e-07\\
27.96	-8.975e-07\\
27.97	-8.603e-07\\
27.98	-8.152e-07\\
27.99	-7.627e-07\\
28	-7.03e-07\\
28.01	-6.366e-07\\
28.02	-5.642e-07\\
28.03	-4.862e-07\\
28.04	-4.035e-07\\
28.05	-3.168e-07\\
28.06	-2.27e-07\\
28.08	-4.123e-08\\
28.1	1.465e-07\\
28.11	2.388e-07\\
28.12	3.288e-07\\
28.13	4.157e-07\\
28.14	4.987e-07\\
28.15	5.77e-07\\
28.16	6.499e-07\\
28.17	7.167e-07\\
28.18	7.769e-07\\
28.19	8.299e-07\\
28.2	8.755e-07\\
28.21	9.133e-07\\
28.22	9.431e-07\\
28.23	9.649e-07\\
28.24	9.787e-07\\
28.25	9.845e-07\\
28.26	9.827e-07\\
28.27	9.734e-07\\
28.28	9.57e-07\\
28.29	9.341e-07\\
28.3	9.052e-07\\
28.31	8.708e-07\\
28.32	8.316e-07\\
28.33	7.883e-07\\
28.34	7.416e-07\\
28.35	6.923e-07\\
28.37	5.886e-07\\
28.4	4.313e-07\\
28.41	3.81e-07\\
28.42	3.328e-07\\
28.43	2.872e-07\\
28.44	2.446e-07\\
28.45	2.053e-07\\
28.46	1.696e-07\\
28.47	1.377e-07\\
28.48	1.097e-07\\
28.49	8.555e-08\\
28.5	6.526e-08\\
28.51	4.865e-08\\
28.52	3.551e-08\\
28.53	2.555e-08\\
28.54	1.84e-08\\
28.55	1.366e-08\\
28.57	9.515e-09\\
28.61	7.837e-09\\
28.62	5.615e-09\\
28.63	1.621e-09\\
28.64	-4.62e-09\\
28.65	-1.354e-08\\
28.66	-2.551e-08\\
28.67	-4.085e-08\\
28.68	-5.982e-08\\
28.69	-8.257e-08\\
28.7	-1.092e-07\\
28.71	-1.397e-07\\
28.72	-1.741e-07\\
28.73	-2.122e-07\\
28.74	-2.536e-07\\
28.75	-2.982e-07\\
28.76	-3.455e-07\\
28.77	-3.951e-07\\
28.79	-4.988e-07\\
28.82	-6.562e-07\\
28.83	-7.063e-07\\
28.84	-7.54e-07\\
28.85	-7.986e-07\\
28.86	-8.393e-07\\
28.87	-8.755e-07\\
28.88	-9.065e-07\\
28.89	-9.316e-07\\
28.9	-9.503e-07\\
28.91	-9.622e-07\\
28.92	-9.667e-07\\
28.93	-9.636e-07\\
28.94	-9.527e-07\\
28.95	-9.338e-07\\
28.96	-9.068e-07\\
28.97	-8.718e-07\\
28.98	-8.29e-07\\
28.99	-7.785e-07\\
29	-7.208e-07\\
29.01	-6.563e-07\\
29.02	-5.856e-07\\
29.03	-5.091e-07\\
29.04	-4.277e-07\\
29.05	-3.421e-07\\
29.06	-2.53e-07\\
29.07	-1.614e-07\\
29.11	2.125e-07\\
29.12	3.033e-07\\
29.13	3.912e-07\\
29.14	4.754e-07\\
29.15	5.551e-07\\
29.16	6.296e-07\\
29.17	6.982e-07\\
29.18	7.603e-07\\
29.19	8.155e-07\\
29.2	8.632e-07\\
29.21	9.033e-07\\
29.22	9.355e-07\\
29.23	9.596e-07\\
29.24	9.756e-07\\
29.25	9.837e-07\\
29.26	9.84e-07\\
29.27	9.768e-07\\
29.28	9.625e-07\\
29.29	9.414e-07\\
29.3	9.141e-07\\
29.31	8.812e-07\\
29.32	8.433e-07\\
29.33	8.012e-07\\
29.34	7.554e-07\\
29.35	7.067e-07\\
29.37	6.037e-07\\
29.4	4.46e-07\\
29.41	3.953e-07\\
29.42	3.464e-07\\
29.43	3e-07\\
29.44	2.565e-07\\
29.45	2.162e-07\\
29.46	1.795e-07\\
29.47	1.465e-07\\
29.48	1.173e-07\\
29.49	9.208e-08\\
29.5	7.07e-08\\
29.51	5.305e-08\\
29.52	3.894e-08\\
29.53	2.81e-08\\
29.54	2.019e-08\\
29.55	1.48e-08\\
29.56	1.15e-08\\
29.58	9.147e-09\\
29.61	8.219e-09\\
29.62	6.412e-09\\
29.63	2.977e-09\\
29.64	-2.574e-09\\
29.65	-1.068e-08\\
29.66	-2.174e-08\\
29.67	-3.609e-08\\
29.68	-5.4e-08\\
29.69	-7.565e-08\\
29.7	-1.012e-07\\
29.71	-1.306e-07\\
29.72	-1.639e-07\\
29.73	-2.009e-07\\
29.74	-2.414e-07\\
29.75	-2.852e-07\\
29.76	-3.317e-07\\
29.77	-3.807e-07\\
29.79	-4.837e-07\\
29.82	-6.415e-07\\
29.83	-6.921e-07\\
29.84	-7.406e-07\\
29.85	-7.862e-07\\
29.86	-8.281e-07\\
29.87	-8.657e-07\\
29.88	-8.982e-07\\
29.89	-9.25e-07\\
29.9	-9.457e-07\\
29.91	-9.595e-07\\
29.92	-9.662e-07\\
29.93	-9.654e-07\\
29.94	-9.567e-07\\
29.95	-9.401e-07\\
29.96	-9.154e-07\\
29.97	-8.827e-07\\
29.98	-8.421e-07\\
29.99	-7.938e-07\\
};
\end{axis}
\end{tikzpicture}%
    \caption{Course of the instantaneous energy difference over time for the MP-based computation. The time step was $0.01 \,s$ and the end time was $30\,s$}
    \label{fig: EnergyDiffPlot}
\end{figure}




Finally, a convergence analysis was carried out. The analytical solution was taken as benchmark for all integration schemes and a relative error $e_{Phi}$ was taken as a measurement of the deviation from the analytical solution. The relative error was calculated as follows:

\begin{align}
    e_\phi = \frac{| \phi_\mathrm{NUM}(t=T) - \phi_\mathrm{ANA}(t=T)| }{| \phi_{ANA}(t=T)|}
\end{align}

where $t$ is the time variable and $T$ is the total simulation time. $\phi_{NUM}$ is the numerically calculated angle and $\phi_{ANA}$ is the analytically calculated. This quantity was computed at different time steps, ranging from $10^{-6} \, s$ to $5\cdot 10^{-3} \,s$. The resulting MSEs were plotted against the time step, both in logarithmic scale, in order to study the experimental order of convergence (EOC) of the integration methods. This is shown in Figure \ref{fig: ConvergencePlot}.


\begin{figure}[h]
    \centering
    \setlength{\figH}{0.3\textheight}
    \setlength{\figW}{0.6\textwidth}
    % This file was created by matlab2tikz.
%
%The latest updates can be retrieved from
%  http://www.mathworks.com/matlabcentral/fileexchange/22022-matlab2tikz-matlab2tikz
%where you can also make suggestions and rate matlab2tikz.
%
\begin{tikzpicture}

\begin{axis}[%
width=4.521in,
height=3.566in,
at={(0.758in,0.481in)},
scale only axis,
xmin=-14,
xmax=-5,
xlabel style={font=\color{white!15!black}},
xlabel={log(deltaT) [-]},
ymin=-16,
ymax=-2,
ylabel style={font=\color{white!15!black}},
ylabel={log(MSE) [-]},
axis background/.style={fill=white},
legend style={at={(0.03,0.97)}, anchor=north west, legend cell align=left, align=left, draw=white!15!black}
]
\addplot[only marks, mark=*, mark options={}, mark size=1.5000pt, color=red, fill=red] table[row sep=crcr]{%
x	y\\
-5.29831736654804	-3.34888343218445\\
-6.90775527898214	-5.19641096507771\\
-7.60090245954208	-5.91761490078087\\
-9.21034037197618	-7.54923216163939\\
-11.5129254649702	-9.85677520310696\\
-9.90348755253613	-8.24513522372546\\
-13.8155105579643	-12.1598554391161\\
};
\addlegendentry{EE}

\addplot[only marks, mark=*, mark options={}, mark size=1.5000pt, color=green, fill=green] table[row sep=crcr]{%
x	y\\
-5.29831736654804	-3.89905963933227\\
-6.90775527898214	-5.30578914182906\\
-7.60090245954208	-5.97201274492776\\
-9.21034037197618	-7.55964916517509\\
-11.5129254649702	-9.85729659429855\\
-9.90348755253613	-8.2500546406524\\
-13.8155105579643	-12.1593942697469\\
};
\addlegendentry{EI}

\addplot[only marks, mark=*, mark options={}, mark size=1.5000pt, color=blue, fill=blue] table[row sep=crcr]{%
x	y\\
-5.29831736654804	-7.06581278896353\\
-6.90775527898214	-8.57210772672604\\
-7.60090245954208	-9.25280583306758\\
-9.21034037197618	-10.8523570801876\\
-11.5129254649702	-13.1527265815506\\
-9.90348755253613	-11.5442729724134\\
-13.8155105579643	-15.4550903133967\\
};
\addlegendentry{MP}

\addplot [color=red, dashed, forget plot]
  table[row sep=crcr]{%
-5.29831736654804	-3.48651629820425\\
-6.90775527898214	-5.13786414312144\\
-7.60090245954208	-5.84906094915229\\
-9.21034037197618	-7.50040879406949\\
-11.5129254649702	-9.86295344501753\\
-9.90348755253613	-8.21160560010034\\
-13.8155105579643	-12.2254980959656\\
};
\addplot [color=green, dashed, forget plot]
  table[row sep=crcr]{%
-5.29831736654804	-3.7813928418328\\
-6.90775527898214	-5.35364730597459\\
-7.60090245954208	-6.03078044700671\\
-9.21034037197618	-7.60303491114851\\
-11.5129254649702	-9.85242251632242\\
-9.90348755253613	-8.28016805218062\\
-13.8155105579643	-12.1018101214963\\
};
\addplot [color=blue, dashed, forget plot]
  table[row sep=crcr]{%
-5.29831736654804	-7.00614113570796\\
-6.90775527898214	-8.59727906525393\\
-7.60090245954208	-9.28254487217082\\
-9.21034037197618	-10.8736828017168\\
-11.5129254649702	-13.1500865381797\\
-9.90348755253613	-11.5589486086337\\
-13.8155105579643	-15.4264902746425\\
};
\end{axis}
\end{tikzpicture}%
    \caption{Comparison  of the experimental mean squared error as a function of the time step for the explicit Euler (EE), implicit Euler (EI) and the middle-point rule (MP). The total simulation time and the starting conditions were chosen equal for all methods.}
    \label{fig: ConvergencePlot}
\end{figure}



\end{document}